%
% The Lion's Commentary, file ch25.tex, version 1.4, 18 May 1994
%
\se{The File ``tty.c''}

In this, the last chapter, the intricacies of interactive terminal handlers
are finally unveiled, including:

\bd
\item[(a)] the handling of the ``erase'' and
kill characters;

\item[(b)] the conversion of characters
during input and output for
upper case only terminals;

\item[(c)] the insertion of delays after
various special characters such
as ``carriage return''.
\ed

The routines ``gtty'' (8165), ``stty''
(8183), ``sgtty'' 82a1) and ``ttystty''
(8577) were dealt within the previous
chapter.

\sbs{flushtty (8252)}

The purpose of this procedure is to
``normalise'' the queries associated with
a particular terminal. Its effect is
to terminate transmission to the
terminal forthwith and to throw away any
accumulated input characters.

\bd
\item[8258:] Throw away everything in the
``cooked'' input queue;

\item[8259:] ditto for the output queue;

\item[8260:] Wakeup any process waiting to
extract a character from the
``raw'' input queue;

\item[8261:] ditto for the output queue;

\item[8263:] Raise the processor priority to
prevent an interrupt from the
terminal while ...

\item[8264:] the ``raw'' input queue is flushed,

\item[8265:] the ``delimiter count'' is properly
set to zero.
\ed

\noindent ``flushtty'' is called by ``wflushtty''
(see below) and ``ttyinput'' (8346, 8350)
when either:

\bd
\item[(a)] the terminal is not operating in
``raw'' mode and a ``quit'' or
``delete'' character is received
from the terminal; or

\item[(b)] the ``raw'' input queue has grown
unreasonably large (presumably
because no process is reading
input from the terminal);
\ed

\sbs{wflushtty (8217)}

This procedure waits until the queue of
characters for a terminal is empty
(because they've all been sent!) and
then calls ``flushtty'' to clean up the
input queues.

``wflushtty'' is called (3053) by
``klclose''. This does not happen very
often -- in fact only when all files
referencing the terminal are closed
i.e. usually only when the user logs
off.

It is also called by ``ttystty'' (8589)
just before the terminal environment
parameters are adjusted.

\sbs{Character Input} 

For a program requesting input from a
terminal, there is a chain of procedure
calls which extends to ``ttread'' ...

\sbs{ttread (8535)}

\bd
\item[8541:] Check that the terminal is
{\bf logically} active;

\item[8543:] If there are characters in the
``cooked'' input queue {\bf or} a call on
``canon'' (8274) is successful ...

\item[8544:] transfer characters from the
``cooked'' input queue until either
it is empty or enough characters
have been transferred to suit the
user's requirements.
\ed

\sbs{canon (8274)}

This procedure is called by ``ttread''
(8543) to transfer characters from the
``raw'' input queue to the ``cooked'' input
queue (after processing ``erase'' and
``kill'' characters and, in the case of
upper case only terminals, processing
``escaped'' characters, i.e. characters
preceded by the character `\verb+\+'). ``canon''
returns a non-zero value if the
``cooked'' input queue is no longer
empty.

\bd
\item[8284:] If the number of delimiters in
the ``raw'' input queue is zero
then ...

\item[8285:] if the terminal is logically
inactive, then just return;

\item[8286:] otherwise go to ``sleep''.
\ed

\noindent Note that delimiters in this context
are characters of all ones (octal value
is 377) and are inserted by ``ttyinput''
(8358).

\bd
\item[8291:] Set ``bp'' to point to the third
character of the work array,
``canonb'';

\item[8292:] Begin a loop (extending to line
8318) which removes one character
from the ``raw'' queue per cycle;

\item[8293:] If the character is a delimiter,
reduce the delimiter count by one
and exit the loop i.e. go to line
8319;

\item[8297:] If the terminal is not operating
in ``raw'' mode ...

\item[8298:] If the previous character (note
the ``bp[--1]'' notation!) was not a
backslash, `\verb+\+', execute the code
from line 8299 to 8307, otherwise
execute the code beginning at
line 8309.
\ed

\noindent {\bf Previous character was not a backslash}

\bd
\item[8299:] If the characters is an ``erase'' and ...

\item[8300:] if there is at least one charater to
erase, backup the pointer ``bp'';

\item[8302:] Start on the next cycle of the loop
beginning at line 8292;

\item[8304:] If the character is a ``kill'',
throw away all the characters
accumulated for the current line,
by going back to line 8290;

\item[8306:] If the character is an ``eot'' (004)
(usually generated at the terminal as ``control-D''),
ignore it (and do not put it inot ``canonb'') and
start on the next cycle;

(If this character occurs at
the beginning of a line, then
subsequently ``ttread'' (8544) will
find no characters in the
``cooked'' input queue i.e. it will
read a zero length record, which
then leads to the program receiving the normal ``end of file''
indication.)
\ed

\noindent{\bf Previous character was a backslash}

\bd
\item[8309:] If ``maptab[c]'' is non-zero, and either\\
``maptab[c] == c'' or the
terminal is upper case only, then ...

\item[8310:] if the last character but one was
{\bf not} a backslash (`\verb+\+'), then
replace ``c'' by ``maptab[c]'' and
back up ``bp'' (so that the
backslash will be erased).
\ed

\noindent{\bf Character ready}

\bd
\item[8315:] Move ``c'' into the next character
in ``canonb'', and if this array is
now full, leave the loop.
\ed

\noindent{\bf Line completed}

\bd
\item[8319:] At this point, an input line has
been assembled in the array
``canonb'';

\item[8322:] Shift the contents of ``canonb''
into the ``cooked'' input queue,
and return a ``successful'' result.
\ed

\sbs{Notes}

\bd
\item[(A)] The reason why ``bp'' starts (8291)
at the third character of ``canonb'' can
be found on line 8310.

\item[(B)] A number of subtleties in
the handling of backslashes (which the reader
will no doubt have encountered in his
practical use of UNIX) are still not
immediately      apparent.        Since
``maptab[c]'' is zero for ``c == `\verb+\+' ''
(octal value of 134), all backslashes
get copied into ``canonb''. A single
backslash will be subsequently overwritten if the following character is
to be asserted (as in the case of `\#'
or `@' or eot (004), or if the case of
an alphabetic character is to be
changed for an upper case only terminal.
\ed

\sbs{ttyinput (8333)}

``canon'' removes characters from the
``raw'' input queue. They are put there
in the first place by ``ttyinput'' which
is called by ``klrint'' (8087) whenever
an input character is received from the
hardware controller.

The parameters passed to ``ttyinput'' are
a character and a reference to a ``tty''
structure.

\bd
\item[8342:] If the character is a ``carriage
return'' and the terminal operates
with a ``carriage return'' only
(instead of a ``carriage return''
``line feed'' pair) change the
character to a ``new line'';

\item[8344:] If the terminal is not operating
in ``raw'' mode and the character
is a ``quit'' or ``delete'' (7958)
then call ``signal'' (3949) to send
a software interrupt to every
process which has the terminal as
its controlling terminal, flush
all the queues associated with
the terminal, and return;

\item[8349:] If the ``raw'' input queue has
grown excessively large, flush
{\bf all} the queues for the terminal
and return. (This may seem a
trifle harsh at first sight but
it will usually be what is
required.);

\item[8353:] If the terminal has a limited
character set, and the character
is an upper case alphabetic,
translate it into lower case;

\item[8355:] Insert the character into the
``raw'' input queue;

\item[8356:] If the terminal is operating in
``raw'' mode, or the character was
a ``new line'' or ``eot'' then ...

\item[8357:] ``wakeup'' any process waiting for
input from the terminal, place a
delimiter character (all ones)
also in the ``raw'' queue and
increment the delimiter count
Note this is one point where possible failure of ``putc'' (when
there is no buffer space) is
explicitly recognised. A failure
occurring here would explain why
the test on line 8316 may sometimes succeed.

\item[8361:] Finally, if the input character
is to be echoed i.e. the terminal
is running in full duplex mode,
insert a copy of the character
into the output queue, and and
arrange to have it transmitted
(``ttstart'') back to the terminal.
\ed

\sbs{Character Output -- ttwrite (8550)}

This procedure is called via ``klwrite''
(8067) when output is to be sent to the
terminal.

\bd
\item[8556:] If the terminal is
inactive, do nothing;

\item[8558:] Loop for each character to be
transmitted ...

\item[8560:] While there are still an adequate
number of characters queued for
transmission to the terminal ...

\item[8561:] call ``ttstart'' just in case it is
time to send another character to
the terminal;

\item[8562:] Setting the ``ASLEEP'' flag here
(also in\\
``wflushtty'' (8224)) is
rather pointless since is is
never interrogated and never
reset until the file is closed;

\item[8563:] Go to sleep. In the meanwhile the
interrupt handler will be draining characters from the output
queue and sending them down the
line to the terminal;

\item[8566:] Call ``ttyoutput'' to insert the
character in the output queue and
arrange to have it transmitted;

\item[8568:] Call ``ttstart'' again, for luck.
\ed

\sbs{ttstart}

This procedure is called whenever it
seems reasonable to try and send the
next character to the terminal. It
often achieves nothing useful.

\bd
\item[8514:] See the comment on line 8499.
This code is not relevant here;

\item[8518:] If the controller is not ready
(i.e. bit 7 of the transmitter
stalus register is not set) or
the necessary delay following the
preious character has not yet
elapsed, do nothing;

\item[8520:] Remove a character from the output queue. If ``c'' is positive,
the queue was not empty (as
expected) ...

\item[8521:] If ``c'' is less than ``0177'' it is
a character to be transmitted ...

\item[8522:] After setting the parity bit from
the corresponding element of the
array ``partab'', write ``c'' to the
transmitter data buffer register
to initiate the hardware operation;

\item[8524:] Otherwise (``c'' $>$ 0177)
the character was inserted in the output
queue to signal a delay. Call
``timeout'' (3845) to make an entry
in the ``callout'' list. The
result of this will be to initiate an execution of ``ttrstrt''
(8486) after ``c \& 0177'' clock
ticks . It will be seen that
``ttrstrt'' calls ``ttstart'' again,
and that the manipulation of the ``TIMEOUT''
flag (8524, 8491) will
ensure that if another execution
of ``ttstart'' is initiated in the
interim, on behalf of the same
terminal, it will (8518) return
without doing anything.
\ed

\sbs{ttrstrt (8486)}

See the comment above for line 8524.

\sbs{ttyoutput (8373)}

This procedure has more comments in the
source code and hence requires less
explanation than some others. Note the
use of recursion (8392) to generate a
string of blanks in place of a tab
character. Other recursive calls are
on lines 8403 and 8413.

\sbs{Terminals with a restricted character set}

\bd
\item[8400:] ``colp'' points to a string of
pairs of characters. If the character to be output matches the
second character of any of these
pairs, the charactcr is replaced
by a backslash followed by the first character of the pair;

\item[8407:] Lower case alphabetics are converted to upper case alphabetics
by the addition of a constant.
\ed

\noindent Note. The conversion here should be
compared wth the handling of the
reverse problem on input. Here we have
an algorithm which clearly trades space
(no table analogue to ``maptab'') for
time (a serial search through the
string on line 8400). A space conserving
approach could be adopted in
``canon'' but the problem is rather more
complicated there.

\bd
\item[8414:] Insert the character into the
output queue. If perchance,
``putc'' fails for lack of buffer
space, don't worry about inserting any subsequent delay, or
updating the system's idea of the
current printing column;

\item[8423:] Set ``colp'' to point to the
``t\_col'' character of the ``tty''
structure, i.e. ``*colp'' has a
value which is the ordinal number
of the column which has just been
printed;

\item[8424:] Set ``ctype'' to the element of
``partab'' corresponding to the
output character ``c'';


\item[8426:] Mask out the significant bits of
``ctype'' and use the result as the
``switch'' index;

\item[8428:] (Case 0) The common situation!
Increment ``t\_col'';

\item[8431:] (Case 1) Non-printing characters.
This group consists of the first,
third and fourth octet of the
ASCII character set, plus ``so''
(016), ``si'' (017) and ``del''
(0177). Don't increment ``t\_col'';

\item[8434:] (Case 2) Backspace. Decrement
``t\_col'' unless it is already
zero;

\item[8439:] (Case 3) Newline. Obviously
``t\_col'' should be set to zero.
The main problem is to calculate
the delay which should ensue
before another character is sent.

For a Model 37 teletype, this
depends on how far the print
mechanism has progressed across
the page. The value chosen is at
least a tenth of a second (six
clock ticks) and may be as much
as ((132/16) + 3)/60 =   0.19
seconds.

For a VT05, the delay is 0.1
second. For a DECwriter it is
zero because the terminal
incorporates buffer storage and
has a double speed ``catch up''
print mode;

\item[8451:] (Case 4) Horizontal tab. Assign
the value of bits 10, 11 of
``t\_flags'' to ``ctype'';

\item[8453:] For the only non-trivial case
recognised\\
(``c'' ==  1 or Model 37
teletype), calculate the the
number of positions to the next
tab stop (via the obscure calculation of line 8454). If this
turns out to be four columns or
less, take it as zero;

\item[8458:] Round ``*colp'' (i.e. the value
pointed to by ``colp''!) to the
next multiple of 8 less one;

\item[8459:] Increment ``*colp'' to be an exact
multiple of eight;

\item[8462:] (Case 5) Vertical Motion. If bit
14 is set in ``t\_flags'', make the
delay as long as possible, i.e.
0177 or 127 clock ticks, i.e.
just over two seconds;

\item[8467:] (Case 6) Carriage Return. Assign
the value of bits 12, 13 of
``t\_flags'' to ``ctype'';

\item[8469:] For the first class, allow a
delay of five clock ticks;

\item[8472:] For the second class, allow a
delay of ten clock ticks;

\item[8475:] Set the ``*colp'' (the last column
printed) to zero.
\ed

Before leaving the file ``tty.c'', there
are two matters which deserve further
examination.

\sbs{A. The test for 'TTLOWAT' (Line 8074)}

On line 8074 in ``klxint'', a test is
made whether to restart any processes
waiting to send output to the terminal.
The test is successful if the number of
characters is zero or if it is equal to
``TTLOWAT''.

If the number of characters is between
these values, no ``wakeup'' is performed
until the queue is completely empty,
with the strong likelihood that there
will then be a hiatus in the flow of
output to the terminal. Since temporary interruptions to the flow of
output are quite frequently observed in
practice and represent a source of
occasional irritation if nothing more,
one may reasonably enquire ``is there
any way the character count can get
from being greater than ``TTLOWAT'' to
below it, without this being detected
at line 8074?''

Quite clearly there is, since each call
on ``ttstart'' can decrement the queue
size, and only one such call is followed by the test. Thus if the call on
``ttstart'' from one of ``ttrstrt'' (8492)
or ``ttwrite'' (8568) happens to cross
the boundary, a delay will result. The
probability that this will happen is
small, but finite and hence the event
is likely to be observed in any reasonably long output sequence.

There are two other situations in which
``ttstart'' is called which seem to be
satisfactory. At ``ttwrite'' (8561) the
queue is at its maximum extent; and at
``ttyinput'' (8363) there is a preceding
call on ``ttyoutput'' which usually (but
not invariably!) will have added a
character to the output queue.

\sbs{B. Inactive Terminals}

When the last special file for a terminal is closed, ``klclose'' (8055) is
called and resets (8059) the ``ISOPEN''
and ``CARR\_ON'' flags. However the ``read
enable'' bit of the receiver control
status register is not reset, so that
incoming characters may still be
received and will be stored away (8087)
in the terminal's ``raw'' input queue by
``klrint'' (8078), and ``ttyinput'' (8333),
which do not test the ``CARR\_ON'' flag,
to see if the terminal is logically
connected.

These characters may accumulate for a
long time and clog up the character
buffer storage. Only when the ``raw''
input queue reaches 256 characters
(``TTYHOG'', 8349) will the contents of
this queue be thrown away. It does seem
therefore, that a statement to disable
reader interrupts should be included in
``klclose'' before line 8058.

\newpage
\vfill

\sbs{Well, that's all, folks ...}

Now that you, oh long-suffering,
exhausted reader have reached this
point, you will have no trouble in
disposing of the last remaining file,
``mem.c'' (Sheet 90). And on this note,
we end this discussion of the UNIX
Operating System Source Code.

Of course there are lots more device
drivers for your patient examination,
and in truth the whole UNIX Timesharing System Source Code has hardly
been scratched. So this is not really

\begin{center}
{\bf \large THE END}
\end{center}
