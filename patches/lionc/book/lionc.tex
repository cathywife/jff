%
% The Lion's Commentary, file lionc.tex, version 1.3, 18 May 1994
%
\documentclass[a4paper,twoside]{article}
\usepackage{fancyhdr}
\usepackage[dvipdfmx]{hyperref} % Or pdftex, xetex, ps2pdf
\hypersetup{pdfborder={0 0 0},
            pdftitle={A COMMENTARY ON THE SIXTH EDITION UNIX OPERATING SYSTEM},
            pdfauthor={John Lions}}

\font\twlrm = cmr10 scaled \magstep1

% Set the page dimensions

\setlength{\textwidth}{16.6cm}
\setlength{\oddsidemargin}{-0.25cm}
\setlength{\evensidemargin}{-0.25cm}
\setlength{\textheight}{24.1cm}
\setlength{\topmargin}{-1.3cm}
\setlength{\columnsep}{1cm}

% Shut LaTeX up about overful hboxes

\tolerance=1000
\hfuzz=15pt

\begin{document}

% I don't feel like typing much....

\newcommand{\bc}{\begin{center}}
\newcommand{\ec}{\end{center}}
\newcommand{\bd}{\begin{description}}
\newcommand{\ed}{\end{description}}
\newcommand{\be}{\begin{enumerate}}
\newcommand{\ee}{\end{enumerate}}
\newcommand{\bi}{\begin{itemize}}
\newcommand{\ei}{\end{itemize}}
\newcommand{\bt}{\begin{tabbing}}
\newcommand{\et}{\end{tabbing}}
\newcommand{\se}[1]{\section{#1}}
\newcommand{\sbs}[1]{\subsection{#1}}
\newcommand{\ssb}[1]{\subsubsection{#1}}

% Set up the headings for the document

\renewcommand{\headrulewidth}{0cm}
\renewcommand{\sectionmark}[1]{\markboth{#1}{}}
\renewcommand{\subsectionmark}[1]{\markright{#1}}
\rhead[{\it UNIX 6th Edition Commentary}]{\it \leftmark}
\lhead[{\it \leftmark}]{\it UNIX 6th Edition Commentary}
\chead{}
\lfoot[{\thepage}]{}
\cfoot{}
\rfoot[]{\thepage}

% Give the title page

\pagestyle{empty}
%
% The Lion's Commentary, file title.tex, version 1.3, 15 May 1994
%
\begin{center}
{\Huge \bf A COMMENTARY ON THE\\
\medskip
SIXTH EDITION \\
\bigskip
\smallskip
UNIX OPERATING SYSTEM }
\end{center}

\vspace{5cm}

\begin{center}
{\LARGE J. Lions. \\
Department of Computer Science \\
\medskip
The University of New South Wales}
\end{center}

\vspace{8cm}

\noindent This booklet has been produced for students at the
University of New South Wales taking courses 6.602B and 6.657G.
It is intended as a companion to, and commentary on,
the booklet {\sl UNIX Operating System Source Code, Level Six}.

\medskip
\noindent The UNIX Software System was written by K. Thompson and
D. Ritchie of Bell Laboratories, Murray Hill, NJ. It has been made
available under a license from the Western Electric Company.

\medskip
\noindent This document may contain information covered by one or more
licenses, copyrights and non-disclosure agreements. Circulation of
this document is restricted to holders of a license for the UNIX
Software System from Western Electric. All other circulation or
reproduction is prohibited.

\medskip
\begin{center}
\copyright~~Copyright 1977 J. Lions
\end{center}


\cleardoublepage			% Blank page for two-up printing

\pagestyle{fancy}
\twocolumn
\pagenumbering{roman}

% Now the table of contents

\tableofcontents
\cleardoublepage			% Blank page for two-up printing

% And the preface

%
% The Lion's Commentary, file preface.tex, version 1.5, 18 May 1994
%
\section*{Preface}

This book is an attempt to explain in
detail the nucleus of one of the most
interesting computer operating systems
to appear in recent years.

It is the UNIX Time-sharing System,
which runs on the larger models of
Digital Equipment Corporation's PDP11
computer system, and was developed by
Ken Thompson and Dennis Ritchie at Bell
Laboratories. It was first announced to
the world in the July, 1974 issue of
the ``Communications of the ACM''.

Very soon in our experience with UNIX,
it suggested itself as an interesting
candidate for formal study by students,
for the following reasons:

\bi
\item it runs on a system which is already available to us;

\item it is compact and accessible;

\item it provides an extensive set of very
 usable facilities;

it is intrinsically interesting, and
 in fact breaks new ground in a
 number of areas.
\ei

Not least amongst the charms and virtues of the UNIX Time-sharing System is
the compactness of its source code.
The source code for the permanently
resident ``nucleus'' of the system when
only a small number of peripheral devices is represented, is comfortably
less than 9,000 lines of code.

It has often been suggested that 1,000
lines of code represents the practical
limit in size for a program which is to
be understood and maintained by a single individual.
Most operating systems either exceed
this limit by one or even two orders of
magnitude, or else offer the user a
very limited set of facilities, i.e.
either the details of the system are
inaccessible to all but the most determined, dedicated and long-suffering
student, or else the system is rather
specialised and of little intrinsic
interest.

There seem to be three main approaches
to teaching Operating Systems.
First there is the ``{\it general principles}''
approach, wherein fundamental principles are expounded, and illustrated by
references to various existing systems,
(most of which happen to be outside the
students' immediate experience). This
is the approach advocated by the COSINE
Committee, but in our view, many students are not mature or experienced
enough to profit from it.

The second approach is the ``{\it building
block}'' approach, wherein the students
are enabled to synthesize a small scale
or ``toy'' operating system for themselves. While undoubtedly this can be a
valuable exercise, if properly organised, it cannot but fail to encompass
the complexity and sophistication of
real operating systems, and is usually
biased towards one aspect of operating
system design, such as process synchronisation.

The third approach is the ``{\it case study}''
approach. This is the one originally
recommended for the Systems Programming
course in ``Curriculum '68'', the report
of the ACM Curriculum Committee on Computer Science, published in the March,
1968 issue of the ``Communications of
the ACM''.

Ten years ago, this approach, which
advocates devoting ``most of the course
to the study of a single system'' was
unrealistic because the cost of providing adequate student access to a suitable system was simply too high.

Ten years later, the economic picture
has changed significantly, and the
costs are no longer a decisive disadvantage if a minicomputer system can be
the subject of study. The considerable
advantages of the approach which undertakes a detailed analysis of an existing system are now attainable.

In our opinion, it is highly beneficial
for students to have the opportunity to
study a working operating system in all
its aspects.

Moreover it is undoubtedly good for
students majoring in Computer Science,
to be confronted at least once in their
careers, with the task of reading and
understanding a program of major dimensions.

In 1976 we adopted UNIX as the subject
for case study in our courses in
Operating Systems at the University of
New South Wales. These notes were
prepared originally for the assistance
of students in those courses (6.602B
and 6.657G).

The courses run for one semester each.
Before entering either Course, students
are presumed to have studied the PDP11
architecture and assembly language, and
to have had an opportunity to use the
UNIX operating system during exercises
for earlier courses.

In general, students seem to find the
new courses more onerous, but much more
satisfying than the previous courses
based on the ``general principles''
approach of the COSINE Committee.

Some mention needs to be made regarding
the documentation provided by the
authors of the UNIX system. As reproduced for use on our campus, this
comprises two volumes of A4 size paper,
with a total thickness of 3 cm, and a
weight of 1250 grams.

A first observation is that the whole
documentation is not unreasonably transportable in a student's brief case.
However it must not be assumed that
this amount of documentation, which is
written in a fresh, terse, whimsical
style, is necessarily inadequate.

In fact the second observation (which
is only made after considerable experience) is that for reference purposes,
the documentation is remarkably
comprehensive. However there is plenty
of scope for additional tutorial
material, one part of which, it is
hoped, is satisfied by these notes.

The actual UNIX operating system source
code is recorded in a separate companion volume entitled ``UNIX Operating
System Source Code'', which was first
printed in July, 1976. This is a specially edited selection of code from
the Level Six version of UNIX, as
received by us in December, 1975.

During 1976, an initial version of the
present notes was distributed in
roneoed form, and only in the latter
part of the year were the facilities of
the ``nroff'' text formatting program
exploited. The opportunity has
recently been taken to revise and
``nroff'' the earlier material, to make
some revisions and corrections, and to
integrate them into their present form.

A decision had to be made quite early
regarding the order of presentation of
the source code. The intention was to
provide a reasonably logical sequence
for the student who wanted to learn the
whole system. With the benefit of
hindsight, a great many improvements in
detail are still possible, and it is
intended that these changes will be
made in some future edition.

It is our hope that this book will be
of interest and value to many students
of the UNIX Time-sharing System.
Although not prepared primarily for use
as a reference work, some will wish to
use it as such. The indices provided at
the end should go some of the way
towards satisfying the requirement for
reference material at this level.

Since these notes refer to proprietary
material administered by the Western
Electric Company, they can only be made
available to licensees of the UNIX
Time-sharing System and hence are
unable to be published through more
usual channels.

Corrections, criticism and suggestions
for improvement of these notes will be
very welcome.

\bigskip

{\noindent \large \bf Acknowledgments}

\smallskip

The preparation of these notes has been
encouraged and supported by many of my
colleagues and students including David
Carrington, Doug Crompton, Ian Hayes,
David Horsfall, Peter Ivanov, Ian Johnstone, Chris Maltby, Dave Milway, John
O'Brien and Greg Rose.

Pat Mackie and Mary Powter did much of
the initial typing, and Adele Green has
assisted greatly in the transfer of the
notes to ``nroff'' format.

David Millis and the Publications Section of the University of New South
Wales have assisted greatly with the
mechanics of publication, and Ian Johnstone and the Australian Graduate
School of Management provided facilities for the preparation of the final
draft.

Throughout this project, my wife
Marianne has given me unfailing moral
support and much practical support with
proof-reading.

Finally Ken Thompson and Dennis Ritchie
started it all.

To all the above, I wish to express my
sincere thanks.

The co-operation of the ``nroff'' program
must also be mentioned. Without it,
these notes could never have been produced in this form. However it has
yielded some of its more enigmatic
secrets so reluctantly, that the
author's gratitude is indeed mixed.
Certainly ``nroff'' itself must provide a
fertile field for future practitioners
of the program documenter's art.

\bt
\hspace{5cm} \= John Lions \\
\> Kensington, NSW \\
\> May, 1977 \\
\et

\cleardoublepage			% Blank page for two-up printing
\pagenumbering{arabic}
\setcounter{section}{0}

% Then do the chapters

\startcomponent ch1
\product mkiv

\chapter{基础知识}

介绍下基本知识。

\page[yes]
第二页。

\page[yes]
第三页。
\page[yes]
第四ss页 aaa 中文。\crlf
中文中文\crlf
嗯,中文好

\stopcomponent


%
% The Lion's Commentary, file ch2.tex, version 1.5, 18 May 1994
%
\se{Fundamentals}

UNIX runs on the larger models of the
PDP11 series of computers manufactured
by Digital Equipment Corporation. This
chapter provides a brief summary of
certain selected features of these computers
with particular reference to the
PDP11/40.

If the reader has not previously made
the acquaintance of the PDP11 series
then he is directed forthwith to the
``PDP11 Processor Handbook'', published
by DEC.

A PDP11 computer consists of a processor
(also called a CPU connected to
one or more memory storage units and
peripheral controllers via a bidirectional
parallel communication line
called the ``Unibus''.

\sbs{The Processor}

The processor, which is designed around a
sixteen bit word length for instructions,
data and program addresses, incorporates
a number of high speed registers.

\sbs{Processor Status Word}

This sixteen bit register has subfields
which are interpreted as follows:

\begin{tabular}{ll}\\
{\bf bits} & {\bf description} \\
14,15 & current mode (00 = kernel;) \\
12,13 & previous mode (11 = user;) \\
5,6,7 & processor priority (range 0..7) \\
4 & trap bit \\
3 & N, set if the previous \\
  & result was negative \\
2 & Z, set if the previous \\
  & result was zero \\
1 & V, set if the previous \\
  & result gave an overflow \\
0 & C, set if the previous \\
  & operation gave a carry \\
\end{tabular}

The processor can operate in two different
modes: kernel and user. Kernel
mode is the more privileged of the two
and is reserved by the operating system
for its own use. The choice of mode
determines:

\bi
\item The set of memory management segmentation
registers which is used
to translate program virtual
addresses to physical addresses;

\item The actual register used as r6, the
``stack pointer'';

\item Whether certain instructions such as
``halt'' will be obeyed.
\ei

\sbs{General Registers}

The processor incorporates a number of
sixteen bit registers of which eight
are accessible at any time as ``general
registers''. These are known as
r0, r1, r2, r3, r4, r5, r6 and r7.

The first six of the general registers
are available for use as accumulators,
address pointers or index registers.
The convention in UNIX for the use of
these registers is as follows:

\bd
\item[r0, r1] are used as temporary accumulators
during expression evaluation, to return results from a
procedure, and in some cases to
communicate actual parameters during a procedure call;

\item[r2, r3, r4] are used for local
variables during procedure execution.
Their values are almost
always stored upon procedure
entry, and restored upon procedure
exit;

\item[r5] is used as the head pointer to
a ``dynamic chain'' of procedure
activation records stored in the
current stack. It is referred to
as the ``environment pointer''.
\ed

The last two of the ``general registers''
do have a special significance and are
to all intents, ``special purpose'':

\bd
\item[r6] (also known as ``sp'') is used as
the stack pointer. The PDP11/40
processor incorporates two
separate registers which may be
used as ``sp'', depending on whether
the processor is in kernel or user
mode. No other one of the general
registers is duplicated in this
way;

\item[r7] (also known as ``pc'') is used as
the program  instruction address register.
\ed

\sbs{Instruction Set}

The PDP11 instruction set includes double,
single and zero operand instructions.
Instruction length is usually
one word, with some instructions being
extended to two or three words with
additional addressing information.

With single operand instructions, the
operand is usually called the ``destination'';
with double operand instructions, the two operands are called the
``source'' and ``destination''. The various
modes of addressing are described later.

The following instructions have been
used in the file ``m40.s'' i.e. the file
of assembly language support routines
for use with the 11/40 processor. Note
that N, Z, V and C are the condition
codes i.e. bits in the processor status
word (``ps''), and that these are set as
side effects of many instructions
besides just ``bit'', ``cmp'' and ``tst''
(whose stated function is to set the
condition codes).

\bd
\item[adc] Add the contents of the C bit to
the destination;

\item[add] Add the source to the destination;

\item[ash] Shift the contents of the defined
register left the number of times
specified by the shift count. (A
negative value implies a right
shift.);

\item[ashc] Similar to ``ash'' except that two
registers are involved;


\item[asl] Shift all bits one place to the
left. Bit 0 becomes 0 and bit 15
is loaded into C;


\item[asr] Shift all bits one place to the
right. Bit 15 is replicated and
bit 0 is loaded into C;

\item[beq] Branch if eaual, i.e. if Z = l;


\item[bge] Branch if greater than or equal
to, i.e. if\\
N = V;

\item[bhi] Branch if higher, i.e if C = 0 and
Z = 0;

\item[bhis] Branch if higher or the same, i.e.
if C = 0;


\item[bic] Clear each bit to zero in the
destination that corresponds to a
non-zero bit in the source;


\item[bis] Perform an ``inclusive or'' of
source and destination and store
the result in the destination;


\item[bit] Perform a logical ``and'' of the
source and destination to set the
condition codes;

\item[ble] Branch if greater than or
equal to, i.e if Z = 1 or N = V;


\item[blo] Branch if lower (than zero),
if C = l;


\item[bne] Branch if not equal (to zero),
i.e. if Z = 0;


\item[br] Branch to a location within the
range (.~-128,\\
.~+127) where ``.'' is the current location;

\item[clc] Clear C;

\item[clr] Clear destination to zero;


\item[cmp] Compare the source and destination
to set the condition codes. N is
set if the source value is less
than the destination value;


\item[dec] Subtract one from the contents of
the destination;


\item[div] The 32 bit two's complement
integer stored in rn and r(n+l)
(where n is even) is divided by
the source operand. The quotient
is left in rn, and the remainder
in r(n+l);


\item[inc] Add one to the contents of the
destination;


\item[jmp] Jump to the destination;

\item[jsr] Jump to subroutine. Register
values are shuffled as follows:

pc, rn, --(sp) = dest., pc, rn


\item[mfpi] Push onto the current stack the
value of the designated word in
the ``previous'' address space;


\item[mov] Copy the source value to the destination;


\item[mtpi] Pop the current stack and store
the value in the designated word
in the ``previous'' address space;


\item[mul] Multiply the contents of rn and
the source. If n is even, the product
is left in rn and r(n+l);


\item[reset] Set the INIT line on the Unibus
for 10 milliseconds. This will
have the effect of reinitialising
all the device controllers;


\item[ror] Rotate all bits of the destination
one place to the right. Bit 0 is
loaded into C, and the previous
value of C is loaded into bit 15;


\item[rts] Return from subroutine. Reload pc
from rn, and reload rn from the
stack;


\item[rtt] Return from interrupt or trap.
Reload both pc and ps from the
stack;

\item[sbc] Subtract the carry bit from
the destination;


\item[sob] Subtract one from the designated
register. If the result is not
zero, branch back ``offset'' words;


\item[sub] Subtract the source from the destination;


\item[swab] Exchange the high and low order
bytes in the destination;


\item[tst] Set the condition codes, N and Z,
according to the contents of the
destination;


\item[wait] Idle the processor and release the
Unibus until a hardware interrupt
occurs.
\ed

The ``byte'' version of the following
instructions are used in the file
``m40.s'', as well as the ``word'' versions
described above:

\bt
\hspace{2cm} \= bis \hspace{2cm} \= inc \\
\> clr \> mov \\
\> cmp \> tst \\
\et


\sbs{Addressing Modes}

Much of the novelty and complexity of
the PDP11 instruction set lies in the
variety of addressing modes which may
be used for defining the source and
destination operands.


The addressing modes which are used in
``m40.s'' are described below.

\bd
\item[Register Mode:] The operand resides in
one of the general registers, e.g.

\begin{verbatim}
    clr r0
    mov rl,r0
    add r4,r2
\end{verbatim}


In the following modes, the designated
register contains an address value
which is used to locate the operand.

\item[Register Deferred Mode:] The register
contains the address of the operand,
e.g.

\begin{verbatim}
    inc (rl)
    asr (sp)
    add (r2),rl
\end{verbatim}

\item[Autoincrement Mode:] The register contains
the address of the operand. As a
side effect, the register is incremented
after the operation, e.g.

\begin{verbatim}
    clr  (rl)+
    mfpi (r0)+
    mov  (r1)+,r0
    mov  r2,(r0)+
    cmp (sp)+,(sp)+
\end{verbatim}

\item[Autodecrement Mode:] The register is
decremented and then 
operand, e.g.

\begin{verbatim}
    inc -(r0)
    mov -(r1),r2
    mov (r0)+,-(sp)
    clr -(sp)
\end{verbatim}

\item[Index Mode:] The register contains a
value which is added to a sixteen bit
word following the instruction to form
the operand address, e.g.

\begin{verbatim}
    clr  2(r0)
    movb 6(sp),(sp)
    movb _reloc(r0),r0
    mov  -10(r2),(rl)
\end{verbatim}

Depending on your viewpoint, in this
mode the register is either an index
register or a base register. The
latter case actually predominates in
``m40.s''. The third example above is
actually one of the few uses of a
register as an index register. (Note
that ``\_reloc'' is an acceptable variable
name.)

There are two addressing modes whose
use is limited to the following two
examples:

\begin{verbatim}
    jsr pc,(r0)+
    jmp *0f(r0)
\end{verbatim}

The first example involves the use of
the ``{\it autoincrement deferred}'' mode.
(This occurs in the routine ``call'' on
lines 0785, 0799.) The address of a
routine intended for execution is to be
found in the word addressed by r0, i.e.
two levels of indirection are involved.
The fact that r0 is incremented as a
side effect is not relevant in this
usage.

The second example (which occurs on
lines 1055, 1066) is an instance of the
``{\it index deferred}'' mode. The destination
of the ``jump'' is the content of the
word whose address is labelled by ``0f''
{\it plus} the value of r0 (a small positive
integer). This is a standard way to
implement a multi-way switch.


The following two modes use the program
counter as the designated register to
achieve certain special effects.

\item[Immediate Mode:] This is the pc autoincrement
mode. The operand is thus
extracted from the program string, i.e.
it becomes an immediate operand, e.g.

\begin{verbatim}
    add $2,r0
    add $2,(rl)
    bic $17,r0
    mov $KISA0,r0
    mov $77406,(rl)+
\end{verbatim}

\item[Relative Mode:] This is the pc index
mode. The address relative to the
current program counter value is
extracted from the program string and
added to the pc value to form the absolute
address of the operand, e.g.

\begin{verbatim}
    bic $340,PS
    bit $l,SSR0
    inc SSR0
    mov (sp),KISA6
\end{verbatim}
\ed

It may be noted that each of the modes
``index'', ``index deferred'', ``immediate''
and ``relative'' extends the instruction
size by one word.

The existence of the ``autoincrement''
and ``autodecrement'' modes, together
with the special attributes of r6, make
it conveniently possible to store many
operands in a stack, or LIFO list,
which grows downwards in memory. There
are a number of advantages which flow
from this: code string lengths are
shorter and it is easier to write
position independent code.

\sbs{Unix Assembler}

The UNIX assembler is a two pass assembler
without macro facilities. A full
description may be found in the ``UNIX
Assembler Reference Manual'' which is
contained in the ``UNIX Documents''

The following brief notes should be of
some assistance:

\bd
\item[(a)] a string of digits may define a
constant number. This is assumed
to be an octal number unless the
string is terminated by a period
(``.''), when it is interpreted as
a decimal number.


\item[(b)] The character ``/'' is used to
signify that the rest of the
line is a comment;

\item[(c)] If two or more statements occur
on the same line, they must be
separated by semicolons;

\item[(d)] The character ``.'' is used to
denote the current location;

\item[(e)] UNIX assembler uses the characters \$
and ``*'' where the DEC
assemblers use ``\#'' and ``@''
respectively.

\item[(f)] An identifier consists of a set
of alphanumeric characters
(including the underscore).
Only the first eight characters
are significant and the first
may not be numeric;

\item[(g)] Names which occur in ``C'' programs
for variables which are to
be known globally, are modified
by the addition of a prefix consisting
of a single underscore.
Thus for example the variable
``\_regloc'' which occurs on line
1025 in the assembly language
file, ``m40.s'', refers to the
same variable as ``regloc'' at
line 2677 of the file, ``trap.c'';

\item[(h)] There are two kinds of statement
labels: name labels and numeric
labels. The latter consist of a
single digit followed by a
colon, and need not be unique.
A reference to ``nf'' where ``n'' is
a digit, refers to the first
occurrence of the label ``n:''
found by searching forward.

A reference to ``nb'' is similar
except that the search is conducted in
the backwards direction;

\item[(i)] An assignment statement of the
form

\bc
identifier = expression
\ec

associates a value {\bf and} type with
the identifier. In the example

\bc
. = 60\verb+^+.
\ec

the operator '\verb+^+' delivers the
value of the first operand and
the type of the second operand
(in this case, ``location'');

\item[(j)] The string quote symbols
are ``$<$'' and ``$>$''.

\item[(k)] Statements of the form

\bc
.globl x, y, z
\ec

serve to make the names ``x'', ``y''
and ``z'' external;

\item[(l)] The names ``\_edata'' and ``\_end''
are loader pseudo variables
which the define the size of the
data segment, and the data segment
plus the bss segment
respectively.
\ed


\sbs{Memory Management}

Programs running on the PDP11 may
address directly up to 64K bytes (32K
words) of storage. This is consistent
with an address size of sixteen bits.
Since it is economical and not unreasonable
to do so the larger PDP11
models may be equipped with larger
amounts of memory (up to 256K bytes for
the PDP11/40) plus a mechanism for
converting sixteen bit {\it virtual} (program)
addresses into {\it physical} addresses of
eighteen bits or more. The mechanism,
which is known as the memory management
unit, is simpler on the PDP11/40 than
on the 11/45 or the 11/70.

On the PDP11/40 the memory management
unit consists of two sets of registers
for mapping virtual addresses to physical
addresses. These are known as
``active page registers'' or ``segmentation
registers''. One set is used when
the processor is in user mode and the
other set, in kernel mode. Changing the
contents of these registers changes the
details of these mappings. The ability
to make these changes is a privilege
that the operating system keeps firmly
to itself.

\sbs{Segmentation Registers.}

Each set of segmentation registers is
composed of eight pairs, each consisting
of a ``{\it page address register}'' (PAR)
and a ``{\it description register}''
(PDR).

Each pair of registers controls the
mapping of one {\it page} i.e. one eighth
part of the virtual address space which
8K bytes (4K words).


Each page may be regarded as an aggregate
of 128 blocks, each of 64 bytes
(32 words). This latter size is the
``grain size'' for the memory mapping
function, and as a practical consequence,
it is also the ``grain size'' for
memory allocation.

Any virtual address belongs to one page
or other. The corresponding physical
address is generated by adding the
relative address within the page to the
contents of the corresponding PAR to
form an extended address (18 bits on
the PDP11/40 and 11/45; 22 bits on the
11/70).

Thus each page address register acts as
a relocation register for one page.

Each page can be divided on a 32 word
boundary into two parts, an upper part
and lower part. Each such part has a
size which is a multiple of 32 words.
In particular one part may be null, in
which case the other part coincides
with the whole page.

One of the two parts is deemed to contain
valid virtual addresses. Addresses
in the remaining part are declared
invalid. Any attempt to reference an
invalid address will be trapped by the
hardware. The advantage of this scheme
is that space in the physical memory
need only be allocated for the valid
r)art of a page.


\sbs{Page Description Register}

The page description register defines:

\bd
\item[(a)] the size of the lower part of
the page. (The number stored is
actually the number of 32 word
blocks less one);

\item[(b)] a bit which is set when the
upper part is the valid part.
(Also known as the ``expansion
direction'' bit);

\item[(c)] access mode bits defining ``no
access'' or ``read only access'' or
``read/write access''.
\ed

Note that if the valid part is null,
this fact must be shown by setting the
access bits to ``no access''.

\sbs{Memory Allocation}

The hardware does not dictate the way
areas in physical memory which
correspond to the valid parts of pages
should be allocated (except to the
extent that they must begin and end on
a 32 word boundary). These areas may be
allocated in any order and may overlap
to any extent.

In practice the allocation of areas of
physical memory is much more disciplined
as we shall see in Chapter Seven.
Areas for pages which are related are
most often allocated contiguously and
in the order of their page numbers, so
that all the segment areas associated
with a single program are contained
within one or at most two large areas
of physical memory.


\sbs{Memory Management Status Registers}

In addition to the segmentation registers,
on the PDP11/40 there are two
memory management status registers:

\bd
\item[SR0] contains abort error flags and
other essential information for
the operating system. In particular
memory management is enabled
when bit 0 of SR0 is on;

\item[SR2] is loaded with the 16 bit virtual
address at the beginning of
each instruction fetch.
\ed


\sbs{``i'' and ``d'' Spaces}

In the PDP11/45 and 11/70 systems,
there are additional sets of segmentation
registers. Addresses created using
the pc register (r7) are said to belong
to ``i'' space, and are translated by a
different set of segmentation registers
from those used for the remaining
addresses which are said to belong to
``d'' space.

The advantage of this arrangement is
that both ``i'' and ``d'' spaces may occupy
up to 32K words, thus allowing the maximum
space which can be allocated to a
program to be increased to twice the
space available on the PDP11/40.


\sbs{Initial Conditions}

When the system is first started after
all the devices on the Unibus have been
reinitialised, the memory management
unit is disabled and the processor is
in kernel mode.

Under these circumstances, virtual
(byte) addresses in the range 0 to 56K
are mapped into identically valued
physical addresses. However the highest
page of the virtual address space is
mapped into the highest page of the
physical address space, i.e. on the
PDP11/40 or 11/45, addresses in the
range

\bc
0160000 to 0177777
\ec

\noindent are mapped into the range

\bc
0760000 to 0777777
\ec


\sbs{Special Device Registers}

The high page of physical memory is
reserved for various special registers
associated with the processor and the
peripheral devices. By sacrificing one
page of memory space in this way, the
PDP11 designers have been able to make
the various device registers accessible
without the need to provide special
instruction types.

The method of assignment of addresses
to registers in this page is a black
art: the values are hallowed by tradition
and are not to be questioned.

%
% The Lion's Commentary, file ch3.tex, version 1.5, 18 May 1994
%
\se{Reading ``C'' Programs}

Learning to read programs written in
the ``C'' language is one of the hurdles
that must be overcome before you will
be able to study the source code of
UNIX effectively.

As with natural languages, reading is
an easier skill to acquire than writing.
Even so you will need to be careful lest
some of the more subtle points pass you by.

There are two of the ``UNIX Documents''
which relate directly to the ``C''
language:

``C Reference Manual'', by Dennis Ritchie

``Programming in C -- A Tutorial'',
 by Brian Kernighan

You should read them now, as far as you
can, and return to reread them from
time to time with increasing comprehension.

Learning to write ``C'' programs is not
required. However if you have the
opportunity, you should attempt to
write at least a few small programs.
This does represent the accepted way to
learn a programming language, and your
understanding of the proper use of such
items as:

semicolons; \\
``='' and ``=='' \\
``\{'' and ``\}'' \\
``$++$'' and ``$--$'' \\
declarations; \\
register variables; \\
``if'' and ``for'' statements \\


You will find that ``C'' is a very
convenient language for accessing and
manipulating data structures and
character strings, which is what a large
part of operating systems is about. As
befits a terminal oriented language,
which requires concise, compact expression,
``C'' uses a large character set
and makes many symbols such as ``*'' and
``\&'' work hard. In this respect it
invites comparison with APL.

There many features of ``C'' which are
reminiscent of PL/1, but it goes well
beyond the latter in the range of
facilities provided for structured programming.

\sbs{Some Selected Examples}

The examples which follow are taken
directly from the source code.

\sbs{Example 1}

The simplest possible procedure, which
does nothing, occurs twice(!) in the
source code as ``nullsys'' (2864) and
``nulldev'' (6577), sic.

\newpage
\begin{verbatim}
  6577 nulldev ()
       {
       }
\end{verbatim}

While there are no parameters, the
parentheses, ``('' and ``)'', are still
required. The brackets ``\{'' and ``\}''
delimit the procedure body, which is
empty.

\sbs{Example 2}

The next example is a little less
trivial:

\begin{verbatim}
  6566 nodev ()
       {
        u.u_error = ENODEV;
       }
\end{verbatim}


The additional statement is an assignment
statement. It is terminated by a
semicolon which is part of the statement,
not a statement separator as in
Algol-like languages.

``ENODEV'' is a defined symbol, i.e. a
symbol which is replaced by an associated
character string by the compiler
preprocessor before actual compilation.
``ENODEV'' is defined on line 0484 as 19.
The UNIX convention is that defined
symbols are written in upper case, and
all other symbols in lower case.

``='' is the assignment operator, and
``u.u\_error'' is an element of the structure
``u''. (See line 0419.) Note the use
of ``.'' as the operator which selects an
element of a structure. The element
name is ``u\_error'' which may be taken as
a paradigm for the way names of structure
elements are constructed in the
UNIX source code: a distinguishing
letter is followed by an underscore
followed by a name.

\sbs{Example 3}

\begin{verbatim}
  6585 bcopy (from, to, count)
       int *from, *to;
       {
        register *a, *b, c;
        a = from;
        b = to;
        c = count;
        do
          *b++ = *a++;
        while (--cc);
       }
\end{verbatim}

The function of this procedure is very
simple: it copies a specified number of
words from one set of consecutive
locations to another set.

There are three parameters. The second
line

\begin{verbatim}
   int *from, *to;
\end{verbatim}

\noindent specifies that the first two variables
are pointers to integers. Since no
specification is supplied for the third
parameter, it is assumed to be an
integer by default.

The three local variables, a, b, and c,
have been assigned to registers,
because registers are more accessible
and the object code to reference them
is shorter. ``a'' and ``b'' are pointers to
integers and ``c'' is an integer. The
register declaration could have been
written more pedantically as

\begin{verbatim}
   register int *a, *b, c;
\end{verbatim}

\noindent to emphasise the connection with
integers.

The three lines beginning with ``do''
should be studied carefully. If ``b'' is
a ``pointer to integer'' type, then

\begin{verbatim}
    *b
\end{verbatim}

\noindent denotes the integer pointed to. Thus to
copy the value pointed to by ``a'' to the
location designated by ``b'', we could
write

\begin{verbatim}
    *b = *a;
\end{verbatim}

\noindent If we wrote instead

\begin{verbatim}
    b = a;
\end{verbatim}

this would make the value of ``b'' the
same as the value of ``a'', i.e. ``b'' and
``a'' would point to the same place.
Here at least, that is not what is
required.

Having copied the first word from
source to destination, we need to
increase the values of ``b'' and ``a'' so
that the point to the next words of
their respective sets. This can be done
by writing

\begin{verbatim}
    b = b+1; a = a+1;
\end{verbatim}

\noindent but ``C'' provides a shorter notation
(which is more useful when the variable
names are longer) viz.

\begin{verbatim}
    b++; a++;
\end{verbatim}

Now there is no difference between the
statements ``b++;'' and ``++b;'' here.

However ``b++'' and ``++b'' may be used as
terms in an expression, in which case
they are different. In both cases the
effect of incrementing ``b'' is retained,
but the value which enters the expression is
the initial value for ``b++'' and
the final value for ``++b''.

The ``$--$'' operator obeys the same rules
as the ``++'' operator, except that it
decrements by one. Thus ``$--$c'' enters an
expression as the value after decrementation.

The ``++'' and ``$--$'' operators are very
useful, and are used throughout UNIX.
Occasionally you will have to go back
to first principles to work out exactly
what their use implies. Note also
there is a difference between {\tt *b++} and {\tt (*b)++}.

These operators are applicable to
pointers to structures as well as to
simple data types. When a pointer
which has been declared with reference
to a particular type of structure is
incremented, the actual value of the
pointer is incremented by the size of
the structure.

\noindent We can now see the meaning of the line

\begin{verbatim}
    *b++ = *a++;
\end{verbatim}

\noindent The word is copied and the pointers are
incremented, all in one hit.

\noindent The line

\begin{verbatim}
    while (--c);
\end{verbatim}

\noindent delimits the end of the set of statements
which began after the ``do''. The
expression in parentheses ``$--$c'', is
evaluated and tested (the value tested
is the value after decrementation). If
the value is non-zero, the loop is
repeated, else it is terminated.

Obviously if the initial value for
``count'' were negative, the loop would
not terminate properly. If this were a
serious possibility then the routine
would have to be modified.

\sbs{Example 4}

\begin{verbatim}
  6619 getf (f)
       {
         register *fp, rf;
         rf = f;
         if (rf < 0 || rf >= NOFILE)
           goto bad;
         fp = u.u_ofile[rf];
         if (fp != NULL)
         return (fp);
       bad:
         u.u_error = EBADF;
         return (NULL);
       }
\end{verbatim}

The parameter ``f'' is a presumed
integer, and is copied directly into
the register variable ``rf''. (This pattern will become so familiar that we
will now cease to remark upon it.)

\noindent The three simple relational expressions

\begin{verbatim}
  rf < 0    rf >=NOFILE    fp != NULL
\end{verbatim}

\noindent are each accorded the value one if
true, and the value zero if false. The
first tests if the value of ``rf'' is
less than zero, the second, if ``rf'' is
greater than the value defined by
``NOFILE'' and the third, if the value of
``fp'' is not equal to ``NULL'' (which is
defined to be zero).

The conditions tested by the ``if''
statements are the arithmetic expressions contained within parentheses.

If the expression is greater than zero
the test is successful and the following statement is executed. Thus if for
instance, ``fp'' had the value 001375,
then

\begin{verbatim}
    fp != NULL
\end{verbatim}

\noindent is true, and as a term in an arithmetic
expression, is accorded the value one.
This value is greater than zero, and
hence the statement

\begin{verbatim}
    return(fp);
\end{verbatim}

\noindent would be executed, to terminate further
execution of ``getf'', and to return the
value of ``fp'' to the calling procedure
as the result of ``getf''.


The expression

\begin{verbatim}
    rf < 0 || rf >= NOFILE
\end{verbatim}


\noindent is the logical disjunction (``or'') of
the two simple relational expressions.

An example of a ``goto'' statement and
associated label will be noted.


``fp'' is assigned a value, which is an
{\bf address}, from the ``rf''-th element of
the array of integers ``u\_ofile'', which
is embedded in the structure ``u''.


The procedure ``getf'' returns a value to
its calling procedure. This is either
the vale of ``fp'' (i.e. an address) or
``NULL''.

\sbs{Example 5}

\begin{verbatim}
  2113 wakeup (chan)
       {
         register struct proc *p;
         register c, i;
         c= chan;
         p= &proc[0];
         i= NPROC;
         do {
             if (p->p_wchan == c) {
               setrun(p);
             }
             p++;
         } while (--i);
       }
\end{verbatim}

There are a number of similarities
between this example and the previous
one. We have a new concept however, an
array of structures. To be just a
little confusing, in this example it
turns out that both the array and the
structure are called ``proc'' (yes, ``C''
allows this). They are declared on
Sheet 03 in the following form:

\begin{verbatim}
  0358 struct proc
       {
         char p_stat;
         ..........
         int p_wchan;
         ..........
       } proc[NPROC];
\end{verbatim}

``p'' is a register variable of type
pointer to a structure of type ``proc''.

\begin{verbatim}
    p = &proc[0];
\end{verbatim}


\noindent assigns to ``p'' the address of the first
element of the array ``proc''. The
operator ``\&'' in this context means ``the
address of''.


Note that if an array has n elements,
the elements have subscripts 0, 1, .., (n-1).
Also it is permissible to write
the above statement more simply as

\begin{verbatim}
    p = proc;
\end{verbatim}

There are two statements in between the
``do'' and the ``while''.
The first of these could be rewritten
more simply as

\begin{verbatim}
    if (p->p wchan == c) setrun (p);
\end{verbatim}

\noindent i.e. the brackets are superfluous in
this case, and since ``C'' is a free form
language, the arrangement of text
between lines is not significant.


\noindent The statement

\begin{verbatim}
    setrun (p);
\end{verbatim}


\noindent invokes the procedure ``setrun'' passing
the value of ``p'' as a parameter (All
parameters are passed by value.). The relation

\begin{verbatim}
    p->p_wchan == c
\end{verbatim}


\noindent tests the equality of the value of ``c''
and the value of the element ``p\_wchan''
of the structure pointed to by ``p''.
Note that it would have been wrong to
have written

\begin{verbatim}
    p.p_wchan == c
\end{verbatim}

\noindent because ``p'' is not the {\bf name} of a structure.

The second statement, which cannot be
combined with the first, increments ``p''
by the size of the ``proc'' structure,
whatever that is. (The compiler can
figure it out.)

In order to do this calculation
correctly, the compiler needs to know
the kind of structure pointed at. When
this is not a consideration, you will
notice that often in similar situations, ``p'' will be declared simply as

\begin{verbatim}
    register *p;
\end{verbatim}

\noindent because it was easier for the
programmer, and the compiler does not insist.

The latter part of this procedure could
have been written equivalently but less
efficiently as

\begin{verbatim}
      ............
      i = 0;
      do
        if (proc[i].p_wchan == c)
          setrun (&proc[i]);
      while (++i < NPROC);
\end{verbatim}

\sbs{Example 6}

\begin{verbatim}
  5336 geterror (abp)
       struct buf *abp;
       {
         register struct buf bp;
         bp = abp;
         if (bp->b flags & B_ERROR)
           if ((u.u_error=bp->b_error)==0)
             u.u_error = EIO;
       }
\end{verbatim}



This procedure simply checks if there
has been an error, and if the error
indicator ``u.u\_error'' has not been set,
sets it to a general error indication

``B\_ERROR'' has the value 04 (see line
4575) so that, with only one bit set,
it can be used as mask to isolate bit
number 2. The operator ``\&'' as used in

\begin{verbatim}
    bp->b_flags & B_ERROR
\end{verbatim}

\noindent is the bitwise logical conjunction
(``and'') applied to arithmetic values.

The above expression is greater than
one if bit 2 of the element ``b\_flags''
of the ``buf'' structure pointed to by
``bp'', is set.

Thus if there has been an error, the
expression

\begin{verbatim}
    (u.u_error) = bp->b_error)
\end{verbatim}

\noindent is evaluated and compared with zero.
Now this expression includes an assignment operator ``=''.
The value of the expression is the value of ``u.u\_error''
{\bf after} the value of ``bp-$>$b\_flags'' has
been assigned to it.

This use of an assignment as part of an
expression is useful and quite common.


\sbs{Example 7}

\begin{verbatim}
  3428 stime ()
       {
         if (suser()) {
           time[0] = u.u_ar0[R0];
           time[1] = u.u_ar0[R1];
           wakeup (tout);
         }
       }
\end{verbatim}

In this example, you should note that
the procedure ``suser'' returns a value
which is used for the ``if'' test. The
three statements whose execution
depends on this value are enclosed in
the brackets ``\{'' and ``\}''.

Note that a call on a procedure with no
parameters must still be written-with a
set of empty parentheses, sic.

\begin{verbatim}
    suser ()
\end{verbatim}

\sbs{Example 8}

``C'' provides a conditional expression.
Thus if ``a'' and ``b'' are integer variables,

\begin{verbatim}
    (a > b ? a : b)
\end{verbatim}

\noindent is an expression whose value is that of
the larger of ``a'' and ``b''.

However this does not work if ``a'' and
``b'' are to be regarded as unsigned
integers. Hence there is a use for the
procedure

\begin{verbatim}
  6326 max (a, b)
       char *a, *b;
       {
         if (a > b)
           return(a);
         return(b);
       }
\end{verbatim}

The trick here is that ``a'' and ``b'',
having been declared as pointers to
characters are treated for comparison
purposes as unsigned integers.

The body of the procedure could have
been written as

\begin{verbatim}
    max (a, b)
    char *a, *b;
    {
      if (a > b)
        return(a);
      else
        return(b);
    }
\end{verbatim}

\noindent but the nature of ``return'' is such that
the ``else'' is not needed here!

\sbs{Example 9}


Here are two quickies which introduce
some different and exotic looking
expressions. First:

\begin{verbatim}
  7679 schar()
       {
         return *u.u_dirp++ & 0377);
       }
\end{verbatim}

\noindent where the declaration

\begin{verbatim}
    char *u_dirp;
\end{verbatim}

\noindent is part of the declaration of the
structure ``u''.

``u.u\_dirp'' is a character pointer.
Therefore the value of ``*u.u\_dirp++'' is
a character. (Incrementation of the
pointer occurs as a side effect.)

When a character is loaded into a sixteen bit register, sign extension may
occur. By ``and''ing the word with 0377
any extraneous high order bits are
eliminated. Thus the result returned
is simply a character.

Note that any integer which begins with
a zero (e.g. 0377) is interpreted as an
octal integer.

The second example is:

\begin{verbatim}
  1771 nseg(n)
       {
         return ((n+127)>>7);
       }
\end{verbatim}

The value returned is n divided by 128
and rounded up to the next highest
``integer''.



Note the use of the right shift operator ``$>>$'' in preference to the division
operator ``/''.

\sbs{Example 10}

Many of the points which have been
introduced above are collected in the
following procedure:

\begin{verbatim}
  2134 setrun (p)
       {
         register struct proc *rp;
         rp = p;
         rp->p_wchan = 0;
         rp->p_stat = SRUN;
         if (rp->p_pri < curpri)
           runrun++;
         if (runout != 0 &&
             (rp->p_flag & SLOAD) == 0) {
            runout = 0;
            wakeup (&runout);
         }
       }
\end{verbatim}


\noindent Check your understanding of ``C'' by
figuring out what this one does.

There are two additional features you
may need to know about:

``\&\&'' is the logical conjunction (``and'')
for relational expressions. (Cf. ``\verb+||+''
introduced earlier.)


The last statement contains the expression

\begin{verbatim}
    &runout
\end{verbatim}

which is syntactically an address variable but semantically just a unique bit
pattern.

This is an example of a device which is
used throughout UNIX. The programmer
needed a unique bit pattern for a particular purpose. The exact value did
not matter as long as it was unique.
An adequate solution to the problem was
to use the address of a suitable global
variable.

\sbs{Example 11}

\begin{verbatim}
  4856 bawrite (bp)
       struct buf *bp;
       {
         register struct buf *rbp;
         rbp = bp;
         rbp->b_flags =| B_ASYNC;
         bwrite (rbp);
       }
\end{verbatim}

The second last statement is interesting because it could have been written
as

\begin{verbatim}
   rbp->b_flags = rbp->b_flags | B_ASYNC;
\end{verbatim}

In this statement the bit mask
``B ASYNC'' is ``or''ed into
``rbp-$>$b\_flags''. The symbol ``\verb+|+'' is the
logical disjunction for arithmetic
values.


This is an example of a very useful
construction in UNIX, which can save
the programmer much labour. If ``O'' is
any binary operator, then


\begin{verbatim}
    x = x O a;
\end{verbatim}

\noindent where ``a'' is an expression, can be
rewritten more succinctly as

\begin{verbatim}
    x =O  a;
\end{verbatim}

A programmer using this construction
has to be careful about the placement
of blank characters, since

\begin{verbatim}
    x =+ 1;
\end{verbatim}

\noindent is different from

\begin{verbatim}
    x = +1;
\end{verbatim}

\noindent What is to be the meaning of

\begin{verbatim}
    x =+1;        ?
\end{verbatim}

\sbs{Example 12}

\begin{verbatim}
  6824 ufalloc ()
       {
         register i;
         for (i=0; i<NOFILE; i++)
           if (u.u_ofile[i]==NULL) {
             u.u_ar0[R0] = i;
             return (i);
           }
         u.u_error = EMFILE;
         return (-1);
       }
\end{verbatim}

This example introduces the ``for''
statement, which has a very general
syntax making it both powerful and compact.

The structure of the ``for'' statement is
adequately described on page 10 of the
``C Tutorial'', and that description is
not repeated here.

The Algol equivalent of the above ``for''
statement would be

\begin{verbatim}
   for i:=1 step 1 until NOFILE-1 do
\end{verbatim}

The power of the ``for'' statement in ``C''
derives from the great freedom the programmer
has in choosing what to include
between the parentheses. Certainly
there is nothing which restricts the
calculations to integers, as the next
example will demonstrate.


\sbs{Example 13}

\begin{verbatim}
  3949 signal (tp, sig)
       {
         register struct proc *p;
         for (p=proc;p<&proc[NPROC];p++)
           if (p->p ttyp == tp)
             psignal (p,sig);
       }
\end{verbatim}


In this example of the ``for'' statement,
the pointer variable ``p'' is stepped
through each element of the array
``proc'' in turn.

Actually the original code had

\begin{verbatim}
   for (p=&proc[0];p<&proc[NPROC];p++)
\end{verbatim}

\noindent but it wouldn't fit on the line! As
noted earlier, the use of ``proc'' as an
alternative to the expression
``\&proc[0]'' is acceptable in this context.

This kind of ``for'' statement is almost
a cliche in UNIX so you had better
learn to recognise it. Read it as

\medskip

{\it for p = each process in turn}

\medskip

\noindent Note that ``proc[NPROC]'' is the address
of the (NPROC+1)-th element of the
array (which does not of course exist)
i.e. it is the first location beyond
the end of the array.


At the risk of overkill we would point
out again that whereas in the previous
example

\begin{verbatim}
    i++;
\end{verbatim}

meant add one to the integer ``i'', here

\begin{verbatim}
    p++;
\end{verbatim}

means ``skip p to point to the next structure''.


\sbs{Example 14}

\begin{verbatim}
  8870 lpwrite ()
       {
        register int c;
        while ((c=cpass()) >= 0)
          lpcanon(c);
       }
\end{verbatim}


This is an example of the ``while''
statement, which should be compared
with the ``do ... while ...'' construction encountered earlier. (Cf. the
``while'' and ``repeat'' statements of Pascal.)

\noindent The meaning of the procedure is

{\it Keep calling ``cpass'' while the
result is positive, and pass the
result as a parameter to a call on
lpcanon.} 

Note the redundant ``int'' in the
declaration for ``c''. It isn't always
omitted!


\sbs{Example 15}

The next example is abbreviated from
the original:

\begin{verbatim}
  5861 seek ()
       {
         int n[2];
         register *fp, t;
         fp = getf (u.u_ar0[R0]);
         ...........
         t = u.u_arg[1];
         switch (t) {

         case 1:
         case 4:
           n[0] =+ fp->f_offset[0];
           dpadd (n, fp->f_offset[1]);
           break;

         default:
           n[0] =+ fp->f_inode->i size0 & 0377;
           dpadd(n,fp->f_inode->i_size1);

         case 0:
         case 3:
           ;
         }
         ...........
       }
\end{verbatim}

Note the array declaration for the two
word array ``n'', and the use of getf
(which appeared in Example 4).

The ``switch'' statement makes a multiway branch depending on the value of
the expression in parentheses. The
individual parts have ``case labels'':

\bi
\item If ``t'' is one or  four,  then  one
set of actions is in order.

\item If ``t'' is zero or  three,  nothing
is to be done at all.

\item If ``t'' is anything  else,  then  a
set  of actions labelled ``default''
is to be executed.
\ei

Note the use of ``break'' as an escape to
the next statement after the end of the
``switch''   statement.    Without    the
``break'',  the normal execution sequence
would be followed within  the  ``switch''
statement.

Thus  a  ``break''  would   normally   be
required  at  the  end of the ``default''
actions.  It has  been  omitted  safely
here  because  the only remaining cases
actually have null  actions  associated
with them.

The two non-trivial  pairs  of  actions
represent  the  addition  of one 32 bit
integer to another.  The later versions
of the ``C'' compiler will support ``long''
variables and make this  sort  of  code
much easier to write (and read).


Note also that in the expression

\begin{verbatim}
    fp->f_inode->i_size0
\end{verbatim}

\noindent there are two levels of indirection.

\sbs{Example 16}

\begin{verbatim}
  6672 closei (ip, rw)
       int *ip;
       {
         register *rip;
         register dev, maj;
         rip = ip;
         dev = rip->i_addr[0];
         maj = rip->i_addr[0].d major;
         switch (rip->i_mode&IFMT) {

         case IFCHR:
           (*cdevsw[maj].d_close)(dev,rw);
           break;
         
         case IFBLK:
           (*bdevsw[maj].d_close)(dev,rw);
         }
         iput(rip);
       }
\end{verbatim}

This example has a number of interesting features.

The declaration for ``d\_major'' is

\begin{verbatim}
    struct {
       char d_minor;
       char d_major;
    }
\end{verbatim}


\noindent so that the value assigned to ``maj''  is
the   hiqh  order  byte  of  the  value
assigned to ``dev''.


In this example, the ``switch'' statement
has  onlv  two  non-null  cases, and no
``default''.  The actions for the  recognised cases, e.g.

\begin{verbatim}
     (*bdevsw[maj].d_close)(dev,rw);
\end{verbatim}

\noindent look formidable


First it should be noted that this is a
procedure  call,  with parameters ``dev''
and ``rw''.

Second  ``bdevsw''  (and  ``cdevsw'')   are
arrays  of  structures, whose ``d\_close''
element is a  pointer  to  a  function,
i.e.

\begin{verbatim}
     bdevsw[maj]
\end{verbatim}

\noindent is the name of a structure, and

\begin{verbatim}
     bdevsw[maj].d_close
\end{verbatim}


\noindent is an element of that  structure  which
happens  to be a pointer to a function,
so that

\begin{verbatim}
     *bdevsw[maj].d_close
\end{verbatim}

\noindent is the name of a function.   The  first
pair  of  parentheses  is  ``syntactical
sugar'' to put the compiler in the right
frame of mind!

\sbs{Example 17}

We offer the following as a final example:

\begin{verbatim}
  4043 psig ()
       {
         register n, p;
         .........
         switch (n) {

         case SIGQIT:
         case SIGINS:
         case SIGTRC:
         case SIGIOT:
         case SIGEMT:
         case SIGEPT:
         case SIGBUS:
         case SIGSEG:
         case SIGSYS:
             u.u arg[0] = n;
             if (core())
               n =+ 0200;
         }
         u.u_arg[0]=(u.u_ar0[R0]<<8) | n;
         exit ();
       }
\end{verbatim}

Here  the  ``switch''   selects certain
values for ``n'' for which the one set of
actions should be carried out.

An alternative would have been to write
a ``monster'' ``if'' statement such as

\begin{verbatim}
   if (n==SIGQUIT || n==SIGINS || ...
              ... || n==SIGSYS)
\end{verbatim}

\noindent but that would  not  have  been  either
transparent or efficient.

Note the addition of an octal  constant
to ``n'' and the method of composing a 16
bit value from two eight bit values.

%
% The Lion's Commentary, file ch4.tex, version 1.4, 18 May 1994
%
\se{An Overview}

The purpose of this chapter is to survey 
the source code as a whole i.e. to
present the ``wood'' before the ``trees''


Examination of the source code will
reveal that it consists of some 44 distinct 
files, of which:

\bi
\item two are in assembly language, and
have names ending in ``s'';

\item 28 are in the ``C'' language and
have names ending in ``c'';

\item 14 are in the ``C'' language, but
are not intended for independent
compilation, and have names ending
in ``h''.
\ei

The files and their contents were
arranged by the programmers presumably
to suit their convenience and not for
ours. In many ways the divisions
between files is irrelevant to the
present discussion and might well be
abolished entirely.

As mentioned already in Chapter One,
the files have been organised into five
sections. As far as was possible, the
sections were chosen to be of roughly
equal size, to cluster files which are
strongly associated and to separate
files which are only weakly associated.


\sbs{Variable Allocation}

The PDP11 architecture allows efficient
access to variables whose absolute
address is known, or whose address
relative to the stack pointer can be
determined exactly at compile time.


There is no hardware support for multiple 
lexical levels for variable
declarations such as are available in
block structured languages such as
Algol or Pascal. Thus ``C'' as implemented 
on the PDP11 supports only two
lexical levels: global and local.


Global variables are allocated statically;
local variables are allocated
dynamically within the current stack
area or in the general registers (r2,
r3 and r4 are used in this way).


\sbs{Global Variables}

In UNIX with very few exceptions, the
declarations for global variables have
been all gathered into the set of ``h''
files. The exceptions are:

\bd
\item[(a)] the static variable ``p'' (2180)
declared in ``swtch'' which is
stored globally, but is accessible 
only from within the procedure 
``swtch'' (Actually ``p'' is
a very popular name for local
variables in UNIX.);

\item[(b)] a number of variables such as
``swbuf'' (4721) which are referenced 
only by procedures within
a single file, and are declared
at the beginning of that file.
\ed

Global variables may be declared
separately within each file in which
they are referenced. It is then the job
of the loader, which links the compiled
versions of the program files together
to match up the different declarations
for the same variable.

\sbs{The `C' Preprocessor}


If global declarations must be repeated
in full in each file (as is required by
Fortran, for instance) then the bulk of
the program is increased, and modifying
a declaration is at best a nuisance,
and at worst, highly error-prone.

These difficulties are avoided in UNIX
by use of the preprocessor facility of
the ``C'' compiler. This allows declarations 
for most global variables to be
recorded once only in one of the few
``h'' files.


Whenever the declaration for a particular 
global variable is required the
appropriate ``h'' file can then be
``included'' in the file being compiled.

UNIX also uses the ``h'' files as vehicles 
for lists of standard definitions
for many symbolic names which represent
constants and adjustable parameters,
and for declaration of some structure
types.

For example, if the file bottle.c
contains a procedure ``glug'' which
global variable called
``gin'' which is declared in the file
``box.h'' then a statement:

\begin{verbatim}
        #include "box.h"
\end{verbatim}

\noindent must be inserted at the beginning of
the file ``bottle.c'' When the file
``bottle.c'' is compiled, all declarations 
in ``box.h'' are compiled, and
since they are found before the beginning 
of any procedure in ``bottle.c''
they are flagged as external in the
relocatable module which is produced.

When all the object modules are linked
together, a reference to ``gin'' will be
found in every file for which the
source included ``box.h'' All these
references will be consistent and the
loader will allocate a single space for
``gin'' and adjust all the references
accordingly.


\sbs{Section One}

Section One contains many of the ``h''
files and the assembly language files.

It also contains a number of files concerned 
with system initialisation and
process management.

\sbs{The First Group of `.h' Files}

\bd
\item[param.h] [Sheet 01] contains no variable 
declarations, but many definitions 
for operating system constants
and parameters, and the declarations
for three simple structures. The
convention will be noted of using
``upper case only'' for defined constants.


\item[systm.h] [Sheet 02; Chapter 19] consists 
entirely of declarations (with
definitions of the structures ``callout''
and ``mount'' as side-effects).
Note that none of the variables is
initialised explicitly, and hence
all are initialised to zero.

The dimensions for the first three
arrays are parameters defined in
param.h. Hence any file which
``includes'' ``systm.h'' must have
previously included ``param.h''.

\item[seg.h] [Sheet 03] contains a few
definitions and one declaration,
which are used for referencing the
segmentation registers. This file
could be absorbed into ``param.h'' and
``systm.h'' without any real loss.

\item[proc.h] [Sheet 03; Chapter 7] contains 
the important declaration for
``proc'' which is both a structure
type and an array of such structures.
Each element of the ``proc''
structure has a name which begins
with ``p\_'' and no other variable is
so named. Similar conventions are
used for naming the elements of the
other structures.

The sets of values for the first two
elements, ``p\_stat'' and ``p\_flag''
have individual names which are
define.


\item[user.h] [Sheet 04; Chapter 7] contains 
the declaration for the very
important ``user'' structure, plus a
set of defined values for ``u\_error''.

Only one instance of the ``user''
structure is ever accessible at one
time. This is referenced under the
name ``u'' and is in the low address
part of a 1024 byte area known as
the ``per process data area''.

In general the complete ``h'' files are
not analysed in detail later in this
text. It is expected that the reader
will refer to them from time to time
(with increasing familiarity and understanding).
\ed


\sbs{Assembly Language Files}

There are two files in assembly
language which comprise about 10\% of
the source code. A reasonable acquaintance 
with these files is necessary.

\bd
\item[low.s] [Sheet 05, Chapter 9] contains
information, including the trap vector,
for initialising the low
address part of main memory. This
file is generated by a utility program 
called ``mkconf'' to suit the set
of peripheral devices present at a
particular installation.

\item[m40.s] [Sheets 06..14; Chapters 6, 8,
9, 10, 22] contains a set of routines 
appropriate to the PDP11/40,
to carry out a variety of specialised 
functions which cannot be
implemented directly in ``C''.
\ed

Sections of this file are introduced
into the discussion as and where
appropriate. (The largest of the
assembler procedures, ``backup'' has
been left to the reader to survey as
an exercise.)

There is an alternative to ``m40.s''
which is not presented here, namely
``m45.s'' which is used on PDP11/45's
and 70's.


\sbs{Other Files in Section One}

\bd
\item[main.c] [Sheets 15..17; Chapters 6,
7] contains ``main'' which performs
various initialisation tasks to get
UNIX running. It also contains
``sureg'' and ``estabur'' which set the
user segmentation registers.


\item[slp.c] [Sheets 18..22; Chapters 6, 7,
8, 14] contains the major procedures
required for process management
including ``newproc'', ``sched'',
``sleep'' and ``swtch''.


\item[prf.c] [Sheets 23, 24; Chapter 5]
contains ``panic'' and a number of
other procedures which provide a
simple mechanism for displaying initialisation 
messages and error messages 
to the operator.

\item[malloc.c] [Sheet 25; Chapter 5] contains 
``malloc'' and ``mfree'' which are
used to manage memory resources.
\ed


\sbs{Section Two}

Section Two is concerned with traps,
hardware interrupts and software interrupts.


Traps and hardware interrupts introduce
sudden switches into the CPU's normal
instruction execution sequence. This
provides a mechanism for handling special 
conditions which occur outside the
CPU's immediate control.

Use is made of this facility as part of
another mechanism called the ``system
call'' whereby a user program may execute 
a ``trap'' instruction to cause a
trap deliberately and so obtain the
operating system's attention and assistance.


The software interrupt (or ``signal'' is
a mechanism for communication between
processes, particularly when there is
``bad news''.

\bd
\item[reg.h] [Sheet 26; Chapter 10] defines
a set of constants which are used in
referencing the previous user mode
register values when they are stored
in the kernel stack.

\item[trap.c] [Sheets 26..28; Chapter 12]
contains the ``C'' procedure ``trap''
which recognises and handles traps
of various kinds.

\item[sysent.c] [Sheet 29; Chapter 12] contains 
the declaration and initialisation 
of the array ``sysent'' which
is used by ``trap'' to associate the
appropriate kernel mode routine with
each system call type.

\item[sysl.c] [Sheets 30..33; Chapters 12,
13] contains various routines associated 
with system calls, including
``exec'' ``exit'' ``wait'' and ``fork''.

\item[sys4.c] [Sheets 34..36; Chapters 12,
13, 19] contains routines for
``unlink'', ``kill'' and various other
minor system calls.

\item[clock.c] [Sheets 37, 38; Chapter 11]
contains ``clock'' which is the
handler for clock interrupts, and
which does much of the incidental
housekeeping and basic accounting.

\item[sig.c] [Sheets 39..42; Chapter 13]
contains the procedures which handle
``signals'' or ``software interrupts''
These provide facilities for interprocess 
communication and tracing.
\ed

\sbs{Section Three}

Section Three is concerned with basic
input/output operations between the
main memory and disk storage.

These operations are fundamental to the
activities of program swapping and the
creation and referencing of disk files.


This section also introduces procedures
for the use and manipulation of the
large (512 byte) buffers.

\bd
\item[text.h] [Sheet 43; Chapter 14]
defines the ``text'' structure and
array. One ``text'' structure is used
to define the status of a shared
text segment.

\item[text.c] [Sheets 43, 44; Chapter 14]
contains the procedures which manage
the shared text segments.

\item[buf.h] [Sheet 45; Chapter 15] defines
the ``buf'' structure and array, the
structure ``devtab'' and names for
the values of ``b\_error'' All these
are needed for the management of the
large (512 byte) buffers.

\item[conf.h] [Sheet 46; Chapter 15]
defines the arrays of structures
``bdevsw'' and ``cdevsw'' which specify
the device oriented procedures
needed to carry out logical file
operations.

\item[conf.c] [Sheet 46; Chapter 15] is
generated, like ``low.s'' by the
``mkconf'' utility to suit the set of
peripheral devices present at a particular 
installation. It contains
the initialisation for the arrays
``bdevsw'' and ``cdevsw'' which control
the basic i/o operations.

\item[bio.c] [Sheets 47..53; Chapters 15,
16, 17] is the largest file after
``m40.s'' It contains the procedures
for manipulation of the large
buffers, and for basic block
oriented i/o.

\item[rk.c] [Sheets 53, 54; Chapter 16] is
the device driver for the RK11/K05
disk controller.
\ed


\sbs{Section Four}

Section Four is concerned with files
and file systems.

A file system is a set of files and
associated tables and directories
organised onto a single storage device
such as a disk pack.


This section covers the means of
creating and accessing files;
locating files via directories
organising and maintaining
file systems.
It also includes the code for an exotic
breed of file called a ``pipe''.


\bd
\item[file.h] [Sheet 55; Chapter 18]
defines the ``file'' structure and
array.

\item[filsys.h] [Sheet 55; Chapter 20]
defines the ``filsys'' structure which
is copied to and from the ``super
block'' on ``mounted'' file systems.


\item[ino.h] [Sheet 56] describes the
structure of ``inodes'' as recorded on
the ``mounted'' devices. Since this
file is not ``included'' in any other,
it really exists for information
only.


\item[inode.h] [Sheet 56; Chapter 18]
defines the ``inode'' structure and
array. ``inodes'' are of fundamental
importance in managing the accesses
of processes to files.

\item[sys2.c] [Sheets 57..59; Chapters 18,
19] contains a set of routines associated
with system calls including
``read'', ``write'', ``creat'', ``open'' and
``close''
 
\item[sys3.c] [Sheets 60, 61; Chapters 19, 20]
contains a set of routines associated
with various minor system
calls.
 
\item[rdwri.c] [Sheets 62, 63; Chapter 18]
contains intermediate level routines
involved with reading and writing
files.
 
 
\item[subr.c] [Sheets 64, 65; Chapter 18]
contains more intermediate level
routines for i/o, especially ``bmap''
which translates logical file
pointers into physical disk
addresses.

\item[fio.c] [Sheets 66..6; Chapters 18,
19] contains intermediate level routines 
for file opening, closing and
control of access.

\item[alloc.c] [Sheets 69..72; Chapter 20]
contains procedures which manage the
allocation of entries in the ``inode''
array and of blocks of disk storage.

\item[iget.c] [Sheets 72..74; Chapters 18,
19, 20] contains procedures concerned 
with referencing and updating
``inodes''.

\item[nami.c] [Sheets 75, 76; Chapter 19]
contains the procedure ``namei'' which
searches the file directories.


\item[pipe.c] [Sheets 77, 78; Chapter 21]
is the ``device driver'' for ``pipes''
which are a special form of short
disk file used to transmit information 
from one process to another.
\ed


\sbs{Section Five}

Section Five is the final section. It
is concerned with input/output for the
slower, character oriented peripheral
devices.

Such devices share a common buffer
pool, which is manipulated by a set of
standard procedures.


The set of character peripheral devices
are exemplified by the following:

\begin{tabular}{ll}\\
{\bf KL/DL11} & interactive terminal\\
{\bf PC11}    & paper tape reader/punch\\
{\bf LP11}    & line printer\\
\end{tabular}


\bd
\item[tty.h] [Sheet 79; Chapters 23, 24]
defines the ``clist'' structure (used
as a list head for character buffer
queues), the ``tty'' structure (stores
relevant data for controlling an
individual terminal), declares the
``partab'' table (used to control
transmission of individual characters 
to terminals) and defines names
for many associated parameters.

\item[kl.c] [Sheet 80; Chapters 24, 25] is
the device driver for terminals connected 
via KL11 or DL11 interfaces.

\item[tty.c] [Sheets 81..85; Chapters 23,
24, 25] contains common procedures
which are independent of the attaching 
interfaces, for controlling
transmission to or from terminals,
and which take into account various
terminal idiosyncrasies.

\item[pc.c] [Sheets 86,87; Chapter 22] is
the device handler for the PC11
paper tape reader/punch controller.

\item[lp.c] [Sheets 88, 89; Chapter 22] is
the device handler for the LP11 line
printer controller.

\item[mem.c] [Sheet 90] contains procedures
which provide access to main memory
as though it were an ordinary file.
This code has been left to the
reader to survey as an exercise.
\ed

%
% The Lion's Commentary, file ch5.tex, version 1.5, 18 May 1994
%
\section*{Section One}

{\sf Section One contains many of the global declaration
files and the assembly language files.

It also comtains a number of files concerned with
system initialisation and process management.}


\se{Two Files}

This chapter is intended to provide a
gentle introduction to the source code
by looking at two files in Section One
which can be isolated reasonably well
from the rest.


The discussion of these files supplements the discussion of Chapter Three
and includes a number of additional
comments regarding the syntax and
semantics of the ``C'' language.


\subsection{The File `malloc.c'}

This file is found on Sheet 25 of the
Source code, and consists of just two
procedures:

\begin{verbatim}
   malloc (2528)   mfree (2556)
\end{verbatim}


These are concerned with the allocation
and subsequent release of two kinds of
memory resources, namely:

\bd
\item[main memory] in units of 32 words (64 bytes);

\item[disk swap area] in units of 256 words (512 bytes).
\ed

For each of these two kinds of
resource, a list of available areas is
maintained within a resource ``map''
(either ``coremap'' or ``swapmap''). A
pointer to the appropriate resource
``map'' is always passed to ``malloc'' and
``mfree'' so that the routines themselves
do not have to know the kind of
resource with which they are dealing.

Each of ``coremap'' and ``swapmap'' is an
array of structures of the type ``map''
as declared at line 2515. This structure consists of two character pointers
i.e. two unsigned integers.

The declarations of ``coremap'' and
``swapmap'' are on lines 0203, 0204.
Here the ``map'' structure is completely
ignored -- a regrettable programming
short-cut which is possible because it
is not detected by the loader. Thus the
actual numbers of list elements in
``coremap'' and ``swapmap'' are ``CMAPSIZ/2''
and ``SMAPSIZ/2'' respectively.


\subsection{Rules for List Maintenance}

\bd
\item[(a)] Each available area is defined
 by its size and relative address
 (reckoned in the units appropriate to the resource);

\item[(b)] The elements of each list are
 arranged at all times in order
 of increasing relative address.
 Care is taken that no two list
 elements represent contiguous
 areas -- the alternative course,
 to merge the two areas into a
 single larger area is always
 taken;
\item[(c)] The whole list can be scanned by
 looking at successive elements
 of the array, starting with the
 first, until an element with a
 zero size is encountered. This
 last element is a ``sentinel''
 which is not part of the list
 proper.
\ed

The above rules provide a complete
specification for ``mfree'', and a
specification for ``malloc'' which is
complete except in one respect:
We need to specify how the
resource allocation is actually
made when there exists more than
one way of performing it.

The method adopted in ``malloc'' is one
known as ``First Fit'' for reasons which
should become obvious.

As an illustration of how the resource
``map'' is maintained, suppose the following
three resource areas were available:

\bi
\item an area of size 15 beginning at
location 47 and ending at location
61;

\item an area of size 13 spanning
addresses 27 to 39 inclusive;

\item an area of size 7 beginning at
location 65.
\ei

\noindent Then the ``map'' would contain:

\begin{center}
\begin{tabular}{ccc}
{\bf Entry} & {\bf Size} & {\bf Address}\\
0 & 13 & 27\\
1 & 15 & 47\\
2 & 7 & 65\\
3 & 0 & ??\\
4 & ?? & ??\\
\end{tabular}
\end{center}

If a request for a space of size 7 were
received, the area would be allocated
starting at location 27, and the ``map''
would become:

\begin{center}
\begin{tabular}{ccc}
{\bf Entry} & {\bf Size} & {\bf Address}\\
0 & 6 & 34\\
1 & 15 & 47\\
2 & 7 & 65\\
3 & 0 & ??\\
4 & ?? & ??\\
\end{tabular}
\end{center}

If the area spanning addresses 40 to 46
inclusive is returned to the available
list, the ``map'' would become:

\begin{center}
\begin{tabular}{ccc}
{\bf Entry} & {\bf Size} & {\bf Address}\\
0 & 28 & 34\\
1 & 7 & 65\\
2 & 0 & ??\\
3 & ?? & ??\\
\end{tabular}
\end{center}

Note how the number of elements has
actually decreased by one because of
amalgamation though the total available
resources have of course increased.

Let us now turn to a consideration of
the actual source code.


\subsection{malloc (2528)}

The body of this procedure consists of
a ``for'' loop to search the ``map'' array
until either:

\bd
\item[(a)] the end of the list of available
 resources is encountered; or

\item[(b)] an area large enough to honour
 the current request is found;
\ed

\bd
\item[2534:] The ``for'' statement initialises
 ``bp'' to point to the first element
	of the resource map. At
 each succeeding iteration ``bp'' is
 incremented to point to the next
 ``map'' structure.
\ed

Note that the continuation condition\\
``bp-$>$m\_size'' is an expression, which becomes zero with the
sentinel is referenced. This
expression could have been written 
equivalently but more transparently as ``bp-$>$m\_size$>$0''.

Note also that no explicit test for the
end of the array is made. (It can be
shown that this latter is not necessary
provided CMAPSIZ, SMAPSIZ $\ge$ 2 * NPROC !)

\bd
\item[2535:] If the list element defines an
 area at least as large as that
 requested, then ...

\item[2536:] Remember the address of the first
 unit of the area;

\item[2537:] Increment the address stored in
 the array element;

\item[2538:] Decrement the size stored in the
 element and compare the result
 with zero (i.e. was it an exact
 fit?);

\item[2539:] In the case of an exact fit, move
 all the remaining list elements
 (up to and including the sentinel) down one place.

Note that ``(bp-l)'' points to the
structure before the one referenced by ``bp'';

\item[2542:] The ``while'' continuation condition does {\bf not}
test the equality of ``(bp-l)-$>$m\_size'' and\\
bm-$>$m\_size !


The value tested is the value assigned to\\
``(bp-$>$m\_size'' copied from ``bp-$>$m\_size''.

(You are forgiven for not
recognising this at once.);


\item[2543:] Return the address of the area.
 This represents the end of the
 procedure and hence very definitely the end of the ``for'' loop.

Note that a value of zero
returned means ``no luck'' This is
based on the assumption that no
valid area can ever begin at
location zero.
\ed

\subsection{mfree (2556)}

This procedure returns the area of size
``size'' at address ``aa'' to the ``resource map''
designated by ``mp''. The body of
the procedure consists of a one line
``for'' statement, followed by a multiline ``if'' statement.

\bd
\item[2564:] The semicolon at the end of this
 line is extremely significant,
 terminating as it does the empty
 statement. (It would aid legibility if this character were moved
 to a line on its own, as is done
 on line 2394.)

Depending on your point of view,
this statement demonstrates
either the power or the obscurity
of the ``C'' language. Try writing
equivalent code to this statement
in another language such as Pascal or PL/1.

Step ``bp'' through the list until
an element is encountered either
with an address greater than the
address of the area being
returned.

 i.e. not ``bp-$>$m\_addr $\le$ a''

\noindent or which indicates the end of the list

 i.e. not ``bp-$>$m\_size != 0'';


\item[2565:] We have now located the element
 in front of which we should
 insert the new list element. The
 question is: Will the list grow
 larger by one element or will
 amalgamation keep the number of
 elements the same or even reduce
 it by one?

If ``bp $>$ mp'' we are not trying to
insert at the beginning of the
list. If\\
(bp-l)-$>$m\_addr+(bp-l)-$>$m\_size==a

\noindent then the area being return abuts
the previous element in the list;

\item [2566:] Increase the size of the previous
list element by the size of the
area being returned;

\item[2567:] Does the area being returned also
 abut the next element of the
 list? If so

\item[2568:] Add the size of the next element
 of the list to the size of the
 previous element;

\item[2569:] Move all the remaining list elements (up to the one containing
 the final zero size) down one
 place.
 
 Note that if the test on line
 2567 fortuitously gives a true
 result when ``bp-$>$m\_size'' is zero
 no harm is done;

\item[2576:] This statement is reached if the
 test on line 2565 failed i.e. the
 area being returned cannot be
 amalgamated with the previous
 element on the list.

 Can it be amalgamated with the
 next element? Note the check that
 the next element is not null;

\item[2579:] Provided the area being returned
 is genuinely non-null (perhaps
 this test should have been made
 sooner?) add a new element to the
 list and push all the remaining
 elements up one place.
\ed


\subsection{In conclusion ...}

The code for these two procedures has
been written very tightly. There is
little, if any, ``fat'' which could be
removed to improve run time efficiency.
However it would be possible to write
these procedures in a more transparent
fashion.


If you feel strongly on this point,
then as an exercise, you should rewrite
``mfree'' to make its function more
easily discernible.

Note also that the correct functioning
of ``malloc'' and ``mfree'' depends on
correct initialisation of ``coremap'' and
``swapmap''. The code to do this occurs
in the procedure ``main'' at lines 1568,
1583.


\subsection{The File `prf.c'}

This file is found on Sheets 23 and 24,
and contains the following procedures:

\begin{verbatim}
    printf (2340)     panic (2416)
    printn (2369)     prdev (2433)
    putchar (2386)    deverror (2447)
\end{verbatim}

The calling relationship between these
procedures is illustrated below:

\newpage
\begin{verbatim}
             panic    deverror
                 |      |
                 |    prdev
                 |      |
                  \    /
                  printf
                    |
                  printn
                    |
                  putchar
\end{verbatim}

\sbs{printf (2340)}

The procedure ``printf'' provides a
direct, unsophisticated low-level,
unbuffered way for the operating system
to send messages to the system console
terminal. It is used during initialisation and to report hardware errors or
the imminent collapse of the system.


(These versions of ``printf'' and
``putchar'' run in kernel mode and are
similar to, but not the same as, the
versions invoked by a ``C'' program which
runs in user mode. The latter versions
of ``printf'' and ``putchar'' live in the
library ``/lib/libc.a''. You may still
find it useful to read the sections
``PRINTF(III)'' and ``PUTCHAR(III)'' of the
UPM at this point.)


\bd
\item[2340:] The programmer must have been
 carried away when he declared all
 the parameters for this procedure. In fact the procedure
 body only contains references to
 ``xl'' and ``fmt''.
\ed


This serves to reveal one of the facts
of ``C'' programming. The rules for
matching parameters in procedure calls
and procedure declarations are not
enforced, not even with respect to the
numbers of parameters.

Parameters are placed on the stack in
{\bf reverse} order. Thus when ``printf'' is
called ``fmt'' will be nearer to the ``top
of stack'' than ``xl'', etc.

\begin{verbatim}
        |   .   |
        ---------
        |   .   |
        ---------
        |   .   |        stack grows down
        ---------
        |   .   |
        ---------
        |  x2   |
        ---------
        |  xl   |
        ---------
        |  fmt  |
        ---------
        |   .   |        top of stack
        ---------
\end{verbatim}

``xl'' has a higher address then ``fmt''
but a lower address then ``x2'', because
stacks grow downwards on the PDP11.

\bd
\item[2341:] ``fmt'' may be interpreted as a
 constant character pointer. This
 declaration is (almost)
 equivalent to

 ``char *fmt;''

The difference is that here the
value of ``fmt'' cannot be changed;

\item[2346:] ``adx'' is set to point to ``xl''.
 The expression ``\&xl'' is the
 address of ``xl''. Note that since
 ``xl'' is a stack location, this
 expression cannot be evaluated at
 compile time.

(Many of the expressions you will
find elsewhere involving the
addresses of variables or arrays
are effective because they {\bf can} be
evaluated at compile or load
time.);

\item[2348:] Extract into the register ``c''
 successive characters from the
 format string;

\item[2349:] If ``c'' is not a `\%' then ...

\item[2350:] If ``c'' is a null character
 (`\verb+\+0'), this indicates the end of
 the format string in the normal
 way, and ``printf'' terminates;

\item[2351:] Otherwise call ``putchar'' to send
 the character to the system console terminal;

\item[2353:] A `\%' character has been seen.
 Get the next character (it had
 better not be the `\verb+\+0'!);

\item[2354:] If this character is a `d' or `l'
 or `o', call ``printn'' passing as
 parameters the value referenced
 by ``adx'' and either the value ``8''
 or ``10'' depending on whether ``c''
 is `o' or not. (The `d' and `l'
 codes are clearly equivalent.)

``printn'' expresses the binary
numbers as a set of digit characters according to the radix 
supplied as the second parameter;

\item[2356:] If the editing character is `s',
 then all but the last character
 of a null terminated string is to
 be sent to the terminal. ``adx''
 should point to a character
 pointer in this case;

\item[2361:] Increment ``adx'' to point to the
 next word in the stack i.e. to
 the next parameter passed to
 ``printf'';

\item[2362:] Go back to line 2347 and continue
 scanning the format string.
 Enthusiasts for structured programming will prefer to replace
 lines 2347 and this by
 ``while (1) \{'' and ``\}''
respectively .
\ed

\sbs{printn (2369)}

This procedure calls itself recursively
in order to generate the required
digits in the required order. It might
be possible to code this procedure more
efficiently but not more completely.
(Anyway, in view of the implementation
of ``putchar'', efficiency is hardly a
consideration here.)

Suppose n = A*b + B where A = ldiv(n,b)
and where B = lrem(n,b) satisfies
$0 \le B < b$. Then in order to display the
value for n, we need to display the
value for A followed by the value for B.


The latter is easy for b = 8 or 10: it
consists of a single character. The
former is easy if A = 0. It is also
easy if ``printn'' is called recursively.
Since A $<$ n, the chain of recursive
calls must terminate.

\bd
\item[2375:] Arithmetic values corresponding
 to digits are conveniently converted to their corresponding
 character representations by the
 addition of the character `0'.
\ed


The procedures ``ldiv'' and ``lrem'' treat
their first parameter as an unsigned
integer (i.e. no sign extension, when a
16 bit value is extended to a 32 bit
value before the actual division operation). They may be found beginning on
lines 1392 and 1400 respectively.


\sbs{putchar (2386)}

This procedure transmits to the system
console the character which was passed
as a parameter.

It illustrates in a small way the basic
features of i/o operations on the PDP11
computer.

\bd
\item[2391:] ``SW'' is defined on line 0166 as
 the value ``0177570''. This is the
kernel address of a read only
processor register which stores
the setting of the console switch
register.

The meaning of the statement is
clear: get the contents at location 0177570 and see if they are
zero. The problem is to express
this in ``C''. The code

if (SW == 0)

would not have conveyed this
meaning. Clearly ``SW'' is a
pointer value which should be
dereferenced. The compiler might
have been changed to accept

if (SW-$>$ == 0)

but as it stands, this is syntactically incorrect. By inventing a
dummy structure, with an element
``integ'' (see line 0175), the programmer has found a satisfactory
solution to his problem.
\ed

Several other examples of this programming device will be found in this
procedure and elsewhere.


In hardware terms, the system console
terminal interface consists of four 16
bit control registers which are given
consecutive addresses on the Unibus
beginning at kernel address 0177560
(see the declaration for ``KL'' on line
0165.) For a description of the formats
and usage of these registers, see
Chapter Twenty-Four of the ``PDP11 Peripherals Handbook''.

In software terms, this interface is
the unnamed structure which is defined
beginning on line 2313, with four elements which name the four control
registers. It does not matter that the
structure is unnamed because it is not
necessary to allocate any instances of
it (the one we are interested in is
essentially predefined, at the address
given by ``KL'').

\bd
\item[2393:] While bit 7 of the transmitter
 status register (``XST'') is off,
 keep doing nothing, because the
 interface is not ready to accept
 another character.
\ed

This is a classic case of ``busy waiting'' where the processor is allowed to
cycle uselessly through a set of
instructions until some externally
defined event occurs. Such waste of
processing power cannot normally be
tolerated but this procedure is only
used in unusual situations.

\bd
\item[2395:] The need for this statement is
 tied up with the statement on
 line 2405;

\item[2397:] Save the current contents of the
 transmitter status register;

\item[2398:] Clear the transmitter status
 register preparatory to sending
 the next character:

\item[2399:] With bit 7 of the control status
 register reset, move the next
 character to be transmitted to
 the transmitter buffer register.
 This initiates the next output
 operation;

\item[2400:] A ``new line'' character needs to
 be accompanied by a ``carriage
 return'' character and this is
 accomplished by a recursive call
 on ``putchar''.

A couple of extra ``delete'' characters are thrown in also to
allow for any delays in completing the carriage return operation
at the terminal;

\item[2405:] This call on ``putchar'' with an
 argument of zero effectively
 results in a re-execution of
 lines 2391 to 2394.

(It is very hard to see why the
programmer chose to use a recursive call here in preference to
simply repeating lines 2393 and
2394, since both code efficiency
and compactness not to mention
clarity seem to have suffered.);

\item[2406:] Restore the contents of the
 transmitter status register. In
 particular if bit 6 was formerly
 set to enable interrupts then
 this resets it.
\ed


\sbs{panic (2419)}

This procedure is called from a number
of locations in the operating system.
(e.g. line 1605). When circumstances
exist under which continued operation
of the system seems undesirable.


UNIX does not profess to be a ``fault
tolerant'' or ``fail soft'' system, and in
many cases the call on ``panic'' can be
interpreted as a fairly unsophisticated
response to a straightforward problem.

However more complicated responses
require additional code, lots of it,
and this is contrary to the general
UNIX philosophy of ``keep it simple''.

\bd
\item[2419:] The reason for this statement is
 given in the comment beginning at
 line 2323;

\item[2420:] ``update'' causes all the large
 block buffers to be written out.
 See Chapter Twenty;

\item[2421:] ``printf'' is called with a format
 string and one parameter, which
 was passed to ``panic'';

\item[2422:] This ``for'' statement defines an
 infinite loop in which the only
 action is a call on the assembly
 language procedure ``idle'' (1284).

``idle'' drops the processor priority to zero, and performs a
``wait''. This is a ``do nothing''
instruction of indefinite duration. It terminates when a
hardware interrupt occurs.

An infinite set of calls on ``idle'' is
better than the execution of a ``halt''
instruction, since any i/o activities
which were under way can be allowed to
complete and the system clock can keep
ticking.

The only way for the operator to
recover from a ``panic'' is to reinitialise the system, (after taking a core
dump, if desired).
\ed


\sbs{prdev (2433), deverror (2447)}

These procedures provide warning messages when errors are occurring in i/o
operations. At this stage, their only
interest is as examples of the use of
``printf''.


\sbs{Included Files}

It will be noted that whereas the file
``malloc.c'' contains no request to
include other files, requests to
include four separate files are
included at the beginning of ``prf.c''.

(The observant reader will note that
these files are presumed to reside one
level higher in the file hierarchy than
``prf.c'' itself.)


The statement on line 2304 is to be
understood as if it were replaced by
the entire contents of the file
``param.h''. This then supplies definitions for the identifiers ``SW'', ``KL''
and ``integ'' which occur in ``putchar''.

We noted earlier that declarations for
``KL'', ``SW'' and ``integ'' occurred on
lines 0165, 0166 and 0175 respectively,
but this would have been meaningless,
if the file ``param.h'' had not been
``included'' in ``prf.c''.

The files ``buf.h'' and ``conf.h'' have
been included to provide declarations
for ``d\_major'', ``d\_minor'', ``b\_dev'' and
``b\_blkno'', which are used in ``prdev''
and ``deverror''.

The reason for the inclusion of the
fourth file, ``seg.h'', is a little
harder to find. In fact it is not
necessary as the code stands, and the
author owes his readers an apology. In
editing the source code, it seemed like
a good idea to move the declaration for
``integ'' from ``seg.h'' to ``param.h''.
Q.E.D.

Note that the variable ``panicstr''
(2328) is also global but since it is
not referenced outside ``prf.c'', its
declaration has not been placed in any
``.h'' file.

%
% The Lion's Commentary, file ch6.tex, version 1.4, 18 May 1994
%
\se{Getting Started}

This chapter considers the sequence of
events which occur when UNIX is
``rebooted'' i.e. it is loaded and initiated in an idle machine.

A study of the initialisation process
is of interest in itself, but more
importantly, it allows a number of
important features of the system to be
presented in an orderly manner.

The operating system may have to be
restarted in the aftermath of a system
crash. It will also have to be restarted frequently for quite ordinary,
operational reasons, e.g. after an
overnight shutdown. If we assume the
latter case, then we can assume that
all the disk files are intact and that
no special circumstance needs to be
recognised or dealt with.

In particular, we can assume there is a
file in the root directory called
``/unix'', which is the object code for
the operating system.

This file began life as a set of source
files such as we are investigating.
These were compiled and linked together
in the normal way to form a single
object program file, and stored in the
root directory.

\sbs{Operator Actions}

Reinitialisation requires operator
action at the processor console. The
operator must:

\bi
\item stop the processor by setting the
``enable/halt'' switch to ``halt'';

\item set the switch register with the
address of the hardware bootstrap
loader program;

\item depress and release the `` load address'' switch;

\item move the ``enable/halt'' switch to
``enable'';

\item depress and release the ``start''
switch.
\ei

This activates the bootstrap program
which is permanently recorded in a ROM
in the processor.

The bootstrap loader program loads a
larger loader program (from block \#0 of
the system disk), which looks for and
loads a file called ``/unix'' into the
low part of memory.

It then transfers control to the
instruction loaded at address zero.

Address zero is occupied by a branch
instruction (line 0508), which branches
to location 000040, which contains a
jump instruction (line 0522), which
jumps to the instruction labelled
``start'' in the file ``m40.s'' (line
0612).

\sbs{start (0612)}

\bd
\item[0613:] The ``enabled'' bit of the memory
 management status register, SR0,
 is tested. If this set, the processor will dwell forever in a
 two instruction loop. This register will normally be cleared when
 the operator activates the
 ``clear'' button on the console
 before starting the system.

A number of reasons have been
suggested for the necessity for
this loop. The most likely is
that in the case of a double bus
timeout error, the processor will
branch to location zero, and in
this situation it should not be
allowed to go further.

\item[0615:] ``reset'' clears and initialises
 all the peripheral device control
 and status registers

{\it The system will now be running in
kernel mode with memory management
disabled.}

\item[0619:] KISA0 and KISD0 are the high core
 addresses of the first pair of
 kernel mode segmentation registers. The first six kernel
 descriptor registers are initialised to 077406, which is the
 description of a full size, 4K
 word, read/write segment.

The first six kernel address
registers are initialised to 0,
0200, 0400, 0600, 01000 and 01200
respectively.

As a result the first six kernel
segments are initialised (without
any reference to the actual size
of UNIX) to point to the first
six 4K word segments of physical
memory. Thus the ``kernel to physical address'' translation is
trivial for kernel addresses in
the range 0 to 0137777;

\item[0632:] ``\_end'' is a loader pseudo variable which defines the extent of
 the program code and data area.
 This value is rounded up to the
 next multiple of 64 bytes and is
 stored in the address register
 for the seventh segment (segment
 \#6).

Note that the address of this
register is stored in ``ka6'', so
that the content of this register
is accessible as ``*ka6'';

\item[0634:] The corresponding descriptor
 register is loaded with a value
 which (since ``USIZE'' is equal to
 16) is the description of a
 read/write segment which is 16 x
 32 = 512 words long.

The value 007406 is obtained by
shifting the octal value 017
eight places to the left and then
``or''ing in the value 6;

\item[0641:] The eighth segment is mapped into
 the highest 4K word segment of
 the physical address space.

It should be noted that with
memory management disabled, the
same translation is already in
force i.e. addresses in the
highest 4K word segment of the
32K program address space are
automatically mapped into the
highest 4K word segment of the
physical address space.
\ed

We may note that from this point on,
all the kernel mode segmentation registers will remain unchanged with the
single exception of the {\bf seventh kernel
segmentation address register}.

This register is explicitly manipulated
by UNIX to point to a variety of
locations in physical memory. Each such
location is the beginning of an area
512 words long, known as a ``per process
data area''.

The seventh kernel address register is
now set to point to the segment which
will become the per process data area
for process \#0.

\bd
\item[0646:] The stack pointer is set to point
 to the highest word of the per
 process data area;


\item[0647:] By incrementing the value of SR0
 from zero to one, the ``memory
 management enabled'' bit is conveniently set.
\ed

{\it From this point, all program addresses
are translated to physical addresses
the memory management hardware.}

\bd
\item[0649:] ``bss'' refers to the second part
 of the program data area, which
 is not initialised by the loader
 (see ``A.OUT(V)'' in the PM). The
 lower and upper limits of this
 area are defined by the loader
 pseudo variables, ``\_edata'' and
 ``\_end'' respectively;

\item[0668:] The processor status word (PS) is
 changed to indicate that the
 ``previous mode'' was ``user mode''.

This prepares the way for the
investigation and initialisation
of the areas of physical memory
which are not part of the kernel
address space. (This involves use
of the special instructions
``mtpi'' and ``mfpi'' (Move To/From
Previous Instruction space)
together with some manipulation
of the user mode segmentation
registers.);

\item[0669:] A call is then made to the procedure ``main'' (1550).
\ed


It will be seen later that ``main'' calls
``sched'' which never terminates. The
need for or use of the last three
instructions of ``start'' (lines 0670,
0671 and 0672) is therefore somewhat
enigmatic. The reason will come later.
In the meantime you might like to
ponder ``why?''. What do these lines do
anyway?


\sbs{main (1550)}

Upon entry to this procedure:

\bd
\item[(a)]  the processor is running at
priority zero, in kernel mode
and with the previous mode shown
as user mode;

\item[(b)] the kernel mode segmentation
registers have been set and the
memory management unit has been
enabled;

\item[(c)] all the data areas used by the
operating system have been initialised;

\item[(d)] the stack pointer (SP or r6)
points to a word which contains
a return address in ``start''.
\ed


\bd
\item[1559:] The first action of ``main'' would
 appear to be redundant, since
 ``updlock'' should have already
 been set to zero as part of the
 initialisation performed by
 ``start'';

\item[1560:] ``i'' is initialised to the ordinal
 of the first 32 word block beyond
 the ``per process data area'' for
 process \#0;

\item[1562:] The first pair of user mode segmentation registers are used to
 provide a ``moving window'' into
 higher areas of the physical
 memory.

At each position of the window an
attempt is made (using ``fuibyte'')
to read the first accessible word
in the window. If this is not
successful, it is assumed that
the end of the physical memory
has been reached.
Otherwise the next 32 word block
is initialised to zero (using
``clearseg'' (0676)) and added to
the list of available memory, and
the window is advanced by 32
words.
\ed

``fuibyte'' and ``clearseg'' are both to be
found in ``m40.s'', ``fuibyte'' will normally return a positive value in the
range 0 to 255. However, in the exceptional case where the memory location
referenced does not respond, the value
--1 is returned. The way this is
brought about is a little obscure, and
will be explained later in Chapter
Ten.)

\bd
\item[1582:] ``maxmem'' defines the maximum
 amount of main memory which may
 be used by a user program. This
 is the minimum of:

\bi
\item the physically available memory (``maxmem'');

\item an installation definable parameter (``MAXMEM'') (0135);

\item the ultimate limit imposed by the PDP11 architecture;
\ei

\item[1583:] ``swapmap'' defines available space
 on the swapping disk which may be
 used when user programs are
 swapped out of main memory. It is
 initialised to a single area of
 size ``nswap'', starting at relative address ``swplo''. Note that
 ``nswap'' and ``swplo'' are initialised in ``conf.c'' (lines 4697,
 4698);

\item[1589:] The significance of this and the
 next four lines will be discussed
 shortly;

\item[1599:] The design of UNIX assumes the
 existence of a system clock which
 interrupts the processor at line
 frequency (i.e 50 Hz or 60 Hz).

There are two possible clock
types available: a line frequency
clock (KW11-L) which has a control register on the Unibus at
address 777546, or a programmable, real-time clock (KW11-P)
located at address 777540 (lines
1509, 1510).

UNIX does not presume which clock
will be present. It attempts to
read the status word for the line
frequency clock first. If successful,
that clock is initialised and the other (if present)
remains unused. If the first
attempt is unsuccessful, then the
other clock is tried. If both
attempts are unsuccessful, there
is a call on ``panic'' which effectively halts the system with an
error message to the operator.
\ed

Since the absence of a clock will be
indicated by a bus timeout error, it is
convenient to make the reference via
``fuiword'', preceded by the setting of a
user mode segmentation register pair
(1599, 1600).


\bd
\item[1607:] Either type of clock is initialised by the statement

*lks = 0115;

As a consequence of this action,
the clock will interrupt the processor within the next 20
 milliseconds. This interrupt may
occur at any time, but it will be
convenient for this discussion to
assume that no interrupt will
occur before initialisation is
complete;

\item[1613:] ``cinit'' (82l4) initialises the
 pool of character buffers. See
 Chapter 23;

\item[1614:] ``binit'' (5055) initialises
 the pool of large buffers. See Chapter 17;

\item[1615:] ``iinit'' (6922) initialises table
 entries for the root device. See
 Chapter Twenty.
\ed

\sbs{Processes}

``process'' is a term which has occurred
more than once already. A definition
which will suit our purposes reasonably
well at present is simply ``a program in
execution''.


Details of the representation of
processes in UNIX will be discussed in
the next chapter. For now we just note
that each process involves a ``proc''
structure from the array called ``proc''
and a ``per process data area'' which
includes one copy of the structure ``u''.


\sbs{Initialisation of proc[0]}

The explicit initialisation of the
structure ``proc[0]'' is performed starting at line 1589. Only four elements
are changed from the overall initial
value of zero:

\bd
\item[(a)] ``p\_stat'' is set to ``SRUN'' which
 implies that process 0 is
 ``ready to run'';

\item[(b)] ``p\_flag'' is set to show both
 ``SLOAD'' and ``SSYS''. The former
 implies that the process is to
 be found in core (it has not
 been swapped out onto the disk),
 and the second, that it should
 never be swapped out;

\item[(c)] ``p\_size'' is set to ``USIZE'';

\item[(d)] ``p\_addr'' is set to the contents
 of the kernel segmentation address register \#6.
\ed


It will be seen that process \#0 has
acquired an area of ``USIZE'' blocks
(exactly the size of a ``per process
data area'') which begins immediately
after the official end (``\_end'') of the
operating system data area.

The ordinal number of the first block
of this area has been stored for future
reference in ``p\_addr''. This area,
which was cleared to zero in ``start''
(0661), contains a single copy of the
user structure called ``u''.

On line 1593, the address of ``proc[0]''
is stored in ``u.u\_procp'', i.e. the
``proc'' structure and the ``u'' structure
are mutually linked.


\sbs{The story continues ...}

\bd
\item[1627:] ``newproc'' (1826) will be discussed in detail in the next
 chapter.

In brief this initialises a
second ``proc'' structure viz.
``proc[1]'', and allocates a second
``per process data area'' in core.
This is a copy of the ``per process data area'' for process \#0,
exact in all but one respect: the
value of ``u.u\_procp'' in the
second area is ``\&proc[1]''.

We should note here that at line
1889, there is a call on ``savu''
(0725) which saves the current
values of the environment and the
stack pointers in ``u.u\_rsav''
before the copy is made.

Also from line 1918 we can see
that the value returned by
``newproc'' will be zero, so that
the statements on lines 1628 to
1635 will not be executed;

\item[1637:] A call is made to ``sched'' (1940)
 which, it may be observed, contains an infinite loop, so that
 it never returns!
\ed

\sbs{sched (1940)}

At this stage we are only interested in
what happens when ``sched'' is entered
for the first time.

\bd
\item[1958:] ``spl6'' is an assembler routine
 (1292) which sets the processor
 priority level to six. (Cf. also
 ``spl0'', ``spl4'', ``spl5'' and ``spl7''
 in ``m40.s'').
\ed

When the processor is at level six,
only devices with priority seven can
interrupt it. The clock whose priority
level is six is thus inhibited from
interrupting the processor between this
point and the subsequent call on ``spl0''
at line 1976.

\bd
\item[1960:] A search is made through ``proc''
 whose status is
``SRUN'' and which is not ``loaded''.
\ed

(Processes \#0 and 1 have status ``SRUN''
and are loaded. All remaining 2193:
processes, have a status of zero, which
is equivalent to ``undefined'' or 
``NULL'' ) .

\bd
\item[1966:] The search fails (``n'' is still
 --1). The flag ``runout'' is made
 non-zero, indicating that there
 are no processes which are both
 ready to run and ``swapped out''
 onto disk;

\item[1968:] ``sleep'' is called (to wait for
 such an event) with a priority
 ``PSWP'' (== --100) for when it
 wakes up, which is in the
 category of ``very urgent''.
\ed

\sbs{sleep (2066)}

\bd
\item[2070:] ``PS'' is the address of the processor status word. The processor
 status is stored in the register
 ``s'' (0164, 0175);

\item[2071:] ``rp'' is set to the address of the
 entry in the array ``proc'' of the
 current process (still ``proc[0]''
 at this stage!);

\item[2072:] ``pri'' is negative, so the ``else''
 branch is taken, setting the
 status of the current process
 (0) to ``SSLEEP''. The reason for
 ``going to sleep'' and the ``awakening priority'' are noted.

\item[2093:] ``swtch'' is then called.
\ed

\sbs{swtch (2178)}

\bd
\item[2184:] ``p'' is a static variable (2180),
 which means that its value is
 initialised to zero (1566) and is
 preserved between calls. For the
 very first call on ``swtch'', ``p''
 is set to point to ``proc[0]'';

\item[2189:] ``savu'' is called to save the
stack pointer and the environment
pointer for the current process
in ``u.u\_rsav'';

\item[2193:] ``retu'' is called
to reset the kernel address
register for segment \#6 to the
value passed as an argument
(this causes a change in the
current process!), and
 to reset the stack and environment pointers to values
 appropriate to the revised
 current process, whose execution
 is about to be resumed.
\ed

The combination of successive calls on
``savu'' and ``retu'' at this point constitutes a so-called ``coroutine jump''
(Cf.
``exchange jump'' on the Cyber or ``Load
PSW'' on the /360 or ``Move Stack'' on the
B6700).

This time however the coroutine jump is
from process 0 to process 0 (not very
interesting!).

\bd
\item[2201:] The set of processes is searched
 to find the process whose state
 is ``SRUN'' and which is loaded and
 for which ``p\_pri'' is a maximum.

The search is successful and process \#1 is found. (N.B. The
state of process \#0 was just
changed from ``SRUN'' to ``SSLEEP''
in ``sleep'' so it no longer satisfies the search criterion);

\item[2218:] Since ``p'' is not ``NULL'', the idle
 loop is not entered;

\item[2228:] ``retu'' (0740) causes a coroutine
 jump to process \#1 which becomes
the current process.

What is process \#1 ? It is a copy
of process \#0, made at a previous
stage of the latter's existence.
\ed

This call on ``retu'' was not preceded by
a call on ``savu'' because the necessary
information has in fact been saved
already. (Where?)

\bd
\item[2229:] ``sureg'' is a routine 1738) which
 copies into the user mode segmentation registers, the values
 appropriate for the current process.
	These have been stored earlier in the arrays ``u.u\_uisa'' and
 ``u.u \_uisd''.
\ed

The very first call on ``sureg'' copies
zeroes and serves no real purpose.

\bd
\item[2240:] The ``SSWAP'' flag is not set, so
 that this enigmatic (2239) section can be
ignored for now;


\item[2247:] Finally ``swtch'' returns with a
 value of ``1''. But where does the ``return'' return to?
{\bf Not to ``sleep'' !}
\ed


The ``return'' follows values set by the
stack pointer and the environment
pointer. These (just before the return)
have values equal to those in force
when the most recent ``savu(u.u\_rsav)''
was performed.

Now process \#1, which is only just
starting has never performed a ``savu'',
but values were stored in ``u.u\_rsav''
before the copy of process \#0 was made
by ``newproc'', which had been called
from ``main''.

Thus in this case, {\it the return from
``swtch'' is made to ``main'', with a value
of one.} (Look over this again, to be
sure you understand!)

\sbs{main revisited}

The story so far: process \#0, having
created a copy of itself in the form of
process \#1, has gone to sleep. As a
result process \#1 has become the
current process and has returned to
``main ''with a value of one. Now read
on ...

\bd
\item[1627:] The statements in ``main'' which
 are conditional on ``newproc'' are
 now executed:

\item[1628:] ``expand'' (2268) finds a new,
 larger area (from USIZE*32 to
 (USIZE+1) *32 words) for process
 \#1, and copies the original data
 area into it.

In this case, the original user
data area consists only of a ``per
process data area'', with zero
length data and stack areas. The
original area is released;

\item[1629:] ``estabur'' is used to set the
 ``prototype'' segmentation registers which are stored in
 ``u.u\_uisa'' and ``u.u\_uisd'' for
 later use by ``sureg''. ``estabur''
 calls ``sureg'' as its last action.


The parameters for ``estabur'' are
the sizes of the text, data and
stack areas plus an indicator to
decide whether the text and data
areas should be in separate
address spaces. (Never true on
the PDP11/40.) The sizes are all
in units of 32 words;

\item[1630:] ``copyout'' (1252) is an assembler
 routine which copies an array in
 kernel space of specified size
 into a region in user space. Here
 the array ``icode'' is copied into
 an area starting at location zero
 in user space;

\item[1635:] The ``return'' is not special. From
 ``main'' it goes to ``start'' (0670)
 where the three last instructions
 have the effect of causing
 {\it execution in user mode of the
 instruction at user mode address
 zero.} i.e. the execution of a
copy of the first instruction in
``icode''. The instructions subsequently executed are copies also
of instructions in ``icode''.
\ed

AT THIS POINT, THE INITIALISATION OF
THE SYSTEM IS COMPLETE.

Process \#1 is running and to all
intents and purposes, is a normal process. Its initial form is (almost)
that which would come from compilation,
loading and execution of the simple,
but non-trivial ``C'' program:

\begin{verbatim}
    char *init "/etc/init";
    main ( ) {
    execl (init, init, 0);
    while (1);
    }
\end{verbatim}

The equivalent assembler program is

\begin{verbatim}
           sys exec
           init
           initp
           br   .
    initp: init
           0
    init:  </etc/init\0>
\end{verbatim}

If the system call on ``exec'' fails
(e.g. the file ``/etc/init'' cannot be
found) the process falls into a tight
loop, and there the processor will
stay, except when the occasional clock
interrupt occurs.


A description of the functions performed by ``/etc/init'' can be found in
the section ``INIT (VIII)'' of the UPM.

%
% The Lion's Commentary, file ch7.tex, version 1.4, 18 May 1994
%
\se{Processes}

The previous chapter traced the
developments which occur after ``the
operating system has been rebooted'',
and in so doing introduced a number of
significant features of the process
concept. One of the aims of this
chapter is to go back and re-explore
some of the same ground more
thoroughly.

There are a number of serious difficulties 
in providing a generally acceptable definition of ``process''. These are
akin to the difficulties faced by the
philosopher who would answer ``what is
life?'' We will be in good company if we
brush the more subtle points lightly
aside.

The definition for ``process'' already
given, ``a program in execution'', does
reasonably well in suggesting what is
intended. However it does not fit the
case of either process \#0 throughout
its life or process \#1 during its first
moments. All other processes in the
system however are clearly associated
with the execution of some program file
or other.

Processes can be introduced into discussions of operating systems at two
levels.

At the upper level, ``process'' is an
important organising concept for
describing the activity of a computer
system as a whole. It is often
expedient to view the latter as the
combined activity of a number of
processes, each associated with a particular program such as
the ``shell'', or
the ``editor''. A discussion of UNIX at
this level is given in the second half
of Ritchie's and Thompson's paper, ``The
UNIX Time-sharing System''.


At this level the processes themselves
are considered to be the active entities in the system, while the 
identities of the true active elements, the
processor and the peripheral devices,
are submerged: the processes are born,
live and die; they exist in varying
numbers; they may acquire and release
resources; they may interact,
cooperate, conflict, share resources;
etc.

At the lower level, ``processes'' are
inactive entities which are acted on by
active entities such as the processor.
By allowing the processor to switch
frequently from the execution of one
process image to another, the impression can
be created that each of the process images is
developing continuously and this leads to the upper level
interpretation.

Our present concern is with the low
level interpretation: with the structure of the process image, with the
details of execution and with the means
for switching the processor between
processes.

The following observations may be made
about processes in the UNIX context:

\bd
\item[(a)] the existence of a process is
implied by the existence of a
non-null structure in the ``proc''
array, i.e. a ``proc'' structure
for which the element ``p\_stat''
is non-null;

\item[(b)] for each process there is a ``per
process data area'' containing a
copy of the ``user'' structure;

\item[(c)] the processor spends its entire
life executing one process or
another (except when it is resting between instructions);

\item[(d)] it is possible for one process
to create or destroy another
process;

\item[(e)] a process may acquire and possess resources of various kinds.
\ed


\sbs{The Process}

Ritchie and Thompson in their paper
define a ``process'' as the execution of
an ``image'', where the ``image'' is the
current state of a pseudo-computer,
i.e. an abstract data structure, which
may be represented in either main
memory or on disk.


The process image involves two or three
physically distinct areas of memory:

\bd
\item[(1)] {\bf the ``proc'' structure}, which is
 contained within the core
 resident ``proc'' array and is
 accessible at all times;

\item[(2)] {\bf the data segment}, which consists of the ``per process data
area'', combined with a segment
containing the user program
data, (possibly) program text,
and stack;

\item[(3)] {\bf the text segment}, which is not
always present, consists of a
segment containing only pure
program text i.e. re-entrant
code and constant data.
\ed

Many programs do not have a separate
text segment. Where one is defined, a
single copy will be shared among all
processes which are executions of the
same particular program.


\sbs{The proc Structure (0358)}

This structure, which is permanently
resident in main memory, contains fifteen
elements, of which eight are characters, six are integers, and one a
pointer to an integer. Each element
represents information that must be
accessible at any time, especially when
the main part of the process image has
been swapped out to disk:

\bi
\item ``p\_stat'' may take one of seven
values which define seven mutually
exclusive states. See lines 0381
to 0387;

\item ``p\_flag'' is an amalgam of six one
bit flags which may be set
independently. See lines 0391 to
0396;

\item ``p\_addr'' is the address of the
data segment:

\bi
\item If the data segment is in main
memory this is a block number;

\item otherwise, if the data segment
has been swapped out, this is a
disk record number;
\ei

\item ``p\_size'' is the size of the data
segment, measured in blocks;

\item ``p\_pri'' is the current process
priority. This may be recalculated
from time to time as a function of
``p\_nice'', ``p\_cpu'' and ``p\_time'';

\item ``p\_pid'', ``p\_ppid'' are numbers
which uniquely identify a process
and its parent;

\item ``p\_sig'', ``p\_uid'', ``p\_ttyp'' are
involved with external communication
i.e. with messages or ``signals'' from outside the process's
normal domain;

\item ``p\_wchan'' identifies, for a
``sleeping'' process (``p\_stat''
equals either ``SSLEEP'' or
``SWAIT''), the reason for sleeping;

\item ``p\_textp'' is either null or a
pointer to an entry in the ``text''
array (4306), which contains vital
statistics regarding the text segment.
\ei

\sbs{The user Structure (0413)}

One copy of the ``user'' structure is an
essential ingredient of each ``per process data area''.
At any one time there
is exactly one copy of the ``user''
structure which is accessible. This
goes under the name ``u'' and is always
to be found at kernel address 0140000
i.e. at the beginning of the seventh
page of the kernel address space.

The ``user'' structure has more elements
than can be conveniently or usefully
introduced here. The comment accompanying
 each declaration on Sheet 04 succinctly suggests the function of each
element.

For the moment you should notice:

\bd
\item[(a)] ``u\_rsav'', ``u\_qsav'' , ``u\_ssav''
 which are two word arrays used
 to store values for r5, r6;

\item[(b)] ``u\_procp'' which gives the
 address of the corresponding
 ``proc'' structure in the ``proc''
 array;

\item[(c)] ``u\_uisa[16]'', ''u\_uisd[16]'' which
 store prototypes for the page
 address and description registers;

\item[(d)] ``u\_tsize'', ``u\_dsize'', ``u\_ssize''
 which are the size of the text
 segment and two parameters
 defining the size of the data
 segment, measured in 32 word
 blocks.
\ed

\noindent The remaining elements are concerned
with:

\bi
\item saving floating point registers
 (not for the\\
PDP11/40);

\item user identification;

\item parameters for input/output operations;

\item file access control;

\item system call parameters;

\item accounting information.
\ei

\sbs{The Per Process Data Area}

The ``per process data area'' corresponds
to the valid part (lower part) of the
seventh page of the kernel address
space. It is 1024 bytes long. The lower
289 bytes are occupied by an instance
of the ``user'' structure, leaving 367
words to be used as a kernel mode stack
area. (Obviously there will be as many
kernel mode stacks as there are
processes.)

While the processor is in kernel mode,
the values of r5 and r6, the environment and stack pointers, should remain
within the range 0140441 to 01437777.
Transition beyond the upper limit would
be trapped as a segmentation violation,
but the lower limit is protected only
by the integrity of the software. (It
may be noted that the hardware stack
limit option is not used by UNIX.)

\sbs{The Segments}

The data segment is allocated as one
single area of physical memory but consists of three distinct parts:

\bd
\item[(a)] a ``per process data area'';

\item[(b)] a data area for the user program. This may be further
 divided into areas for program
 text, initialised data and uninitialised data;

\item[(c)] a stack for the user program.
\ed


The size of (a) is always ``USIZE''
blocks. The sizes of (b) and (c) are given in blocks by ``u.u\_dsize'' and
``u.u\_ssize''. (It may be noted in passing
that the latter two may change during the life of a process.)


A separate text segment containing only
pure text is allocated as one single
area of physical memory. The internal
structure of the segment is not important here.


\sbs{Execution of an Image}

The image currently being executed (and
hence the identity of the current process) is determined by the setting of
the seventh kernel segmentation address
register. If process \#i is the current
process, then the register has the
value ``proc[i].p\_addr''.

It is often desirable to distinguish
between a process being executed in
kernel mode and the same one being executed in user mode. We will use the
terms ``kernel process \#i'' and ``user
process \#i'' to denote ``process \#i executing in kernel mode''
and ``process \#i
executing in user mode'' respectively.

If we chose to associate processes with
particular execution stacks rather than
with an entry in the ``proc'' array, then
we would consider kernel process \#i and
user process \#i to be separate
processes, rather than different
aspects of a single process \#i.

\sbs{Kernel Mode Execution}

The seventh kernel segmentation address
register must be set appropriately.
None of the other kernel segmentation
registers is ever disturbed and so
their values are assumed. As was seen
earlier, the first six kernel pages are
mapped to the first six pages of physical memory, while the eighth is mapped
into the highest page of physical
memory. The size of the seventh segment
is always the same.

In kernel mode the setting of the user
mode segmentation registers is in general
irrelevant. However they are normally
set correctly for the user process.

The environment and stack pointers
point into the kernel stack area in the
seventh page, above the ``user'' structure.

\sbs{User Mode Execution}

Each activation of a user process is
preceded and succeeded by an activation
of the corresponding kernel process.
Accordingly both the user mode and kernel mode registers will be properly set
whenever a process image is being executed in user mode.

The environment and stack pointers
point into the user stack area. This
begins as the upper part of the eighth
user page, but may be extended downwards, e.g. to occupy the whole of
the eighth page and part or all of the
seventh page, etc.

Whereas the setting of the kernel segmentation registers is fairly trivial,
setting the user segmentation registers
is much less so.


\sbs{An Example}

Consider a program on the PDP11/40
which uses 1.7 pages of text, 3.3 pages
of data, and 0.7 pages of stack area.
(Our use of fractions in this example
is admittedly a little crude.) The set
of virtual addresses would be divided
as shown in the following diagram:

\begin{center}
\begin{tabular}{|l|l} \cline{1-1}
888  s1 & Stack \\
888  s1 & area \\
\cline{1-1}
888     & \\
\cline{1-1}
777     & \\
777     & \\
777     & \\
\cline{1-1}
666     & \\
666     & \\
\cline{1-1}
666  d4 & \\
\cline{1-1}
555  d3 & \\
555  d3 & \\
555  d3 & \\
\cline{1-1}
444  d2 & Data \\
444  d2 & \\
444  d2 & area \\
\cline{1-1}
333  d1 & \\
333  d1 & \\
333  d1 & \\
\cline{1-1}
222     & \\
\cline{1-1}
222  t2 & \\
222  t2 & Text \\
\cline{1-1}
111  t1 & \\
111  t1 & area \\
111  t1 & \\
\cline{1-1}
\end{tabular}

\medskip

Virtual Address Space
\end{center}


Two whole pages in the virtual address
space must be allocated to the text
segment, even though the physical area
required is only 1.7 pages.
 
\begin{center}
\begin{tabular}{|l|l} \cline{1-1}
\cline{1-1}
222  t2 & \\
222  t2 & Text \\
\cline{1-1}
111  t1 & \\
111  t1 & area \\
111  t1 & \\
\cline{1-1}
\end{tabular}

\medskip

Text Segment
\end{center}

The data and stack areas require the
dedication of four and one pages of
virtual address space, and 3.3 and 0.7
pages of physical memory respectively.

The whole data segment requires four
and one eighth pages of physical
memory. The extra eighth is for the
``per process data area'' which
corresponds (from time to time) to the
seventh kernel address page.


\begin{center}
\begin{tabular}{|l|l} \cline{1-1}
888  s1 & Stack \\
888  s1 & area \\
\cline{1-1}
666  d4 & \\
\cline{1-1}
555  d3 & \\
555  d3 & \\
555  d3 & \\
\cline{1-1}
444  d2 & Data \\
444  d2 & \\
444  d2 & area \\
\cline{1-1}
333  d1 & \\
333  d1 & \\
333  d1 & \\
\cline{1-1}
\multicolumn{1}{|r|}{ppda} & \\
\cline{1-1}
\end{tabular}

\medskip

Data Segment
\end{center}

\noindent Note the order of the components of the
data segment, and that there is no
embedded unused space.

The user mode segmentation need to be
set to reflect the values in the following table, where ``t'', ``d'' denote the
block numbers of beginning of the text
and data segments respectively:

\medskip

\begin{tabular}{llll}
{\bf Page}  & {\bf Address} & {\bf Size} & {\bf Comment} \\
\hline
1 & t+0    & 1.0 & read only \\
2 & t+128  & 0.7 & read only \\
3 & d+16   & 1.0 & \\
4 & d+144  & 1.0 & \\
5 & d+272  & 1.0 & \\
6 & d+400  & 0.3 & \\
7 & ?      & 0.0 & not used \\
8 & d+400  & 0.7 & grows downward \\
\end{tabular}

\medskip

Note the setting of the eighth address
register. The address prototypes stored
in the array ``u.u\_uisa'' are obtained by
setting ``t'' and ``d'' to zero.

\sbs{Setting the Segmentation Registers}

Prototypes for the user segmentation
registers are set up by ``estabur'' which
is called when a program is first
launched into execution, and again
whenever a significant change in memory
allocation requires it. The prototypes
are stored in the arrays ``u.u\_uisa'',
``u.u\_uisd''.

Whenever process \#i is about to be reactivated, the procedure ``sureg'' is
called to copy the the prototypes into
the appropriate registers. The description registers are copied directly, but
the address registers must be adjusted
to reflect the actual location in physical memory of the area used.


\sbs{estabur (1650)}

\bd
\item[1654:] Various checks on consistency are
 performed, to ensure that the
 requested sizes for the text,
 data and stack are reasonable.

Note that a non-zero value for
``sep'' implies separate mappings
for the text area (``i'' space) and
the data area (``d'' space). This
is never possible on the
PDP11/40;

\item[1664:] ``a'' defines the address of a segment relative to an arbitrary
 base of zero. ``ap'' and ``dp'' point
 to the set of prototype segmentation address and descriptor
 registers respectively.
\ed


The first eight of each of these sets
are intended to refer to ``i'' space, and

\bd
\item[1667:] ``nt'' measures the number of 32
 word blocks needed for the text
 segment. If ``nt'' is non-zero,
 one or more pages must be allocated for this purpose.
\ed

Where more than one page is allocated,
all but the last will consist of 128
blocks (4096 words), and will be read
only, and will have relative addresses
starting at zero and increasing successively by 128.

\bd
\item[1672:] If some fraction of a page of
 text is still to be assigned,
 allocate the appropriate part of
 the next page;

\item[1677:] if ``i'' and ``d'' spaces are being
 used separately, mark the segmentation registers for the 
remaining ``i'' pages as null;

\item[1682:] ``a'' is reset because all remaining addresses refer to the data
area (not the text area) and are
relative to the beginning of this
area. The first ``USIZE'' blocks
of this area are reserved for the
``per process data area'';

\item[1703:] The stack area is allocated from
 the top of the address space
 towards the lower addresses
 (``downwards'');

\item[1711:] If a partial page must be allocated for the stack area, it is
 the high address art of the page
 which is valid. (For text and
 data areas, which grow ``upwards'',
 it is the lower part of a partial
 page which is valid.) This
 requires an extra bit in the
 descriptor, hence ``ED'' (``expansion downwards'');

\item[1714:] If separate ``i'' and ``d'' spaces
 are not used, only the first
 eight of the sixteen prototype
 register pairs will have been
 initialised by this point. In
 this case, the second eight are
 copied from the first eight.
\ed


\sbs{sureg (1739)}

This routine is called by ``estabur''
(1724), ``swtch'' (2229) and ``expand''
(2295), to copy the prototype
segmentation registers into the actual
hardware segmentation registers.


\bd
\item[1743:] Get the base address for the data
 area from the appropriate element
 of the ``proc'' array;

\item[1744:] The prototype address registers
 (of which there are only eight
 for the PDP11/40) are modified by
 the addition of ``a'' and stored in
 the hardware segmentation address
 registers;

\item[1752:] Test if a separate text area has
 been allocated, and if so, reset
 ``a'' to the relative address of
 the text area to the data area.
 (Note this value may be negative!
 Fortunately at this point,
 addresses are in terms of 32 word
 blocks.);

\item[1754:] The pattern of code now followed
 is similar to the beginning of
 the routine, except ...

\item[1762:] a rather obscure piece of code
adjusts the setting of the
address register for segments
which are not ``writable'' i.e.
which presumably are text segments.
\ed

The code in ``estabur'' and ``sureg'' shows
evidence of having been developed in
several stages and is not as elegant as
could be desired.

\sbs{newproc (1826)}

It is now time to take a good look at
the procedure which creates new
processes as (almost exact) replicas of
their creators.


\bd
\item[1841:] ``mpid'' is an integer which is
 stepped through the values 0 to
 32767. As each new process is
 created, a new value for ``mpid''
 is created to provide a unique
distinguishing number for the
process. Since the cycle of
values may eventually repeat, a
check is made that the number is
not still in use; if so a new
value is tried;

\item[1846:] A search is made through the
 ``proc'' array for a null ``proc''
 structure (indicated by ``p\_stat''
 having a null value);


\item[1860:] At this point, the address of the
new entry in the ``proc'' array is
stored as both ``p'' and ``rpp'', and
the address of ``proc'' entry for
the current process is stored both
as ``up'' and ``rip'';

\item[1861:] The attributes of the new process
are stored in the new ``proc''
entry. Many of these are copied
from the current process;

\item[1876:] The new process inherits the open
files of its parent. Increment
the reference count for each of
these;

\item[1879:] If there is a separate text segment increment the associated
reference counts. Notice that
``rip'', ``rpp'' are used for temporary reference here;

\item[1883:] Increment the reference count for
the parent's current directory;

\item[1889:] Save the current values of the
 environment and stack pointers in
``u.u\_rsav''. ``savu'' is an assembler
routine defined at line 0725;

\item[1890:] Restore the values of ``rip'' and
``rpp''. Temporarily change the
value of ``u.u\_procp'' from the
value appropriate to the current
process to the value appropriate
to the new process;

\item[1896:] Try to find an area in main
 memory in which to create the new
 data segment;

\item[1902:] If there is no suitable area in
 main memory, the new copy will
have to be made on disk. The
next section of code should be
analysed carefully because of the
inconsistency introduced at line
1891 i.e.\\
u.u\_procp-$>$p\_addr != *ka6

\item[1903:] Mark the current process as
 ``SIDL'' to head off temporarily
 any further attempt to swap it
 out (i.e. initiated by ``sched''
 (1940));

\item[1904:] Make the new ``proc'' entry consistent,
i.e set rpp-$>$p\_addr = *ka6;

\item[1905:] Save the current values of the
 environment and stack pointers in
 ``u.u\_ssav'';

\item[1906:] Call ``xswap'' (4368) to copy the
 data segment into the disk swap
 area. Because the second parameter is zero, the main memory area
 will not be released;

\item[1907:] Mark the new process as ``swapped
 out'';

\item[1908:] Return the current process to its
 normal state;

\item[1913:] There was room in main memory, so
 store the address of the new
 ``proc'' entry and copy the data
 segment a block at a time;

\item[1917:] Restore the current process'
 ``per process data area'' to its
 previous state;

\item[1918:] Return with a value of zero.
\ed


Obviously ``newproc'' on its own is not
sufficient to produce an interesting
and varied set of processes. The procedure ``exec'' (3020) which is discussed
in Chapter Twelve provides the necessary additional facility: the means for
a process to change its character, to
be reincarnated.

%
% The Lion's Commentary, file ch8.tex, version 1.4, 18 May 1994
%
\se{Process Management}

Process management is concerned with
the sharing of the processor and the
main memory amongst the various
processes, which can be seen as competitors for these resources.

Decisions to reallocate resources are
made from time to time, either on the
initiative of the process which holds
the resource, of for some other reason.


\sbs{Process Switching}

An active process may suspend itself
i.e relinquish the processor, by calling ``swtch'' (2178) which calls ``retu''
(0740).
This may be done for example if a process has reached a point beyond which
it cannot proceed immediately. The process calls ``sleep'' (2066) which calls
``swtch''.

Alternatively a kernel process which is
ready to revert to user mode will test
the variable ``runrun'' and if this is
non-zero, implying that a process with
a higher precedence is ready to run,
the kernel process will call ``swtch''.

``swtch'' searches the ``proc'' table, for
entries for which ``p\_stat'' equals
``SRUN'' and the ``SLOAD'' bit is set in
``p\_flag''. From these it selects the
process for which the value of ``p\_pri''
is a minimum, and transfers control to it.

Values for ``p\_pri'' are recalculated for
each process from time to time by use
of the procedure ``setpri'' (2156). Obviously the algorithm used by ``setpri''
has a significant influence.
A process which has called ``sleep'' and
suspended itself may be returned to the
``ready to run'' state by another process. This often occurs during the
handling of interrupts when the process
handling the interrupt calls ``setrun''
(2134) either directly or indirectly
via a call on ``wakeup'' (2113).


\sbs{Interrupts}

It should be noted that a hardware
interrupt (see Chapter Nine) does not
directly cause a call on ``swtch'' or its
equivalent. A hardware interrupt will
cause a user process to revert to a
kernel process, which as just noted,
may call ``swtch'' as an alternative to
reverting to user mode after the interrupt handling is complete.

If a kernel process is interrupted,
then after the interrupt has been handled, the kernel process resumes where
it had left off regardless. This point
is important for understanding how UNIX
avoids many of the pitfalls associated
with ``critical sections'' of code, which
are discussed at the end of this
chapter.


\sbs{Program Swapping}

In general there will be insufficient
main memory for all the process images
at once, and the data segments for some
of these will have to be ``swapped out''
i.e. written to disk in a special area
designated as the swap area.

While on disk the process images are
relatively inaccessible and certainly
unexecutable. The set of process
images in main memory must therefore be
changed regularly by swapping images in
and out. Most decisions regarding
swapping are made by the procedure
``sched'' (1940) which is considered in
detail in Chapter Fourteen.

``sched'' is executed by process \#0,
which after completing its initial
tasks, spends its time in a double
role: openly as the ``scheduler'' i.e. a
normal kernel process; and surreptitiously as the intermediate process of
``swtch'' (discussed in Chapter Seven).
Since the procedure ``sched'' never terminates,
kernel process \#0 never completes its task, and so the question of
a user process \#0 does not arise.


\sbs{Jobs}

There is no concept of ``job'' in UNIX,
at least in the sense in which this
term is understood in more conventional, batch processing oriented systems.

Any process may ``fork'' a new copy of
itself at any time, essentially without
delay, and hence create the equivalent
of a new job. Hence job scheduling,
job classes, etc. are non-events here.

\sbs{Assembler Procedures}

The next three procedures are written
in assembler and run with the processor
priority level set to seven. These
procedures do not observe the normal
procedure entry conventions so that r5
and r6, the environment and stack
pointers, are not disturbed during procedure entry and exit.

As has already been noted, ``savu'' and
``retu'' can combine to produce the
effect of a coroutine jump. The third
procedure, ``aretu,'' when followed by a
``return'' statement produces the effect
of a non-local ``goto''.


\sbs{savu (0725)}

This procedure is called by ``newproc''
(1889, 1905), ``swtch'' (2189, 2281),
``expand'' (2284), ``trapl'' (2846) and
``xswap'' (4476,4477).

The values of r5 and r6 are stored in
the array whose address is passed as a
parameter.


\sbs{retu (0740)}

This procedure is called by ``swtch''
(2193, 2228) and ``expand'' (2294).

It resets the seventh kernel segmentation address register, and then resets
r6 and r5 from the newly accessible
copy of ``u.u\_rsav'' (which it may be
noted, is at the beginning of ``u'').


\sbs{aretu (0734)}

This procedure is called by ``sleep''
(2106) and ``swtch'' (2242).

It reloads r6 and r5 from the address
passed as a parameter.

\sbs{swtch (2178)}

``swtch'' is called by ``trap'' (0770,
0791), ``sleep'' (2084, 2093), ``expand''
(2287), ``exit'' (3256), ``stop'' (4027)
and ``xalloc'' (4480).

This procedure is unique in that its
execution is in three phases which in
general involve three separate kernel
processes. The first and third of
these processes will be called the
``retiring'' and the ``arising'' processes
respectively. Process \#0 is always the
intermediate process; it may be the
``retiring'' or the ``arising'' process as
well.


Note that the only variables used by
``swtch'' are either registers, or global
or static (stored globally).

\bd
\item[2184:] The static structure pointer,
 ``p'', defines a starting point for
 searching through the ``proc''
 array to locate the next process
 to activate. Its use reduces the
 bias shown to processes entered
 early in the ``proc'' array. If ``p''
 is null, set its value to the
 beginning of the ``proc'' array.
 This should only occur upon the
 very first call on ``swtch'';

\item[2189:] A call on ``savu'' (0725) saves the
 current values of the environment
 and stack pointers (r5 and r6);

\item[2193:] ``retu'' (0740) resets r5 and r6,
 and, most importantly, resets the
 kernel address register 6 to
 address the ``scheduler's'' data
 segment;

\item[2195:] Phase Two begins:

 The code from this line to line
 2224 is only ever executed by
 kernel process \#0. There are two
 nested loops, from which there is
 no exit until a runnable process
 can be found.

 At slack periods, the processor
spends most of its time executing
line 2220. It is only disturbed
thence by an interrupt (e.g. from
the clock);

\item[2196:] The flag ``runrun'' is reset. (It
 is used to indicate that a higher
 priority process than the current
 process is ready to run. ``swtch''
 is about to look for the highest
 priority process.);

\item[2224:] The priority of the ``arising''
 process is noted in ``curpri'' (a
 global variable) for future
 reference and comparison;

\item[2228:] Another call on ``retu'' resets r5,
 r6 and the seventh kernel address
 register to values appropriate
 for the ``arising'' process;

\item[2229:] Phase Three begins:

``sureg'' (1739) resets the user
mode hardware segmentation registers using the stored prototypes
for the arising process;

\item[2230:] The comment which begins here is
 not encouraging. We will return
 to this point again towards the
 end of this chapter;

\item[2247:] If you check, you will find that
none of the procedures which call
``swtch'' directly examines the
value returned here.

Only the procedures which call
``newproc'' which are interested in
this value, because of the way
the child process is first
activated!
\ed


\sbs{setpri (2156)}

\bd
\item[2161:] Process priorities are calculated
 according to the formula

\begin{tabbing}
priority = min(\= 127,\\
\> (time used + PUSER + p\_nice))\\
\end{tabbing}

where
\bd
 \item[(1)] time used = accumulated central
 processor time (usually since the
 process was last swapped in),
 measured in clock ticks divided
 by 16 i.e. thirds of a second.
 (More on this later when we discuss the clock interrupt.);

 \item[(2)] PUSER == 100;

 \item[(3)] ``p\_nice'' is a parameter used to
 bias the process priority. It is
 normally positive and hence
 reduces the process's effective
 precedence.
\ed
\ed

Note the somewhat confusing convention
in UNIX that the lower the priority,
the higher the precedence. Thus a
priority of --10 beats a priority of 100
every time.

\bd
\item[2165:] Set the rescheduling flag if the
 process, whose priority has just
 been recalculated, has less precedence than the current process.
\ed

The sense of the test on line 2165 is
surprising, especially when it is compared with line 2141. We leave it to
the reader to satisfy himself that this
is not an error. (Hint: look at the
parameters for the calls on ``setpri''.)


\sbs{sleep (2066)}

This procedure is called (from nearly
30 different places in the code) when a
kernel process chooses to suspend
itself. There are two parameters:

\bi
\item the reason for sleeping;

\item a priority with which the process
 will run after being awakened.
\ei

If this priority is negative the process cannot be aroused from its sleep
by the arrival of a ``signal''. ``signals''
are discussed in Chapter Thirteen.

\bd
\item[2070:] The current processor status is
saved to preserve the incoming
processor priority and previous
mode information;

\item[2072:] If the priority is non-negative,
a test is made for ``waiting signals'';

\item[2075:] A small critical section begins
here, wherein the process status
is changed and the parameters are
stored in generally accessible
locations (viz. within the array ``proc'').

This code is critical because the
same information fields may be
interrogated and changed by
``wakeup'' (2113) which is frequently called by interrupt
handlers;

\item[2080:] When ``runin'' is non-zero, the
 scheduler (process \#0) is waiting
 to swap another process into main
 memory;

\item[2084:] The call on ``swtch'' represents a
 delay of unknown extent during
 which a relevant external event
 may have occurred. Hence the
 second test on ``issig'' (2085) is
 not irrelevant;

\item[2087:] For negative priority ``sleeps'',
 where the process typically waits
 for freeing of system table
 space, the occurrence of a ``signal'' is not allowed to deflect
 the course of the activity.
\ed


\sbs{wakeup (2113)}

This procedure complements ``sleep''. It
simply searches the set of all
processes, looking for any processes
which are ``sleeping'' for a specified
reason (given as the parameter ``chan''),
and reactivating these individually by
a call on ``setrun''.

\sbs{setrun (2134)}

\bd
\item[2140:] The process status is set to
 ``SRUN''. The process will now be
 considered by ``swtch'' and ``sched''
 as a candidate for execution
 again;

\item[2141:] If the aroused process is more
important (lower priority!) than
the current process, the
rescheduling flag, ``runrun'' is
set for later reference;

\item[2143:] If ``sched'' is sleeping, waiting
for a process to ``swap in'', and
if the newly aroused process is
on disk, wake up ``sched''.
\ed

Since it turns out that ``sched'' is the
only procedure which calls sleep with
``chan'' equal to ``\&runout'', line 2145
could be replaced by the recursive call

\begin{verbatim}
   setrun(&proc[0]);
\end{verbatim}

\noindent or better still, by just

\begin{verbatim}
   rp = &proc[0];
   goto sr;
\end{verbatim}

\noindent where ``sr'' is a label to be inserted at
the beginning of line 2139.

\sbs{expand (2268)}

The comment at the beginning of this
procedure (2251) says most of what
needs to be said about the procedure,
except for the question of ``swapping
out'' when not enough core is available.

Note that ``expand'' takes no particular
notice of the contents of the user data
area or stack area.

\bd
\item[2277:] If the expansion is actually a
 contraction, then trim off the
 excess from the high address end;

\item[2281:] ``savu'' stores the values of r5
 and r6 in ``u.u\_rsav'';

\item[2283:] If sufficient main memory is not
 available ...

\item[2284:] The environment pointer and stack
 pointer are recorded again in
 ``u.u\_ssav''. But note that since
 no new procedures have been
 entered, and since there has been
 no cumulative stack growth, the
 values recorded are the same as
 at line 2281;

\item[2285:] ``xswap'' (4368) copies the core
 image for the process designated
 by its first parameter to disk.

Since the second parameter is
non-zero the main memory area
occupied by the data segment is
returned to the list of available
space.

However the computation continues
using the same area in main
memory until the next call on
``retu'' (2193) in ``swtch''.
\ed

Note also that the call on ``savu'' at
line 2189 in ``swtch'' stores new values
in ``u.u\_rsav'' after the disk image has
been made (and therefore serves no useful purpose since the core image has
already been officially ``abandoned'');

\bd
\item[2286:] The ``SSWAP'' flag is set in the
 process's proc array element.
 (This is not swapped out, so the
 effect is not lost);

\item[2287:] ``swtch'' is called, and the process, still running in its old
 area suspends itself. Since the
 call on ``xswap'' will have
 resulted in the ``SLOAD'' flag
 being switched off, there is no
 way that ``swtch'' will choose the
 process for immediate reactivation.
\ed

Only after the disk image has
been copied back into core again
can the process be activated
again. The ``return'' executed by
``swtch'' is a return to the procedure which called ``expand''.

\sbs{swtch revisited}

What happens to the process when it is
reactivated i.e. it becomes the ``arising'' process in ``swtch''?

\bd
\item[2228:] The stack and environment
 pointers are restored from
 ``u.u\_rsav'' (Note that a pointer
 to ``u'' is also a pointer to
 ``u.u\_rsav'' (0415) but ...

\item[2240:] If the core image was swapped
 out e.g. by ``expand'' ...

\item[2242:] No reliance is placed on the
 values of the stack and environment pointers, and they are reset
\ed


The question is if the values stored
in ``u.u\_ssav'' at line 2284 are the same
as values stored in ``u.u\_rsav'' at line
2281, how did they get to be different?


Presumably this is what ``you are not
expected to understand'' (line 2238) ...
clearly ``xswap'' should be investigated
... the trail finally ends at Chapter
Fifteen ... in the meantime you may
wish to investigate for yourself so
that you may join the ``2238'' club that
much sooner.

\sbs{Critical Sections}

If two or more processes operate on the
same set of data, then the combined
output of the set of processes may
depend on the relative synchronisation
of the various processes.

This is usually considered to be highly
undesirable and to be avoided at all
costs. The solution is usually to
define ``critical sections'' (it is the
programmer's responsibility to recognise these) in the code which is executed by each process. The programmer
must then ensure that at any time no
more than process is executing a
section of code which is critical with
respect to a part1cular set of data.

In UNIX user processes do not share
data and so do not conflict in this
way. Kernel processes however have
shared access to various system data
and can conflict.

In UNIX an interrupt does not cause a
change in process as a direct side
effect. Only where kernel processes
may suspend themselves in the middle of
a critical section by an explicit call
on ``sleep'', does an explicit lock variable which may be observed by a group
of processes) need to be introduced.
Even then the actions of testing and
setting the locks do not usually have
to be made inseparable.

Some critical sections of code are executed by interrupt
handlers. To protect other sections of code whose outcome may
be affected by the handling of
certain interrupts, the processor
priority is raised temporarily high
enough before the critical section is
entered to delay such interrupts until
it is safe, when the processor priority
is reduced again. There are of course
a number of conventions which interrupt
handling code should observe, as will
be discussed later in Chapter Nine.


In passing it may be noted that the
strategy adopted by UNIX works only for
a single processor system and would be
totally inappropriate in a multiprocessor system.

%
% The Lion's Commentary, file ch9.tex, version 1.5, 18 May 1994
%
\section*{Section Two}

{\sf Section Two is concerned with traps,
hardware interrupts and software interrupts.

Traps and hardware interrupts introduce
sudden switches into the CPU's normal
instruction execution sequence. This
provides a mechanism for handling special conditions which occur outside the
CPU's immediate control.

Use is made of this facility as part of
another mechanism called the ``system
call'', whereby a user program may execute a ``trap'' instruction to cause a
trap deliberately and so obtain the
operating system's attention and assistance.

The software interrupt (or ``signal'') is
a mechanism for communication between
processes, particularly when there is
``bad news''.}

\se{Hardware Interrupts and Traps}

In the PDP11 computer, as in many other
computers, there is an ``interrupt''
mechanism, which allows the controllers
of peripheral devices (which are devices external to the CPU) to interrupt
the CPU at appropriate times, with
requests for operating system service.


The same mechanism has been usefully
and conveniently applied to ``traps''
which are events internal to the CPU,
which relate to hardware and software
errors, and to requests for service
from user programs.

\sbs{Hardware Interrupts}

The effect of an interrupt is to divert
the CPU from whatever it was doing and
to redirect it to execute another program.

During a hardware interrupt:

\bi
\item The CPU saves the current processor
 status word (PS) and the current
 program count (PC) in its internal registers;

\item the PC and PS are then reloaded from
 two consecutive words located in
 the low area of main memory. The
 address of the first of these
 two words is known as the
 ``{\bf vector location}'' of the interrupt;

\item finally the original PC and PS values
 are stored into the newly
 current stack. (Whether this is
 the kernel or user stack depends
 on the new value of the PS.)
\ei

Different peripheral devices may have
different vector locations. The actual
vector location for a particular device
is determined by hard wiring, and can
only be changed with difficulty. Moreover there are well entrenched conventions for choosing vector locations for
the various devices.

Thus after the interrupt has occurred,
because the PC has been reloaded, the
source of instructions executed by the
CPU has been changed. The new source
should be a procedure associated with
the peripheral device controller which
caused the interrupt.


Also since the PS has also been
changed, the processor mode may have
changed. In UNIX, the initial mode may
be either ``user'' or ``kernel'', but after
the interrupt, the mode is always ``kernel''. Recall also that a change in mode
implies:

\bd
\item[(a)] a change in memory mappings.
(Note that to avoid any confusion, vector locations are
always interpreted as kernel
mode addresses.);

\item[(b)] a change in stack pointers.
 (Recall that the stack pointer,
SP or r6, is the only special
register which is replicated for
each mode. This implies that
after a mode change, the stack
pointer value will have changed
even though it has not been
reloaded!)
\ed

\sbs{The Interrupt Vector}

For our sample system, the representative peripheral devices chosen are
listed in Table 9.1, along with their
conventional hardware defined vector
locations and priorities.

\begin{center}
\begin{tabular}{llcc}
{\bf vector} & {\bf peripheral} & {\bf interrupt} & {\bf process} \\
{\bf location} & {\bf device} & {\bf priority} & {\bf priority} \\ \hline
060 & teletype input & 4 & 4 \\
064 & teletype output & 4 & 4 \\
070 & paper tape input & 4 & 4 \\
074 & paper tape output & 4 & 4 \\
100 & line clock & 6 & 6 \\
104 & programmable & 6 & 6 \\
    & clock & &\\
200 & line printer & 4 & 4 \\
220 & RK disk drive & 5 & 5 \\
\end{tabular}
\bigskip

{\bf Table 9.1 Interrupt Vector Locations and Priorities}
\end{center}

\sbs{Interrupt Handlers}

Within this selection of UNIX source
code, there are seven procedures known
as ``interrupt handlers'', i.e. which are
executed as the result of, and only as
the result of, interrupts:

\begin{verbatim}
   clock  (3725)  pcrint (8719)
   rkintr (5451)  pcpint (8739)
   klxint (8070)  lpint  (8976)
   klrint (8078)
\end{verbatim}

\noindent ``clock'' will be examined in detail in
Chapter 11. The others are discussed
with the code for their associated devices.

\sbs{Priorities}

An interrupt does not necessarily occur
immediately the peripheral device controller requests it, but only when the
CPU is ready to accept it. It is usually desirable that a request for a low
priority service should not be allowed
to interrupt an activity with a higher
priority.

Bits 7 to 5 of the PS determine the
processor priority at one of eight levels (labelled zero to seven). Each
interrupt also has an associated priority level determined by hardware
wiring. An interrupt will be inhibited as
long as the processor priority is
greater than or equal to the interrupt
priority.


After the interrupt the processor
priority will be determined from the PS
stored in the vector location and this
does not have to be the same as the
interrupt priority. Whereas the interrupt priority is determined by
hardware, it is possible for the
operating system to change the contents
of the vector location at any time.


As a matter of curiosity, it may be
noted that the PDP11 hardware restricts
the possible interrupt priorities to 4,
5, 6 and 7 i.e. levels 1, 2 and 3 are
not supported by the Unibus.

\sbs{Interrupt Priorities}

In UNIX, interrupt handling routines
are initiated at the same priority as
the interrupt priority.

This means that during the handling of
the interrupt, a second interrupt from
a device of the same priority class
will be delayed until the processor
priority is reduced, either by the execution of one of the ``spl'' procedures,
which are intended for just this purpose (see lines 1293 to 1315), or by
reloading the processor status word
e.g. upon returning from the interrupt.

During interrupt handling, the processor priority may be raised temporarily
to protect the integrity of certain
operations. For instance, character
oriented devices such as the paper tape
reader/punch or the line printer interrupt at level four. Their interrupt
handlers call ``getc'' (0930) or ``putc''
(0967), which raise the processor
priority temporarily to level five,
while the character buffer queues are
manipulated.

The interrupt handler for the console
teletype makes use of a ``timeout''
facility. This involves a queue which
is also manipulated by the clock interrupt handler, which runs at level six.
To prevent possible interference, the
``timeout'' procedure (3835) runs at
level seven (the highest possible
level).


Usually it does not make sense to run
an interrupt handler at a processor
priority lower than the interrupt
priority, for this would then risk a
second interrupt of the same type, even
from the same device, before completion
of the processing of the first interrupt. This likely to be at
best inconvenient and at worst disastrous. However the clock
interrupt handler, which
once per second has a lot of extra work
to do, does exactly this.

\sbs{Rules for Interrupt Handlers}

As discussed above, interrupt handlers
need to be careful about the manipulation of the processor priority to avoid
allowing other interrupts to happen
``too soon''. Likewise care needs to be
taken that the other interrupts are not
delayed excessively, lest the performance of the whole system be degraded.
It is important to note that when an
interrupt occurs, the process which is
currently active will very likely not
be the process which is interested in
the occurrence. Consider the following
scenario:


User process \#m is active and initiates
an i/o operation. It executes a trap
instruction and transfers to kernel
mode. Kernel process \#m initiates the
required operation and then calls
``sleep'' to suspend itself to await completion of the operation ...

Some time later, when some other process, user process \#n say, is active,
the operation is completed and an
interrupt occurs. Process \#n reverts to
kernel mode, and kernel process \#n
deals with the interrupt, even though
it may have no interest in or prior
knowledge of the operation.


Usually kernel process \#n will include
waking process \#m as part of its
activity. This will not always be the
case though, e.g. where an error has
occurred and the operation is retried.

Clearly, the interrupt handler for a
peripheral device should not made
references to the current ``u'' structure
for this is not likely to be the
appropriate ``u'' structure. (The
appropriate ``u'' structure could quite
possibly be inaccessible, if it has
been temporarily swapped out to the
disk.)

Likewise the interrupt handler should
not call ``sleep'' because the process
thus suspended will most likely be some
innocent process.

\sbs{Traps}

``Traps'' are like ``interrupts'' in that
they are events which are handled by
the same hardware mechanism, and hence
by similar software mechanisms.

``Traps'' are unlike ``interrupts'' in that
they occur as the result of events
internal to the CPU, rather than externally. (In other systems the
terminology ``internal interrupt'' and ``external
interrupt'' is used to draw this distinction more forcefully.) Traps may
occur unexpectedly as the result of
hardware or power failures, or predictably and reproducibly, e.g. as the
result of executing an illegal instruction or a ``trap'' instruction.


``Traps'' are always recognised by the
CPU immediately. They cannot be delayed
in the way low priority interrupts may
be. If you like, ``traps'' have an
``interrupt priority'' of eight.

``Trap'' instructions may be deliberately
inserted in user mode programs to catch
the attention of the operating system
with a request to perform a specified
service. This mechanism is used as part
of the facility known as ``system
calls''.

Like interrupts, traps result in the
reloading of the PC and PS from a vector location, and the saving of the old
values of the PC and PS in the current
stack. Table 9.2 lists the vector locations for the various ``trap'' types.

\begin{center}
\begin{tabular}{llc}
{\bf vector} & {\bf trap type} & {\bf process} \\
{\bf location} & & {\bf priority} \\ \hline
004 & bus timeout & 7 \\
010 & illegal instruction & 7 \\
014 & bpt-trace & 7 \\
020 & iot & 7 \\
024 & power failure & 7 \\
030 & emulator trap instruction & 7 \\
034 & trap instruction & 7 \\
114 & 11/10 parity & 7 \\
240 & programmed interrupt & 7 \\
244 & floating point error & 7 \\
250 & segmentation violation & 7 \\
\end{tabular}
\bigskip

{\bf Table 9.2 Trap Vector Locations and Priorities}
\end{center}


The contents of Tables 9.1 and 9.2
should be compared with the file
``low.s'' on Sheet 05. As noted earlier,
this file is generated at each installation (along with the file ``conf.c''
(sheet 46)), as the product of the
utility program ``mkconf'', so as to
reflect the actual set of peripherals
installed.


\sbs{Assembly Language `trap'}

From ``low.s'' it appears that traps and
interrupts are handled separately by
the software. However closer examination reveals that ``call'' and ``trap'' are
different entry points to a single code
sequence in the file ``m40.s'' (see lines
0755, 0776). This sequence is examined
in detail in the next chapter.


During the execution of this sequence,
a call is made on a ``C'' language procedure to carry out further specific
processing. In the case of an interrupt, the ``C'' procedure
is the interrupt handler specific to the particular
device controller.

 In the case of a trap, the ``C'' procedure is another procedure called
 ``trap'' (yes, the word ``trap'' is definitely overworked!), which in the case
process of a system error will most likely call
priority ``panic'' and in the case of a ``system
 call'', will invoke (indirectly via
 ``trapl''(2841)) the appropriate system
 call procedure.

\sbs{Return}

Upon completion of the handling of an
interrupt or trap the code follows a
common path ending in an ``rtt'' instruction (0805). This reloads both the PC
and PS from the current stack, i.e. the
kernel stack, in order to restore the
processor environment that existed
before the interrupt or trap.

%
% The Lion's Commentary, file ch10.tex, version 1.5, 17 May 1994
%
\se{The Assembler ``Trap'' Routine}

The principal purpose of this chapter
is to examine the assembly language
code in ``m40.s'' which is involved in
the handling of interrupts and traps.

This code is found between lines 0750
and 0805, and has two entry points,
``trap'' (0755) and ``call'' (0766). There
are several different and relevant
paths through this code and we shall
trace some examples of these.

\sbs{Sources of Traps and Interrupts}

The discussion in Section One introduced three places where the occurrence
of a trap or interrupt was expected:

\bd
\item[(a)] ``main'' (1564) calls ``fuibyte''
 repeatedly until a negative
 value is returned. This will
 occur after a ``bus timeout
 error'' has been encountered with
 a subsequent trap to vector
 location 4 (line 0512);

\item[(b)] The clock has been set running
 and will generate an interrupt
 every clock tick i.e. 16.7 or 20
 milliseconds;

\item[(c)] Process \#1 is about to execute a
 ``trap'' instruction as part of
 the system call on ``exec''.
\ed


\sbs{fuibyte (0814), fuiword (0844)}

``main'' uses both ``fuibyte'' and ``fuiword''.
Since the former is more complicated in a non-essential way, we leave
it to the reader, and concentrate on the latter.

``fuiword'' is called (1602) when the
system is running in kernel mode with
one argument which is an address in
user address space. The function of the
routine is to fetch the value of the
corresponding word and to return it as
a result (left in r0). However if an
error occurs, the value --1 is to be
returned.

Note that with ``fuiword'', there is an
ambiguity which does not occur with
``fuibyte'', namely a returned value of
--1 may not necessarily be an error
indication but the actual value in the
user space. Convince yourself that for
the way it is used in ``main'', this does
not matter.

Also the code does not distinguish
between a ``bus timeout error'' and a
``segmentation error''.

The routine proceeds as follows:

\bd
\item[0846:] The argument is moved to r1;

\item[0848:] ``gword'' is called;

\item[0852:] The current PS is stored on the
 stack;

\item[0853:] The priority level is raised to 7
 (to disable interrupts);

\item[0854:] The contents of the location
 nofault (1466) are saved in the
 stack;

\item[0855:] ``nofault'' is loaded with
 address of the routine ``err'';

\item[0856:] An ``mfpi'' instruction is used to
 fetch the word from user space.
\ed

{\bf If nothing goes wrong} this value will
 left on the kernel stack.

\bd
\item[0857:] The value is transferred from the
 stack to r0;

\item[0876:] The previous values of ``nofault''
 and PS are restored;
\ed


{\bf Now suppose something does go wrong}
with the ``mfpi'' instruction, and a bus
time-out does occur.

\bd
\item[0856:] The ``mfpi'' instruction will be
 aborted. PC will point to the
 next instruction (0857) and a
 trap via vector location 4 will
 occur;

\item[0512:] The new PC will have the value of
 ``trap''. The new PS will indicate:

present mode = kernel mode

previous mode = kernel mode

priority = 7;

\item[0756:] The next instruction executed is
 the first instruction of ``trap''.
 This saves the processor status
 word two words beyond the current
 ``top of stack''. (This is not
 relevant here.);

\item[0757:] ``nofault'' contains the address of 
 ``err'' and is non-zero;

\item[0765:] Moving 1 to SR0 reinitialises the
 memory management unit;

\item[0766:] The contents of ``nofault'' are
 moved on top of the stack,
 {\bf overwriting} the previous contents, which was the return
 address in ``gword'';

\item[0767:] The ``rtt'' returns, not to ``gword''
 but to the first word of ``err'';

\item[0880:] ``err'' restores ``nofault'' and PS,
 skips the return to ``fuiword'',
 places --1 in r0, and returns
 directly to the calling routine.
\ed

\sbs{Interrupts}

Suppose the clock has interrupted the
processor.


Both clock vector locations, 100 and
104, have the same information. PC is
set to the address of the location
labelled ``kwlp'' (0568) and PS is set to
show:

 present mode = kernel mode

 previous mode = kernel or user mode

 priority = 6


\noindent Note.
The PS will contain the true previous mode, regardless of the value
picked up from the vector location.

\bd
\item[0570:] The vector location contains a
 new PC value which is the address
 of the statement labelled ``kwlp''.
 This instruction is a subroutine
 call on ``call'' via r0.

 After the execution of this
 instruction, r0 is left with the
 address of the code word after
 the instruction which contains
 ``\_clock'', i.e. r0 contains {\bf the
 address of the address} of the
 ``clock'' routine in the file
 ``clock.c'' (3725).
\ed

\sbs{call (0776)}

\bd
\item[0777:] Copy PS onto the stack;

\item[0779:] Copy r1 onto the stack;

\item[0780:] Copy the stack pointer for the
 previous address space onto the
 stack. (This is only significant
 if the previous mode was user
 mode).

This represents a {\bf special case} of
the ``mfpi'' instruction. See the
``PDP11 Processor Handbook'', page
6-20;

\item[781:] Copy the copy of PS onto the
 stack and mask out all but the
 lower five bits. The resulting
 value designates the cause of the
 interrupt (or trap). The original value of the PS had to be
 captured quickly;

\item[0783:] Test if the previous mode is kernel or user.

{\bf If the previous mode is kernel
mode} the branch is taken (0784).
PS is changed to show the previous
mode as user mode (0798);

\item[0799:] The specialised interrupt handling routine pointed to by r0 is
entered. (In this case it is the
routine ``clock'', which is discussed in detail in the next
chapter.)

\item[0800:] When the ``clock'' routine (or some
other interrupt handler) returns,
the top two words of the stack
are deleted. These are the
masked copy of the PS and the
copy of the stack pointer;

\item[0802:] r1 is restored from the stack;

\item[0803:] Delete the copy of PS from the
 stack;

\item[0804:] Restore the value of r0 from the
 stack;

\item[0805:] Finally the ``rtt'' instruction
returns to the ``kernel'' mode
routine that was interrupted;

{\bf If the previous mode was user mode}
it is not certain that the interrupted routine will be resumed
immediately;

\item[0788:] After the specialised interrupt
routine (in this case ``clock'')
returns, a check (``runrun $>$ 0'')
is made to see if any process of
higher priority than the current
process is ready to run. If the
decision is to allow the current
process to continue, then it is
important that it be not interrupted as
it restores its registers prior to the ``return from
interrupt'' instruction. Hence
before the test, the processor
priority is raised to seven (line
0787), thus ensuring that no more
interrupts occur until user mode
is resumed. (Another interrupt
may occur immediately thereafter,
however.)
\ed


If ``runrun $>$ 0'', then another, higher
priority, process is waiting. The processor priority is reset to 0, allowing
any pending interrupt to be taken. A
call is then made to ``swtch'' (2178), to
allow the higher priority process to
proceed. When the process returns from
``swtch'', the program loops back to
repeat the test.


The above discussion obviously extends
to all interrupts. The only part which
relates specifically to the clock
interrupt is the call on the specialised routine ``clock''.


\sbs{User Program Traps}

The ``system call'' mechanism which
enables user mode programs to call on
the operating system for assistance,
involves the execution by the user mode
program of one of 256 versions of the
``trap'' instruction. (The ``version'' is
the value of the low order byte of the
instruction word.)

\bd
\item[0518:] Execution of the trap instruction in a user mode program
 causes a trap to occur to vector
 location 34 which causes the PC
 to be loaded with the value of
 the label ``trap'' (lines 0512,
 0755). A new PS is set which
 indicates

present mode = kernel mode

previous mode = user mode

priority = 7

\item[0756:] The next instruction executed is
 the first instruction of ``trap''.
 This saves the processor status
 word in the stack two words
 beyond the current ``top of
 stack''.

It is important to save the PS as
soon as possible, before it can
be changed, since it contains
information defining the type of
trap that occurred. The somewhat
unconventional destination of the
``move'' is to provide compatibility with the
handling of interrupts, so that the same code can
be used further on;

\item[0757:] ``nofault'' will be zero so the
 branch is not taken;

\item[0759:] The memory management status
registers are stored just in case
they will be needed, and the
memory management unit is reinitialised;

\item[0762:] A subroutine entry is made to
 ``call'' using r0. (This neatly
 stores the old value of r0 in the
 stack, but not a return address.
 The new value is the address of
 the address of the routine to be
 entered next (in this case the
 ``trap'' routine in the file
 ``trap.c'' (2693));

\item[0772:] The stack pointer is adjusted to
 point to the location which
 already contains the copy of PS;

\item[0773:] The CPU priority is set to zero;

{\bf From here the same path as for an
interrupt is followed.}
\ed


\sbs{The Kernel Stack}

The state of the kernel stack at the
time that the ``trap'' procedure (``C''
version) or one of the specialised
interrupt handling routines is entered,
is shown in Figure 10.1.

\begin{center}
\begin{tabular}{lrl|c|l}
 & & & .... & {\bf Previous top of stack} \\ \hline
(rps & 2)  & 7      & ps   & old PS \\
(r7  & 1)  & 6      & pc   & old PC (r7) \\
(r0  & 0)  & 5 --$>$ & r0   & old r0 \\
     &     & 4      & nps  & new PS after trap \\
(r1  & --2) & 3      & r1   & old r1 \\
(r6  & --3) & 2      & sp   & old SP for previous mode \\
     &     & 1      & dev  & masked new PS \\
     &     & 0 --$>$ & tpc  & return address in ``call'' \\ \hline
(r5  & --6  & --1     & (r5) & old r5 \\
(r4  & --7  & --2     & (r4) & old r4 \\
(r3  & --8  & --3     & (r3) & old r3 \\
(r2  & --9  & --4     & (r2) & old r2 \\
\end{tabular}
\bigskip

Figure 10.1
\end{center}

Columns (2) and (3) give the positions
of stack words relative to the positions in the stack of the words
labelled ``r0'' and ``tpc'' respectively.

Columns (1) and (2) define (or explain)
the contents of the file ``reg.h'' (Sheet 26).

``dev'', ``sp'', ``r1'', ``nps'' ``r0'', ``pc'' and
``ps'' in that order are the names of the
parameters used in the declaration of
the procedures ``trap'' (2693) and
``clock'' (3725).

Note that just before entry to ``trap''
(``C'' version) or the other interrupt
handling routines, the values for the
registers r2, r3, r4 and r5 have not
yet been saved in the stack. This is
performed by a call on ``csv'' (lg20)
which is automatically included by the
``C'' compiler at the beginning of every
compiled procedure. The form of the
call on ``csv'' is equivalent to the
assembler instruction

\begin{verbatim}
   jsr r5,csv
\end{verbatim}

This saves the current value of r5 on
the stack and replaces it by the
address of the next instruction in the
``C'' procedure.

\bd
\item[1421:] This value of r5 is copied into r0;

\item[1422:] the current value of the stack
 pointer is copied into r5.
\ed


Note that at this point, r5 points to a
stack location containing the previous
value of r5 i.e. it points to the
beginning of a chain of pointers, one
per procedure, which ``thread'' the
stack. When a ``C'' procedure exits, it
actually returns to ``cret'' (1430) where
the value of r5 is used to restore the
stack and r2, r3 and r4 to their earlier condition (i.e. as they were
immediately prior to entering the procedure). For this reason r5 is often
called the {\bf environment pointer}.

%
% The Lion's Commentary, file ch11.tex, version 1.4, 17 May 1994
%
\se{Clock Interrupts}

The procedure ``clock'' (3725) handles
interrupts from either the line frequency time clock
(type \mbox{KW11-L}, interrupt vector address 100) or the
programmable real-time clock (type KW11-P,
interrupt vector address 104).

UNIX requires that at least one of
these should be available. (If both are
present, only the line time clock is
used.)

Whichever clock is used, interrupts are
generated at line frequency (i.e. with
a 50 Hz power supply, every 20 milliseconds).
The clock interrupt priority level is six, higher than for any
other peripheral device on our typical
system, so that there will usually be
very little delay in the initiation of
``clock'' once the interrupt has been
requested by the clock controller.


\sbs{clock (3725)}

The function of ``clock'' is one of general housekeeping:

\bi
\item the display register is updated
(PDP11/45 and 11/70 only);

\item various accounting values such as
the time of day, accumulated processing times
and execution profiles are maintained;

\item processes sleeping for a fixed
time interval are awakened as per
schedule;

\item core swapping activity is initiated once per second.
\ei

``clock'' breaks most of the rules for
peripheral device handlers: it does
reference the current ``u'' structure,
and it also runs at a low priority for
some of the time. It abbreviates its
activity if a previous execution has
not yet completed.

\bd
\item[3740:] ``display'' is a no-op on the
 PDP11/40;

\item[3743:] The array ``callout'' (0265) is an
 array of ``NCALL'' (0143) structures
of type ``callo'' (0260).
The ``callo'' structure contains
three elements: an incremental
time, an argument and the address
of a function. When the function
element is not null, the function
is to be executed with the supplied
argument after a specified time.


(For the systems under study, the
only function ever executed in
this way is ``ttrstrt'' (8486),
handler. (See Chapter 25.));

\item[3748:] If the first element of the list
 is null, the whole list is null;

\item[3750:] The ``callout'' list is arranged in
 the desired order of execution.
 The time recorded is the number
of clock ticks between events.
Unless the first time (the time
before the next event) is already
zero, (meaning that the execution
is already due) this time should
be decremented by one.


If this time has already been
counted to zero, decrement the
next time unless it is already
zero also, etc. i.e. decrement
the first non-zero time in the
list. All the leading entries
with zero times represent operations which are already due. (The
operations are actually carried
out a little later.);

\item[3759:] Examine the previous processor
status word, and if the priority
was non-zero, bypass the next
section, which executes those
operations which are due;

\item[3766:] Reduce the processor priority to
five (other level six interrupts
may now occur);

\item[3767:] Search the ``callout'' array looking
for operations which are due and execute them;

\item[3773:] Move the entries for operations
which are still not yet due, to
 the beginning of the array;

\item[3787:] The code from here until line
3797 is executed, whatever the
previous processor priority, at
either priority level five or
six;

\item[3788:] If the previous mode was ``user
 mode'', then increment the user
 time counter, and if an execution
 profile is being accumulated,
 call ``incupc'' (a895) to make an
 entry in a histogram for the user
 mode program counter (PC).

``incupc'' is written in assembler,
presumably for efficiency and
convenience. A description of
what it does may be found in the
section ``PROFIL(II)'' of the UPM.
See also the procedure ``profil''
(3667);

\item[3792:] If the previous mode was not user
mode, increment the system (kernel) time counter for the process.
\ed

The code just described performs the
basic time accounting for the system.
Every clock tick results in the incrementing of either ``u.u\_utime'' or
``u.u\_stime'' for some process. Both
``u.u\_utime'' and ``u.u\_stime'' are
initialised to zero in ``fork'' (3322). Their
values are interrogated in ``wait''
(3270). The values will go negative
after 32K ticks (about 10 hours)!

\bd
\item[3795:] ``p\_cpu'' is used in determining
 process priorities. It is a character value which is always
 interpreted as a positive integer
 (0 to 255). When it is moved to a
 special register, sign extension
 occurs so that 255, for instance,
 becomes like --1. Adding one then
 leaves a zero result. In this
 case the value is reduced to --1
 again, and stored as 255
 unsigned. Note that in the other
 places where ``p\_cpu'' is referenced (2161, 3814), the top eight
 bits are masked off after the
 value has been transferred to a
 special register;

\item[3797:] Increment ``lbolt'' and if it
exceeds ``HZ'', i.e. a second or
more has elapsed ...

\item[3798:] Then provided the processor was
 not previously running at a nonzero priority, do a whole lot of
 housekeeping;

\item[3800:] Decrement ``lbolt'' by ``HZ'';

\item[3801:] Increment the time of day accumulator;

\item[3803:] The events which follow may take
 some time, but they may reasonably be interrupted to service
 other peripherals. So the processor priority is dropped below all
 the device priority levels i.e.
 below {\bf four}.

However there is now a possibility of another clock interrupt
before this activation of the
``clock'' procedure is completed.
By setting the processor priority
to {\bf one} rather than to {\bf zero}, a
second activation of ``clock'' will
not attempt to execute the code
from line 3804 on also. Note however that
to the hardware, priority one is functionally the same
as priority zero;

\item[3804:] If the current time (measured in
 seconds) is equal to the value
 stored in ``tout'', wake all
 processes which have elected to
 suspend themselves for a period
 of time via the ``sleep'' system
 call i.e. via the procedure
 ``sslep'' (5979).
\ed

``tout'' stores the time at which the
next process is to be awakened. If
there is more than one such process,
then the remainder, which will have
been disturbed, must reset ``tout''
between them. This mechanism, while
quite effective, will not be efficient
if the number of such processes ever
becomes large.


In this situation, a mechanism similar
to the ``callout'' array (see 3767) would
need to be provided. (In fact, how difficult would it be to merge the two
mechanisms? What would be the disadvantages ??);

\bd
\item[3806:] When the last two bits of
 ``time[1]'' are zero i.e. every
 four seconds, reset the scheduling flag ``runrun'' and wake up
 everything waiting for a ``lightning bolt''. (``lbolt'' represents a
 general event which is caused
 every four seconds, to initiate
 miscellaneous housekeeping. It is
 used by ``pcopen'' (8648).);

\item[3810:] For all currently defined
 processes:

increment ``p\_time'' up to a maximum
of 127 (it is only a character
variable);

decrement ``p\_cpu'' by ``SCHMAG''
(3707) but do not allow it to go
negative. Note that as discussed
earlier (line 3795) ``p\_cpu'' is
treated as a positive integer in
the range 0 to 255;

if the processor priority is currently set
at a depressed value, recalculate it.
\ed


Note that ``p\_cpu'' enters into the calculation of process priorities,
``p\_pri'', by ``setpri'' (2156). ``p\_pri''
is used by ``swtch'' (2209) in choosing
which process, from among those which
are in core (``SLOAD'') and ready to run
(``SRUN''), should next receive the CPU's
attention.

``p\_time'' is used to measure how long
(in seconds) a process has been either
in core or swapped out to disk.
``p\_time'' is set to zero by ``newproc''
(1869), by ``sched'' (2047) and by
``xswap'' (4386). It is used by ``sched''
(1962, 2009) to determine which
processes to swap in or out.

\bd
\item[3820:] If the scheduler is waiting to
 rearrange things, wake it up.
 Thus the normal rate for scheduling decisions is once per second;

\item[3824:] If the previous mode before the
 interrupt was ``user mode'', store
 the address of ``r0'' in a standard
 place, and if a ``signal'' has been
 received for the process, call
 ``psig'' (4043) for the appropriate
 action.
\ed


\sbs{timeout (3845)}

This procedure makes new entries in the
``callout'' array. In this system it is
only called from the routine ``ttstart''
(8505), passing the procedure ``ttrstrt''
(3486). Note that ``ttrstrt'' calls
``ttstart'', which may call ``timeout'',
for a thoroughly incestuous relationship!

Note also that most of ``timeout'' runs
at priority level seven, to avoid clock
interrupts.


%
% The Lion's Commentary, file ch12.tex, version 1.5, 17 May 1994
%
\se{Traps and System Calls}

This chapter is concerned with the way
the system handles traps in general and
system calls in particular.

There are quite a number of conditions
which can cause the processor to
``trap''. Many of these are quite
clearly error conditions, such as
hardware or power failures, and UNIX
does not attempt any sophisticated
recovery procedures for these.

The initial focus for our attention is
the principal procedure in the file
``trap.c''.

\sbs{trap (2693)}

The way that this procedure is invoked
was explored in Chapter Ten. The
assembler ``trap'' routine carries out
certain fundamental housekeeping tasks
to set up the kernel stack, so that
when this procedure is called, everything appears to be kosher.

The ``trap'' procedure can operate as
though it had been called by another
``C'' procedure in the normal way with
seven parameters

\begin{center}
 dev, sp, rl, nps, r0, pc, ps.
\end{center}

(There is a special consideration which
should be mentioned here in passing.
Normally all parameters passed to ``C''
procedures are passed by value. If the
procedure subsequently changes the
values of the parameters, this will not
affect the calling procedure directly.

However if ``trap'' or the interrupt
handlers change the values of their
parameters, the new values will be
picked up and reflected back when the
``previous mode'' registers are
restored.)

The value of ``dev'' was obtained by capturing the value of the processor
status word immediately after the trap
and masking out all but the lower five
bits. Immediately before this, the processor status word had been set using
the prototype contained in the
appropriate vector location.

Thus if the second word of the vector
location was ``br7+n;'' (e.g. line 0516)
then the value of ``dev'' will be n.

\bd
\item[2698:] ``savfp'' saves the floating point
 registers (for the PDP11/40, this
 is a no-op!);

\item[2700:] If the previous mode is ``user
 mode', the value of ``dev'' is
 modified by the addition of the
 octal value 020 (2662);

\item[2701:] The stack address where r0 is
 stored is noted in ``u.u\_ar0'' for
 future reference. (Subsequently
the various register values can
be referenced as ``u.u\_ar0[Rn]''.);

\item[2702:] There is now a multi-way ``switch''
 depending on the value of ``dev''.
\ed

At this point we can observe that UNIX
divides traps into three classes,
depending on the prior processor mode
and the source of the trap:

\bd
\item[(a)] kernel mode;

\item[(b)] user mode, not due to a ``trap''
 instruction;

\item[(c)] user mode, due to a ``trap''
 instruction.
\ed


\sbs{Kernel Mode Traps}

The trap is unexpected and with one
exception, the reaction is to ``panic''.
The code executed is the ``default'' of
the ``switch'' statement:

\bd
\item[2716:] Print:

\bi
\item the current value of the seventh
kernel segment address register
(i.e. the address of the current
per process data area);

\item the address of ``ps'' (which is in
the kernel stack); and

\item the trap type number;
\ei

\item[2719:] ``panic'', with no return.
\ed

Floating point operations are only used
by programs, and not by the operating
system. Since such operations on the
PDP11/45 and 11/70 are handled asynchronously, it is possible that when a
floating point exception occurs, the
processor may have already switched to
kernel mode to handle an interrupt.

Thus a kernel mode floating point
exception trap can be expected occasionally and is the concern of the
current user program.

\bd
\item[2793:] Call ``psignal'' (3963) to set a
 flag to show that a floating
 point exception has occurred;

\item[2794:] Return.
\ed

This raises an interesting question: ``Why are the kernel mode
and user mode floating point
exceptions handled slightly di�ferently?''


\sbs{User Mode Traps}

Consider first of all a trap which is
not generated as the result of the execution of a ``trap instruction''. This
is regarded as a probable error for
which the operating system makes no
provision apart from the possibility of
a ``core dump''. However the use program
itself may have anticipated it and provided for it.

The way this provision is made and
implemented is the subject of the next
chapter. At this stage, the principal
requirement is to ``signal'' that the
trap has occurred.

\bd
\item[2721:] A bus error has occurred while
 the system is in user mode. Set
 ``i'' to the value ``SIGBUS'' (0123);

\item[2723:] The ``break'' causes a branch out
 of the ``switch'' statement to line
 2818;

\item[2733:] Apart from the one special case
noted, the treatment of illegal
instructions is the same at this
level as for bus errors;

\item[2739:]
\item[2743:]
\item[2747:]
\item[2796:] Cf. the comment for line 2721.
\ed


Note that cases ``4+USER'' (power fail)
and\\
``7+USER'' (programmed interrupt) are
handled by the ``default'' case (line
2715).

\bd
\item[2810:] This represents a case where
 operating system assistance is
 required to extend the user mode
 stack area.

The assembler routine ``backup''
(1012) is used to reconstruct the
situation that existed before
execution of the instruction that
caused the trap.

``grow'' (4136) is used to do the
actual extension.
\ed

The procedure ``backup'' is non-trivial
and its comprehension involves a careful consideration of various aspects of
the PDP11 architecture. It has been
left for the interested reader to pursue privately.

As noted for the PDP11/40, ``backup'' may
not always succeed because the processor does not save enough information to
resolve all possibilities.

\bd
\item[218:] Call ``psignal'' (3963) to set the
 appropriate ``signal''. (Note that
 this statement is only reached
 from those cases of the ``switch''
 which included a ``break'' statement.);

\item[2821:] ``issig'' checks if a ``signal'' has
 been sent to the user program,
 either just now or at some earlier time and has not yet been
 attended to;

\item[2822:] ``psig'' performs the appropriate
 actions. (Both ``issig'' and ``psig''
 are discussed in detail in the
 next chapter.);

\item[2823:] Recalculate the priority for the
 current process.
\ed

\sbs{System Calls}


User mode programs use ``trap'' instructions as part of the ``system call''
mechanism to call upon the operating
system for assistance.

Since there are many possible ``versions'' of the ``trap'' instruction, the
type of assistance requested can be and
is encoded as part of the ``trap''
instruction.


Parameters which are part of a system
call may be passed from the user program in different ways:

\bd
\item[(a)] via the special register r0;

\item[(b)] as a set of words embedded in
 the program string following the
 ``trap'' instruction;

\item[(c)] as a set of words in the
 program's data area. (This is
 the ``indirect'' call.)
\ed


Indirect calls have a higher overhead
than direct system calls. Indirect
calls are needed when the parameters
are data dependent and cannot be determined at compile time.

Indirect calls may sometimes be avoided
if there is only one data dependent
parameter, which is passed via r0. In
choosing which parameters should be
passed via r0, the system designers
have presumably been guided by their
own experience, since the pattern
doesn't satisfy the law of least astonishment.

The ``C'' compiler does not give special
recognition to system calls, but treats
them in the same way as other procedures. When the loader comes to
resolve undetermined references, it
satisfies these with library routines
which contain the actual ``trap''
instructions.

\bd
\item[2752:] The error indicators are reset;

\item[2754:] The user mode instruction which
 caused the trap is retrieved and
 all but the least significant six
 bits are masked off. The result
 is used to select an entry from
 the array of structures,
``sysent''. The address of the
selected entry is stored in
``callp'';


\item[2755:] The ``zeroeth'' system call is the
``indirect'' system call, in which
the parameter passed is actually
the address in the user program
data space of a system call
parameter sequence.
\ed

Note the separate uses of ``fuword'' and
``fuiword''. The distinction between
these is unimportant on the PDP11/40,
but is most important on machines with
separate ``i'' and ``d'' address spaces;

\bd
\item[2760:] ``i=077'' simulates a call on the
very last system call (2975),
which results in a call on
``nosys'' (2855), which results in
an error condition which will
usually be fatal for the user
mode program;

\item[2762:]
\item[2765:] The number of arguments specified
in ``sysent'' is the actual number
provided by the user programmer,
or that number less one if one
argument is transferred via r0.
The arguments are copied from the
user data or instruction area
into the five element array
``u.u\_arg''. (From ``sysent'' (Sheet
29) it would seem that four elements would have been sufficient
for ``u\_area[ ]'' -- is this an
allowance for future inflation?);

\item[2770:] The value of the first argument
is copied into ``u.u\_dirp'', which
seems to function mainly as a
convenient temporary storage
location;

\item[2771:] ``trapl'' is called with the
 address of the desired system
 routine. Note the comment beginning on line 2828;

\item[2776:] When an error occurs, the ``c-bit''
in the old processor status word
is set (see line 2658) and the
error number is returned via r0.
\ed

\sbs{System Call Handlers}

The full set of system calls may be
reviewed in the file ``sysent.c'' on
Sheet 29, but more relevantly, these
are discussed in full detail in Section
II of the UPM.

The procedures which handle the system
calls are found mostly in the files
``sysl.c'', sys2.c``, sys3.c'' and
``sys4.c''.

Two important ``trivial'' procedures are
``nullsys'' (2855) and ``nosys'' (2864)
which are found in the file ``trap.c''.


\sbs{The File `sysl.c'}

This file contains the procedures for
five system calls, of which three will
be considered now, and two (``rexit'' and
``wait'') will be deferred to the next
chapter.

The first procedure in this file, and 
also the first system call we have
encountered, is ``exec''.

\sbs{exec (3020)}

This system call, \#11, changes a process executing
one program into a process executing a different program.
See Section ``EXEC(II)'' of the UPM.
This is the longest and one of the most
important system calls.

\bd
\item[3034:] ``namei'' (6618) (which is discussed in detail in Chapter 19)
converts the first argument
(which is a pointer to a character
 string defining the name of 
 the new program) into an ``inode''
 reference. (``inodes'' are essential parts of the file
 referencing mechanism.);

\item[3037:] Wait if the number of ``exec''s
 currently under way is too large
 (See the comment on line 3011.);

\item[3040:] ``getblk(NODEV)'' results in the
 allocation of a 512 byte buffer
 from the pool of buffers. This
 buffer is used temporarily to
 store in core, that information
 which is currently in the user
 data area, and which is needed to
 start the new program. Note that
 the second argument in ``u.u\_arg''
 is a pointer to this information;

\item[3041:] ``access'' returns a non-zero
result if the file is not executable. The second
condition examines whether the file is a
directory or a special character file.
(It would seem that by making
this test earlier, e.g. just
after line 3036, the efficiency
of the code could be improved.);

\item[3052:] Copy the set of arguments from
 the user space into the temporary
 buffer;

\item[3064:] If the argument string is too
large to fit in the buffer, take
an error exit;

\item[3071:] If the number of characters in
the argument string is odd, add
an extra, null character;

\item[3090:] The first four words (8 bytes) of
 the named file are read into
 ``u.u\_arg''. The interpretation of
 these words is indicated in the
 comment beginning on line 3076
 and, more fully, in the section
 ``A.OUT(V)'' of the UPM.

Note the setting of ``u.u\_base'',
``u.u\_count'', ``u.u\_offset'' and
``u.u\_segflg'' preparatory to the
read operation;

\item[3095:] If the text segment is not to be
protected, add the text area size
to the data area size, and set the former to
zero;

\item[3105:] Check whether the program has a
``pure'' text area, but the program
file has already been opened by
some other program as a data
file. If so, take the error exit;

\item[3127:] When this point is reached, the
 decision to execute the new program is irrevocable i.e. there is
 no longer the opportunity to
 return to the original program
 with an error flag set;

\item[3129:] ``expand'' here actually implies a
major contraction, to the ``per
process data'' area only;

\item[3130:] ``xalloc'' takes care of allocating
(if necessary) and linking to the
text area;

\item[3158:] The information stored in the
 buffer area is copied into the
 stack in the user data area of
 the new program;

\item[3186:] The locations in the kernel stack
which contain copies of the ``previous'' values of the registers in
user mode are set to zero, except
for r6, the stack pointer, which
was set at line 3155;

\item[3194:] Decrement the reference count for
 the ``inode'' structure;

\item[3195:] Release the temporary buffer;

\item[3196:] Wake up any other process waiting
 at line 3037.
\ed


\sbs{fork (3322)}

A call on ``exec'' is frequently preceded
by a call on ``fork''. Most of the work
for ``fork'' is done by ``newproc'' (1826),
but before the latter is called, ``fork''
makes an independent search for a slot
in the ``proc'' array, and remembers the
place as ``p2'' (3327).

``newproc'' also searches ``proc'' but
independently. Presumably it always
locates the same empty slot as ``fork'',
since it does not report the value
back. (Why is there no confusion on
this point?)

\bd
\item[3335:] For the new process, ``fork''
returns the value of the parent's
process identification, and initialises various accounting
parameters;

\item[3344:] For the parent process, ``fork''
returns the value of the child's
process identification, and {\bf skips}
the user mode program counter by
one word.
\ed

\noindent Note that the values finally returned
to a ``C'' program are slightly different
from the above. Refer to the section
FORK(II) of the UPM.

\sbs{sbreak (3354)}

This procedure implements system call
\#17 which is described in the Section
``BREAK (II)'' of the UPM. The comment at
the head of the procedure has confused
more than one reader: clearly the identifier
``break'' is used in ``C'' programs
(leave an enclosing program loop) in an
entirely different way from that
intended here (change the size of the
program data area).

``sbreak'' has clear similarities with
the procedure ``grow'' (4136) but unlike
the latter, it is only invoked explicitly and
may in fact cause a contraction of the data area as well as an
expansion (depending on the new desired
size).

\bd
\item[3364:] Calculate the new size for the
 data area (in 32 word blocks);

\item[3371:] Check that the new size is consistent with the memory
segmentation constraints;

\item[3376:] The area is shrinking. Copy the
 stack area down into the former
 data area. Call ``expand'' to trim
 off the excess;

\item[3386:] Call ``expand'' to increase the
 total area. Copy the stack area
 up into the new part, and clear
the areas which were formerly
occupied by the stack.
\ed

The following procedures which are also
contained in ``sysl.c'' are described in
Chapter 13:

\begin{verbatim}
  rexit (3205)     wait (3270)
  exit  (3219)
\end{verbatim}

\sbs{The Files `sys2.c' and `sys3.c'}

``sys2.c'' and ``sys3.c'' are mainly concerned with the file system and
input/output, and they have been
relegated to Section Four of the
operating system source code.


\sbs{The File `sys4.c'}

All the procedures in this file implement system calls.
The following procedures are described in Chapter 13:

\begin{verbatim}
    ssig (3614)     kill (3630)
\end{verbatim}

The following procedures are straightforward and have been left for the
amusement and edification of the reader:

\begin{verbatim}
    getswit (3413) sync (3486)
    gtime (3420)   getgid (3472)
    stime (3428)   getpid (3480)
    setuid (3439)  nice (3493)
    getuid (3452)  times (3656)
    setgid (3460)  profil (3667)
\end{verbatim}

The following procedures which are concerned with file systems, are described
later:

\begin{verbatim}
    unlink (3510)   chown (3575)
    chdir (3538)    smdate (3595)
    chmod (3560)
\end{verbatim}

%
% The Lion's Commentary, file ch13.tex, version 1.4, 17 May 1994
%
\se{Software Interrupts}

The principal concern of this chapter
is the content of the file ``sig.c'',
which appears on Sheets 39 to 42. This
file introduces a facility for communication between processes. In particular
it provides for the course of one ``user
mode'' process to be interrupted,
diverted or terminated by the action of
another process or as the result of an
error or operator action.

In this discussion the term ``software
interrupt'' has been deliberately used
in place of the term ``signal''. This
latter has been eschewed because it has
obtained connotations in the UNIX
milieu which are rather different from
the usage of ordinary English.

UNIX recognises 20 (``NSIG'', line 0113)
different types of software interrupts,
of which (as the reader may discover
for himself by perusal of the the Section ``SIGNAL (II)'' of the UPM)
thirteen
have standard names and associations.
Interrupt type \#0 is interpreted as ``no
interrupt''.

Within the ``per process data area'' of
each process is an array, ``u.u\_signal'',
of ``NSIG'' words. Each word corresponds
to a different software interrupt type
and defines the action which should be
taken if the process encounters that
kind of software interrupt:

\begin{center}
\begin{tabular}{ll}
u\_signal[n] & when interrupt \#n occurs\\ \hline
zero & the process will terminate itself; \\
\\
odd non-zero & the software interrupt is ignored;\\
\\
even non-zero & the value is taken as the address\\
              & in user space of a procedure which\\
	      & which should be executed\\
	      & forthwith.\\
\end{tabular}
\end{center}

Interrupt type \#9 (``SIGKILL'') is especially distinguished
because UNIX ensures that ``u.u\_signal[9]'' remains
zero until the very end of a process's
existence, so that if a process is ever
interrupted for that reason, it will
always terminate itself.


\sbs{Anticipation}

Each process can set the contents of
the array\\
``u.u\_signal[ ]'' (with the
exception of ``u.u\_signal[9]'' as just
noted) in anticipation of future interrupts so that the appropriate action is
taken. The user programmer does this
via the ``signal'' system call (see ``SIGNAL (II)'' of the UPM).

Thus if for example the programmer
wishes to ignore software interrupts of
type \#2 (which result if the user hits
the ``delete'' key on his terminal), he
should set ``u.u\_signal[2]'' to one by
executing the system call

\begin{center}
 ``signal (2,1);''
\end{center}

\noindent from his ``C'' program.


\sbs{Causation}

An interrupt is ``caused'' for a process
quite simply by setting the value of
``p\_sig'' (0363) in the process's ``proc''
entry, to the type number appropriate
to the interrupt (i.e. a value in the
range 1 to ``NSIG''--1).

``p\_sig'' is always directly accessible,
even when the affected process and its
``per process data area'' have been
swapped out to disk. Obviously this
mechanism only allows one interrupt per
process to be outstanding at any one
time. The outstanding interrupt will
always be the most recent one, unless
one of the interrupts was of type \#9,
which always prevails.


\sbs{Effect}

The effect of a software interrupt
never takes place immediately. It may
occur after only some slight delay if
the affected process is currently running, or possibly after a considerable
delay if the affected process is
suspended and has been swapped out.

The action dictated by the interrupt is
always inflicted on the affected process by {\bf itself}, and hence can only
occur when the affected process is
active.

Where the effect is to execute a user
defined procedure, the kernel mode process adjusts the user mode stack to
make it appear that the procedure had
been entered and immediately interrupted
(hardware style) before executing the first instruction. The system
then returns from kernel mode to user
mode in the usual manner. The result
of all this is that the next user mode
instruction which is executed is the
first instruction of the designated
procedure.

\sbs{Tracing}

The software interrupt facility has
been extended to provide a powerful but
somewhat inefficient mechanism whereby
a parent process may monitor the progress of one or more child processes.

\sbs{Procedures}

Since the interrelationships of the
procedures associated with software
interrupts are somewhat confusing at
first sight, it is worthwhile introducing the procedures briefly before
plunging in with both feet ....


\sbs{A. Anticipation}

``ssig'' (3614) implements system call
\#48 (``signal'') to set the value in one
element of the array ``u.u\_signal''.


\sbs{B. Causation}

``kill'' (3630) implements system call
\#37 (kill) to cause a specified
interrupt to a process defined by its
process identifying number.

``signal'' (3949) causes a specified
interrupt to be caused for all
processes controlled and/or initiated
from a specified terminal.


``psignal'' (3963) is called by ``kill''
(3649) and ``signal'' (3955) (also ``trap''
(2793, 2818) and ``pipe'' (7828)) to do
the actual setting of ``p\_sig''.


\sbs{C. Effect}

``issig'' (3991) is called by ``sleep''
( 2085), ``trap'' (2821) and ``clock''
(3826) to enquire whether there is an
outstanding non-ignorable software
interrupt for the active process just
waiting to happen.

``psig'' (4043) is called whenever
``issig'' returns a non-zero result
(except in ``sleep'' where things are a
little more complex) to implement the
action triggered by the interrupt.

``core'' (4094) is called by ``psig'' if a
core dump is indicated for a terminating process.

``grow'' (4136) is called by ``psig'' to
enlarge the user mode stack area if
necessary.

``exit'' (3219) terminates the currently
active process.

\sbs{D. Tracing}

``ptrace'' (4164) implements the ``ptrace''
system call \#26.

``stop'' (4016) is called by ``issig''
(3999) for a process which is being
traced to allow the supervising parent
to have a ``look-see''.

``procxmt'' (4204) is a procedure called
from ``stop'' (4028) whereby the child
carries out certain operations related
to tracing, at the behest of the
parent.

\sbs{ssig (3614)}

This procedure implements the ``signal''
system call.

\bd
\item[3619:] If the interrupt reason is out of
 range or is equal to ``SIGKILL''
 (9), take an error exit;

\item[3623:] Capture the initial value in
``u.u\_signal[a]'' for return as the
result of the system call;

\item[3624:] Set the element of ``u.u\_signal''
to the desired value ...

\item[3625:] If an interrupt for the current
reason is pending, cancel it. (It
is not clear why this step should
be necessary or even desirable.
Any suggestions??)
\ed


\sbs{kill (3630)}

This procedure implements the ``kill''
system call to cause a specified type
of software interrupt to another designated process.

\bd
\item[3637:] If ``a'' is non-zero, it is the
 process identifying number of a
 process to be interrupted. If
 ``a'' is zero, then all processes
 originating from the same terminal as the current process are to
 be interrupted;

\item[3639:] Consider each entry in the ``proc''
table in turn and reject it if:
it is the current process (3640);
it is not the designated process
(3642);
no particular process was designated (``a'' == 0) but it does not
have the same controlling terminal or it is
one of the two initial processes (3644);
the user is not the ``super user''
and the user identities do not
match (3646);

\item[3649:] For any process that survives the
above tests, call ``psignal'' to
change ``p\_sig''.
\ed


\sbs{signal (3949)}

For every process, if it is controlled
by the specified terminal (denoted by
``tp''), hit it with ``psignal''.


\sbs{psignal (3963)}

\bd
\item[3966:] Reject the call if ``sig'' is too
 large (but why not if negative?? ``kill'' does not
check this parameter before passing it to ``psignal''.
Admittedly the ``kill'' command could only result in a
positive value for ``sig'' ...);

\item[3971:] If the current value of ``p\_sig''
 is NOT set to ``SIGKILL'', then
 overwrite it (i.e. once a process
 has been ``killed outright'' there
 is no way to revive it.);

\item[3973:] Seems to be an error here ... for
 ``p\_stat'' read ``p\_pri'' ... improve
 the priority of the process if it
 is not too good;

\item[3975:] If the process is waiting for a
 non-kernel event i.e. it called
 ``sleep'' (2066) with a positive
 priority, then set it running
 again.
\ed

\sbs{issig (3991)}

\bd
\item[3997:] If ``p\_sig'' is non-zero, then ...

\item[3998:] If the ``tracing'' flag is on, call
 ``stop'' (this topic will be resumed later);

\item[4000:] Return a zero value if ``p\_sig'' is
 zero. (This apparently redundant
 test is necessary because ``stop''
 may reset ``p\_sig'' as a side
 effect.);

4003. If the value in the corresponding
 element of ``u.u\_signal'' is even
 (may be zero) return a non-zero
 value;

\item[4006:] Otherwise return a zero value.
\ed


The comment regarding the frequency of
calls on ``issig'' which occurs on lines
3983 to 3985 needs some clarification.
At least one call on ``issig'' is a part
of every execution of ``trap'' but only
of one interrupt routine (``clock'',
which calls ``issig'' only once per
second). In cases where ``pri'' is positive, ``sleep'' (2073, 2085) calls
``issig'' before and after calling
``swtch''.


\sbs{psig (4043)}

This procedure is only called if
``u.u\_signal[n]'' was found by ``issig'' to
have an even value. If this value is
found (4051) to be non-zero, it is
taken as the address of a user mode
function which has to be executed.

\bd
\item[4054:] Reset ``u.u\_signal[n]'' except
the case where the interrupt
for an illegal instruction or
trace trap;

\item[4055:] Calculate the user space
 addresses of the lower of two
 words which are to be inserted
 into the user mode stack ...

\item[4056:] Call ``grow'' to check the current
 user mode stack size, and to
 extend it (downwards!) if necessary;

\item[4057:] Put the values of the processor
 status register and the program
 counter which were captured at
 the time of the ``trap'' or
 hardware interrupt (in the case
 of a ``clock'' interrupt) into the
 user stack, and update the
 ``remembered'' values of r6, r7 and
 the processor status word. Upon
 returning to user mode, execution
 will resume at the beginning of
 the designated procedure. When
 this procedure returns, the procedure which was originally
 interrupted will be resumed;

\item[4066:] If ``u.u\_signal[n]'' is zero, then
 for the interrupt types listed,
 generate a core image via the
 procedure ``core'';

\item[4079:] Store a value in ``u.u\_arg[0]''
 composed of the low order byte of
 the remembered value of r0, and
 of ``n'', which records the interrupt type and whether a core
 image was successfully created;

\item[4080:] Call ``exit'' for the process
to terminate itself.
\ed


\sbs{core (4094)}

This procedure copies the swappable
program image into a file called ``core''
in the user's current directory. A
detailed explanation of this procedure
must wait until the material of Sections Three and Four, which deal with
input/output and file systems, have
been covered.


\sbs{grow (4136)}

The parameter, ``sp'', of this procedure
defines the address of a word which
should be included in the user mode
stack.

\bd
\item[4141:] If the stack already extends far
 enough, simply return with a zero
 value.

Note that this test relies on the
idiosyncrasies of 2's complement
arithmetic, and if both

\begin{center}
\verb+|+sp\verb+|+ $> 2^{15}$
\end{center}

\noindent and

\begin{center}
\verb+|+u.u\_size $* 64$\verb+|+ $> 2^{15}$
\end{center}

\noindent the decision to extend the stack
may be taken wrongly at this
juncture;

\item[4143:] Calculate the stack size increment needed to include the new
 stack point plus a 20*32 word
 margin;


\item[4144:] Check that this value is in fact
 positive (i.e. we are not dealing
 with a failure of the test on
 line 4141.);

\item[4146:] Check that the new stack size
 does not conflict with the memory
 segmentation constraints (``estabur'' sets ``u.u\_error'' if they do)
 and reset the segmentation register prototypes;

\item[4148:] Get a new, enlarged data area,
 copy the stack segments (32 words
 at a time) into the high end of
 the new data area, and clear the
 segments which now become the
 stack expansion;

\item[4156:] Update the stack size,
 ``u.u\_ssize'' and return a ``successful'' result.
\ed


\sbs{exit (3219)}

This procedure is called when a process
is to terminate itself.

\bd
\item[3224:] Reset the ``tracing'' flag;

\item[3225:] Set all of the values in the
 array ``u.u\_signal'' (including
 ``u.u\_signal[SIGKILL]'') to one so
 that no future execution of
 ``issig'' will ever be followed by
 execution of ``psig'';

\item[3227:] Call ``close'' (6643) to close all
 the files which the process has
 open. (For the most part, ``closing'' simply involves decrementing
 a reference count.);

\item[3232:] Reduce the reference count for
 the current directory;

\item[3233:] Sever the process's connection
 with any text segment;

\item[3234:] A place is needed to store ``per
 process'' information until the
 parent process can look at it. A
 block (256 words) in the swap
 area of the disk is a convenient
 place;

\item[3237:] Find a suitable buffer (256
 words) and ...

\item[3238:] Copy the {\bf lower half} of the ``u''
 structure into the buffer area;

\item[3239:] Write the buffer into the swap
 area;

\item[3241:] Enter the core space occupied by
 the process into the free list.
 (This space is of course still in
 use, but the use will terminate
 before any other process gets to
 dip into the free list again.
 This could not be done any
 sooner, because, as will be seen
 later, both ``getblk'' and ``bwrite''
 can call ``sleep'', during which
 all sorts of things might happen.
 In view of all this, it might be
 reasonable if the statement

\begin{center}
 expand (USIZE);
\end{center}

\noindent were inserted after line 3226.);

\item[3243:] Set the process state to ``zombie''
 (i.e. ``a corpse said to be
 revived by witchcraft''\\
(O.E.D.));

\item[3245:] The remaining code searches the
 ``proc'' array to find the parent
 process and to wake it up, to
 make any children ``wards of the
 state'', and, if they have
 ``stopped'' for tracing, to release
 them. Finally the code includes
 (for this process) a last call on
 ``swtch''.
\ed


Before going on to consider tracing,
there are two routines which are
closely associated with ``exit'', which
can be conveniently disposed of now.


\sbs{rexit (3205)}

This procedure implements the ``exit''
system call, \#1. It simply salvages the
low order byte of the user supplied
parameter and saves it in ``u.u\_arg[0]''.
which is in the lower half of the ``u''
structure i.e. the part that is written
to the ``swap area'' as a ``zombie''.


\sbs{wait (3270)}

For every call on ``exit'', there should
be a matching call on ``wait'' by an
anxious parent or ancestor. The principal
function of the latter procedure, which
implements the ``wait'' system call, is
for the parent or ancestor to find and
dispose of a ``zombie'' child.

``wait'' also has a secondary function,
to look for children which have
``stopped'' for tracing (which is the
next major topic).

\bd
\item[3277:] Search the whole ``proc'' array
 looking for child processes. (If
none exist, take an error exit (line 3317));
\item[3280:] If the child is a ``zombie'':

\bi
\item save the child's process identifying number, to report back to
the parent;

\item read the 256 word record back from the disk swap area, and release
the swap space;

\item reinitialise the ``proc'' array entry;

\item accumulate the various accounting entries;

save the ``u\_arg[0]'' value also to
report back to the parent;
\ei

\item[3300:] Is the child in a ``stopped''
 state? (If so, wait for the discussion on tracing);

\item[3313:] If one or more children were
 found but none were ``zombies'' or
 ``stopped'', ``sleep'' and then look
 again.
\ed

\sbs{Tracing}

The tracing facilities are provided
through a modification and extension of
the software interrupt facilities.
Briefly, if a parent process is tracing
the progress of child process, every
time the child process encounters a
software interrupt, the parent process
is given the opportunity to intervene
as part of the total response to the
interrupt.

The parent's intervention may involve
interrogation of values within the
child process's data areas, including
the ``per process data area''. Subject to
certain constraints, the parent process
may also change values within these
data areas.


The source of the software interrupts
may be the parent process, the user
himself (e.g. by entering ``kill''
commands or ``delete''s through his terminal)
or the child process itself (e.g.
instructions or other maladies).

The communication between child and
parent processes is a kind of ritual
dance:

\bd
\item[(1)] the child experiences a software
 interrupt and ``stops'';

\item[(2)] the waiting parent discovers
 the ``stopped'' child (line 3301), and
revives. Subsequently ...

\item[(3)] the parent may execute the
 ``ptrace'' system call which has
 the effect of leaving a request
 message in the system defined
 structure ``ipc'' (3939) for the
 child process;

\item[(4)] the parent then goes to ``sleep''
 while the child ``wakes up'';

\item[(5)] the child reads the message in
``ipc'' and acts upon it (e.g copying
one of its own values into
``ipc.ip\_data'');

\item[(6)] the child then goes to ``sleep''
 while the parent ``wakes up'';

\item[(7)] the parent inspects the result,
as recorded in ``ipc'', of the
operation;

\item[(8)] steps (3) to (7) may be repeated
 several times in succession.
\ed

Finally the parent may allow the child
to continue its normal execution, possibly without ever knowing that a
software interrupt had occurred.

A discussion of the tracing facility is
contained in the Section ``PTRACE (II)''
of the UPM. To the list of functional
limitations noted in the ``Bugs''
paragraph, we can add the following comments on efficiency:

\bi
\item There should be a mechanism for
 transferring large blocks (e.g.
 up to 256 words at a time) of
 information from the child to
 the parent (though not necessarily in the reverse direction);

\item There should be a proper coroutine
 procedure (analogous to ``swtch'')
 to allow rapid transfer of control between child and parent.
\ei

\sbs{stop (4016)}

This procedure is called by ``issig''
(3999) if the tracing flag (``STRC'',
0395) is set.

\bd
\item[4022:] If your parent is process 1
 (i.e. ``/etc/init''), then call
 ``exit'' (line 4032);

\item[4023:] Otherwise look through ``proc'' for
 your parent ... wake him up ...
 declare yourself ``stopped'' and
 ... call ``swtch'' (Note do NOT
 call ``sleep''. Why?);

\item[4028:] If the tracing flag has been
 reset, or the result of the procedure ``procxmt'' is true, return
 to ``issig'';

\item[4029:] Otherwise start again.
\ed

\sbs{wait (3270) (continued)}

\bd
\item[3301:] If the child process has
 ``stopped'' and ...

\item[3302:] If the ``SWTED'' flag is not set
 (i.e. the parent hasn't noticed
 this child lately) ...

\item[3303:] As an ``aide-memoire'' set the
 ``SWTED'' flag. Set ``u.u\_ar0[R0]'',
 ``u.u\_ar0[R1]'' so that the child
 process status word is returned
 to the parent;

\item[3309:] The ``SWTED'' flag was set. This
 means that the parent, by performing at least two ``waits'' in
 succession without any intervening call on ``ptrace'', is not very
 interested in the child. So
 reset both the ``STRC'' and the
 ``SWTED'' flags and release the
 child. (Note the use of ``setrun''
 (not ``wakeup'') to complement the
 call on ``swtch'' (4027)).
\ed

\sbs{ptrace (4164)}

This procedure implements the ``ptrace''
system call, \#26.

\bd
\item[4168:] ``u.u\_arg[2]'' corresponds to the
 first parameter in the ``C'' program calling sequence. If this is
 zero, a child process is asking
 to be traced by its parent, so
 set the ``STRC'' flag and return.
\ed

Note that this code handles the only
explicit action the child process is
asked to take with respect to tracing.
There is no real reason why even this
action should be taken by the child
process and not by the parent process.
From a security point of view it is
most probably desirable that a child
process should only be traceable if it
gives its permission. On the other
hand, if the child asks to be traced
and is then ignored by the parent, the
child process may be blocked indefinitely. Perhaps the best solution would
be for the ``STRC'' flag to be set only
after explicit action by {\bf both} the
parent {\bf and} the child.

\bd
\item[4172:] Search the ``proc'' table looking
for a process which: is stopped;
matches the given process identifying number;
is a child of the current process;

\item[4181:] Wait for the ``ipc'' structure to
 become available if it is currently in use;

\item[4183:] Copy the parameters into ``ipc'' ...

\item[4187:] reset the ``SWTED'' flag, and ...

\item[4188:] return the child to a ``ready to
run'' state;

\item[4189:] Sleep until ``ipc.ip\_req'' is nonpositive (4212);

\item[4191:] Extract a value that is to be
 returned to the parent process,
 check for errors, unlock ``ipc''
 and ``wake up'' any processes waiting for  ``ipc''.
\ed

Note that the ``sleeps'' on lines 4182,
4190 are for essentially different reasons, and could be differentiated to
good effect by replacing ``\&ipc'' by
``\&ipc.ip\_req'' on lines 4190 and 4213.


\sbs{procxmt (4204)}

This procedure is executed by the child
process under the influence of data
left by the parent in the ipc structure.

\bd
\item[4209:] If ``ipc.ip\_lock'' is set wrongly
 for the current process, then
 certainly the rest of ``ipc''
 should be ignored.
\ed

After ``stop'' (4027) calls ``swtch'', the
chide process is restarted by one of
three calls on ``setrun'' which leave the
``STRC'' and ``SWTED'' flags in the state
indicated:

\begin{center}
\begin{tabular}{lllll}
 & & STRC & SWTED & ipc.ipc\_lock\\ \hline
exit & (3254) & set & set & arbitrary\\
wait & (3310) & reset & reset & arbitrary\\
ptrace & (4188) & set & reset & properly set\\
\end{tabular}
\end{center}


In the third case ``ptrace'' will always
set ``ipc.ip\_lock'' properly, before the
child is restarted, so that there is
then no chance of the test on 4209
into ``ipc'' failing.

In the second case, where the parent
has ignored the child, ``procxmt'' will
never in fact be called.

By executing the statement ``return(0)'';
on line 4210, ``procxmt'' forces
``stop'' to loop back to line 4020. In
the case where the parent has already
died, the test on line 4022 will then
fail, and a call on ``exit'' (4032) will
result.

\bd
\item[4211:] Store the value of ``ipc.ip\_req''
 before resetting the latter,
 ``wake up'' the parent, and select
 the next action as indicated.
\ed

The various actions are adequately
explained in Section ``PTRACE (II)'' of
the UPM, with the one qualification
that cases 1, 2 and 4, 5 are documented
the wrong way around (i.e. ``I'' and ``D''
spaces respectively, not ``D'' and ``I''!).

%
% The Lion's Commentary, file ch14.tex, version 1.5, 17 May 1994
%
{\noindent \Large \bf Section Three}

{\noindent \sf Section Three
is concerned with basic
input/output operations between the
main memory and disk storage.

These operations are fundamental to the
activities of program swapping and the
creation and referencing of disk files.

This section also introduces procedures
for the use and manipulation of the
large (512 byte) buffers.
}

\se{Program Swapping}

UNIX, like all time-sharing systems,
and some multiprogramming systems uses
``program swapping'' (also called ``rollin/roll-out'')
 to share the limited
resource of the main physical memory
among several processes.

Processes which are suspended may be
selectively ``swapped out'' by writing
their data segments (including the ``per
process data'') into a ``swap area'' on
disk

The main memory area which was occupied
can then be reassigned to other
processes, which quite probably will be
``swapped in'' from the ``swap area''.

Most of the decisions regarding ``swapping out'', and all the decisions
regarding ``swapping in'', are made by
the procedure ``sched''. ``Swapping in'' is
handled by a direct call (2034) on the
procedure ``swap'' (5196), whereas ``swapping out'' is handled by a call (2024)
on ``xswap'' (4368).

For those archaeologists who like to
ponder the ``bones'' of earlier versions
of operating systems, it seems that
originally ``sched'' called ``swap''
directly to ``swap out'' processes,
rather than via ``xswap''. The extra procedure (one of several to be found in
the file ``text.c'') has been necessitated by the implementation of the
sharable ``text segments''.

It is instructive to estimate how much
extra code has been necessitated by the
text segment feature: in ``text.c'' are
four procedures ``xswap'', ``xalloc'',
``xfree'' and ``xccdec'', which manipulate
an array of structures called ``text'',
which is declared in the file ``text.h''.
Additional code has also been added to
``sysl.c'' and ``slp.c''.


\sbs{Text Segments}

Text segments are segments which contain only ``pure'' code and data i.e.
code and data which remain unaltered
throughout the program execution, so
that they may be shared amongst several
processes executing the same program.

The resulting economies in space can be
quite substantial when many users of
the system are executing the same program simultaneously e.g. the editor or
the ``shell''.

Information about text segments must be
stored in a central location, and hence
the existence of the ``text'' array. Each
program which shares a text segment
keeps a pointer to the corresponding
text array element in ``u.u\_textp''.

The text segment is stored at the
beginning of the code file. The first
program to begin execution causes a
copy of the text segment to be made in
the ``swap'' area.

When subsequently no programs are left
which reference the text segment, the
resources absorbed by the text segment
are released. The main memory resource
is released whenever there are no programs which reference the text segment
currently in main memory; the ``swap''
area is released in general whenever
there are no programs left running
which reference the text segment.


The numbers in each of these states are
denoted by ``x\_ccount'' and ``x\_count''
respectively. Decrementing these
numbers is handled by the routines
``xccdec'' and ``xfree'' which also take
care of releasing resources when the
counts reach zero. (``xccdec'' is called
whenever a program is swapped out or
terminates. ``xfree'' is called by ``exit''
whenever a program terminates.)

\sbs{sched (1940)}

Process \#0 executes ``sched''. When it is
not waiting for the completion of an
input/output operation that it has initiated, it spends most of its time
waiting in one of the following situations:

\bd
 \item[A. (runout)]
 None of the processes which are
 swap\-ped out is ready to run, so
 that there is nothing to do. The
 situation may be changed by a call
 to ``wakeup'', or to ``xswap'' called
 by either ``newproc'' or ``expand''.

 \item[B. (runin)]
 There is at least one process
 swapped out and ready to run, but
 it hasn't been out more than 3
 seconds and/or none of the
 processes presently in main memory
 is inactive or has been there more
 than 2 seconds. The situation may
be changed by the effluxion of
time as measured by ``clock'' or by
a call to ``sleep''.
\ed

\noindent When either of these situations terminate:

\bd
\item[1958:] With the processor running at
 priority six, so that the clock
 can't interrupt and change values
 of ``p\_time'', a search is made for
 the process which is ready to run
 and has been swapped out for the
 longest time;

\item[1966:] If there is no such process then
situation A holds;

\item[1976:] Search for a main memory area of
 adequate size to hold the data
 segment. If an associated text
 segment must be present also but
 is not currently in main memory,
 the area is increased by the size
 of the text segment;

\item[1982:] If an area of adequate size is
 available the program branches to
 ``found2'' (2031). (Note that the
 program does not handle the case
 where there is sufficient space
 for both text and data segments
 but in distinct areas of main
 memory. Would it be worth while
 to extend the code to cover this
 possibility?);

\item[1990:] Search for a process which is in
 main memory, but which is not the
 scheduler or locked (i.e. already
 being swapped out), and whose
 state is ``SWAIT'' or ``SSTOP'' (but
 {\bf not}\\
``SSLEEP'') (i.e. the process
 is waiting for an event of low
 precedence, or has stopped during
 tracing (see Chapter Thirteen)).
 If such a process is found, go to
 line 2021, to swap the image out.

Note that there seems to be a
bias here against processes whose
``proc'' entries are early in the
``proc'' array;

\item[2003:] If the image to be swapped in has
 been out less than 3 seconds,
 then situation B holds;

\item[2005:] Search for the process which is
 loaded, but is not the scheduler
 or locked, whose state is ``SRUN''
 or ``SSLEEP'' (i.e. ready to run,
 or waiting for an event of high
 precedence) and which has been in
 main memory for the longest time;

\item[2013:] If the process image to be
 swapped out has been in main
 memory for less than 2 seconds,
 then situation B holds.

The constant ``2'' here (also the
``3'' on line 2003) is somewhat
arbitrary. For some reason the
programmer has departed from his
usual practice of naming such
constants to emphasise their origins;

\item[2022:] The process image is flagged as
 not loaded and is swapped out
 using ``xswap'' (4368).

Note that the ``SSWAP'' flag is not
set here because the process
swapped out is not the current
process. (Cf. lines 1907, 2286);

\item[2032:] Read the text segment into main
 memory if necessary. Note that
 the arguments for the ``swap'' procedure are:
	\bi
	\item an address within the swap area of the disk;

	\item a main memory address (ordinal
number of a 32 word block);

	\item a size (number of 32 word blocks
to be transferred);

\item a direction indicator
(``B\_READ==1'' denotes ``disk to
main memory'');
	\ei

\item[2042:] Swap in the data segment and ...

\item[2044:] Release the disk swap area to the
 available list, record the main
 memory address, set the ``SLOAD''
 flag and reset the accumulated
 time indicator.
\ed

\sbs{xswap (4368)}

\bd
\item[4373:] If ``oldsize'' data was not supplied, use the current size of
the data segment stored in ``u'';

\item[4375:] Find a space in the disk swap
area for the process's data segment. (Note that the disk swap
area is allocated in terms of 512
character blocks);

\item[4378:] ``xccdec'' (4490) is called (unconditionally!) to decrease the
count, associated with the text
segment, of the number of ``in
main memory'' processes which
reference that text segment. If
the count becomes zero, the main
memory area occupied by the text
segment is simply returned to the
available space. (There is no
need to copy it out, since, as we
shall see, there will be a copy
already in the disk swap area);

\item[4379:] The ``SLOCK'' flag is set while the
process is being swapped out.
This is to prevent ``sched'' from
attempting to ``swap out'' a process which is already in the process of being ``swapped out''.
(This can only happen if ``swapping out'' was started initially
by some routine other than
``sched'' e.g. by ``expand'');

\item[4382:] The main memory image is released
 except when ``xswap'' is called by
 ``newproc'';

\item[4388:] If ``runout'' is set, ``sched'' is
 waiting for something to ``swap
 in'', so wake it up.
\ed

\sbs{xalloc (4433)}

``xalloc'' is called by ``exec'' (3130),
when a new program is being initiated,
to handle the allocation of, or linking
to, the text segment. The argument,
``ip'', is a pointer to the ``mode'' of the
code file. At the time of this call,
``u.u\_arg[1]'' contains the text segment
size in bytes.

\bd
\item[4439:] If there is no text segment,
 return immediately;

\item[4441:] Look through the ``text'' array for
 both an unused entry and an entry
 for the text segment. If the
 latter can be found, do the bookkeeping and go to ``out'' (4474);

\item[4452:] Arrange to copy the text segment
 into the disk swap area. Initialise the unused text entry, and
 get space in the disk swap area;

\item[4459:] Change the space occupied by the
 process to one large enough to
 contain the ``per process data''
 area and the text segment;

\item[4460:] The call on ``estabur'' is necessary to set the user mode
segmentation registers before reading the code file;

\item[4461:] A UNIX process can only initiate
 one input/output operation at a
 time. Hence it is possible to
 store i/o parameters at standard
 locations in the ``u'' structure,
 viz. ``u.u\_count'', ``u.u\_offset[ ]'' and
``u.u\_base'';



\item[4462:] The octal value 020 (decimal 16)
 is an offset into the code file;

\item[4463:] Information is to be read into
the area beginning at location
zero in the user address space;

\item[4464:] Read the text segment part of the
code file into the current data
segment;

\item[4467:] ``Swap out'' the data segment
(minus the ``per process data'')
into the disk swap area reserved
for the text segment;

\item[4473:] ``Shrink'' the data segment -- it is
 about to be swapped out;

\item[4475:] ``sched'' always ``swaps in'' the
 text segment before the data segment i.e. there is no mechanism
 for bringing the text segment
 into main memory once the data
 segment is present. If the text
segment is not in main memory,
get back into step by ``swapping
out'' the data segment to disk.
\ed

It will be noted that the code to handle text segments is very conservative
whenever the situation starts to get
complicated. For example, the ``panic''
(4451) when no more text entries are
available would seem to be a rather
extreme reaction. However the strategy
of being generous with ``text'' array
space is quite likely to be less expensive than the code needed to do
``better''. What do you think?


\sbs{xfree (4398)}

``xfree'' is called by ``exit'' (3233);
when a process is being terminated, and
by ``exec'' (3128), when a process is
being transmogrified.

\bd
\item[4402:] Set the text pointer in the
 ``proc'' entry to ``NULL'';

\item[4403:] Decrement the main memory count
 and if it is now zero ...

\item[4406:] and if the text segment has not
 been flagged to be saved, ...

\item[4408:] Abandon the image of the text
 segment in the disk swap area;

\item[4411:] Call ``iput'' (7344) to decrement
 the ``inode'' reference count and
 if necessary delete it.
\ed


``ISVTX'' (5695) is a mask which defines
the ``sticky bit'' mentioned in section
``CHMOD(I)'' of the UPM. If this bit is
set, the disk copy of the text segment
is allowed to remain in the disk swap
area even when no programs are running
which reference it, in the expectation
that it will be required again shortly.
This is an efficient device for commonly used programs such as the ``shell''
or the editor.

%
% The Lion's Commentary, file ch15.tex, version 1.5, 17 May 1994
%
\se{Introduction to Basic I/O}

There are three files whose contents
need to be thoroughly absorbed before
the subject of UNIX input/output is
broached in detail.

\sbs{The File `buf.h'}

This file declares two structures
called ``buf'' (4520) and ``devtab''
(4551). Instances of the structure
``buf'' are declared as 'bfreelist
(4567) and as the array ``buf'' (!)
(4535) with ``NBUF'' elements.

The structure ``buf'' is possibly
misnamed because it is in fact a {\bf buffer
header} (or buffer control block). The
buffer areas proper are allocated
separately and declared (4720) as

\begin{verbatim}
   ``char buffers [NBUF] [514];''
\end{verbatim}

Pointers from the ``buf'' array to the
``buffers'' array are set up by the procedure ``binit''.


Other instances of the structure ``buf''
are declared as ``swbuf'' (4721) and
``rrkbuf'' (5387). No 514 character
buffer areas are associated with
``bfreelist'' or ``swbuf'' or ``rrkbuf''.

The ``buf'' structure may be divided into
three parts:

\bd
\item[(a) flags] These convey status information and are contained within
 a single word. Masks for setting these flags are defined as
 ``B\_WRITE'', ``B\_READ'' etc. in lines 4572 to 4586.

\item[(b) list pointer] Forward and backward pointers for two doubly
 linked lists, which we shall
 refer to as the ``b''-list and the
 ``av''-list.

\item[(c)  i/o parameters] A set of values
 associated with the actual data
 transfer.
\ed

\sbs{devtab (4551)}

The ``devtab'' structure has five words,
the last four of which are forward and
backward pointers.

One instance of ``devtab'' is declared
within the device handler for each
block type of peripheral device. For
our model system the only block device
is the RK05 disk, and ``rktab'' is
declared as a ``devtab'' structure at
line 5386.

The ``devtab'' structure contains some
status information for the the device
and serves as a list head for:

\bd
\item[(a)] the list of buffers associated
 with the device, and simultaneously on the ``av''-list;

\item[(b)] the list of outstanding i/o
 requests for the device.
\ed

\sbs{The File `conf.h'}

The file ``conf.h'' declares:

\bi
\item yet another way to dissect an
integer into two parts (``d\_minor''
and ``d\_major''). Note that
``d\_major'' corresponds to ``hibyte'' (0180);

\item two arrays of structures;

\item two integer variables, ``nlkdev''
and ``nchrdev''.
\ei

The two arrays of structures, ``bdevsw''
and ``cdevsw'', are declared but not
dimensioned or initialised in ``conf.h''.
The initialisation of these arrays is
performed in the file ``conf.c''.

\sbs{The File `conf.c'}

This file, along with ``low.s'', is generated individually at each installation (to reflect the set of peripherals
actually installed) by the program
``mkconf''. (In our case, ``conf.c''
reflects the representative devices for
our model system.)

This file initialises the following:

\begin{verbatim}
  bdevsw  (4656)   swapdev (4696)
  cdevsw  (4663)   swplo   (4637)
  rootdev (4635)   nswap   (4698)
\end{verbatim}

\sbs{System Generation}

System generation at a UNIX installation consists mainly of:

\bi
\item running ``mkconf'' with appropriate
 input;

\item recompiling the output files (created
as ``c.c'' and ``l.s'');

\item reloading the system with the revised
 object files.
\ei

This process only takes a few minutes
(not the several hours of some other
operating systems). Note that ``bdevsw''
and ``cdevsw'' are defined differently in
``conf.c'' from elsewhere, namely as a
one dimensional array of pointers to
functions which return integer values.
This quietly ignores the fact that, for
example, ``rktab'' is not a function, and
relies on the linking program not to
enquire too closely into the nature of
the work which it is performing.

\sbs{swap (5196)}

Before plunging into all the detail of
the file ``bio.c'', it will be instructive as well as convenient to examine
one routine which was introduced earlier, namely ``swap''.

The buffer head ``swbuf'' was declared to
control swapping input/output, which
must share access to the disk with
other activity. No element of ``buffers''
is associated with ``swbuf''. Instead the
core area occupied (or to be occupied)
by the program serves as the data
buffer.

\bd
\item[5200:] The address of the flags in
``swbuf'' is transferred to the
register variable ``fp'' for convenience and economy;

\item[5202:] The ``B\_BUSY'' flag is tested, and
if it is on, a swap operation is
already under way, so that the
``B\_WANTED'' flag is set and the
process must wait via a call on ``sleep''.
\ed

Note that the code loop on lines
5202 to 5205 runs at priority
level six, i.e. one higher than
the disk interrupt priority.

Can you see why this is necessary? Under what conditions will
the ``B\_BUSY'' flag be set?

\bd
\item[5206:] The flags are set to reflect:

\bi
\item ``swbuf'' is in use (``B\_BUSY'');

\item physical i/o implying a large
transfer direct to/from the user
data segment\\
(``B\_PHYS'');

\item whether the operation is read or
write. (``rdflg'' is a parameter to
``swap'');
\ei

\item[5207:] The ``b\_dev'' field is initialised.
(Presumably this could have been
performed once during initialisation rather than every time
``swbuf'' is used, i.e. in ``binit''.);

\item[5208:] ``b\_wcount'' is initialised. Note
 the negative value and the effective multiplication by 32;

\item[5210:] The hardware device controller
requires a full physical address
(18 bits on the PDP 11/40). The
block number of a 32 word block
must be converted into two parts:
the low order ten bits are
shifted left six places and
stored as ``b\_addr'', and the
remaining six high order bits as
``b\_xmem''. (On the PDP 11/40 and
11/45 only two of these bits are
significant.);

\item[5212:] A mouthful at first glance! Shift
``swapdev'' eight places to the
right to obtain the major device
number. Use the result to index
``bdevsw''. From the structure
thus selected, extract the strategy routine and execute it with
the address of ``swbuf'' passed as
a parameter;

\item[5213:] Explain why this call on ``spl6''
is necessary;

\item[5214:] Wait until the i/o operation is
 complete. Note that the first
 parameter to ``sleep'' is in effect
 the address of ``swbuf'';

\item[5216:] Wakeup those processes (if any)
 which are waiting for ``swbuf'';

\item[5218:] Reset the process or priority to
 zero, thus allowing any pending
 interrupts to ``happen'';

\item[5219:] Reset both the ``B\_BUSY'' and
 ``B\_WANTED'' flags.
\ed

\sbs{Race Conditions}

The code for ``swap'' has a number of
interesting features. In particular it
displays in microcosm the problems of
race conditions when several processes
are running together.

\bigskip

\noindent Consider the following scenario:

No swapping is taking place when process A initiates a swapping operation.
Denoting ``swbuf.b\_flags'' by simply
``flags'', we have initially

\begin{verbatim}
  flags == null
\end{verbatim}

\noindent Process A is not delayed at line 5204,
initiates its i/o operation and goes to
sleep at line 5215. We now have

\begin{verbatim}
  flags == B_BUSY | B_PHYS | rdflg
\end{verbatim}

\noindent which was set at line 5206.


Suppose now while the i/o operation is
proceeding, process B also initiates a
swapping operation. It too begins to
execute ``swap'', but finds the ``B\_BUSY''
flag set, so it sets the ``B\_WANTED''
flag (5203) and goes to sleep also
(5204). We now have

\begin{verbatim}
  flags == B_BUSY | B_PHYS | rdflg | B_WANTED
\end{verbatim}

At last the i/o operation completes.
Process C takes the interrupt and executes ``rkintr'', which calls (5471)
``iodone'' which calls (5301) ``wakeup'' to
awaken process A and process B.
``iodone'' also sets the ``B\_DONE'' flag
and resets the ``B\_WANTED'' flag so that

\begin{verbatim}
  flags == B_BUSY | B_PHYS | rdflg | B_DONE
\end{verbatim}

What happens next depends on the order
in which process A and process B are
reactivated. (Since they both have the
same priority, ``PSWP'', it is a toss-up
which goes first.)

\bd
\item[Case (a):] Process A goes first.
``B\_DONE'' is set so no more sleeping is
needed. ``B\_WANTED'' is reset so there is
no one to ``wakeup''. Process A tidies up
(5219), and leaves ``swap'' with

\begin{verbatim}
  flags == B_PHYS | rdflg | B_DONE
\end{verbatim}

Process B now runs and is able to initiate its i/o operation without further
delay.

\item[Case (b):] Process B goes first. It
 finds ``B\_BUSY'' on, so it turns the
 ``B\_WANTED'' flag back on, and goes to
 sleep again, leaving

\begin{verbatim}
  flags == B_BUSY | B_PHYS | rdflg |
           B_DONE | B_WANTED
\end{verbatim}

Process A starts again as in Case (a),
but this time finds ``B\_WANTED'' on so it
must call ``wakeup'' (5217) in addition
to its other chores. Process B finally
wakes again and the whole chain completes.
\ed

Case (b) is obviously much less efficient than case (a). It would seem that
a simple change to line 5215 to read

\begin{verbatim}
  sleep (fp, PSWP-1);
\end{verbatim}

\noindent would cost virtually nothing and ensure
that Case (b) never occurred!

The necessity for the raising of processor priority at various points
should be studied: for example if line
5201 was omitted and if process B had
just completed line 5203 when the ``i/o
complete'' interrupt occurred for Process A's operation, then ``iodone'' would
turn off ``B\_WANTED'' and perform
``wakeup'' before process B went to sleep ... forever! A bad scene.

\sbs{Reentrancy}

Note also the assumption made above,
that both process A and process B could
execute ``swap'' simultaneously. All UNIX
procedures are in general ``re-entrant''
(which means multiple simultaneous executions are possible). How would UNIX
have to change if re-entrancy were not
allowed?

\sbs{For the Uninitiated}

We can now return to complete an investigation started
in Chapter Eight concerning ``aretu'' and ``u.u\_ssav'':

After setting ``u.u\_ssav'' (2284),
``expand'' calls (2285) ``xswap'',
which calls (4380) ``swap'',
which calls (5215) ``sleep'',
which calls (2084) ``swtch'',
which {\bf resets} ``u.u\_rsav'' (2189).

Thus in fact ``u.u\_rsav'' finally gets
reset to a value appropriate to four
procedure calls deeper than that for
``u.u\_ssav''.

\sbs{Additional Reading}

The article ``The UNIX I/O System'' by
Dennis Ritchie is highly pertinent.

%
% The Lion's Commentary, file ch16.tex, version 1.3, 16 May 1994
%
\se{The RK Disk Driver}

The RK disk storage system employs a
removable disk cartridge containing a
single disk, which is mounted inside a
drive with moving read/write heads.

The device designated RK11-D consists
of a disk controller together with a
single drive. Additional drives, designated RK05, up to a total of seven, may
be added to a single RK11-D.

A requirement for more than eight
drives would require an additional controller with a different set of UNIBUS
addresses. Also the code in the file
``rk.c'' would have to be modified to
handle the case of two or more controllers. This case is most unlikely
because requirements for large amounts
of on-line disk storage will be more
economically provided otherwise e.g.
by the RP04 disk system.


\begin{center}
\begin{tabbing}
Cartridge \= capacity: \= 1,228,800 words\\
\> (4800 512 byte records)\\
Surfaces/cartridge: \> \> 2\\
Tracks/surface: \> \>200 (plus 3 spare)\\
Sectors/Track: \> \> 12\\
Words/Sector: \> \> 256\\
Recording density: \> \> 2040 bpi maximum\\
Rotation speed: \> \> 1500 rpm\\
Half revolution: \> \> 20 msecs\\
Track positioning:\\
\> 10 msecs (one track)\\
\> 50 msecs (average)\\
\> 85 msecs (worst case)\\
Interrupt Vector Address: 220\\
Priority Level: \> \> 5\\
\end{tabbing}
\end{center}

\begin{tabular}{lll}
\multicolumn{3}{c}{\bf Unibus Register Addresses}\\
Drive Status        & RKDS & 777400 \\
Error               & RKER & 777402 \\
Control Status      & RKCS & 777404 \\
Word Count          & RKWC & 777406 \\
Current bus address & RKBA & 777410 \\
Disk address        & RKDA & 777412 \\
Data Buffer         & RKDB & 777416 \\
\end{tabular}

\begin{center}
 Table 16.1 RK Vital Statistics
\end{center}

The average total access time is 70
milllseconds. With multi-drive subsystems,
seeking by one drive may be overlapped with reading or writing by
another drive. However this feature is
not used by UNIX because of bugs which
existed at one time in the hardware
controller.


In initiating a data transfer, RKDA,
RRBA and RKC are set, and then RKCS is
set. Upon completion, status information is available in RKCS, RRER and
RKDS. When an error occurs, UNIX simply
calls ``deverror'' (2447) to display RKER
and RKDS on the system console, without
any attempt at analysis. An operation
is repeated up to ten times before an
error is reported by the device driver.


The register formats which are
described fully in the ``PDP11 Peripherals Handbook'' are reflected in the
program code at several points. The
following summaries suffice to describe
the features used by UNIX:

\begin{tabular}{ll}
\multicolumn{2}{c}{\bf Control Status Register (RKCS)}\\ \hline
\multicolumn{1}{c}{bit} & \multicolumn{1}{c}{description}
\\
15 & Set when any bit of RKER (the\\
   & Error Register) is set;\\
\\
7  & Set when the control is no\\
   & longer engaged in actively\\
   & executing a function and is ready\\
   & to accept a command;\\
\\
6  & When set, the control will issue\\
   & an interrupt to vector address\\
   & 220 upon operation completion or\\
   & error;\\
\\
5--4 & Memory Extension. The two most\\
   & significant bits of the 13 bit\\
   & physical bus address. (The other\\
   & 16 bits are recorded in RKBA.);\\
\\
3--1 & Function to be performed:\\
\\
   & CONTROL RESET: 000\\
   & WRITE: 001\\
   & READ: 010\\
   & etc.,\\
\\
0 & Initiate the function designated\\
 & by bits 1 to 3 when set. (write\\
 & only);\\
\\
\multicolumn{2}{c}{\bf Word Count Register (RKWC)}\\ \hline
\end{tabular}

Contains the twos complement of the
number of words to be transferred.

\bigskip

\begin{tabular}{ll}
\multicolumn{2}{c}{\bf Disk Address Register (RKDA)}\\ \hline
\multicolumn{1}{c}{bit} & \multicolumn{1}{c}{description}
\\
15--13 & Drive number (0 to 7)\\
12--5 & Cylinder number (0 to 199)\\
 4 & Surface number (0,1)\\
 3-0 & Sector address (0 to 11)\\
\end{tabular}
 

\sbs{The file `rk.c'}

This file contains the code which is
specific to the RK disk system, i.e.
which is the RK ``device driver''.


\sbs{rkstrategy (5389)}

The strategy routine is called, e.g.
from ``swap'' (5212), to handle both read
and write requests.

\bd
\item[5397:] The test and call on ``mapalloc''
 here is a ``no-op'' except on the
 PDP11/70 system;

\item[5399:] The code from here to line 5402
 appears to be unnecessarily devious! See the discussion of
 ``rkaddr'' below. If the block
 number is too large, set the
 ``B\_ERROR'' flag and report ``completion'';

\item[5407:] Link the buffer into a FIFO list
 for the controller. The list is
singly linked, uses the ``av\_forw''
pointer of the ``buf'' structures,
and has head and tail pointers in
``rktab''. Interrupts from disk
devices may not be allowed after
the first step;

\item[5414:] If the RK controller is not
 currently active, wake it up via
 a call on ``rkstart'' (5440), which
 checks that there is something to
 do (5444), flags the controller
 as busy (5446) and calls
 ``devstart'' (5447), passing as
 parameters:

\bi
\item a pointer to the first enqueued
buffer header;

\item the address of the RKDA disk
address register. (The value
passed is in effect 0177412. See
lines 5363, 5382.);

\item a ``disk address'' computed by
``rkaddr'';

\item zero (not really important in our
discussion, and may be ignored).

\ei
\ed


\sbs{rkaddr (5420)}

The code in this procedure incorporates
a special feature for files which
extend over more than one disk drive.
This feature is described in the UPM
Section ``RK(IV)''. Its usefulness seems
to be restricted.

The value returned by ``rkaddr'' is formatted for direct transmission to the
control register, RKDA.


\sbs{devstart (5096)}

This procedure when called for the RK
disk loads appropriate values into the
registers RKDA, RKBA, RKWC and RKCS in
succession. Only the last value needs
to be computed at this stage.

The calculation, though messy in
appearance, is straight forward. Note
that ``hbcom'' is zero and ``rbp-$>$b\_xmem''
contains the two high order bits of the
physical core address. The loading of
RKCS initialises the disk controller
i.e. the operation is now entirely
under the control of the hardware.

``devstart'' returns to ``rkstart'' (5448),
which returns to ``rkstrategy'' (5416),
which resets the processor priority and
returns to ``swap'' (5213), which ...

\sbs{rkintr (5451)}

This procedure is invoked to handle the
interrupts which occur when RK disk
operations are completed.

\bd
\item[5455:] Check for a false alarm!

\item[5459:] Inspect the error bit; if set ...

\item[5460:] Call ``deverror'' (2447) to display
a message on the system console
terminal;

\item[5461:] Clear the internal registers of
 the disk controller and ...

\item[5462:] Wait till this is completed (usually a few microseconds);

\item[5463:] If the operation has been retried
 less than ten times, call
 ``rkstart'' to try again. Otherwise
 give up and report an error;

\item[5469:] Set the ``retry'' (!) count back to
 zero, remove the current operation from the ``actf'' list, and
 complete the operation by calling
 ``iodone'';

\item[5472:] ``rkstart'' is called unconditionally here. If the call is not
 necessary (because the ``actf''
 list is empty) ``rkstart'' will
 return immediately (5444).
\ed


\sbs{iodone (5018)}

This routine is primarily concerned
with the return of resources when a
block i/o operation has completed. It:

\bi
\item frees up the Unibus map (for 11/70's,
 if appropriate);

\item sets the ``B\_DONE'' flag;

\item releases the buffer if the i/o was
 asynchronous, or else resets the
 ``B\_WANTED'' flag and wakes up any
 process waiting for the i/o
 operation to complete.
\ei


%
% The Lion's Commentary, file ch17.tex, version 1.5, 16 May 1994
%
\se{Buffer Manipulation}

In this chapter we look at the file
``bio.c'' in detail. It contains most of
the basic routines used to manipulate
buffer headers and buffers (4535,
4720).


Individual buffer headers are tagged by
a device number ``b\_dev'', (4527) and a
block number ``b\_blkno'', (4531). (Note
the way in which the latter is declared
as an unsigned integer.)

Buffer headers may be linked simultaneously into two lists:

\bd
\item[the ``b''-lists] are lists, one per
device controller, which link
together buffers associated with
that device type;

\item[the ``av''-list] is a list of buffers
which may be detached from their
current use and converted to an
alternate use.
\ed

Both the ``av''-list and the various
``b''-lists are doubly linked to facilitate insertion and deletion at any
point.

\sbs{Flags}

If a buffer is withdrawn temporarily
from the ``av''-list, then its ``B\_BUSY''
flag is raised.

If the contents of a buffer correctly
reflect the information that is or
should be stored on disk, then the
``B\_DONE'' flag is raised.

If the ``B\_DELWRI'' flag is raised, the
contents of the buffer are more up to
date than the contents of the
corresponding disk block, and hence the
buffer must be written out before it
can be reassigned.


\sbs{A Cache-like Memory}

It will be seen that the large buffers
in UNIX are manipulated in a way which
is analogous to the operation of
hardware cache attached to the main
memory of a computer e.g. the PDP11/70.

Buffers are not assigned to any particular program or file, except for very
short intervals at a time. In this way
a relatively small number of buffers
can be shared effectively amongst a
large number of programs and files.

Information is left in the buffers
until the buffer is needed i.e. immediate ``write through'' is avoided if only
part of the buffer has recently been
changed. Programs which read or write
records which are small compared with
the buffer size are then not penalised
unduly.

Finally when programs are terminated
and files are closed, the problems of
ensuring that the program's buffers are
flushed properly (problems which have
plagued other operating systems) have
largely disappeared.

There is one area of practical concern:
if the decision ``when to write'' is left
to the operating system alone, then
some buffers may not be written out for
a very long time. Accordingly there is
a utility program which runs twice per
minute and forces all such buffers to
be written out unconditionally. This
limits the likely amount of damage that
a sudden system crash may cause.


\sbs{clrbuf (5038)}

This routine zeros out the first 256
words (512 bytes) of the buffer. Note
that the parameter passed to ``clrbuf''
is the address of the buffer header.
``clrbuf'' is called by ``alloc'' (6982).

\sbs{incore (4899)}

This routine searches for a buffer that
is already assigned to a particular
(device, block number) pair. It
searches the circular ``b''-list whose
head is the ``devtab'' structure for the
device type. If a buffer is found, the
address of the buffer header is
returned. ``incore'' is called by
``breada'' (4780, 4788).

\sbs{getblk (4921)}

This routine performs the same search
as ``incore'' but goes further in that if
the initial search is unsuccessful, a
buffer is allocated from the ``av''-list
(available list).

By a call on ``notavail'' (4999), the
buffer is removed from the ``av''-list
and flagged as ``B\_BUSY''.



``getblk'' is more suspicious of its
parameters than ``incore''. It is called
by

\begin{verbatim}
  exec   (3040)        writei (6304)
  exit   (3237)        iinit  (6928)
  bread  (4758)        alloc  (6981)
  breada (4781,4789)   free   (7016)
  smount (6123)        update (7216)
\end{verbatim}


\bd
\item[4940:] At this point the required buffer
 has been located by searching the
 ``b''-list. Either it is ``B\_BUSY''
 in which case a ``sleep'' must be
 taken (4943), or else it is
 appropriated (4948);

\item[4953:] If the required buffer has not
been located, and if the
``av''-list is empty, set the
``B\_WANTED'' flag for the ``av''-list
and go to ``sleep'' (4955);

\item[4960:] If the ``av''-list is not empty,
 select the first member, and if
 it represents a ``delayed write''
 arrange to have it written out
 asynchronously (4962);

\item[4966:] ``B\_RELOC'' is a relic! (See 4583);

\item[4967:] The code from here until 4973
unconditionally removes the
buffer from the ``b''-list for its
current device type and reinserts
it into the bn-list for the new
device type. Since this will frequently be a ``no-op'' i.e. the new
and old device type will be the
same, it would seem desirable to
insert a test

\begin{verbatim}
 if (bp->b_dev == dev)
\end{verbatim}

\noindent before executing lines 4967 to
4974.

Note the special handling for calls where\\
``dev == NODEV'' (--1).
(Such calls incidentally are made
without a second parameter - tut!
tut! See e.g. 3040.)
\ed

``bfreelist'' serves as the ``devtab''
structure for the ``b''-list for ``NODEV''.

\sbs{brelse (4869)}

This procedure takes the buffer passed
as a parameter and links it back into
the ``av''-list.

Any process which is either waiting for
the particular buffer or any available
buffer is woken up.

Note however that since both ``sleeps''
(4943, 4955) are at the same priority,
if two processes are waiting -- one for
the particular buffer and one for any
buffer -- it will be a toss-up which
will get it.

By giving the first priority over the
second (e.g. by biasing by one) the
race should be resolved more satisfactorily. The disadvantage of such a
change might be that it could lead to a
deadlock situation in certain rather
peculiar circumstances.

If an error has occurred e.g. upon
reading information into the buffer
the information in the buffer may be
incorrect. The assignment on line 4883
ensures that the information in the
buffer will not be mistakenly retrieved
subsequently. The ``B\_ERROR'' flag is
set e.g. by ``rkstrategy'' (5403) and
``rkintr'' (5467).

To see how this could occur, consider
what happens to a buffer when a disk
i/o operation is completed:

\bd
\item[5471] ``rkintr'' calls ``iodone'';
\item[5026] ``iodone'' sets the ``B\_DONE'' flag;
\item[5028] ``iodone'' calls ``brelse'';
\item[4387] ``brelse'' resets the ``B\_WANTED'',
``B\_BUSY'' and ``B\_ASYNC'' flags
but not the ``B\_DONE'' flag;

. . . . . . . . . . . .

\item[4948] ``getblk'' finds the buffer and
 calls ``notavail'';
\item[5010] ``notavail'' sets the ``B\_BUSY'' flag;
\item[4759] ``bread'' (which called ``getblk'')
 finds the ``B\_DONE'' flag set and exits.
\ed

Note that buffer headers are removed
from the ``av''-list by ``notavail'' and
are returned by ``brelse''. Buffer
headers are moved from one ``b''-list to
another by ``getblk''.


\sbs{binit (5055)}

This procedure is called by ``main''
(1614) to initialise the buffer pool.
Empty, doubly linked circular lists are
set up:

\bi
\item for the ``av''-list (``bfreelist'' is
 head);

\item the ``b''-list for null devices (``dev
 == NODEV'') (``bfreelist'' is again
 head);

\item a ``b''-list for each major device
 type.
\ei

\noindent For each buffer:

\bi
\item the buffer header is linked into the
 ``b''-list for the device ``NODEV'' (--1);

\item the address of the buffer is set in
 the header (5067);

\item the buffer flags are set as ``B\_BUSY''
 (this doesn't seem to be really
 necessary) (5072);

\item the buffer header is linked into the
 ``av''-list by a call on ``brelse'' (5073);
\ei


The number of block devices is recorded
as ``nblkdev''. This is used for checking
values for ``dev'' in ``getblk'' (4927),
``getmdev'' (6192) and ``openi'' (6720).
Inspection of ``bdevsw'' (4656) shows
that ``nblkdev'' will be set to eight
whereas the value one is what is really
required.

This result could be obtained by ``editing'' as follows:

\begin{verbatim}
    /5084/m/5081/ "nblkdev=i;
    /5083/m/5077/ "i++
\end{verbatim}

\sbs{bread (4754)}

This is the standard procedure for
reading from block devices. It is
called by:

\begin{verbatim}
  wait   (3282)      iinit  (6927)
  breada (4799)      alloc  (6973)
  statl  (6051)      ialloc (7097)
  smount (6116)      iget   (7319)
  readi  (6258)      iupdat (7386)
  writei (6305)      itrunc (7426, 7431)
  bmap   (6472,6488) namei  (7625) 
\end{verbatim}

``getblk'' finds a buffer. If the
``B\_DONE'' flag is set no i/o is needed.

\sbs{breada (4773)}

This procedure has an additional parameter, as compared with ``bread''. It is
called only by ``readi'' (6256).

\bd
\item[4780:] Check if the desired block has
 already been assigned to a
 buffer. (It may not yet be
 available, but at least is it
 there?);

\item[4781:] If not initiate the necessary
read operation but don't wait for
it to finish;

\item[4788:] Look around for the ``read ahead''
 block. If it is not there, allocate a buffer (4789) but release
 it (4791) if the buffer is
 already ready;

\item[4793:] The ``read ahead'' block is not
 ready, so initiate an asynchronous read operation;

\item[4798:] If a buffer was assigned to the
 current block call ``bread'' to
 wrap it up, else...

\item[4800:] Wait for the completion of the
 operation which was started at
 line 4785.
\ed


\sbs{bwrite (4809)}

This is the standard procedure for
writing to block devices. It is called
by ``exit'' (3239), ``bawrite'' (4863),
``getblk'' (4963), ``bflush'' (5241),
``free'' (7021), ``update'' (7221) and
``iupdat'' (7400). N.B. ``writei'' calls
``bawrite'' (6310)!

\bd
\item[4820:] If the ``B\_ASYNC'' flag is not set,
 the procedure does not return
 until the i/o operation is completed;

\item[4823:] If the ``B\_ASYNC''  is set, but
``B DELWRI'' {\bf was} not set (note
``flag'' is set at line 4816) call
``geterror'' (5336) to check on the
error flag. (If ``B\_DELWRI'' was
set, and there is an error, sending the error indication to the
right process is ``too hard.'').
The call (4824) on ``geterror''
will only report errors related
to the initiation of the write
operation.
\ed

\sbs{bawrite (4856)}

This procedure is called by ``writei''
(6310) and ``bdwrite'' (4845). ``writei''
calls either ``bawrite'' or ``bdwrite''
depending on whether the block to be
written has been wholly or partially
filled.


\sbs{bdwrite (4836)}

This procedure is called by ``writei''
(6311) and ``bmap'' (6443, 6449, 6485,
6500 and 6501 !).

\bd
\item[4844:] Don't delay the write if the device is a magnetic tape drive ...
 keep everything in order;

\item[4847:] Set the ``B\_DONE'', ``B\_DELWRI''
 flags and call ``brelse'' to link
 the buffer into the ``av''-list.
\ed

\sbs{bflush (5229)}

This procedure is called by ``update''
(7201), which is called by ``panic''
(2420), ``sync'' (3489) and ``sumount''
(6150).

``bflush'' searches the ``av''-list for
``delayed write'' blocks and forces them
to be written out asynchronously.

Note that as ``notavail'' adjusts the
links of the ``av''-list, the search
(which runs at processor priority six)
is reinitiated after each ``delayed
write'' block is encountered.

Note also that since it happens that
``bflush'' is only called by ``update''
with ``dev'' equal to ``NODEV'', line 5238,
in particular, could be simplified.


\sbs{physio (5259)}

This routine is called to handle ``raw''
input/output i.e. operations which
ignore the normal 512 character block
size.

``physio'' is called by ``rkread'' (5476)
and ``rkwrite'' (5483) which appear as
entries in the array ``cdevsw'' (4684)

``Raw i/o'' is not an essential feature
of UNIX. For disk devices it is used
mainly for copying whole disks and
checking the integrity of the file system as a whole (see e.g. ICHECK (VIII)
in the UPM), where it is convenient to
read whole tracks, rather than single
blocks, at a time.

Note the declaration of ``strat'' (5261).
Since the actual parameter used e.g.
``rkstrategy'' (5389) does not return any
value, is this form of declaration
really necessary?

%
% The Lion's Commentary, file ch18.tex, version 1.4, 16 May 1994
%
{\noindent \Large Section Four}

{\sf Section Four is concerned with files
and file systems.

A file system is a set of files and
associated tables and directories
organised onto a single storage device
such as a disk pack.

This section covers the means of
creating and accessing files,
locating files via directories, and
organising and maintaining
file systems.

It also includes the code for an exotic
breed of file called a ``pipe''.
}


\se{File Access and Control}


A large part of every operating system
seems to be concerned with data management and file management, and UNIX
turns out to be no exception.


Section Four of the source code contains thirteen files.


The first four contain common declarations needed by various of the other
routines:

\bd
\item[``file.h''] describes the structure
of the ``file'' array;

\item[``filsvs.h''] describes the structure
of the ``super block'' for ``mounted''
file systems;

\item[``ino.h''] describes the structure of
``inodes'' recorded on ``mounted''
devices;

\item[``inode.h''] describes the structure
of the ``inode'' array;
\ed

The next two files, ``sys2.c'' and
``sys3.c'' contain code for system calls.
(``sys1.c'' and ``sys4.c'' were presented
in Section Two).

Tne next five files, ``rdwri.c'',
``subr.c'', ``fio.c'', ``alloc.c'' and
``iget.c'', together present the principal routines for file management, and
provide a link between the i/o oriented
system calls and the basic i/o routines.

The file ``nami.c'' is concerned with
searching directories to convert file
pathnames into ``inode'' references.

Finally, ``pipe.c'' is the ``device
driver'' for pipes.

\sbs{File Characteristics}


A UNIX file is conceptually a named
character string, stored on one of a
variety of peripheral devices (or in
the main memory), and accessible via
mechanisms appropriate to the usual
peripheral devices.

It will be noted that there is no
record structure associated with UNIX
files. However ``newline'' characters may
be inserted into the file to define
substrings analogous to records.
 
UNIX carries the ideas of device
independence to their logical extreme
by allowing the file name in effect to
determine uniquely all relevant attributes of the file.


\sbs{System Calls}

The following system calls are provided
expressly for file manipulation:

\begin{center}
\begin{tabular}{llllll}
{\bf \#} & {\bf Name} & {\bf Line \hspace{0.5cm}} & {\bf \#} & {\bf Name} & {\bf Line}\\ \hline
3 & read & 5711 & 14 & mknod & 5952\\
4 & write & 5720 & 15 & chmod & 3560\\
5 & open & 5765 & 16 & chown & 3575\\
6 & close & 5846 & 19 & seek & 5861\\
8 & creat & 5781 & 21 & mount & 6086\\
9 & link & 5909 & 22 & umount & 6144\\
10 & unlink & 3510 & 41 & dup & 6069\\
12 & chdir & 3538 & 42 & pipe & 7723\\
\end{tabular}
\end{center}


\sbs{Control Tables}

The arrays ``file'' and ``inode'' are
essential components of the file access
mechanism.

\sbs{file (5507)}

The array ``file'' is defined as an array
of structures (also named ``file'').

An element of the ``file'' array is considered to be unallocated if ``f\_count''
is zero.

Each ``open'' or ``creat'' system call
results in the allocation of an element
of the ``file'' array. The address of
this element is stored in an element of
the calling process's array
``u.u\_ofile''. It is the index of the
newly allocated element of the latter
array which is passed back to the user
process. Descendants of a process
created by ``newproc'' inherit the
contents of the parent's ``u.u\_ofile''
array.


Each element of ``file'' includes a
counter, ``f\_count'', to determine the
number of current processes which
reference it.

``f\_count'' is incremented by ``newproc''
(1878), ``dup'' (6079) and ``falloc''
(6857); it is decremented by ``closef''
(6657) and (if the file can't be
opened) by ``openl'' (5836).

The ``f\_flag'' (5509) of the ``file'' element notes whether the file is open for
reading and/or writing or whether it is
a ``pipe'' or not. (Further discussion of
``pipes'' will be deferred till Chapter
Twenty-One.)

The ``file'' structure also contains a
pointer, ``f\_inode'' (5511) to an entry
in the ``inode'' table, and a 32 bit
integer, ``f\_offset'' (5512), which is a
logical pointer to a character within
the file.


\sbs{inode (5659)}

``inode'' is defined as an array of
structures (also named ``inode'').


An element of the ``inode''
is considered to be unallocated if the reference
count, ``i\_count'', is zero.

At each point in time, ``inode'' contains
a single entry for each file which may
be referenced for normal i/o operations, or which is being executed or
which has been executed and has the
``sticky'' bit set, or which is the working directory for some process.


Several ``file'' table entries may point
to a single ``inode'' entry. The inode
entry describes the general disporition
of the file.

\sbs{Resources Required}

Each file requires the dedication of
certain system resources. When a file
exists, but is not being referenced in
any way, it requires:

\bd
\item[(a)] a directory entry (16 characters
 in a directory file);

\item[(b)] a disk ``inode'' entry (32 characters in a table stored on the
disk);

\item[(c)] zero, one or more blocks of disk
 storage (512 characters each).
\ed


\noindent In addition if the file is being referenced for some purpose, it requires

\bd
\item[(d)] a core ``inode'' entry (32 characters in the ``inode'' array);
\ed


\noindent Finally if a user program has ``opened''
the file for reading or writing, a
number of resources are required:

\bd
\item[(e)] a ``file'' array entry (8 characters);

\item[(f)] an entry in the user program's
``u.u\_ofile'' array (one word per
file, pointing to a ``file'' array
entry);
\ed


Mechanisms have to be set up for allocating and deallocating each of these
resources in an orderly manner. The
following table gives the names of the
principal procedures involved:

\begin{center}
\begin{tabular}{lll}
{\bf resource} & {\bf obtain} & {\bf free} \\ \hline
directory entry & namei & namei\\
disk ``inode'' entry & ialloc & ifree\\
disk storage block & alloc & free\\
core ``inode'' entry & iget & iput\\
``file'' table entry & falloc & closef\\
``u\_ofile'' entry & ufalloc & close\\
\end{tabular}
\end{center}

\sbs{Opening a File}

When a program wishes to reference a
file which already exists, it must
``open'' the file to create a ``bridge'' to
the file. (Note that in UNIX,
processes usually inherit the open
files of their parents or predecessors,
so that often all needed files are
already implicitly open.) If the file
does not already exist, it must be
``created''.


This second case will be investigated
first:


\sbs{creat (5781)}

\bd
\item[5786:] ``namei'' (7518) converts a pathname into an ``inode'' pointer.
 ``uchar'' is the name of a procedure which recovers the pathname, character by character,
 from the user program data area;

\item[5787:] A null ``inode'' pointer indicates
 either an error or that no file
 of that name already exists;

\item[5788:] For error conditions, see ``CREAT
 (II)'' in the UPM;

\item[5790:] ``maknode'' (7455) creates a core
 ``inode'' via a call on ``ialloc''
 and then initialises it and
 enters it into the appropriate
 directory. Note the explicit
 resetting of the ``sticky'' bit
\ed


\sbs{openl (5804)}

This procedure is called by ``open''
(5774) and ``creat'' (5793, 5795), passing values of the third parameter,
``trf'', of 0, 2 and 1 respectively. The
value 2 represents the case where no
file of the desired name already
exists.

\bd
\item[5812:] The second parameter, ``mode'', can
 take the values 01 (``FREAD''), 02
(``FWRITE'') or 03 (``FREAD \verb+|+ FWRITE'')
when ``trf'' is 0, but only 02 otherwise;

Whete a file of the desired name
already exists, check the access
permissions for the desired
mode(s) of activity via calls on
``access'' (6746), which may set
``u.u\_error'' as a side-effect;

\item[5824:] If the file is being ``created'',
 eliminate its previous contents
 via a call on ``itrunc'' (7414).
 The code here could be improved
 by changing the test to ``(trf ==
 1)''. Verify that this would be
 so.

\item[5826:] ``prele'' (7882) is used to
 ``unlock'' ``inodes''. Where, you
 may ask, did the ``inode'' get
 ``locked'', and why?

\item[5827:] Note that ``falloc'' (6847) calls
 ``ufalloc'' (6824) as the first
 thing it does;

\item[5831:] ``ufalloc'' leaves
 the user file
identifying number
``u.u\_ar0[R0]''. Why does this
statement occur where it does,
instead of after line 5834?

\item[5832:] ``openi'' (6702) is called to call
 handlers for special files, in
 cae any device specific actions
 are required (for disk files
 there is no action);

\item[5839:] In the case of an error while
 making the ``file'' array entry,
 the ``inode'' entry is released by
 a call on ``iput''.
\ed

It will be seen that responsibility is
quite widely distributed. The ``file''
table entry is initialised by ``falloc''
and ``openl''; the ``inode'' table entry,
by ``iget'', ``ialloc'' and ``maknode''.

Note that ``ialloc'' clears out the
``i\_addr'' array of a newly allocated
``inode'' and ``itrunc'' does the same for
a pre-existing ``inode'', so that after
the ``creat'' system call, there are no
disk blocks associated with the file,
now classed as ``small''.


\sbs{open (5763)}

We now turn to consider the case where
a program wishes to reference a file
which already exists.

``namei'' is called (5770) with a second
parameter of zero to locate the named
file. (``u.u\_arg[0]'' contains the
address in the user space of a character string which defines a file path
name.)

``u.u\_arg[1]'' has to be incremented by
one, because there is a mismatch
between the user programming conventions and the internal data representations.)


\sbs{openl revisited}

``trf'' is now zero, so access permissions are checked (5813) but the existing file (if any) is not deallocated
(5824).

What is a little disconcerting here is
that, apart from the call on ``falloc''
(5827), there is no direct call on any
of the ``resource allocation'' routines.
Of course, for an existing file, neither directory entry nor disk ``inode''
entry nor disk blocks need be allocated. The core ``inode'' entry is allocated (if necessary) as a side-effect
of the call on ``namei'', but ... where
is it initialised?


\sbs{close (5846)}

The ``close'' system call is used to
sever explicitly the connection between
a user program and a file and thus can
be regarded as the inverse of ``open''.


The user program's file identification
is passed via r0. The value is validated by
``getf'' (6619), the ``u.u\_ofile''
entry is erased, and a call is made on ``closef''.


\sbs{closef (6643)}

``closef'' is called by ``close'' (5854)
and by ``exit'' (3230). (The latter is
more common since most files do not get
closed explicitly but only implicitly
when the user program terminates.)


\bd
\item[6649:] If the file is a pipe, reset the
mode of the pipe and ``wakeup'' any
process which is waiting for the
pipe, either for information or
for space;

\item[6655:] If this is the last process to
 reference the file, call ``closei''
 (6672) to handle any special end
 of file processing for special
 files and then call ``iput'';

\item[6657:] Decrement the ``file'' entry reference count. If this now zero, the
 entry is no longer allocated.
\ed

\sbs{iput (7344)}


``closei'', as its last action calls
``iput''. This routine is in fact called
from many places, whenever a connection
to a core ``inode'' is to be severed and
the reference count decremented.


\bd
\item[7350:] If the reference count is one at
 this point, the ``inode'' is to be
 released. While this is happening, it should be locked.

 \item[7352:] If the number of ``links'' to the
 file is zero (or less) the file
 is to be deallocated (see below);

\item[7357:] ``iupdat'' (7374) updates the
 accessed and update times as
recorded on the disk ``inode'';

\item[7358:] ``prele'' unlocks the ``inode''. Why
 should it be called here as well
 as at line 7363?
\ed

\sbs{Deletion of Files}


New files are automatically entered
into the file directory as permanent
files as soon as they are ``opened''.
Subsequent ``closing'' of a file does not
automatically cause its deletion. As
was seen at line 7352, deletion will
occur when the field ``i\_nlink'' of the
core ``inode'' entry is zero. This field
is set to one initially by ``maknode''
(7464) when the file is first created.
It may be incremented by the system
call ``link'' (5941) and decremented by
the system call ``unlink'' (3529).


Programs which create temporary ``work
files'' should remove these files before
terminating, by executing an ``unlink''
system call. Note that the ``unlink''
call does not of itself remove the
file. This can only happen when the
reference count (``i\_count'') is about to
be decremented to zero (7350, 7362).

To minimise the problems associated
with ``temporary'' files which survive
program or system crashes, programmers
should observe the conventions that:

\bd
\item[(a)] temporary files should be
 ``unlinked'' immediately after
 they are opened;

\item[(b)] temporary files should always be
 placed in the ``tmp'' directory.
 Unique file names can be generated by incorporating the
 process's identifying number into the file name (See ``getpid''
 (3480)).
\ed

\sbs{Reading and Writing}

It is of interest to work through an
abbreviated summary of the code which
is invoked when a user process performs
a ``read'' system call before examining
the code in detail.

\begin{verbatim}
   .... read (f, b, n); /*user program/

                 {trap occurs}
   2693 trap

                 {system call #3}
   5711 read ();
   5713   rdwr (PREAD);
\end{verbatim}



Execution of the system call by the
user process results in the activation
of ``trap'' running in kernel mode.
``trap'' recognises system call \#3, and
calls (via ``trapl'') the routine ``read'',
which calls ``rdwr''.

\begin{verbatim}
  5731 rdwr

  5736 fp = getf (u.u_ar0[R0];
  5743 u.u_base = u.u_arg[0];
  5744 u.u_count = u.u_arg[1];
  5745 u.u_segflg = 0;
  5751 u.u_offset[1] = f2->f offset[1];
  5752 u.u_offset[0] = fp->f offset[0];
  5754 readi(fp->f inode);
  5756 dpadd(fp->f offset,
              u.u_arg[1]-u.u_count);
\end{verbatim}

``rdwr'' includes much code which is common to both ``read'' and ``write'' operations. It converts, via ``getf'' (6619),
the file identification supplied by the
user process into the address of an
entry in the ``file'' array.

Note that the first parameter of the
system call is passed in a different
way from the remaining two parameters.


``u.u\_segflg'' is set to zero to indicate
that the operation destination is in
the user address space. After ``readi''
is called with a parameter which is an
``inode'' pointer, the final accounting
is performed by adding the number of
characters requested for transfer less
the residual number not transferred
(left in ``u.u\_count'') to the file
offset.

\begin{verbatim}
  6221 readi

  6239 lbn = lshift (u.u_offset, -9);
  6240 on = u.u_offset[1] & 0777;
  6241 n = min (512 - on, u.u_count);
  6248 bn = bmap(ip, lbn);
  6250 dn = ip->i_dev;
  6258 bp = bread (dn, bn);
  6260 iomove (bp, on, n, B READ);
  6261 brelse (bp);

\end{verbatim}

``readi'' converts the file offset into
two parts: a logical block number,
``lbn'', and an index into the block,
``on''. The number of characters to be
transferred is the minimum of
``u.u\_count'' and the number of characters left in the block (in which case
additional block(s) must be read (not
shown)) (and the number of characters
remaining in the file (this case is not
shown)).

``dn'' is the device number which is
stored within the ``inode''. ``bn'' is the
actual block number on the device
(disk), which is computed by ``bmap''
(6415) using ``lbn''.

The call on ``bread'' finds the required
block, copying it into core from disk
if necessary. ``iomove'' (6364)
transfers the appropriate characters to
their destination, and performs
accounting chores.



\sbs{rdwr (5731)}

``read'' and ``write'' perform similar
operations and share much code. The
two system calls, ``read'' (5711) and
``write'' (5720), call ``rdwr'' immediately
to:

\bd
\item[5736:] Convert the user program file
 identification to a pointer in
 the file table;

\item[5739:] Check that the operation (read or
 write) is in accordance with the
 mode with which the file was
 opened;

\item[5743:] Set up various standard locations
 in ``u'' with the appropriate
 parameters;

\item[5746:] ``pipes'' get special treatment
 right from the start!

\item[5755:] Call ``readi'' or ``writei'' as
 appropriate;

 \item[5756:] Update the file offset by, and
set the value returned to the
user program to, the number of
characters actually transferred.
\ed

\sbs{readi (6221)}

\bd
\item[6230:] If no characters are to be
transferred, do nothing;

\item[6232:] Set the ``inode'' flag to indicate
that the ``inode'' has been accessed;

\item[6233:] If the file is a character special file, call the appropriate
 device ``read'' procedure, passing
 the device identification as
 parameter;

\item[6238:] Begin a loop to transfer data in
 amounts up to 512 characters at a
 time until (6262) either an irrecoverable error condition has
 been encountered or the requested
 number of characters has been
 transferred;

\item[6239:] ``lshift'' (1410) concatenates the
 two words of the array
 ``u.u\_offset'', shifts right by
 nine places, and truncates to 16
 bits. This defines the ``logical
block number'' of the file which
is to be referenced;

\item[6240:] ``on'' is a character offset within
 the block;

\item[6241:] ``n'' is determined initially as
the minimum of the number of
characters beyond ``on'' in the
block, and the number requested
for transfer. (Note that ``min''
(6339) treats its arguments as
unsigned integers.)

\item[6242:] If the file is not a special
 block file then ...

\item[6243:] Compare the file offset with the
 current file size;

\item[6246:] Reset ``n'' as the minimum of the
 characters requested and the
 remaining characters in the file;

\item[6248:] Call ``bmap'' to convert the logical block number for the file to
 a physical block number for its
 host device. There will be more
 on ``bmap'' shortly. For now, note
 that ``bmap'' sets ``rablock'' as a
 side effect;

\item[6250:] Set ``dn'' as the device identification from the ``inode'';

\item[6251:] If the file is a special block
 file then ...

\item[6252:] Set ``dn'' from the ``i\_addr'' field
 of the ``inode'' entry. (Presumably
 this will nearly always be the
 same as the ``i\_dev'' field, so why
 the distinction?)

\item[6253:] Set the ``read ahead block'' to the
 next physical block;

\item[6255:] If the blocks of the file are
 apparently being read sequentially then ...

\item[6256:] Call ``breada'' to read the desired
 block and to initiate reading of
 the ``read ahead block'';

\item[6258:] else just read the desired block;

\item[6260:] Call ``iomove'' to transfer information from the buffer to the
 user area;

\item[6261:] Return the buffer to the
 ``av''-list.
\ed


\sbs{writei}

\bd
\item[6303:] If less than a full block is
 being written the previous contents of the buffer must be read
 so that the appropriate part can
 be preserved, otherwise just get
 any available buffer;

\item[6311:] There is no ``write ahead'' facility, but there is a ``delayed
 write'' for buffers whose final
 characters have not been changed;

\item[6312:] If the file offset now points
 beyond the recorded end of file
 character, the file has obviously
 grown bigger!

\item[6318:] Why is it necessary/desirable to
 set the ``IUPD'' flag again? (See
 line 6285.)
\ed


\sbs{iomove (6364)}

The comment at the beginning of this
procedure says most of what needs to be
said. ``copyin'', ``copyout'', ``cpass'' and
``passc'' may be found at lines 1244,
1252, 6542 and 6517 respectively.


\sbs{bmap (6415)}

A general description of the function
of ``bmap'' may be found on Page 2 of
``FILE SYSTEM (V)'' of the UPM.

\bd
\item[6423:] Files of more than 2**15 blocks
 (2**24 characters) are not supported;

\item[6427:] Start with the ``small'' file algorithm (file is not greater than
 eight blocks i.e. 4096 characters);

\item[6431:] If the block number is 8 or more,
the ``small'' file must converted
into a large file. Note this is
a side effect of ``bmap'', and
should occur only when ``bmap'' has
been called by ``writei'' (and
never by ``readi'' -- see line
6245). Thus all files start life
as ``small'' files and are never
explicitly changed to ``large''
files. Note also that the change
is irreversible!

\item[6435:] ``alloc'' (6956) allocates a block
on device ``d'' from the device's
free list. It then assigns a
buffer to this block and returns
a pointer to the buffer header;

\item[6438:] The eight buffer addresses in the
``i\_addr'' array for the ``inode''
are copied into the buffer area
and then erased;

\item[6442:] ``i\_addr[0]'' is set to point to
 the buffer which is set up for a
 ``delayed'' write;

\item[6448:] The file is still small. Get the
next block if necessary;

\item[6456:] Note the setting of ``rablock'';
\ed

\sbs{Leftovers}

You should investigate the following
procedures for yourself:

\begin{verbatim}
   seek  (5861)   statl (6045)
   sslep (5979)   dup   (6069)
   fstat (6014)   owner (6791)
   stat  (6028)   suser (6811)
\end{verbatim}

%
% The Lion's Commentary, file ch19.tex, version 1.4, 16 May 1994
%
\se{File  Directories and Directory Files}

As we have seen, much important  information  about
individual files is contained in the ``inode'' tables.   If  the
file  is currently accessible, or being
accessed, the relevant  information  is
held  in  the core ``inode'' table.  If a
file is on  disk  (more  generally,  on
some  file  system volume) and is not
currently accessible, then the relevant
``inode''  table  is  the one recorded on
the disk (file system volume).


Notably absent from the  ``inode''  table
is any information regarding the ``name''
of the file.  This  is  stored  in  the
directory files.

\sbs{The Directory Data Structure}

Each file must have at least one  name.
A file may have more than one  distinct
name, but the  same  name  may  not  be
shared  by  tw  distinct  files,  i.e.
each name must define a unique file.


A name may be multipart.  When written,
the parts or components of the name are
separated by slashes (``/'').  The  order
of components within a name is significant i.e. 
``a/b/c''  is  different  from ``a/c/b''.

If file  names  are  divided  into  two
parts:  an initial part or ``stem'' and a
final part or ``ending'', then two  files
whose  names  have  identical stems are
usually relate in some way.  They  may
reside  on  the  same  disk,  they  may
belong to the same user, etc.
Users make initial reference  to  files
by  quoting  the file name, e.g. in the
``open''  system  call.    An   important
operating  sysem function is to decode
the name into the corresponding ``inode''
entry.   To  do  this, UNIX creates and
maintains a directory  data  structure.
This   structure  is  equivalent  to  a
directed graph with named edges.


In its purest form, the graph is a tree
i.e.  it  has  a single root node, with
exactly one path between the  root  and
any  node.  More  commonly in UNIX (but
not so commonly in other operating systems) the graph is a lattice  which may
be obtained from a tree  by  coalescing
one or more groups of leaves.

In this case, while there is still only
one path between the root and any interior node, thee may be more  than  one
path  between  the  root  and  a  leaf.
Leaves are nodes without successors and
correspond to  data  files.  Interior
nodes are  nodes  with  successors  and
correspond to directory files.

The name for a file  is  obtained  from
the  names  of  the  edges  of the path
between   the   root   and   the   node
corresponding  to  the file.  (For this
reason, the name is often  referred  to
as  a ``pathname''.) If there are several
paths, then the file has several names.


\sbs{Directory Files}

A directory file is  in  many  respects
indistinguishable  from a non-directory
file.  However it contains  information
which  is  used in locating other files
and hence its  contents  are  carefully
protected,  and  are manipulated by the
operating system alone.


In  every  file,  the  information   is
stored  as  one  or  more 512 character
blocks.  Each block of a directory file
is  divided  into  32  *  16  character
structures.  Each structure consists of
a 16 bit ``inode'' table pointer and a 14
character name.  The ``inode'' pointer is
to  the  ``inode'' table on the same disk
or file  system  volume  as  the  files
which  the  directory references. (More
on this later.)  An  ``inode''  value  of
zero defines a null entry in the directory.

The procedures which  reference  directories are:

\begin{verbatim}
   namei  (7518) search directory
   link   (5909) create alternate name
   wdir   (7477) write directory entry
   unlink (3510) delete name
\end{verbatim}

\sbs{namei (7518)}

\bd
\item[7531:] ``u.u\_cdir'' defines the ``inode'' of
      a process's current directory.  A
      process  inherits  its   parent's
      current    directory   at   birth
      (``newproc'', 1883).   The  current
      directory  may  be  changed using
      the ``chdir'' (3538) system call;
\item[7532:] Note that ``func'' is  a  parameter
      to  ``namei''  and is always either
      ``uchar'' (7689) or ``schar'' (7679);
\item[7534:] ``iget'' (7276) is called to:
\bi
\item wait  until  such  time  as   the
``inode''  corresponding to ``dp'' is
no longer locked;

\item check that  the  associated  file system is still mounted;
\item increment the reference count;
\item lock the ``inode'';
\ei

\item[7535:] Multiple slashes are  acceptable! (i.e.\\
 ``////a///b/'' is the same as ``/a/b'');

\item[7537:] Any attempt to replace or  delete
      the  current working directory or
      the  root  directory  is  bounced
      immediately!

\item[7542:] The  label  ``cloop''   marks   the
      beginning  of a program loop that
      extends to line 7667.  Each cycle
      analyses a component of the pathname (i.e. a string terminated by
      a  null  character or one or more
      slashes).  Note that a  name  may
      be  constructed  from  many  different characters (7571);

\item[7550:] The end of the pathname has  been
      reached  (successfully).   Return
      the current value of ``dp'';

\item[7563:] ``search''  permission  for  directories  is  coded in the same way
      as ``execute'' permission for other files;

\item[7570:] Copy the name into a more  accessible  location before attempting
      to  match  it  with  a  directory
      entry.    Note  that  a  name  of
      greater than ``DIRSIZ''  characters
      is truncated;

\item[7589:] ``u.u\_count'' is set to the  number
      of entries in the directory;

\item[7592:] The  label  ``eloop''   marks   the
      beginning of a program loop which
      extends to line 7647. Each cycle
      of the loop handles a single
      directory entry;

\item[7600:] If the directory has been
      searched (linearly!) without
      matching the supplied pathname
      component, then there must be an
      error unless:

\bd
     \item[(a)] this is the last component of
      the pathname, i.e. ``c==`\verb+\+0' '';

      \item[(b)] the file is to be created,
      i.e. ``flag == 1''; ard

      \item[(c)] the user program has ``write''
      permission for the directory;
\ed

\item[7606:] Record the ``inode'' address for
      the directory for the new file in
      ``u.u\_pdir'';

\item[7607:] If a suitable slot for a new
      directory entry has previously
      been encountered (7642), store
      the value in ``u.u\_offset[1]'';
      else set the ``IUPD'' fIag for the
      ``dp'' designated ``inode'' (but why?);

\item[7622:] When appropriate, read a new
      block from the directory file
      (note the use of ``bread'') (why
      not ``breada''?), after carefully
      releasing any previously held
      buffer;

\item[7636:] Copy the eight words of the
      directory entry into the array
      ``u.u\_dent''. The reason for copying before comparing is obscure!
      Can this actually be more efficient? (The reason for copying
      the whole directory at all is
      rather perplexing to the author
      of these notes.);

\item[7645:] This comparison makes efficient
      use of a single character pointer
      register variable, ``cp''. The
      loop would be even more efficient
      if word by word comparison were
      used;

\item[7647:] The ``eloop'' cycle is terminated
      by one of:

\begin{verbatim}
   return(NULL);    (7610)
   goto out;        (7605, 7613)
\end{verbatim}

\noindent a successful match  so  that  the
branch  to  ``eloop'' (7647) is not
taken;

\item [7657:] If the  name  is  to  be  deleted
(``flag==2''),  if the pathname has
been completed, and if  the  user
program has ``write'' access to the
directory, then return a  pointer
to the directory ``inode'';

Save  the  device  identity  temporarily (why not in the register
``c''?) and call ``iput''  (7344)  to
unlock  ``dp'',  to  decrement  the
reference count on  ``dp''  and  to
perform  any  consequent processlng;

\item[7664:] Revalidate ``dp'' to point  to  the
      ``inode'' for the next level file;

\item[7665:] ``dp==NULL''   shouldn't    happen,
since the directory says the file
exists!  However  ``inode''   table
overflows   and  i/o  errors  can
occur,  and  sometimes  the  file
system  may  be left in an inconsistent  state  after  a   system
crash.
\ed

\sbs{Some Comments}


``namei'' is a key procedure which  would
seem  to  have been written very early,
to have been  thoroughly  debugged  and
then  to  have  been  left  essentially
unchanged.   The   interface    between
``namei''  and  the rest of the system is
rather complex,  and  for  that  reason
alone,  it  would not win the prize for
``Procedure of the Year''.


``namei'' is  called  thirteen  times  by
twelve different procedures:

\begin{tabular}{llll}\\
{\bf line} & {\bf routine} & \multicolumn{2}{l}{\bf parameters} \\ \hline
3034 & exec & uchar & 0 \\
3543 & chdir & uchar & 0 \\
5770 & open & uchar & 0 \\
5914 & link & uchar & 0 \\
6033 & stat & uchar & 0 \\
6097 & smount & uchar & 0 \\
6186 & getmdev & uchar & 0 \\
6976 & owner & uchar & 0 \\
\\
5786 & creat & uchar & 1 \\
5928 & link & uchar & 1 \\
5958 & mknod & uchar & 1 \\
\\
3515 & unlink & uchar & 2 \\
\\
4101 & core & schar & 1 \\
\end{tabular}

\bigskip

\noindent It will be seen that:
\bd
\item[(a)] there are two calls from ``link'';

\item [(b)] the calls can  be  divided  into
four  categories,  of  which the
first is by far the largest;

\item[(c)] the  last  two  categories  have
only one representative each;

\item[(d)] in particular, there is only one
call   involving   the   routine
``schar'', which is always  for  a
file  called  ``core''.  (If  this
case were handled as  a  special
case   e.g.   where  the  second
parameter  had  the  value  ``3'',
then  the  ``uchar''s  and ``schar''
could be eliminated.)
\ed


\noindent ``namei'' may terminate in a  variety  of
ways:

\bd
\item[(a)]  if there has been an error, then
     a  ``NULL''  value is returned and
     the variable ``u.u\_error'' is set.

(Most errors result in a  branch
to  the  label  ``out''  (7669) so
that reference  counts  for  the
inodes are properly maintained
(7670). This is not necessary if
the  failure  occurs  in  ``iget''
(7664).);



\item[(b)]  if ``flag==2'' (i.e. the  call  is
     from    ``unlink''),   the   value
     returned   (in    normal    circumstances)    is   an   ``inode''
     pointer for the parent directory
     of the named file (7660);

\item[(c)]  if ``flag==1'' (i.e. the  call  is
     from   ``creat''   or   ``link''  or
     ``mknod'', and a  file  is  to  be
     created  if  it does not already
     exist) and  if  the  named  file
     does  not  exist,  then a ``NULL''
     value  is  returned  (7610).  In
     this   case  a  pointer  to  the
     ``inode'' for the directory  which
     will  point  to the new file, is
     left in ``u.u\_pdir'' (7606). (Note
     also    that   in   this   case,
     ``u.u\_offset''  is  left  pointing
     either  at  an  empty  directory
     entry  or  at  the  end  of  the
     directory file.);

\item[(d)] if  in the remaining cases,  the
file  exists, an ``inode'' pointer
for the file is returned (7551).
The  ``inode''  is  locked and the
reference count has been  incremented.   A  call  to  ``iput'' is
needed subsequently to undo both
these side effects.
\ed



\sbs{link (5909)}

This procedure implements a system call
which enters a new name for an existing
file  into  the  directory   structure.
Arguments  to  the  procedure  are  the
existing and the new names of the file;

\bd
\item[5914:] Look up the existing file name;

\item[5917:] If the file already has 127  different names, quit in disgust;

\item[5921:] If the existing file turns out to
      be  a  directory,  then  only the
      super-user may rename it;

\item[5926:] Unlock the existing file  ``inode''
      This  is  locked  when  the first
      call on ``namei''  does  an  ``iget''
      (7534, 7664).

Under what conditions  would  the
failure  to  unlock  the  ``inode''
here be disastrous?  The  chances
that the existing file would be a
directory  encountered   in   the
search  for  the  new  name would
seem slight, if  not  impossible.
Most  probably  the relevant circumstance is where the system  is
attempting  to recreate an alternative file name or alias,  which
{\bf already} exists;

\item[5927:] Search  the  directory  for   the
      second  name,  with the intention
      of creating a new entry;

\item[5930:] There is an  existing  file  with
      the second name;

\item[5935:] ``u.u\_pdir'' is set as a side effect
of  the  call  on ``namei'' (5928).
Check that the directory  resides
on the same device as the file;

\item[5940:] Write a new directory entry  (see
      below);

\item[5941:] Increase the ``link'' count for the
      file.
\ed


\sbs{wdir (7477)}

This procedure enters a new name into a
directory.   It  is  called  by  ``link''
(5940)  and  ``maknode''  (7467)  with  a
pointer  to a (core) ``inode'' as parameter.


The sixteen characters of the directory
entry  are  copied  into  the structure
``u.u\_dent'', and written from there into
the directory file. (Note that the previous content of ``u.u\_dent''  will  have
been  the name of the last entry in the
directory file.)


The procedure assumes that  the  directory  file  has  already been searched,
that the ``inode'' for the dlrectory file
has already been allocated and that the
values of ``u.u offset''  have  been  set
appropriately.

\sbs{maknode (7455)}

This procedure is  called  from  ``core''
(4105),   ``creat''  (5790)  and  ``mknod''
(5966),  after  a  previous   call   on
``namei'' with a second parameter of one,
has revealed that no file of the specified name existed.


\sbs{unlink (3510)}

This procedure implements a system call
which  deletes  a  file  name  from the
directory structure.  (When all  references  to  a file are deleted, the file
itself will be deleted.)

\bd
\item[3515:] Search for a file with the specified
 name,  and  if  it  exists,
return a pointer to  the  ``inode''
of  the  immediate  parent directory;


\item[3518:] Unlock the parent directory;

\item[3519:] Get an  ``inode''  pointer  to  the
file itself;

\item[3522:] Unlinking directories is  forbidden, except for super-users;

\item[3528:] Rewrite the directory entry  with
      the ``inode'' value set to zero;

\item[3529:] Decrement the ``link'' count.
\ed


Note that there is no attempt to reduce
the size of a directory below its ``high
water'' mark.


\sbs{mknod (5952)}

This procedure, which implements a system call
of the same name, is only executable by the super-user. As explained
in  the Section ``MKNOD(II)'' of the UPM,
this system  call  is  used  to  create
``inodes'' for special files.

``mknod''  also  solves  the  problem  of
``where  do directories come from''?  The
second parameter passed to  ``mknod''  is
used,  without modification or restriction to set ``i\_mode''.
(Compare  ``creat'' (5790)  and  ``chmod''  (3569)).  This is
the only way an ``inode'' can get flagged
as a directory, for instance.

In  such  cases,  the  third  parameter
passed  to  ``mknod'' {\bf must} be zero.  This
value is copied into ``i\_addr[0]'' (as is
appropriate for special files), and, if
non-zero, will be accepted uncritically
by  ``bmap''  (6447). It might be prudent
to insert a test

\begin{verbatim}
   if (ip->i_mode & (IFCHR & IFBLK) != 0)
\end{verbatim}

\noindent before  line  5969,  rather  than  rely
indefinitely  on  the  infallibility of
the super-user.


\sbs{access (6746)}

This  procedure  is  called  by  ``exec''
(3041),  ``chdir'' (3552), ``core'' (4109),
``openl'' (5815,  5817),  ``namei''  (7563,
7664,  7658) to check access permission
to  a  file.  The   second   parameter,
``mode'',  is  equal  to  one of ``IEXEC'',
``IWRITE'' and ``IREAD'', with octal values
of 0100, 0200 and 0400 respectively.

\bd
\item[6753:] ``write'' permission is  denied  if
      the  file  is  on  a  file system
      volume which has been mounted  as
      ``read  only''  or  if  the file is
      functioning as the  text  segment
      for an executing program;

\item[6763:] the suer-user may not execute  a
file unless it is ``executable'' in
at least one of the  three  ``permission''  groups.  In  any  other
situation he  is  always  allowed
access;

\item[6769:] If the user is not the  owner  of
      the  file, shift ``m'' three places
      to the right so that  group  permissions will be operative ... If
      the groups don't match, shift ``m''
      again;

\item[6774:] Compare ``m'' and the  access  permissions.
\ed


Note that there is an anomaly  here  in
that  if  a  file has a ``mode'' of 0077,
the owner cannot reference it  at  all,
but  everyone  else can. This situation
could  be  changed  satisfactorily   by
inserting a statement

\begin{verbatim}
   m =|  (m |  (m >> 3)) >> 3;
\end{verbatim}

\noindent after line 6752,  and  replacing  lines
6764, 6765 by

\begin{verbatim}
   if (m & IEXEC && (m & ip->i_mode) == 0)
\end{verbatim}

%
% The Lion's Commentary, file ch20.tex, version 1.3, 15 May 1994
%
\se{File  Systems}

In most computer systems more than  one
peripheral  storage  device is used for
the storage of files. It is now  necessary  to  discuss  a  number of matters
pertaining to the management by UNIX of
the whole set of files and file storage
devices. First, some definitions:

\bd
\item[file system:] an integrated collection  of files with a hierarchical
system of directories recorded  on
a  single  block  oriented storage
device;

\item[storage device:] a device which can
store information (especially disk
pack or DECtape, etc.);

\item[access  device:]  a  mechanism  for
transferring   information  to  or
from a storage device;

\item[a  storage   device]    is    only
{\bf accessible} if it is inserted in an
access device.  In this situation
reference to the storage device is
made via a reference to the access
devce;

\item[a storage device] is acceptable  as
a {\bf file system volume} if:

\bd
\item[(a)]  information   is   recorded   as
     addressable  blocks of 512 characters  each,   which   can   be
     independently read or written.

(Note  IBM  compatible  magnetic
tape  does not satisfy this condition.);

\item[(b)] the information recorded  on  the
     device  satisfies  certain  consistency criteria:

block \#1 is formatted as a ``super block'' (see below);

blocks \#2 to \#(n+1)  (where n  is
recorded  in  the ``super block'')
contain an ``inode''  table  which
references all files recorded on
the storage device, and does not
reference any other files;

directory files recorded on  the
storage  device  reference  all,
and  only,  files  on  the  same
storage device, i.e. a file system volume
constitutes  a  self-contained  set  of files,
directories and ``inode'' table;
\ed

\item[a file system volume] is mounted if
the presence of the storage device
in an access device has been  formally  recognised by the operating
system.
\ed


\sbs{The `Super Block'  (5561)}

The ``super block'' is always recorded as
block  \#l on the storage device. (Block
\#0 is always ignored and  is  available
for  miscellaneous uses not necessarily
concerned with UNIX.)

The ``super block'' contains  information
used  in allocating resources, viz. the
storage blocks and the entries  in  the
``inode'' table recorded on the file system. While the file  system  volume  is
mounted  a copy of the ``super block'' is
maintained in core and  updated  there.
To  prevent  the  storage  device  copy
becoming too far out of date, its  contents are written out at regular intervals.


\sbs{The 'mount' table (0272)}

The ``mount'' table  contained  an  entry
for  each  mounted  file system volume.
Each entry defines the device on  which
the  file  system  volume is mounted, a
pointer to the buffer which stores  the
``super  block''  for  the device, and an
``inode'' pointer.  The table  is  referenced as follows:

\bd
\item[iinit (6922)] which  is  called  by
``main''  (1615), makes an entry for
the root device;

\item[smount (6086)]  is  a  system  call
which makes entries for additional
devices;

\item[iget (7276)] searches  the  ``mount''
table  if it encounters an ``inode''
with the `IMOUNT' flag set;

\item[getfs (7167)] searches the  ``mount''
table to find and return a pointer
to the ``super block'' for a particular device;

\item[update (7201)] is  called  periodically  and  searches  the  ``mount''
table to locate information  which
should be written from core tables
into the tables maintained on  the
file system volumes;

\item[sumount (6144)] is  a  system  call
which  deletes  entries  from  the
table.
\ed

\sbs{iinit (6922)}

This routine is called by ``main'' (1615)
to initialise the ``mount'' table entry
for the root device.

\bd
\item[6926:] Call the ``open'' routine for the
      root device. Note that ``rootdev''
      is defined in ``conf.c'' (4695);

\item[6931:] Copy the contents of the root
      device ``super block'' into a
      buffer area not associated with
      any particular device;

\item[6933:] The zeroeth entry in the ``mount''
      table is assigned to the root
      device. Only two of the three
      elements are explicitly initialised. The third, the ``inode''
      pointer, will never be referenced;

\item[6936:] The ``locks'' stored in the ``super
      block'' are explicitly reset.
      (These locks may have been set
      when the ``super block'' was last
      written onto the file system
      volume);

\item[6938:] The root device is mounted
      in a ``writable'' state;

\item[6939:] The system sets its idea of the
      current time and date from the
      time recorded in the ``super
      block''. (If the system has been
      stopped for an appreciable
      period, the computer operator
      will need to reset the contents
      of ``time''.)
\ed

\sbs{Mounting}

From an operational view point, ``mounting''  a  file  system  volume  involves
placing it in a suitable access device,
readying  the device, and then entering
a command such as
parameters.

\begin{verbatim}
    ``/etc/mount /dev/rk2 /rk2''
\end{verbatim}

\noindent to the ``shell'', which forks  a  program
to perform a ``mount'' system call, passing pointers to the two file  names  as
parameters.


\sbs{smount (6086)}

\bd
\item[6093:] ``getmdev'' decodes the first argument  to  locate a block oriented
      access device;

\item[6096:] ``u.u\_dirp'' is  reset  preparatory
to  calling ``namei'' to decode the
second  file  name.   (Note   that
``u.u\_dirp''  is  set  by ``trap'' to
``u.u\_arg[0]'' (2770);

\item[6100:] Check that the file named by  the
      second  parameter  is in a satisfactory condition,  i.e.  no  one
      else  is  currently accessing the
      file, and that the file is not  a
      special  file  (block  or character);

\item[6103:] Search the ``mount'' table  looking
  for      an      empty      entry
  (``mp-$>$m bufp==NULL'') or an  entry
  already   made  for  the  device.
  (The ``mount''  data  structure  is
  defined at line 0272);


\item[6111:] ``smp'' should point to a  suitable
entry in the ``mount'' table;

\item[6113:] Perform  the  appropriate  ``open''
routine, with the device name and
a read/write flag  as  arguments.
(As  was  seen  earlier,  for the
RK05 disk the ``open'' routine is a
``no-op'');


\item[6116:] Read block \#1 from the device.
This block is the ``super block'';

\item[6124:] Copy the ``super block'' into a
buffer associated with ``NODEV'',
from the buffer associated with
``d''. The second buffer will not
 be released again until the device is unmounted;

\item[6130:] ``ip'' points to  the  ``inode''  for
the   second  named  file.   This
``inode''   is   now   flagged   as
``IMOUNT''.  The  effect of this is
to force ``iget'' (7292) to  ignore
the  normal contents of the file,
while the file system  volume  is
mounted. (In practice, the second
file is  an  empty  file  created
especially for this purpose.)
\ed


\sbs{Notes}

\bd
\item[1.] The ``read/write'' status of a mounted
device  depends  only on the parameters
provided to ``smount''.   No  attempt  is
made to sense the hardware ``read/write''
status. Thus if a disk is readied  with
``write  protect'' on, but is not mounted
``read only'', then the system will  complain vigorously.

\item[2.] The ``mount'' procedure does not carry
out  any  kind of label checking on the
``mounted'' file system volume.  This  is
reasonable  in  a  situation where file
system volumes are  rarely  rearranged.
However in situations where volumes are
mounted and remounted frequently,  some
means  of  verifying  that  the correct
volume  has  been  mounted  would  seem
desirable.   (Further,  if a file system
volume contains sensitive  information,
it  may  be  desirable  to include some
form of password  protection  as  well.
There  is  room  in  the  ``super block''
(5575) for the storage of a name and an
encrypted password.)
\ed

\sbs{iget (7276)}

This  procedure  is  called  by  ``main''
(1616,1618),  ``unlink'' (3519), ``ialloc''
(7078) and ``namei''  (7534,  7664)  with
two  parameters which together uniquely
identify a  file:  a  device,  and  the
``inode'' number of a file on the device.
``iget'' returns a reference to an  entry
in the core ``inode'' table.


When ``iget'' is called, the core ``inode''
table  is  searched  first to see if an
entry already exists for  the  file  in
the  core  ``inode''  table. If not, then
``iget'' creates one.

\bd
\item[7285:] Search the core ``inode'' table ...

\item[7286:] If an entry  for  the  designated
file already exists ...

\item[7287:] Then if it is locked go to sleep;


\item[7290:] Try again. (Note the whole  table
  needs  to  be searched again from
  the beginning, because the  entry
  may have vanished!);

\item[7292:] If the IMOUNT flag  is  on  ...
this  is an important possibility
for which we will delay the  discussion;

\item[7302:] If the ``IMOUNT'' flag is not  set,
increase  the  ``inode''  reference
count, set the ``ILOCK''  flag  and
return a pointer to the ``inode'';


\item[7306:] Make a note of  the  first  empty
      slot in the ``inode'' table;

\item[7309:] If the  ``inode''  table  is  full,
      send  a  message to the operator,
      and take an error exit;

\item[7314:] At this point, a new entry is  to
      be made in the ``inode'' table;

\item[7319:] Read the block which contains the
      file  system volume ``inode''. Note
      the use  of  ``bread''  instead  of
      ``readi'',   the   assumption  that
      ``inode''  information  begins   in
      block \#2  and the convention that
      valid ``inode''  numbers  begin  at
      one (not zero);

\item[7326:] A read error at this point isn't
      very well reported to the rest of       
      the system; 

\item[7328:] Copy the relevant ``inode'' 
information.  This code makes implicit
      use of the contents of  the  file
      ``ino.h''  (Sheet  56), which isn't
      referenced explicitly anywhere.
\ed

\noindent Let us now return to  unfinished  business:

\bd
\item[7292:] The ``IMOUNT'' flag is found to  be
      set.   This   flag   was  set  by
``smount'',  when  a  file   system
volume was mounted;

\item[7293:] Search the ``mount'' table to  find
      the  entry  which  points  to the
      curent    ``inode''.     (Although
      searching  this  table  is  not a
      horrendous overhead, it does seem
      possible  that  a  ``back pointer''
      could be conveniently  stored  in
      in   the   ``inode''  e.g.  in  the
      ``i\_lastr'' field. This would  save
      both time and code space.;

\item[7396:] Reset  ``dev''  and  ``ino''  to  the
      mounted  device  number  and  the
      ``inode'' number of the root directory  on  the mounted file system
      volume.  Start again.
\ed


Clearly,  since  ``iget''  is  called  by
``namei''  (7534,  7664),  this technique
allows the whole directory structure on
the  mounted  file  system volume to be
integrated into the pre-existing directory   structure.   If  we  momentarily
ignore  the  possible   deviations   of
directory  structures  away  from  tree
structures, we have the situation where
a  leaf  of  the existing tree is being
replaced by an entire subtree.


\sbs{getfs (7167)}

There is little that needs to be said
 about this procedure in addition to the
author's comment. This procedure is called by

\begin{verbatim}
   access   (6754)     ialloc   (7072)
   alloc    (6961)     ifree    (7138)
   free     (7004)     iupdat   (7383)
\end{verbatim}


Note the  cunning  use  of  ``n1'',  ``n2''
which   are   declared   as   character
pointers  i.e.  as  unsigned  integers.
This allows only one sided tests on the
two variables at line 7177.

\sbs{update (7201)}

The function of this procedure, in  its
broadest   terms,  is  to  ensure  that
information on the file system  volumes
is  kept  up  to date.  The comment for
this procedure (beginning on line 7190)
describes the three main sub-functions,
(in the reverse order!).

``update'' is the whole business  of  the
``sync''  system call (3486). This may be
invoked via the ``sync''  shell  command.
Alternatively  there is a standard system program
which runs continuously and
whose  only  function is to call ``sync''
every 30 seconds.  (See  ``UPDATE(VIII)''
in the UPM.)


``update'' is called by ``sumount''  (6150)
before   a   file   system   volume  is
unmounted, and by ``panic'' (2420) as the
last   action   of  the  system  before
activity ceases.

\bd
\item[7207:] If another execution of  ``update''
      is under way, then just return;

\item[7210:] Search the ``mount'' table;

\item[7211:] For each mounted volume, ...

\item[7213:] Unless the file  system  has  not
      been  recently  modified  or  the
      ``super block'' is  locked  or  the
      volume  has  been  mounted  ``read
      only'' ...

\item[7217:] Update the ``super block'', copy it
      into   a  buffer  and  write  the
      buffer out onto the volume;

\item[7223:] Search the ``inode'' table, and for
      each  non-null  entry,  lock  the
      entry and call ``iupdat'' to update
      the  ``inode''  entry on the volume
      if appropriate;

\item[7229:] Allow  additional  executions  of
      ``update'' to commence;

\item[7230:] ``bflush'' (5229)  forces  out  any
      ``delayed write'' blocks.
\ed

\sbs{sumount (6144)}

This system call deletes an entry for a
mounted  device from the ``mount'' table.
The purpose of this call is  to  ensure
that  traffic to and from the device is
terminated properly, before the storage
device  is  physically removed from the
access device.

\bd
\item[6154:] Search the ``mount'' table for  the
      appropriate entry;

\item[6161:] Search the ``inode'' table for  any
      outstanding  entries for files on
      the device. If  any  such  exist,
      take  an  error  exit, and do not
      change the ``mount'' table entry;

\item[6168:] Clear the ``IMOUNT'' flag.
\ed


\sbs{Resource Allocation}

Our attention now turns to the  management  of the resources of an individual
FSV (file system volume).


Storage blocks are allocated  from  the
free  list by ``alloc'' at the request of
``bmap''.  Storage blocks are returned to
the  free  list by ``free'' at the behest
of ``itrunc'' (which is called by ``core'',
``openl'' and ``iput'').


Entries in the FSV ``inode''  tables  are
made  by  ``ialloc'',  which is called by
``maknode'' and ``pipe''. Entries  in  this
table  are  cancelled by ``ifree'', which
is called by ``iput''.


The ``super block'' for the FSV  is  central  to  the  resource management procedures.  The ``super block'' (5561) contains:

\bi
\item size  information (total resources available);

\item list of up to 100  available  storage blocks;

\item list of up to 100  available  ``inode'' entries;

\item locks to control manipulation of  the above lists;

\item flags;

\item current date of last update.
\ei


If the list in core of available
``inode'' entries for the file system
volume ever becomes exhausted, then the
entire table on the FSV is read and
searched to rebuild the list. Conversely if the available ``inode'' table
overflows, additional entries are simply forgotten to be rediscovered later.


A different strategy is used for the
list of available storage blocks.
These blocks are arranged in groups of
up to one hundred blocks. The first
block in each group (except the very
first) is used to store the addresses
of the blocks belonging to the previous
group. Addresses of blocks in the last
incomplete group are stored in the
``super block''.

The first entry in the first list of
block numbers is zero, which acts as a
sentinel. Since the whole list is subject to a LIFO discipline, discovery of
a block number of zero in the list signifies that the list is in fact empty.


\sbs{alloc (6965)}

This is called by ``bmap'' (6435, 6448,
6468, 6480, 6497) whenever a new
storage block is needed to store part
of a file.

\bd
\item[6961:] Convert knowledge of the device
      name into a pointer to the ``super
      block'';

\item[6962:] If ``s\_flock'' is set, the list of
      available blocks is currently
      being updated by another process;

\item[6967:] Obtain the block  number  of  the
      next available storage block;

\item[6968:] If the last block number  on  the
      list  is zero, the entire list is
      now empty;

\item[6970:] ``badblock''  (7040)  is  used   to
      check   that   the  block  number
      obtained from the list seems reasonable;

\item[6971:] If the list of  available  blocks
      in   the  ``super  block''  is  now
      empty,  then   the   block   just
      located    will    contain    the
      addresses of the  next  group  of

\item[6972:] Set ``s\_flock'' to delay any  other
  process from getting a ``no space''
  indication  before  the  list  of
  available  blocks  in  the ``super
  block'' can be replenished;

\item[6975:] Determine  the  number  of  valid
entries in the list to be copied;

\item[6978:] Reset  ``s\_flock'',  and   ``wakeup''
      anyone waiting;

\item[6982:] Clear  the  buffer  so  that  any
      information  recorded in the file
      by default will be all zeros;

\item[6983:] Set the ``modified'' flag to ensure
      that  the  ``super  block'' will be
      written out by ``update'' (7213).
\ed

\sbs{itrunc (7414)}

This  procedure  is  called  by  ``core''
(4112),   ``openl''   (5825)  and  ``iput''
(7353). In the  first  two  cases,  the
contents  of the ``file'' are about to be
replaced.  In the third case, the  file
is about to be abandoned.

\bd
\item[7421:] If the file  is  a  character  or
      block  special file then there is
      nothing to do;

\item[7423:] Search  backwards  the  list   of
      block   numbers   stored  in  the
      ``inode'';

\item[7425:] If the file is large,  then  an
indirect fetch is needed. (A double indirect fetch is needed  for
blocks    numbered    seven   and
higher.);


\item[7427:] Reference all {\bf 257} elements of the
  buffer  in  reverse  order. (Note
  this seems to be the  only  place
  where  characters  \#512,  \#513 of
  the buffer area  are  referenced.
  Since  they  will presumably contain zero, they  will  contribute
  nothing to the calculation. Hence
  if  ``510''  were  substituted  for
  ``512''  here,  and  again  on line
  7432, a general  improvement  all
  round would result (?));


\item[7438:] ``free''  returns   an   individual
      block to the available list;


\item[7439:] This is  the  end  of  the  ``for''
      statement   commencing   on  line
      7427.   (Likewise  the  statement
      which  begins  at  7432  ends  at
      7435.);


\item[7443:] Clear the entry in ``i\_addr[ ]'';


\item[7445:] Reset size information, and  flag
      the ``inode'' as ``updated''.
\ed



\sbs{free (7000)}


This procedure is  called  by  ``itrunc''
(7435, 7438, 7442) to reinsert a simple
storage block into the  available  list
for a device.

\bd
\item[7005:] It is not clear why the ``s\_fmod'' flag
is set here as well as at the end of  the  procedure  (line
7026). Any suggestions?

\item[7006:] Observe the locking protocol;


\item[7010:] If  no  free  blocks   previously
      existed  for  the device, restore
      the situation by setting up a one
      element  list containing an entry
      for block \#0.   This  value  will
      subsequently be interpreted as an
      ``end of list'' sentinel;

\item[7014:] If  the  available  list  in  the
      ``super block'' is already full, it
      is time to write it out onto  the
      FSV. Set ``s\_flock'';


\item[7016:] Get a buffer, associated with the
      block  now  being  entered in the
	free list;

\item[7019:] Copy the contents  of  the  super
      block  list,  preceded by a count
      of the number  of  valid  blocks,
      into   the   buffer;   write  the
      buffer;  unset   the   lock   and
      ``wakeup'' anybody waiting,


\item[7025:] Add the  returned  block  to  the
      available list.
\ed


\sbs{iput (7344)}


This procedure is one of the most popular  in UNIX (called from nearly thirty
different places) and its use will have
already been frequently observed.


In essence  it  simply  decrements  the
reference  count for the ``inode'' passed
as a parameter, and then calls  ``prele''
(7882) to reset the ``inode'' lock and to
perform any necessary ``wakeup''s.


``iput'' has an important side effect. If
the  reference  count  is  going  to be
reduced to  zero,  then  a  release  of
resources  is  indicated.  This  may be
simply the core ``inode'', or  both  that
and  the  file itself, if the number of
links is also zero.



\sbs{ifree (7134)}


This  procedure  is  called  by  ``iput''
(7355)  to  return a FSV ``inode'' to the
available list maintained in the ``super
block''.  If  this  list is already full
(as noted above)  or  if  the  list  is
locked  (using  ``s\_ilock'') the information is simply discarded.

\sbs{iupdat (7374)}


This procedure  is  called  by ``statl''
(6050),   ``update''  (7226)  and  ``iput''
(7357) to revise a  particular  ``inode''
entry  on a FSV. It does nothing if the
corresponding  core  ``inode''   is   not
flagged (``IUPD'' or ``IACC'');


The ``IUPD'' flag may be set by one of

\begin{verbatim}
  unlink (3530)        bmap (6452,6467)
  chmod  (3570)        itrunc  (7448)
  chown  (3583)        maknode (7462)
  link   (5942)        namei   (7609)
  writei (6285,6318)   pipe    (7751)
\end{verbatim}

\noindent The ``IACC'' flag may be set by one of

\begin{verbatim}
  readi  (6232)       maknode (7462)
  writei (6285)       pipe    (7751)
\end{verbatim}


The flags are reset by ``iput'' (7359).

\bd
\item[7383:] Forget it, if the  FSV  has  been
mounted as ``read only'';


\item[7386:] Read the appropriate  block  containing  the  FSV  ``inode'' entry.
      As observed earlier with  respect
      to  ``iget'',  note  the the use of
      ``bread'' instead of  ``readi'',  the
      assumption that the ``inode'' table
      begins at block \#2 and  the  convention    that   valid   ``inode''
      numbers begin at one;


\item[7389:] Copy  the  relevant   information
      from the core ``inode'';


\item[7391:] If appropriate, update  the  time
      of last access;


\item[7396:] If appropriate, update  the  time
      of last modification;


\item[7400:] Write the updated block  back  to
      the FSV.
\ed

%
% The Lion's Commentary, file ch21.tex, version 1.3, 15 May 1994
%
\se{Pipes}

A ``pipe''  is  a  FIFO  character  list,
which is managed by UNIX as yet another
variety of file.


One group of processes may ``write'' into
a  ``pipe''  and another group may ``read''
from the same ``pipe''. Hence ``pipe''s may
be,  and are used, primarily for interprocess communication.

By  exploiting   the   concept   of   a
``filter'',  which  is  a  program  which
reads an input file and  transforms  it
into  an  output  file,  and  by  using
``pipes'' to link two or more programs of
this  type  together,  UNIX  offers its
users a surprisingly comprehensive  and
sophisticated set of facilities.


\sbs{pipe (7723)}

A ``pipe'' is created as a result of a 
system call on the ``pipe'' procedure.

\bd
\item[7728:] Allocate an ``inode'' for the  root device;

\item[7731:] Allocate a ``file'' table entry;

\item[7736:] Remember the ``file''  table  entry
as  ``r''  and  allocate  a  second
``file'' table entry;

\item[7744:] Return user file  identifications
      in R0 and R1;

\item[7746:] Complete  the  entries   in   the
      ``file''   array  and  the  ``inode''
      entry.
\ed


\sbs{readp (7758)}

``pipes'' are different from other  files
in  that  two separate offsets into the
file are kept -- one for  ``read''  operations 
and  one for ``write'' operations.
The ``write'' offset is actually the same
as the file size.

\bd
\item[7763:] the parameter passed  to  ``readp''
      is  a  pointer  to a ``file'' array
      entry,  from  which  an   ``inode''
      pointer can be extracted;

\item[7768:] ``plock'' (7862) ensures that  only
      one  operation  takes  place at a
      time: either ``read'' or ``write'';

\item[7776:] If a process wishing to write  to
      a ``pipe'' has been blocked because
      the pipe was  ``full''  (or  rather
      because  the  valid  part  of the
      file had reached the file limit),
      it will have signified its predicament by  setting  the  ``IWRITE''
      flag in ``ip-$>$i\_mode'';

\item[7786:] Release the lock before going  to
      sleep;

\item[7787:] ``i\_count'' is the number  of  file
      table  entries  pointing  at  the
      ``inode''. If  this  is  less  than
      two,  then the group of ``writers''
      must be extinct;

\item[7789:] A process waiting for input  will
raise  the ``IREAD'' flag.  Since a
pipe cannot  be  full  and  empty
simultaneously,  no more than one
of the flaqs ``IWRITE'' or  ``IREAD''
should be set at any one time;

\item[7799:] ``prele''  unlocks  the  file   and
      ``wakes  up''  any  process waiting
      for the pipe.
\ed

\sbs{writep (7805)}

The structure of this procedure  echoes
that of ``readp'' in many respects.

\bd
\item[7828:] Note that a ``writer'', which finds
      that  there are no more ``readers''
      left, receives a ``signal'' just in
      case  he  is  not  monitoring the
      result of his ``write'' operation.

(A  ``reader''  in  the   analogous
situation receives a zero character count as the  result  of  the
read,  and  this  is the standard
end-of-file indication.)

\item[7835:] The ``pipe'' size is not allowed to
      grow  beyond ``PIPSIZ'' characters.
      As long as ``PIPSIZ'' (7715) is  no
      greater  than 4096, the file will
      not be  converted  to  a  ``large''
      file.  This  is  highly desirable
      from  the  viewpoint  of   access
      efficiency.

(Note that  ``PIPSIZ''  limits  the
``write''  offset pointer value. If
the ``read'' offset pointer is  not
far  behind,  the true content of
the ``pipe'' may be quite small).
\ed



\sbs{plock (7862)}

Lock  the  ``inode''  after  waiting   if
necessary.  This procedure is called by
``readp'' (7768) and ``writep'' (7815).


\sbs{prele (7882)}

Unlock the ``inode'' and ``wake''
any waiting processes. This procedure is called
by several others (especially  ``iput''),
in addition to ``readp'' and ``writep''.

%
% The Lion's Commentary, file ch22.tex, version 1.2, 15 May 1994
%
{\sf Section Five is the final section: last
but not least. It is concerned with
input/output for the slower, character
oriented peripheral deviees.

Such devices share a common buffer
pool, which is manipulated by a set of
standard procedures.

The set of character oriented peripheral devices are exemplified by the
following:

\bi
\item KL/DL11 interactive terminal
\item PC11 paper tape reader/punch
\item LP11 line printer.
\ei
}

\se{Character Oriented Special Files}

Character oriented peripheral devices
are relatively slow ($<$ 1000 charaeters
per second) and involve character by
character transmission of variable
length, usually short, records.

A device handler (as its name suggests)
is the software part of the interface
between a device and the general system. In general, the device handler is
the only part of the software which
recognises the idiosyncrasies of a particular device.

As far as possible or reasonable, a
single device driver is written to
serve many separate devices of similar
types, and, where appropriate, several
such devices simultaneously. The group
of ``interactive terminals'' (with keyboard input and a serial printer or
visual display output) can just be
coerced with difficulty into a single
device driver, as the reader may judge
during his perusal of the file ``tty.c''.

The standard UNIX device handlers for
character devices make use of the procedures ``putc'' and ``getc'' which store
and retrieve characters into and from a
standard buffer pool. This will be
described in more detail in Chapter
Twenty-Three.

The ``PDP11 Peripherals Handbook'' should
be consulted for more complete information on the device controller hardware
and the devices themselves.

\sbs{LP11 Line Printer Driver}

This driver is to be found in the file
``lp.c'' (Sheets 88, 89). Much of the
complexity of this driver is contained
in the proeedure ``lpcanon'' (8879).
This procedure is involved in the
proper handling of special characters
and this is a separate issue from the
one we wish to study first.

Initially one may ignore ``lpcanon'' by
assuming that all calls upon it (lines
8859, 8865, 8875) are simply replaced
by similar calls upon ``lpoutput''
(8986). ``lpcanon'' acts as a ``final
filter'' for characters going to the
line printer: handling code conversions, special format characters, etc.

\sbs{lpopen (8850)}

When a line printer file is opened, the
normal calling sequence is followed:

``open'' (5774) calls ``open1'',
which (5832) calls ``openi'', which
(6716) calls, in the case of a
character      special      file,
``cdevsw[..].d\_open''. In the case
of the line printer, this latter
translates (4675) to ``lpopen''.

\bd
\item[8853:] Take the error exit if either
another line printer file is
already open, or if the line
printer is not ready (e.g the
power is off, or there is no
paper, or the printer drum gate
is open, or the temperature is
too high, or the operator has
switched the printer off-line.)

\item[8857:] Set the ``lp11.flag'' to indicate
that the file is open, the
printer has a ``form feed'' capability and lines are to be
indented by eight characters.
\ed

\sbs{Notes}

\bd
\item[(a)] ``lp11'' is a seven word structure
defined beginning at line 8829. The
first three words of the structure in
fact constitute a structure of type
``clist'' (7908). Only the first element
is explicitly manipulated in ``lp.c''.
The next two are used implicitly by
``putc'' and ``getc''.

\item[(b)] ``flag'' is the fourth element of
the structure. The remaining three elements are

\begin{tabular}{ll} \\
``mcc'' &      maximum character count\\
``ccc'' &     current character count\\
``mlc'' &     maximum line count\\
\end{tabular}

\bigskip

\item[(c)] The line printer controller has
two registers on the UNIBUS.
\ed

\noindent {\large \bf Line Printer Status Register (``lpsr'')}

\bd
\item[bit 15] Set when an error condition
exists (see above);

\item[bit 7 ``DONE''] Set when the printer
controller is ready to receive the next character;

\item[bit 6 ``IENABLE''] Set to allow ``DONE'' or ``Error''
to cause an interrupt;
\ed

\noindent {\large \bf Line Printer Data Buffer Register (``lpbuf'')}

Bits 6 through 0 hold the seven bit
ASCII code for the character to be
printed. This register is ``write only''.

\bd
\item[8858:] Set the ``enable interrupts'' bit
in the line printer status register;

\item[8859:] Send a ``form feed'' (or ``new
page'') character to the printer,
to ensure that characters which
follow will start on a new page.
(As already noted above, at this
stage we are ignoring ``lpcanon''
and assuming line 8859 to be simply ``lpoutput (FORM)''. ``lpcanon''
does things like suppressing all
but the first ``form feed'' in a
string of ``form feed''s and ``new
line''s, to avoid wasting paper.);
\ed

\sbs{lpoutput (8986)}

This procedure is called with a character to be printed, as a parameter.

\bd
\item[8988:] ``lp11.cc'' is a count of the 
number of characters waiting to be sent to the line
printer. If this is already large enough
(``LPHWAT'', 8819), ``sleep'' for a
while (so as not to flood the
character buffer pool);

\item[8990:] Call ``putc'' (0967) to store the
character in a safe place. (The
function of ``putc'' and its companion ``getc'' is a major topic to
be discussed in Chapter Twenty Three.) It should be noted that
no check is made that ``putc'' was
successful in storing the character. (There may have been no
space in the character buffers.)
In practice there seems to be no
real problem here, but one can
wonder.

\item[8991:] Raise the processor priority
sufficiently to inhibit the interrupts from the line printer, call
``lpstart'' and then drop the
priority again.
\ed

\sbs{lpstart (8967)}

While the line printer is ready, and
while there are still characters stored
away in the ``safe place'', keep sending
characters to the printer controller.

The presumption is that while the controller
is building up a set of characters for a complete line, the ``DONE''
bit will reset faster than the CP can
feed characters to the controller.

However once a print cycle has been
initiated, the ``DONE'' bit will not be
reset again for a period of the order
of 100 milliseconds (depending on the
speed of the printer).

Note that during this series of data
transfers, interrupts will be inhiblted
and so ``lpint'' will not be getting into
the act whenever the ``DONE'' bit is set,
except possibly once at the very end
when the processor priority is reduced
again.

\sbs{lpinit(8976)}

This procedure is called to handle
interrupts from the line printer. As
mentioned above, most potential interrupts are ignored by the processor.
Those interrupts which are accepted by
the CP will be associated with either

\bd
\item[(a)] completion of a print cycle; or

\item[(b)] the printer going ready after a
period during which the ``Error''
bit was set; or

\item[(c)] the last transfer in a series of
	character transfers;
\ed

\bd
\item[8980:] Start transferring characters
into the printer buffer again;

\item[8981:] Wakeup the process waiting to
feed characters to the printer if
the number of characters waiting
to be sent is either zero or
exactly ``LPLWAT'' (8818).
\ed

This latter condition is somewhat puzzling
in that it will only {\bf occasionally}
be satisfied. The intention surely is
``if the number of characters in the
list is getting low, start refilling''.
However if ``lpstart'' carries out a
series of transfers without interruption (at least by ``lpint'') the number
of characters could go from a value
greater than ``LPLWAT'' to one less than
this without this test ever being made.
Accordingly the waiting process will
not be awakened until the list is completely empty.
The result could be frequently to delay the initiation of the
next print cycle, and hence to allow
the printer to run below its rated
capacity.

One solution to this problem is to
change entirely the buffering strategy
for line printers. A less drastic
change would involve inventing a new
flag, ``lp11.wflag'' say, replacing lines
8981, 8982 by something like

\begin{verbatim}
  if (lp11.cc <= LPLWAT && lp11.wflag)
  {       wakeup (&lp11);
          lp11.wflag = 0
  }
\end{verbatim}

\noindent and replacing line 8989 by

\begin{verbatim}
  {       lp11.wflag++;
          sleep(&lp11, LPRI);
  }
\end{verbatim}

\sbs{lpwrite (8870)}

This is the procedure which is invoked
as a result of the write system call:

``write'' (5722) calls ``rdwr'',
which (5755) calls ``writei'',
which        (6287)         calls
``cdevsw[..].d write'',       which
translates (4675) to ``lpwrite''.

``lpwrite'' takes the non-null characters
of a null terminated string recorded in
the user area, and passes them to
``lpoutput'' (via ``lpcanon'') one at a
time.

The list of procedure calls which leads
to the invocation of this procedure is
similar to that for ``lpopen''. A ``form
feed'' character is output to clear the
current page, and the ``open'' flag is
reset.

\sbs{Discussion}

``lpwrite'' is called one or more times
to send a string of characters to the
printer. In turn it calls ``lpcanon''
which calls ``lpoutput''. If at any point
too many characters are stored away,
the procss will ``sleep'' in ``lpoutput''.
Sooner or later ``lpoutput'' will continue, will store the character in a
buffer area, and will then call
``lpstart'' to send, if possible, a
string of characters to the printer
controller.

``lpstart'' is called both when more
characters are available to be sent,
and when an interrupt from the printer
is taken.

The majority of calls on ``lpstart'' will
in fact achieve nothing. Occasionally
(usually when the printer has just completed
a print cycle) ``lpstart'' will be
able to send a whole string of characters to the printer controller.

\sbs{lpcanon (8879)}

This procedure interprets characters
being sent to the line printer and make
various modifications, insertions and
deletions. It thus functions as a
filter.

\bd
\item[8884:] The section of code from here to
line 8913 is concerned with character translation when the full
96 character set is not available, and a 64 character set is
in use.

Since the capabilities of a
printer do not usually change
with time, the defined variable
``CAP'' (8840) must be set once and
for all (at a particular installation).

The run-time test on (lp11.flag \& CAP)
could be replaced by a compile-time test on
(CAP) and if the compiler has its
``druthers'', if CAP turns out to
be zero, the whole section of
code to line 8913 could be compiled down to nothing.

The present code could be said
to plan ahead for a situation
where an installation may have
two or more printers of different
types. Even so there is a basic
inconsistency here in the use of
``CAP'', ``IND'' and ``EJECT'' on the
one hand, and ``EJLINE'' and ``MAXCOL'' on the other. In fact since
forms of different sizes are not
uncommonly used on a single
printer, the last two should not
be constants at all, but should
be dynamically settable.

\item[8885:] Lower case alphabetics are
translated by the addition of a
constant, which is conveniently
defined as ``'A' -- 'a''';

\item[8887:] Certain of the remaining characters are special characters which
are printed as a similar character with an overprinted minus
sign, e.g. ``\{'' (8889) is printed as ``\{-'';

\item[8909:] The ``similiar'' character is output
via a recursive call on
``lpcanon'', which will increment
``lp11.ccc'' by one as a side
effect;

\item[8910:] Decrement the current character
count (for the same effect as a
``back space'' character) and ...

\item[8911:] Prepare to output a minus sign;

\item[8915:] The ``switch'' statement beginning here
extends to line 8963. Certain characters involved in vertical
and horizontal spacing are given special interpretations
with delayed actions;

\item[8917:] For a horizontal tab character, round the current character
count up to the next multiple of eight. Do not output
any blank characters immediately;

\item[8921:] For a ``form feed'' or ``new line'' character, if:

(a) the printer does not have a ``page
restore'' capability; or

(b) the current line is not empty; or

(c) some lines have been completed
since the last ``form feed'' character, then ...

\item[8925:] reset ``lp11.mcc'' to zero;

\item[8926:] Increment the completed line
count;

\item[8927:] Convert a ``new line'' character to
a ``form feed'' if sufficient lines
have been completed on the
current page, and the printer has
a ``form feed'' capability;

\item[8929:] Output the character, and if
was a ``form feed'', reset
number of completed lines
zero;
\ed

\noindent Examination of this code will show
that:

\bd
\item[(a)] Any string of ``form feed''s or
``new line''s which begins with a
``form feed'', will, if sent to a
printer with ``form feed'' capability, be reduced to a single
``form feed'';

\item[(b)] A ``form feed'' character sent to
a printer without the ``form
feed'' capability, will cause a
new line to be started but will
be passed on otherwise without
comment.
\ed

\bd
\item[8934:] For ``carriage return''s, {\bf and},
note, ``form feed''s and ``new
line''s, reset the current character count to zero or eight,
depending on ``IND'', and return;

\item[8949:] For all other characters ...

\item[8950:] If a string of ``backspace''s (real
or contrived) and/or ``carriage
returns'' has been received, output
a single ``carriage return''
and reset the maximum character
count to zero;

\item[8954:] Provided the
count does not exceed the maximum
line length, output blank characters to
bring the maximum character count to the
current character count. (Perhaps these two
variables would be more accurately called the
``actual character count'' and the ``logical character count''.);

\item[8959:] Output the actual character.
\ed

\sbs{For idle readers: A suggestion}

It will be observed that backspaces for
overprinting or underscoring characters
introduce separate print cycles, and
where these features are in heavy use,
the effective output rate of the
printer may be drastically reduced. If
this is considered a serious problem,
``lpcanon'' could be rewritten to ensure
that no more than two print cycles are
used per line in such cases.

\sbs{PC-11 Paper Tape Reader/Punch Driver}

This driver is to be found in the file
``pc.c'' on Sheets 86, 87. It is simpler
than the line printer driver in that
there is ro routine analogous to
``lpcanon''. However it is more complicated in that there is both an input
and an output device which can be
simultaneously and independently
active.

A description of the operation of this
device is inciuded in the document ``The
UNIX I/O System'' by D. Ritchie. Certain
special features may be noted:

\bd
\item[(1)] Only one process may open the file
for reading at a time, but there is no
limit on the nmber of writers;

\item[(2)] This routine pays a little more
attention to error conditions than the
line printer driver, but the treatment
is still not exhaustive;

\item[(3)] ``passc'' (8695) knows how many
characters are required and returns a
negative valie when ``enough'' is
reached;

\item[(4)] ``pcclose'' is careful to flush out
any remaining characters in the input
queue if and only if it believes the
device was opened for input.
\ed

%
% The Lion's Commentary, file ch23.tex, version 1.4, 16 May 1994
%
\se{Character Handling}

Buffering for character special devices
is provided via a set of four word
blocks, each of which provides storage
for six characters. The prototype
storage block is ``cblock'' (8140) which
incorporates a word pointer (to a similar
structure) along with the six characters.

Structures of type ``clist'' (7908) which
contain a character counter plus a head
and tail pointer are used as ``headers''
for lists of blocks of type ``cblock''.

``cblock''s which are not in current use
are linked via their head pointers into
a list whose head is the pointer
``cfreelist'' (3149). The head pointer
for the last element of the list has
the value ``NULL''.

A list of ``cblock''s provides storage
for a list of characters. The procedure
``putc'' may be used to add a character
to the tail of such a list, and ``getc'',
to remove a character from the head of
such a list.

Figures 23.1 through 23.4 illustrate
the development of a list as characters
are deleted and added.

%
% The Lion's Commentary, file fig23_1.tex, version 1.2, 15 May 1994
%
\vspace{0.6cm}
\setlength{\unitlength}{0.0109in}%
\begin{picture}(245,155)(40,525)
\thicklines
\put(185,540){\framebox(25,140){}}
\put(195,650){\makebox(0,0)[lb]{\raisebox{0pt}[0pt][0pt]{\twlrm g}}}
\put(195,629){\makebox(0,0)[lb]{\raisebox{0pt}[0pt][0pt]{\twlrm h}}}
\put(195,608){\makebox(0,0)[lb]{\raisebox{0pt}[0pt][0pt]{\twlrm i}}}
\put(195,587){\makebox(0,0)[lb]{\raisebox{0pt}[0pt][0pt]{\twlrm j}}}
\put(195,566){\makebox(0,0)[lb]{\raisebox{0pt}[0pt][0pt]{\twlrm k}}}
\put(195,545){\makebox(0,0)[lb]{\raisebox{0pt}[0pt][0pt]{\twlrm l}}}
\put(240,540){\framebox(25,140){}}
\put(250,629){\makebox(0,0)[lb]{\raisebox{0pt}[0pt][0pt]{\twlrm n}}}
\put(250,608){\makebox(0,0)[lb]{\raisebox{0pt}[0pt][0pt]{\twlrm o}}}
\put(250,587){\makebox(0,0)[lb]{\raisebox{0pt}[0pt][0pt]{\twlrm p}}}
\put(250,566){\makebox(0,0)[lb]{\raisebox{0pt}[0pt][0pt]{\twlrm q}}}
\put(250,545){\makebox(0,0)[lb]{\raisebox{0pt}[0pt][0pt]{\twlrm r}}}
\put(247,650){\makebox(0,0)[lb]{\raisebox{0pt}[0pt][0pt]{\twlrm m}}}
\put( 40,540){\framebox(60,60){}}
\put( 90,570){\vector( 1, 0){ 40}}
\put(130,540){\framebox(25,140){}}
\put(140,670){\vector( 1, 0){ 45}}
\put(195,670){\vector( 1, 0){ 45}}
\put( 90,550){\line( 0,-1){ 25}}
\put( 90,525){\line( 1, 0){195}}
\put(285,525){\line( 0, 1){ 25}}
\put(285,550){\vector(-1, 0){ 20}}
\put( 48,584){\makebox(0,0)[lb]{\raisebox{0pt}[0pt][0pt]{\twlrm 14}}}
\put( 49,564){\makebox(0,0)[lb]{\raisebox{0pt}[0pt][0pt]{\twlrm head}}}
\put( 49,545){\makebox(0,0)[lb]{\raisebox{0pt}[0pt][0pt]{\twlrm tail}}}
\put(140,566){\makebox(0,0)[lb]{\raisebox{0pt}[0pt][0pt]{\twlrm e}}}
\put(140,545){\makebox(0,0)[lb]{\raisebox{0pt}[0pt][0pt]{\twlrm f}}}
\end{picture}
\vspace{0.4cm}
\begin{center}
{\large \bf Figure 23.1}
\end{center}
\vspace{0.2cm}


%
% The Lion's Commentary, file fig23_2.tex, version 1.2, 15 May 1994
%
\vspace{0.6cm}
\setlength{\unitlength}{0.0109in}%
\begin{picture}(245,155)(40,525)
\thicklines
\put(185,540){\framebox(25,140){}}
\put(195,650){\makebox(0,0)[lb]{\raisebox{0pt}[0pt][0pt]{\twlrm g}}}
\put(195,629){\makebox(0,0)[lb]{\raisebox{0pt}[0pt][0pt]{\twlrm h}}}
\put(195,608){\makebox(0,0)[lb]{\raisebox{0pt}[0pt][0pt]{\twlrm i}}}
\put(195,587){\makebox(0,0)[lb]{\raisebox{0pt}[0pt][0pt]{\twlrm j}}}
\put(195,566){\makebox(0,0)[lb]{\raisebox{0pt}[0pt][0pt]{\twlrm k}}}
\put(195,545){\makebox(0,0)[lb]{\raisebox{0pt}[0pt][0pt]{\twlrm l}}}
\put(240,540){\framebox(25,140){}}
\put(250,629){\makebox(0,0)[lb]{\raisebox{0pt}[0pt][0pt]{\twlrm n}}}
\put(250,608){\makebox(0,0)[lb]{\raisebox{0pt}[0pt][0pt]{\twlrm o}}}
\put(250,587){\makebox(0,0)[lb]{\raisebox{0pt}[0pt][0pt]{\twlrm p}}}
\put(250,566){\makebox(0,0)[lb]{\raisebox{0pt}[0pt][0pt]{\twlrm q}}}
\put(250,545){\makebox(0,0)[lb]{\raisebox{0pt}[0pt][0pt]{\twlrm r}}}
\put(247,650){\makebox(0,0)[lb]{\raisebox{0pt}[0pt][0pt]{\twlrm m}}}
\put( 40,540){\framebox(60,60){}}
\put( 90,570){\vector( 3,-1){ 40.500}}
\put(130,540){\framebox(25,140){}}
\put(140,670){\vector( 1, 0){ 45}}
\put(195,670){\vector( 1, 0){ 45}}
\put( 90,550){\line( 0,-1){ 25}}
\put( 90,525){\line( 1, 0){195}}
\put(285,525){\line( 0, 1){ 25}}
\put(285,550){\vector(-1, 0){ 20}}
\put( 49,564){\makebox(0,0)[lb]{\raisebox{0pt}[0pt][0pt]{\twlrm head}}}
\put( 49,545){\makebox(0,0)[lb]{\raisebox{0pt}[0pt][0pt]{\twlrm tail}}}
\put(140,545){\makebox(0,0)[lb]{\raisebox{0pt}[0pt][0pt]{\twlrm f}}}
\put( 48,584){\makebox(0,0)[lb]{\raisebox{0pt}[0pt][0pt]{\twlrm 13}}}
\end{picture}
\vspace{0.4cm}
\begin{center}
{\large \bf Figure 23.2}
\end{center}
\vspace{0.2cm}


Initially the list is assumed to contain the fourteen characters
``efghijklmnopqr''. Note that the head
and tail pointers point to {\bf characters}.
If the first character, ``e'', is removed
by ``getc'', the situation portrayed in
Figure 23.1 changes to that of Figure
23.2. The character count has been
decremented and the head pointer has
been advanced by one character position.

If a further character, ``f'', is removed
from the head of the list, the
situation becomes as in Figure 23.3.
The character count has been decremented;
the first ``cblock'' no longer
contains any useful information and has
been returned to ``cfreelist''; and the
head pointer now points to the first
character in the second ``cblock''.

%
% The Lion's Commentary, file fig23_3.tex, version 1.2, 15 May 1994
%
\vspace{0.6cm}
\setlength{\unitlength}{0.0109in}%
\begin{picture}(245,155)(40,525)
\thicklines
\put(185,540){\framebox(25,140){}}
\put(195,650){\makebox(0,0)[lb]{\raisebox{0pt}[0pt][0pt]{\twlrm g}}}
\put(195,629){\makebox(0,0)[lb]{\raisebox{0pt}[0pt][0pt]{\twlrm h}}}
\put(195,608){\makebox(0,0)[lb]{\raisebox{0pt}[0pt][0pt]{\twlrm i}}}
\put(195,587){\makebox(0,0)[lb]{\raisebox{0pt}[0pt][0pt]{\twlrm j}}}
\put(195,566){\makebox(0,0)[lb]{\raisebox{0pt}[0pt][0pt]{\twlrm k}}}
\put(195,545){\makebox(0,0)[lb]{\raisebox{0pt}[0pt][0pt]{\twlrm l}}}
\put(240,540){\framebox(25,140){}}
\put(250,629){\makebox(0,0)[lb]{\raisebox{0pt}[0pt][0pt]{\twlrm n}}}
\put(250,608){\makebox(0,0)[lb]{\raisebox{0pt}[0pt][0pt]{\twlrm o}}}
\put(250,587){\makebox(0,0)[lb]{\raisebox{0pt}[0pt][0pt]{\twlrm p}}}
\put(250,566){\makebox(0,0)[lb]{\raisebox{0pt}[0pt][0pt]{\twlrm q}}}
\put(250,545){\makebox(0,0)[lb]{\raisebox{0pt}[0pt][0pt]{\twlrm r}}}
\put(247,650){\makebox(0,0)[lb]{\raisebox{0pt}[0pt][0pt]{\twlrm m}}}
\put( 40,540){\framebox(60,60){}}
\put(195,670){\vector( 1, 0){ 45}}
\put( 90,550){\line( 0,-1){ 25}}
\put( 90,525){\line( 1, 0){195}}
\put(285,525){\line( 0, 1){ 25}}
\put(285,550){\vector(-1, 0){ 20}}
\put( 90,570){\line( 1, 0){ 50}}
\put(140,570){\line( 0, 1){ 85}}
\put(140,655){\vector( 1, 0){ 45}}
\put( 49,564){\makebox(0,0)[lb]{\raisebox{0pt}[0pt][0pt]{\twlrm head}}}
\put( 49,545){\makebox(0,0)[lb]{\raisebox{0pt}[0pt][0pt]{\twlrm tail}}}
\put( 48,584){\makebox(0,0)[lb]{\raisebox{0pt}[0pt][0pt]{\twlrm 12}}}
\end{picture}
\vspace{0.4cm}
\begin{center}
{\large \bf Figure 23.3}
\end{center}
\vspace{0.2cm}


The question now poses itself: ``how is
the difference between the first and
second situations detected so that the
action taken is always appropriate?'':

The answer (if you have not already
guessed) involves looklng at the value
of the pointer address modulo 8. Since
division by eight is easily performed
on a binary computer, the reason for
the choice of six characters per
``cblock'' should now also be apparent.

The addition of a character to the list
is illustrated in the change between
Figure 23.3  and Figure 23.4.

%
% The Lion's Commentary, file fig23_4.tex, version 1.2, 15 May 1994
%
\vspace{0.6cm}
\setlength{\unitlength}{0.0109in}%
\begin{picture}(305,155)(60,525)
\thicklines
\put(185,540){\framebox(25,140){}}
\put(195,650){\makebox(0,0)[lb]{\raisebox{0pt}[0pt][0pt]{\twlrm g}}}
\put(195,629){\makebox(0,0)[lb]{\raisebox{0pt}[0pt][0pt]{\twlrm h}}}
\put(195,608){\makebox(0,0)[lb]{\raisebox{0pt}[0pt][0pt]{\twlrm i}}}
\put(195,587){\makebox(0,0)[lb]{\raisebox{0pt}[0pt][0pt]{\twlrm j}}}
\put(195,566){\makebox(0,0)[lb]{\raisebox{0pt}[0pt][0pt]{\twlrm k}}}
\put(195,545){\makebox(0,0)[lb]{\raisebox{0pt}[0pt][0pt]{\twlrm l}}}
\put(240,540){\framebox(25,140){}}
\put(250,629){\makebox(0,0)[lb]{\raisebox{0pt}[0pt][0pt]{\twlrm n}}}
\put(250,608){\makebox(0,0)[lb]{\raisebox{0pt}[0pt][0pt]{\twlrm o}}}
\put(250,587){\makebox(0,0)[lb]{\raisebox{0pt}[0pt][0pt]{\twlrm p}}}
\put(250,566){\makebox(0,0)[lb]{\raisebox{0pt}[0pt][0pt]{\twlrm q}}}
\put(250,545){\makebox(0,0)[lb]{\raisebox{0pt}[0pt][0pt]{\twlrm r}}}
\put(247,650){\makebox(0,0)[lb]{\raisebox{0pt}[0pt][0pt]{\twlrm m}}}
\put( 40,540){\framebox(60,60){}}
\put(195,670){\vector( 1, 0){ 45}}
\put( 90,570){\line( 1, 0){ 50}}
\put(140,570){\line( 0, 1){ 85}}
\put(140,655){\vector( 1, 0){ 45}}
\put(295,540){\framebox(25,140){}}
\put(250,670){\vector( 1, 0){ 45}}
\put( 90,550){\line( 0,-1){ 25}}
\put( 90,525){\line( 1, 0){255}}
\put(345,525){\line( 0, 1){130}}
\put(345,655){\vector(-1, 0){ 25}}
\put( 49,564){\makebox(0,0)[lb]{\raisebox{0pt}[0pt][0pt]{\twlrm head}}}
\put( 49,545){\makebox(0,0)[lb]{\raisebox{0pt}[0pt][0pt]{\twlrm tail}}}
\put(305,650){\makebox(0,0)[lb]{\raisebox{0pt}[0pt][0pt]{\twlrm s}}}
\put( 48,584){\makebox(0,0)[lb]{\raisebox{0pt}[0pt][0pt]{\twlrm 13}}}
\end{picture}
\vspace{0.4cm}
\begin{center}
{\large \bf Figure 23.4}
\end{center}
\vspace{0.2cm}


Since the last ``cblock'' ln Figure 23.3
was full, a new one has been obtained
from ``cfreelist'' and linked into the
list of ``cblock''s. The character count
and tail pointer have been adjusted appropriately.

\sbs{cinit (8234)}

This procedure, which is called once by
``main'' (1613), links the set of character buffers into the free list,
``cfreelist'', and counts the number of
character device types.

\bd
\item[8239:] ``ccp'' is the address of the first
word in the array ``cfree'' (8146);

\item[8240:] Round ``ccp'' up to the next
highest multiple of eight, and
mark out ``cblock'' sized pieces,
taking care not to exceed the
boundary of ``cfree''.

Note. In general there will be
``NCLIST -- 1'' (rather than
``NCLIST'') blocks so defined;

\item[8241:] Set the first word of the
``cblock'' to point to the current
head of the free list.

Note that ``c\_next'' is defined on
line 8141, and that the initial
value of ``cfreelist'' is ``NULL''.

\item[8242:] Update ``cfreelist'' to point to
the new head of the list;

\item[8244:] Count the number of character
device types. Upon reference to
``cdevsw'' on Sheet 46, it will be
seen that ``nchrdev'' will be set
to 16, whereas a more appropriate
value would be 10.
\ed

\sbs{getc (0930)}

This procedure is called by

\begin{verbatim}
   flushtty (8258, 8259, 8264)
   canon    (8292)    pcread  (8688)
   ttstart  (8520)    pcstart (8714)
   ttread   (8544)    lpstart (8971)
   pcclose  (8673)
\end{verbatim}

\noindent with a single argument which is the
address of a ``clist'' structure.

\bd
\item[0931:] Copy the parameter to r1 and save
the initial processor status word
and value of r2 on the stack;

\item[0934:] Set the processor priority to
five (higher than the interrupt
priority of a character device);

\item[0936:] r1 points to the first word of a
``clist'' structure (i.e. a character count). Move the second word
of this structure (i.e. a pointer
to the head character) to r2;

\item[0937:] If the list is empty (head
pointer is ``NULL'') go to line 0961;

\item[0938:] Move the head character to r0 and
increment r2 as a side effect;

\item[0939:] Mask r0 to get rid of any
extended negative sign;

\item[0940:] Store the updated head pointer
back in the ``clist'' structure.
(This may have to be altered
later.);

\item[0941:] Decrement the character count and
if this is still positive, go to
line 0947;

\item[0942:] The list is now empty, so reset the head and tail
character pointers to ``NULL''. Go to line 0952;

\item[0947:] Look at the three least significant bits of r2. If these are
non-zero, branch to line 0957
(and return to the calling routine forthwith);

\item[0949:] At this point, r2 is pointing at
the next character position
beyond the ``cblock''. Move the
value stored in the first word of
the ``cblock'' (i.e. at r2 -- 8),
which is the address of the next
``cblock'' in the list to the head
pointer in the ``clist''. (Note
that r1 was incremented as a side
effect at line 0941):

\item[0950:] The last value stored needs to
incremented by two (Consult
Figures 23.2 and 23.3);

\item[0952:] At this point, a
``cblock'' determined by r2 is to be returned to
``cfreelist''. Either r2 points
into the ``cblock'' or just beyond
it. Decrement r2 so that r2 will
point into the ``cblock'';

\item[0953:] Reset the three least significant
bits of r2, leaving a pointer to
the ``cblock'';

\item[0954:] Link the ``cblock'' into ``cfreelist'';

\item[0957:] Restore the values of r2 and PS
from the stack and return;

\item[0961:] At this point the list is known
to be empty because a ``NULL'' head
pointer was encountered. Make
sure that the tail pointer is
``NULL'' also;

\item[0962:] Move --1 to r0 as the result to be
returned when the list is empty.
\ed

\sbs{putc (0967)}

This procedure is called by

\begin{verbatim}
   canon     (8323)
   ttyinput  (8355, 8358)
   ttyoutput (8414, 8478)
   pcrint    (8730)
   pcoutput  (8756)
   lpoutput  (8990)
\end{verbatim}

\noindent with two arguments: a character and the
address of a ``clist'' structure.

``getc'' and ``putc'' have related functions and
the codes for the two procedures are similar in many respects.
For this reason the code for ``putc''
will not be examined in detail, but is
left for the reader.

It should be noted that ``putc'' can fail
if a new ``cblock'' is needed and
``cfreelist'' is empty. In this case a
non-zero value (line 1002) is returned
rather than a zero value (line 0996).

{\bf Note.} The procedures ``getc'' and ``putc''
discussed here are {\bf NOT} directly related
to the procedures d1scussed in the Sections
``GETC(III)'' and ``PUTC(III)'' of the UPM.

\sbs{Character Sets}

UNIX makes use of the full ASCII character set,
which is displayed in Section ``ASCII(V)'' of the UPM. Since
knowledge of this character set is
often assumed without comment, not
always justifiably, some comment here
would seem to be in order.

``ASCII'' is an acronym for ``American
Standard Code for Information Interchange''.

\sbs{Control Characters}

The first 32 of the 128 ASCII characters are non-graphic and are intended
for the control of some aspect of
transmission or display. The control
characters explicitly used or recognised by UNIX are

\noindent\begin{tabular}{llll}\\
{\bf Numeric} & & \multicolumn{1}{c}{\bf Description} & {\bf UNIX} \\
{\bf Value} &	&	& {\bf Name} \\ \hline
004 & eot & end of transmission		& 004 \\
    &    & or (control-D) & \\
010 & bs & back space			& 010 \\
011 & ht & (horizontal) tab		& '\verb+\+t' \\
012 & nl & new line or line feed	& FORM \\
014 & np & new page or form feed	& '\verb+\+r' \\
015 & cr & carriage return		& '\verb+\+n' \\
034 & fs & file separator or quit	& CQUIT \\
040 & sp & forward space or blank	& '~' \\
0177 & del & delete			& CINTR
\end{tabular}


It will be noted that the last two of
these belong to the last 96 characters
or the graphic portion, of the code.

\sbs{Graphic Characters}

There are 96 graphic characters. Two of
these, the space and the delete, are
not ``visible'', and may be ciassified
with the control characters.

The graphic characters may be divided
into three groups of 32 characters,
which may be roughly characterised as

\bi
\item numeric and special characters
\item upper case alphabetic characters
\item lower case alphabetic characters.
\ei

Of course, since there are only 26
alphabetic characters, the latter two
groups include some special characters
as well. In particular, the last group
includes the following six nonalphabetic characters:

\begin{tabular}{lll}\\
140 & `			& reverse apostrophe \\
173 & \{		& left brace \\
174 & \verb+|+		& vertical bar \\
175 & \}		& right brace \\
176 & \verb+~+		& tilde \\
177 &			& delete \\
\end{tabular}

\sbs{Graphic Character Sets}

Devices such as line printers or terminals which support {\bf all} the ASCII
graphic symbols are often said to support the 96 ASCII character set (though
there are only 94 graphics actually
involved).

Devices which support all the ASCII
graphic symbols except those in the
last group of 32, are said to support
the 64 ASCII character set. Such devices lack the lower case alphabetics
and the symbols listed above, namely ``\verb+~+'', ``\{'', ``\verb+|+'' and
``\verb+\}+''. Note that
``delete'', since it is not a visible
character, can still be supported.

Devices in this latter group may be
referred to as ``upper case only''.

Sometimes some of the graphic symbols
may be non-standard, e.g. $\leftarrow$  instead
of \_ and this can be inconvenient,
though not usually fatal.

UNIX prefers, as the reader is no doubt
well aware, to view the world through
``lower case'' spectacles. Alphabetic
characters received from an ``upper case
only'' terminal are translated
immediately upon receipt from upper
case to lower case. A lower case alphabetic may subsequently be translated
back to upper case if it is preceded by
a single backslash. For output to such
a terminal, both upper and lower case
alphabetic characters are mapped to uppercase.

%Equivalences for the five ``upper case''
%special characters are as follows:

The conventions for line printers and
terminals are different because:

\bd
\item[(a)] for line printers, horizontal
alignment is usually important,
and it is possible (without too
much difficulty) to print composite, overstruck characters
(using the minus sign in this
case); and

\item[(b)] for terminals, horizontal alignment is not considered to be so
important; backspacing to provide overstruck characters does
not work on most VDUs; and,
since the same graphic conventions are used for both input
and output, the symbols should
be as convenient to type as possible.
\ed

\sbs{maptab (8117)}

This array is used in the translation
of character input from a terminal preceded by a single backslash, ``\verb+\+''.

There are three characters, 004 (eot),
`\#' and `@', which always have special
meanings and need to be asserted by a
backslash whenever they are to be
interpreted literally. These three
characters occur in ``maptab'' in their
``natural'' locations (i.e. their locations in the ASCII table). Thus for
example `\#' has code 043 and

\begin{verbatim}
   maptab[043] == 043.
\end{verbatim}

The other non-null characters in ``maptab'' are involved in the translation of
input characters from ``upper case only''
devices and do not occur in their
``natural'' locations but in the location
of their equivalent character, e.g. ``\verb+\+\{''
occurs in the natural location for ``\{'',
since ``\verb+\+\{'' will be interpreted as ``\{'',
etc.

Note the situation regarding alphabetic
characters. This is only explicable
when it is remembered that the alphabetic characters are all translated to
lower case before any backslash is
recognised.

\sbs{partab (7947)}

This array consists of 256 characters,
like ``maptab''. Unfortunately the initialisation of ``partab''
was omitted from the UNIX Operating System Source Code
booklet. It is certainly needed, and so
is given now:

\begin{verbatim}
  char partab [] {

0001,0201,0201,0001,0201,0001,0001,0201,
0202,0004,0003,0205,0005,0206,0201,0001,
0201,0001,0001,0201,0001,0201,0201,0001,
0001,0201,0201,0001,0201,0001,0001,0201,
0200 0000,0000,0200,0000,0200,0200,0000,
0000 0200,0200 0000,0200,0000,0000,0200,
0000,0200,0200 0000,0200,0000,0000,0200,
0200,0000,0000,0200,0000,0200,0200,0000,
0200,0000,0000,0200,0000,0200,0200,0000,
0000,0200,0200,0000,0200,0000,0000,0200,
0000,0200,0200,0000,0200,0000,0000,0200,
0200 0000,0000 0200,0000,0200,0200,0000,
0000 0200,0200 0000,0200,0000,0000,0200,
0200,0000,0000,0200,0000,0200,0200,0000,
0200,0000,0000,0200,0000,0200,0200,0000,
0000,0200,0200,0000,0200,0000,0000,0201
   };
\end{verbatim}

Each element of ``partab'' is an eight
bit character, which, with the use of
appropriate bitmasks (0200 and 0177),
can be interpreted as a two part structure:

\begin{tabular}{ll}\\
bit 7    & parity bit;\\
bits 3-5 & not used. Always zero;\\
bits 0-2 & code number.\\
\end{tabular}

\bigskip

The parity bit is appended to the seven
bit ASCII code when a character is
transmitted by the computer, to form an
eight bit code with even parity.

The code number is used by ``ttyoutput''
(8426) to classify the character into
one of seven categories for determining
the delay which should ensue before the
transmission of the next character.
(This is particularly important for
mechanical printers which require time
for the carriage to return from the end
of a line, etc.)

%
% The Lion's Commentary, file ch24.tex, version 1.4, 15 May 1994
%
\se{Interactive Terminals}

Our remaining task, to be completed in
this and the following chapter, is to
consider the code which controls
interactive terminals (or ``terminals'',
for short).

A wide variety of terminals is available and several different types may be
simultaneously attached to a single
computer. Distinguishing characteristics for different classes of terminal
include (besides such non-essential
features as shape, size and colour):

\bd
\item[(a)] transmission speed, e.g. 110
baud for an ASR33 teletype, 300
baud for a DECwriter, 2400 baud
or 9600 baud for a Visual
Display unit (``VD'');

\item[(b)] graphic character set, notably
the full ASCII graphic set and
the 64 graphic subset;

\item[(c)] transmission parity: odd, even,
none or inoperatlve;

\item[(d)] output technique:  serial printer
or visual display;

\item[(e)] miscellaneous: combined carriage
return/line feed character, half
duplex terminal (input characters do not need echoing);
recognition of tab characters;

\item[(f)] characteristic delays for certain control functions, e.g.
carriage returns may not be completed within a single character
transmission time, etc.
\ed

As well as the wide variety of terminals which are available and in use,
there is also a variety of hardware
devices which may be used to interface
a terminal to a PDP 11 computer. For example:

\noindent\begin{tabular}{ll}\\
DL11/KL11 & single line, asynchronous\\
	  & interface; 13 standard\\
	  & transmission rates between\\
	  & 40 and 9600 baud;\\
\\
DJ11 & 16 line, asynchronous, buffered\\
	& serial line multiplexer; 11\\
	& speeds between 75 and 9600 baud,\\
	& selectable in four line groups;\\
\\
DH11 & 16 line, asynchronous, buffered,\\
	&  serial line multiplexer; 14 speeds,\\
	& individually selectable; DMA \\
	& transmission\\
\end{tabular}

Each of the above interfaces will work
in full or half duplex mode; handle 5,
6, 7 or 8 level codes; generate odd,
even or no parity; and generate a stop
code of 1, 1.5 or 2 bits.

In addition to the above asynchronous
interfaces, there are a number of synchronous interfaces, e.g. DQ11.

Each interface has its own control
characteristics and it requires a
separate operating system device
driver. The common code which can be
shared between these is gathered into a
single file ``tty.c'', to be found on
Sheets 81 to 85. A set of common definitions is gathered in the file ``tty.h''
on Sheet 79.

By way of example, Sheet 80 contains
the file ``kl.c'', which constitutes the
device driver for a set of DL11/KL11
interfaces. This device driver always
needs to be present, since one KL11
interface is invariably included in a
system for the the operator's console
terminal.

\sbs{The 'tty' Structure (7926)}

An instance of ``tty'' is associated with
every terminal port to the system (no
matter what type of hardware interface
is used). A ``port'' in this context is a
place to attach a terminal line. Hence
a DL11 supplies only one port, whereas
a DJ11 supplies up to sixteen ports.

The ``tty'' structure consists of sixteen
words and includes:

\noindent\begin{tabular}{lll}\\
A. & t\_dev & fixed for a particular \\
   & t\_addr & terminal port;\\
\\
B. &  t\_speeds & fixed for a particular \\
  & t\_erase & terminal. These values may \\
  & t\_kill & be set by ``stty'' and \\
  & t\_flags & interrogated by ``gtty''; \\
 \\
C. & t\_rawq & list heads for {\bf three} \\
  & t\_canq & character queues: the \\
  & t\_outq & so-called ``raw'' input, \\
 & & ``cooked'' input and the \\
 & & output queues; \\
 \\
D. & t\_state & status information which \\
  & t\_delct & changes frequently during \\
  & t\_col & normal processing; \\
  & t\_char & \\
\end{tabular}

\begin{center}
{\large \bf Table 24.1}
\end{center}

\sbs{ Interactive Terminals}

The reader should study the information
on Sheet 79 carefully. Certain items
listed below are not referenced in any
essential way in the selection of code
examined here.

\begin{verbatim}
   t_char    (7940)    NLDELAY   (7974)
   t_speeds  (7941)    TBDELAY   (7975)
   HUPCL     (7966)    CRDELAY   (7976)
   ODDP      (7972)    WOPEN     (7985)
   EVENP     (7973)    ASLEEP    (7993)
\end{verbatim}

\sbs{Initialisation}

Initialisation of the ``tty'' structures
is the responsibility of the various
``open'' routines in the device drivers,
for example, ``klopen'' (8023).

The items in Group B of Table 24.1 may
be changed by a ``stty'' system call.
The current values may be interrogated
by a ``gtty'' system call.

A description of these is contained in
the sections, ``STTY(II)'' and ``GTTY(II)''
of the UPM. These calls are invoked by
the ``stty'' shell command which is
described in the section ``STTY(I)''.

Since the ``stty'' and ``gtty'' system
calls require a file descriptor as a
parameter, they can only be applied to
an ``open'' character special file.

The two system calls share a good deal
of common code. We will trace the progress of an execution of ``stty'' below
and leave the tracing of a similar execution of ``gtty'' to the reader.

\sbs{stty (8183)}

This procedure implements the ``stty''
system call. It copies three words of
user parameter information into
``u.u\_arg[..]'' using the parameter supplied as a pointer, and then calls
``sgtty''.

\sbs{sgtty (8201)}

\bd
\item[8206:] Get a validated pointer to  a
``file'' array entry;

\item[8209:] Check that the file is a ``character special'';

\item[8213:] Call the appropriate ``d\_sgtty''
routine for the device type. (See
Sheet 46.)
\ed

Note that the ``d\_sgtty'' routine is
``nodev'' for the line printer and paper
tape reader/punch.

\sbs{klsgtty (8090)}

This is an example of a ``d\_sgtty'' routine. It calls ``ttystty'' passing a
pointer to the appropriate ``tty'' structure as a parameter.

\sbs{tysty (8577)}

A call originating from ``stty'' will
have a second parameter of zero.

\bd
\item[8589:] Empty all the queues associated
with the terminal forthwith. They
quite likely contain nonsense;

\item[8591:] Reset the speed information (useful in the
case of a DH11 interface, but of little interest for
the present selection of code);

Reset the ``erase'' character and
the ``kill'' character. (``kill''
here denotes ``throw away the
current input line''.) Note that
if these characters are changed
away from their normal values of
``\#'' and ``@'' respectively, no
corresponding changes are made to
``maptab''. Nor should they!),

\item[8593:] Reset the ``flags'' defining some
relevant terminal characteristics
(see Sheet 79):

\noindent\begin{tabular}{lll}\\
{\bf flag} &  {\bf bit} & {\bf if set ...}\\
 \\
XTABS & 1 & the terminal will not interpret \\
& & horizontal tab characters\\
& &  correctly; \\
 \\
LCASE & 2 & the terminal supports only the \\
& & 64 character ASCII subset; \\
 \\
ECHO & 3 & the terminal is operating in \\
& & full duplex mode, and input \\
& & characters must be echoed \\
& & back; \\
 \\
CRMOD & 4 & upon input, a ``carriage \\
& & return'' is replaced by a \\
& & ``line feed''; upon output, a\\
& & ``line feed'' is replaced by a\\
& & ``carriage return'' and a ``line\\
& &  feed''; \\
 \\
RAW & 5 & input characters are to be \\
& & sent to the program exactly \\
& & as received, without ``erase''\\
& & or``kill'' processing, or\\
& & adjustment for backslash\\
& & characters. \\
\end{tabular}

In addition, the following bits are
interrogated by ``ttyoutput'' (8373) in
choosing the delay which should ensue
after the character indicated is sent,
before sending the next character:


\begin{tabular}{ll}\\
8,9 & line feed;\\
10,11 & horizontal tab;\\
12,13 & carriage return;\\
14 & vertical tab or form feed.\\
\end{tabular}
\ed

\sbs{The DL11/KL11 Terminal Device Handler}

The file ``kl.c'' constitutes the device
handler for terminals connected to the
system via DL11/KL11 interfaces. This
group always has at least one member -- the operator's console terminal. Hence
this device handler will always be
present.

Each DL11/KL11 hardware controller provldes an asynchronous, serial interface
to connect a single terminal to a PDP
11 system. For more complete details
regarding this interface the reader
should consult the ``PDP11 Peripherals
Handbook''.

\sbs{Device Registers}

Each DL11/RL11 unit has a group of four
registers occupying four consecutive
words on the UNIBS. UNIX maps a
structure of type ``klregs'' (8016) onto
each register group.

\noindent\begin{tabular}{ll}\\
\multicolumn{2}{l}{\bf Receiver Status Register (klrcsr)} \\
bit 7 & {\bf Receiver Done.} (A character has\\
      & been transferred into the\\
      & Receiver Data Buffer Register.);\\
\\
bit 6 & {\bf Receiver Interrupt Enable.}\\
      & (When set, an interrupt is\\
      & caused every time bit 7 is set.);\\
\\
bit 1 & {\bf Data terminal ready;}\\
\\
bit 0 & {\bf Reader Enable. Write only.}\\
      & (When set, bit 7 is cleared.).\\
\end{tabular}

\noindent\begin{tabular}{ll}\\
\multicolumn{2}{l}{\bf Receiver Data Buffer Register (klrbuf)} \\
bit 15 & Error indication, when set.\\
\\
bits 7-0 & Received character, Read\\
         & only.\\
\end{tabular}

\noindent\begin{tabular}{ll}\\
\multicolumn{2}{l}{\bf Transmitter Status Register (kltcsr)} \\
bit 7 & {\bf Transmitter ready.} This is\\
      & cleared when data is loaded\\
      & into the Transmitter Data\\
      & Buffer, and is set when the\\
      & latter is ready to receive\\
      & another chatacter;\\
\\
bit 6 & {\bf Transmitter Interrupt Enable.}\\
     & (when set, causes an\\
     & interrupt to be generated\\
     & whenever bit 7 is set.)\\
\end{tabular}

\noindent\begin{tabular}{ll}\\
\multicolumn{2}{l}{\bf Transmitter Data Buffer Register (kltbuf)} \\
bits 7-0 & Transmitted data. Write only.\\
\end{tabular}

\sbs{UNIBUS Addresses}

The Receiver Status Register always has
its lowest address starting on a four
word boundary. (The addresses which
follow are all 18 bit octal addresses.)

\noindent\begin{tabular}{lll}\\
 & {\bf Receiver} & {\bf Transmitter}\\
 & {\bf Status} & {\bf Status}\\
Operator's console & 777560 $\rightarrow$ & 777566\\
\\
Group Two & 776500 $\rightarrow$ & 776506\\
	& 776510 $\rightarrow$ & 77656 \\ \cline{2-3}
	& 776670 $\rightarrow$ & 776676\\
\\
Group Three & 775610 $\rightarrow$ & 775616\\
        & 775620 $\rightarrow$ & 775626 \\ \cline{2-3}
        & 776170 $\rightarrow$ & 776176\\
\end{tabular}

Apart from the operator's console
interface which has its own standard
UNIBUS location, the interfaces are
gathered into two groups (for reasons
which are irrelevant here). Within
each group, by convention, registers
are allocated in consecutive locations
starting at the lowest address.

\sbs{Software Considerations}

``NKL11'' (8011) must be set to define,
for a particular installation, the
number of interfaces in the first two
groups, and ``NDL11'' (8012), the nmber
in the third group. Any hardware
alterations which changed the actual
number of interfaces would have to be
reflected in the software by changing
and recomiling ``kl.c'', and relinking
the operaing system.

It will be seen that ``klopen'' calculates the correct kernel mode address
(16 bits) for the Receiver Status
Register for each interface, and this
is stored (8044) into the the ``t\_addr''
element of the appropriate ``tty'' structure.

\sbs{Interrupt Vector Addresses}

The vector addresses for the first
interface are 060 and 064 (for receiver
and transmitter interrupts, respectively).
Additional DL11/KL11 interfaces have vector addresses which are
always at least 0300, and which are
assigned according to rules which take
into consideration other interfaces
which may be present.

The second word of an interrupt doubiet
is the ``new processor status'' word. The
five low order bits of this word may be
chosen arbitrarily, and are in fact
used to define the minor device number
(cf. a similar use to distinguish the
various kinds of ``traps'' -- see Sheet
05). A masked version of the new processor status word is provided to the
interrupt handling routines as the
parameter ``dev'' (see e.g. line 8070).

\sbs{Source Code}

We can now turn to a detailed study of
the code in the files ``kl.c'' (Sheet 80)
and ``tty.c'' (Sheets 81 to 85). We
shall look first at ``opening'' and
``closing'' terminals as character
special files and the handling of interrupts. Then in the next chapter we
shall look at the receipt of data from
the terminal, and finally transmission
of data to the terminal.

``klread'' (8062), ``klwrite'' (8066) and
``klsgtty'' (8090) have already been discussed above.

\sbs{klopen (8023)}

This procedure is called to ``open'' a
terminal as a character special file.
This call is usually made by the program ``/etc/init'' for each terminal
which is to be active in the system.
Since child processes inherit the open
files of their parents, it is not usually necessary for other processes to
``open'' the device again. It will be
noted that the there is no attempt to
stop two unrelated processes having the
terminal as an open file simultaneously.

\bd
\item[8026:] Check the minor device number;

\item[8030:] Locate the appropriate ``tty''
structure;

\item[8031:] If the process opening the file
has no associated controlling
terminal designate the current
terminal for this role. (Note
that the reference stored is the
address of a ``tty'' structure.):

\item[8033:] Store the terminal device number
in the ``tty'' structure

\item[8039:] Calculate the address of the
appropriate set of device registers for the
terminal and store

\item[8045:] If the terminal is not already
``open'', do some initialisation of
the ``tty'' structure ..

\item[8046:] ``t\_state'' is set to show the file
is ``open'', so that the next three
lines will not be executed if the
file is opened a second time,
possibly undoing the effect of a
``stty'' system call:

``t\_state'' is also set to show
``CARR\_ON'' (``carrier on''). This is
a software flag which shows that
the terminal is logically
enabled, regardless of the true
hardware status of the terminal.
If ``CARR\_ON'' is reset for a terminal, the system {\bf should} ignore
all input from the terminal.

(This does not seem to be
entirely true, and this point
will be taken up again later.);

\item[8047:] The standard terminal is assumed
to be unable to interpret horizontal tabs, to support only the
64 character ASCII subset, to run
in full duplex mode and to
require both ``carriage return''
and ``line feed'' characters to
provide normal ``new line'' processing. (Could this be a Model
33 teletype?);

\item[8048:] The ``erase'' and ``kill'' characters
are set according to the UNIX
convention;

\item[8051:] The Receiver Control Status
register is initialised with the
pattern ``0103'' so that the terminal is made ready, reading is
enabled and receiver interrupts
are enabled;

\item[8052:] The Transmitter Control Status
register is initialised so that
an interrupt will be generated
whenever the interface is ready
to receive another character.
\ed

Note that the ``open'' routine does not
distinguish between the cases where the
file is opened for reading only, or
writing only, or for both reading and
writing.

\sbs{klclose (8055)}

\bd
\item[8057:] Find the address of the appropriate ``tty''
structure in the array
of such structures, ``kl11''
(8015). (This operation may be
observed in all the procedures in
the second column of Sheet 80,
and its relevance should be
noted.)

\item[8058:] ``wflushtty'' (8217) allows   the
output queue for the terminal to
``drain'' and then flushes the
input queue

\item[8059:] ``t\_state'' is reset so that ``ISOPEN''
and\\ ``CARR\_ON'' are no longer
true.
\ed

\sbs{klxint (8070)}

This procedure is executed in response
to a transmitter interrupt. It should
be compared with ``pcpint'' (8739) and
``lpint'' (8976). Note that the parameter
``dev'' is a masked version (low order
five bits preserved) of the ``new processor status'' word in the interrupt
vector. Provided the vector was properly initialised, the minor device
number will be properly identified.

The second part of the test on line
8074 will be discussed at the end of
the next chapter.

\sbs{klrint (8078)}

This procedure is executed in response
to a receiver interrupt. It is not so
readily compared with ``pcrint'' (8719)
although similarities certainly exist.

\bd
\item[8083:] Read the input character from the
Receiver Data Buffer register;

\item[8084:] Enable the receiver for the next
character;

\item[8085:] The comment says ``hardware
botch''. Better believe it;

\item[8086:] Pass the character to ``ttyinput''
to insert it into the appropriate
``raw'' input queue.
\ed

%
% The Lion's Commentary, file ch25.tex, version 1.4, 18 May 1994
%
\se{The File ``tty.c''}

In this, the last chapter, the intricacies of interactive terminal handlers
are finally unveiled, including:

\bd
\item[(a)] the handling of the ``erase'' and
kill characters;

\item[(b)] the conversion of characters
during input and output for
upper case only terminals;

\item[(c)] the insertion of delays after
various special characters such
as ``carriage return''.
\ed

The routines ``gtty'' (8165), ``stty''
(8183), ``sgtty'' 82a1) and ``ttystty''
(8577) were dealt within the previous
chapter.

\sbs{flushtty (8252)}

The purpose of this procedure is to
``normalise'' the queries associated with
a particular terminal. Its effect is
to terminate transmission to the
terminal forthwith and to throw away any
accumulated input characters.

\bd
\item[8258:] Throw away everything in the
``cooked'' input queue;

\item[8259:] ditto for the output queue;

\item[8260:] Wakeup any process waiting to
extract a character from the
``raw'' input queue;

\item[8261:] ditto for the output queue;

\item[8263:] Raise the processor priority to
prevent an interrupt from the
terminal while ...

\item[8264:] the ``raw'' input queue is flushed,

\item[8265:] the ``delimiter count'' is properly
set to zero.
\ed

\noindent ``flushtty'' is called by ``wflushtty''
(see below) and ``ttyinput'' (8346, 8350)
when either:

\bd
\item[(a)] the terminal is not operating in
``raw'' mode and a ``quit'' or
``delete'' character is received
from the terminal; or

\item[(b)] the ``raw'' input queue has grown
unreasonably large (presumably
because no process is reading
input from the terminal);
\ed

\sbs{wflushtty (8217)}

This procedure waits until the queue of
characters for a terminal is empty
(because they've all been sent!) and
then calls ``flushtty'' to clean up the
input queues.

``wflushtty'' is called (3053) by
``klclose''. This does not happen very
often -- in fact only when all files
referencing the terminal are closed
i.e. usually only when the user logs
off.

It is also called by ``ttystty'' (8589)
just before the terminal environment
parameters are adjusted.

\sbs{Character Input} 

For a program requesting input from a
terminal, there is a chain of procedure
calls which extends to ``ttread'' ...

\sbs{ttread (8535)}

\bd
\item[8541:] Check that the terminal is
{\bf logically} active;

\item[8543:] If there are characters in the
``cooked'' input queue {\bf or} a call on
``canon'' (8274) is successful ...

\item[8544:] transfer characters from the
``cooked'' input queue until either
it is empty or enough characters
have been transferred to suit the
user's requirements.
\ed

\sbs{canon (8274)}

This procedure is called by ``ttread''
(8543) to transfer characters from the
``raw'' input queue to the ``cooked'' input
queue (after processing ``erase'' and
``kill'' characters and, in the case of
upper case only terminals, processing
``escaped'' characters, i.e. characters
preceded by the character `\verb+\+'). ``canon''
returns a non-zero value if the
``cooked'' input queue is no longer
empty.

\bd
\item[8284:] If the number of delimiters in
the ``raw'' input queue is zero
then ...

\item[8285:] if the terminal is logically
inactive, then just return;

\item[8286:] otherwise go to ``sleep''.
\ed

\noindent Note that delimiters in this context
are characters of all ones (octal value
is 377) and are inserted by ``ttyinput''
(8358).

\bd
\item[8291:] Set ``bp'' to point to the third
character of the work array,
``canonb'';

\item[8292:] Begin a loop (extending to line
8318) which removes one character
from the ``raw'' queue per cycle;

\item[8293:] If the character is a delimiter,
reduce the delimiter count by one
and exit the loop i.e. go to line
8319;

\item[8297:] If the terminal is not operating
in ``raw'' mode ...

\item[8298:] If the previous character (note
the ``bp[--1]'' notation!) was not a
backslash, `\verb+\+', execute the code
from line 8299 to 8307, otherwise
execute the code beginning at
line 8309.
\ed

\noindent {\bf Previous character was not a backslash}

\bd
\item[8299:] If the characters is an ``erase'' and ...

\item[8300:] if there is at least one charater to
erase, backup the pointer ``bp'';

\item[8302:] Start on the next cycle of the loop
beginning at line 8292;

\item[8304:] If the character is a ``kill'',
throw away all the characters
accumulated for the current line,
by going back to line 8290;

\item[8306:] If the character is an ``eot'' (004)
(usually generated at the terminal as ``control-D''),
ignore it (and do not put it inot ``canonb'') and
start on the next cycle;

(If this character occurs at
the beginning of a line, then
subsequently ``ttread'' (8544) will
find no characters in the
``cooked'' input queue i.e. it will
read a zero length record, which
then leads to the program receiving the normal ``end of file''
indication.)
\ed

\noindent{\bf Previous character was a backslash}

\bd
\item[8309:] If ``maptab[c]'' is non-zero, and either\\
``maptab[c] == c'' or the
terminal is upper case only, then ...

\item[8310:] if the last character but one was
{\bf not} a backslash (`\verb+\+'), then
replace ``c'' by ``maptab[c]'' and
back up ``bp'' (so that the
backslash will be erased).
\ed

\noindent{\bf Character ready}

\bd
\item[8315:] Move ``c'' into the next character
in ``canonb'', and if this array is
now full, leave the loop.
\ed

\noindent{\bf Line completed}

\bd
\item[8319:] At this point, an input line has
been assembled in the array
``canonb'';

\item[8322:] Shift the contents of ``canonb''
into the ``cooked'' input queue,
and return a ``successful'' result.
\ed

\sbs{Notes}

\bd
\item[(A)] The reason why ``bp'' starts (8291)
at the third character of ``canonb'' can
be found on line 8310.

\item[(B)] A number of subtleties in
the handling of backslashes (which the reader
will no doubt have encountered in his
practical use of UNIX) are still not
immediately      apparent.        Since
``maptab[c]'' is zero for ``c == `\verb+\+' ''
(octal value of 134), all backslashes
get copied into ``canonb''. A single
backslash will be subsequently overwritten if the following character is
to be asserted (as in the case of `\#'
or `@' or eot (004), or if the case of
an alphabetic character is to be
changed for an upper case only terminal.
\ed

\sbs{ttyinput (8333)}

``canon'' removes characters from the
``raw'' input queue. They are put there
in the first place by ``ttyinput'' which
is called by ``klrint'' (8087) whenever
an input character is received from the
hardware controller.

The parameters passed to ``ttyinput'' are
a character and a reference to a ``tty''
structure.

\bd
\item[8342:] If the character is a ``carriage
return'' and the terminal operates
with a ``carriage return'' only
(instead of a ``carriage return''
``line feed'' pair) change the
character to a ``new line'';

\item[8344:] If the terminal is not operating
in ``raw'' mode and the character
is a ``quit'' or ``delete'' (7958)
then call ``signal'' (3949) to send
a software interrupt to every
process which has the terminal as
its controlling terminal, flush
all the queues associated with
the terminal, and return;

\item[8349:] If the ``raw'' input queue has
grown excessively large, flush
{\bf all} the queues for the terminal
and return. (This may seem a
trifle harsh at first sight but
it will usually be what is
required.);

\item[8353:] If the terminal has a limited
character set, and the character
is an upper case alphabetic,
translate it into lower case;

\item[8355:] Insert the character into the
``raw'' input queue;

\item[8356:] If the terminal is operating in
``raw'' mode, or the character was
a ``new line'' or ``eot'' then ...

\item[8357:] ``wakeup'' any process waiting for
input from the terminal, place a
delimiter character (all ones)
also in the ``raw'' queue and
increment the delimiter count
Note this is one point where possible failure of ``putc'' (when
there is no buffer space) is
explicitly recognised. A failure
occurring here would explain why
the test on line 8316 may sometimes succeed.

\item[8361:] Finally, if the input character
is to be echoed i.e. the terminal
is running in full duplex mode,
insert a copy of the character
into the output queue, and and
arrange to have it transmitted
(``ttstart'') back to the terminal.
\ed

\sbs{Character Output -- ttwrite (8550)}

This procedure is called via ``klwrite''
(8067) when output is to be sent to the
terminal.

\bd
\item[8556:] If the terminal is
inactive, do nothing;

\item[8558:] Loop for each character to be
transmitted ...

\item[8560:] While there are still an adequate
number of characters queued for
transmission to the terminal ...

\item[8561:] call ``ttstart'' just in case it is
time to send another character to
the terminal;

\item[8562:] Setting the ``ASLEEP'' flag here
(also in\\
``wflushtty'' (8224)) is
rather pointless since is is
never interrogated and never
reset until the file is closed;

\item[8563:] Go to sleep. In the meanwhile the
interrupt handler will be draining characters from the output
queue and sending them down the
line to the terminal;

\item[8566:] Call ``ttyoutput'' to insert the
character in the output queue and
arrange to have it transmitted;

\item[8568:] Call ``ttstart'' again, for luck.
\ed

\sbs{ttstart}

This procedure is called whenever it
seems reasonable to try and send the
next character to the terminal. It
often achieves nothing useful.

\bd
\item[8514:] See the comment on line 8499.
This code is not relevant here;

\item[8518:] If the controller is not ready
(i.e. bit 7 of the transmitter
stalus register is not set) or
the necessary delay following the
preious character has not yet
elapsed, do nothing;

\item[8520:] Remove a character from the output queue. If ``c'' is positive,
the queue was not empty (as
expected) ...

\item[8521:] If ``c'' is less than ``0177'' it is
a character to be transmitted ...

\item[8522:] After setting the parity bit from
the corresponding element of the
array ``partab'', write ``c'' to the
transmitter data buffer register
to initiate the hardware operation;

\item[8524:] Otherwise (``c'' $>$ 0177)
the character was inserted in the output
queue to signal a delay. Call
``timeout'' (3845) to make an entry
in the ``callout'' list. The
result of this will be to initiate an execution of ``ttrstrt''
(8486) after ``c \& 0177'' clock
ticks . It will be seen that
``ttrstrt'' calls ``ttstart'' again,
and that the manipulation of the ``TIMEOUT''
flag (8524, 8491) will
ensure that if another execution
of ``ttstart'' is initiated in the
interim, on behalf of the same
terminal, it will (8518) return
without doing anything.
\ed

\sbs{ttrstrt (8486)}

See the comment above for line 8524.

\sbs{ttyoutput (8373)}

This procedure has more comments in the
source code and hence requires less
explanation than some others. Note the
use of recursion (8392) to generate a
string of blanks in place of a tab
character. Other recursive calls are
on lines 8403 and 8413.

\sbs{Terminals with a restricted character set}

\bd
\item[8400:] ``colp'' points to a string of
pairs of characters. If the character to be output matches the
second character of any of these
pairs, the charactcr is replaced
by a backslash followed by the first character of the pair;

\item[8407:] Lower case alphabetics are converted to upper case alphabetics
by the addition of a constant.
\ed

\noindent Note. The conversion here should be
compared wth the handling of the
reverse problem on input. Here we have
an algorithm which clearly trades space
(no table analogue to ``maptab'') for
time (a serial search through the
string on line 8400). A space conserving
approach could be adopted in
``canon'' but the problem is rather more
complicated there.

\bd
\item[8414:] Insert the character into the
output queue. If perchance,
``putc'' fails for lack of buffer
space, don't worry about inserting any subsequent delay, or
updating the system's idea of the
current printing column;

\item[8423:] Set ``colp'' to point to the
``t\_col'' character of the ``tty''
structure, i.e. ``*colp'' has a
value which is the ordinal number
of the column which has just been
printed;

\item[8424:] Set ``ctype'' to the element of
``partab'' corresponding to the
output character ``c'';


\item[8426:] Mask out the significant bits of
``ctype'' and use the result as the
``switch'' index;

\item[8428:] (Case 0) The common situation!
Increment ``t\_col'';

\item[8431:] (Case 1) Non-printing characters.
This group consists of the first,
third and fourth octet of the
ASCII character set, plus ``so''
(016), ``si'' (017) and ``del''
(0177). Don't increment ``t\_col'';

\item[8434:] (Case 2) Backspace. Decrement
``t\_col'' unless it is already
zero;

\item[8439:] (Case 3) Newline. Obviously
``t\_col'' should be set to zero.
The main problem is to calculate
the delay which should ensue
before another character is sent.

For a Model 37 teletype, this
depends on how far the print
mechanism has progressed across
the page. The value chosen is at
least a tenth of a second (six
clock ticks) and may be as much
as ((132/16) + 3)/60 =   0.19
seconds.

For a VT05, the delay is 0.1
second. For a DECwriter it is
zero because the terminal
incorporates buffer storage and
has a double speed ``catch up''
print mode;

\item[8451:] (Case 4) Horizontal tab. Assign
the value of bits 10, 11 of
``t\_flags'' to ``ctype'';

\item[8453:] For the only non-trivial case
recognised\\
(``c'' ==  1 or Model 37
teletype), calculate the the
number of positions to the next
tab stop (via the obscure calculation of line 8454). If this
turns out to be four columns or
less, take it as zero;

\item[8458:] Round ``*colp'' (i.e. the value
pointed to by ``colp''!) to the
next multiple of 8 less one;

\item[8459:] Increment ``*colp'' to be an exact
multiple of eight;

\item[8462:] (Case 5) Vertical Motion. If bit
14 is set in ``t\_flags'', make the
delay as long as possible, i.e.
0177 or 127 clock ticks, i.e.
just over two seconds;

\item[8467:] (Case 6) Carriage Return. Assign
the value of bits 12, 13 of
``t\_flags'' to ``ctype'';

\item[8469:] For the first class, allow a
delay of five clock ticks;

\item[8472:] For the second class, allow a
delay of ten clock ticks;

\item[8475:] Set the ``*colp'' (the last column
printed) to zero.
\ed

Before leaving the file ``tty.c'', there
are two matters which deserve further
examination.

\sbs{A. The test for 'TTLOWAT' (Line 8074)}

On line 8074 in ``klxint'', a test is
made whether to restart any processes
waiting to send output to the terminal.
The test is successful if the number of
characters is zero or if it is equal to
``TTLOWAT''.

If the number of characters is between
these values, no ``wakeup'' is performed
until the queue is completely empty,
with the strong likelihood that there
will then be a hiatus in the flow of
output to the terminal. Since temporary interruptions to the flow of
output are quite frequently observed in
practice and represent a source of
occasional irritation if nothing more,
one may reasonably enquire ``is there
any way the character count can get
from being greater than ``TTLOWAT'' to
below it, without this being detected
at line 8074?''

Quite clearly there is, since each call
on ``ttstart'' can decrement the queue
size, and only one such call is followed by the test. Thus if the call on
``ttstart'' from one of ``ttrstrt'' (8492)
or ``ttwrite'' (8568) happens to cross
the boundary, a delay will result. The
probability that this will happen is
small, but finite and hence the event
is likely to be observed in any reasonably long output sequence.

There are two other situations in which
``ttstart'' is called which seem to be
satisfactory. At ``ttwrite'' (8561) the
queue is at its maximum extent; and at
``ttyinput'' (8363) there is a preceding
call on ``ttyoutput'' which usually (but
not invariably!) will have added a
character to the output queue.

\sbs{B. Inactive Terminals}

When the last special file for a terminal is closed, ``klclose'' (8055) is
called and resets (8059) the ``ISOPEN''
and ``CARR\_ON'' flags. However the ``read
enable'' bit of the receiver control
status register is not reset, so that
incoming characters may still be
received and will be stored away (8087)
in the terminal's ``raw'' input queue by
``klrint'' (8078), and ``ttyinput'' (8333),
which do not test the ``CARR\_ON'' flag,
to see if the terminal is logically
connected.

These characters may accumulate for a
long time and clog up the character
buffer storage. Only when the ``raw''
input queue reaches 256 characters
(``TTYHOG'', 8349) will the contents of
this queue be thrown away. It does seem
therefore, that a statement to disable
reader interrupts should be included in
``klclose'' before line 8058.

\newpage
\vfill

\sbs{Well, that's all, folks ...}

Now that you, oh long-suffering,
exhausted reader have reached this
point, you will have no trouble in
disposing of the last remaining file,
``mem.c'' (Sheet 90). And on this note,
we end this discussion of the UNIX
Operating System Source Code.

Of course there are lots more device
drivers for your patient examination,
and in truth the whole UNIX Timesharing System Source Code has hardly
been scratched. So this is not really

\begin{center}
{\bf \large THE END}
\end{center}

%
% The Lion's Commentary, file ch26.tex, version 1.4, 15 May 1994
%
\se{Suggested Exercises}

Any operating system design involves
many subjective and ad hoc judgements
on the part of system's designers. At
many places in the UNIX source code
you will find yourself wondering ``Why
did they do it that way?'', ``What would
happen if I changed this?''

The following exercises express some of
these questions. Some can be answered
from an examination of the source code
alone after a study in more depth; others require some experimental probing
and measurement, for which read-only
access to the file ``/dev/kmem'' via terminal will prove invaluable; and still
others really require the construction
and testing of experimental versions of
the operating system.

\sbs{Section One}

\bd
\item[1.1] Devise changes to ``malloc'' (2528)
to implement the Best Fit algorithm.

\item[1.2] Rewrite the procedure ``mfree''
(2556) to render its function more
easily discernible by the reader.

\item[1.3] Investigate the adequacy of the
sizes of the arrays ``coremap'' and
``swapmap'' (0203, 0204).   How should
``CMAPSIZ'' and ``SWAPSIZ'' change when
``NPROC'' is increased?

\item[1.4] Prove that ``malloc'' and ``mfree''
jointly solve the memory aliocation
problem correctly.

\item[1.5] By monitoring the contents of
``coremap'', estimate the efficiency with
which main memory is utilised. Estimate also the cost of compacting ``in
use areas'' of main memory from time to
time to reduce memory fragmentation.

Hence decide whether it would be
worthwhile to extend the present memory
allocation scheme to include memory
compaction.

\item[1.6] In setting the first six kernel
page description registers, UNIX does
not make use of all the hardware protection features that are available
e.g. some pages which contain only pure
text could be made read-only. Devise
changes to the code to maximise the use
of the available hardware protection.

\item[1.7] Compile the program

\begin{verbatim}
    char *init ``/etc/init'';
    main ( ) {
    execl (init, init, 0);
    while (1);
    }
\end{verbatim}

and compare the result with the contents of the array ``icode'' (1516).

\item[1.8] Investigate the size required for
kernel mode stack areas. Hence show
that the 367 word area which is provided is adequate.

\item[1.9] If main memory consists of several
independent memory modules and one of
these, not the last, is down, ``main''
will not include memory modules beyond
the one which is down, in the list of
available space in ``coremap''. Devise
some simple changes to ``main'' to handle
this situation. what other parts of the
system would also need revision?

\item[1.10] Rewrite the routines ``estabur''
(1650) and ``sureg'' (1739) so that they
will work as efficiently as possible on
the PDP11/40. How often are these routines used in practice? Would it really
be worthwhile trying to implement your
improved versions?

\item[1.11] Investigate the overheads involved
in initiating a new process. Perform a
series of measurements for a set of
different sized programs under different conditions.

\item[1.12] Evaluate the following scheme
which is intended by Ken Thompson as
the basis for a revised scheduling
algorithm:

A number ``p'' is kept for each process, stored as ``p\_cpu''.
``p'' is incremented by one every clock tick that the
process is found to be executing. ``p''
therefore accumulates the CPU usage.
Every second, each value of ``p'' is
replaced by four fifths of its value
rounded to the nearest integer. This
means that ``o'' has values which are
bounded by zero and the solution of the
equation  $k = 0.8*(k + HZ)$ i.e.
4*HZ. Hence if HZ is 50 or 60, and ``p''
is integerised, ``p'' can be stored in
one byte.

\item[1.13] The ``proc'' table is always
searched via a direct linear search. As
the table size is increased, the search
overheads also increase. Survey the
alternatives for improving the search
mechanism, when ``NPROC'' is say 300.
\ed

\sbs{Section Two}

\bd
\item[2.1] Explain in detail how the system
reacts to a floating point trap which
occurs when the processor is in kernel
mode.

\item[2.2] When a process dies, a ``zombie''
record is written to disk, and is
subsequently read back by the parent. Devise a scheme for passing back the
necessary information to the parent
which will avoid the overhead of the
two i/o operations.

\item[2.3] Document ``backup'' (1012).

\item[2.4] It is relatively easy using the
``shell'' to set up a set of asynchronous
processes which will flood your terminal with useless output. Trying to stop
these processes individually can be a
problem, since their identifying
numbers may not be known. Use of the
command ``kill 0'' is usually an act of
sheer desperation. Devise an alternative scheme, e.g. based on the use of
messages such as ``kill --99'', which will
be effective, but more selective.

\item[2.5] Design a form of coroutine jump
whlch will cause control to pass more
efficiently between a program which is
being traced, and its parent.
\ed

\sbs{Section Three}

\bd
\item[3.1] Rewrite the procedure ``sched'' to
avoid the use of ``goto'' statements.

\item[3.2] Modify ``sched'' so that the text
segment and data segment for a program
will possibly be allocated in separate
main memory areas if a single large
area is not immediately available.

\item[3.3] If the system crashes and must be
``rebooted'' the contents of the buffers
which were not written out at the time
of the crash are lost.

However if a core dump is taken,
the contents of the buffers can be
obtained and hence the contents of the
disk can be brought completely up to
date. Outline a detalled plan for carrying out this scheme.
How effective do you think it would be?

\item[3.4] Explain why the buffer areas
declared on line 4720 are 514, and not
512, characters long.

\item[3.5] Explain how deadlock situations may
arise if there are too few ``large''
buffers available. What measures can
you suggest to alleviate the problem,
assuming that increasing the number of
buffers is not possible.
\ed

\sbs{Section Four}

\bd
\item[4.1] Devise a scheme for labelling file
system volumes and checking these
labels when the volumes are mounted.

\item[4.2] Discuss the problems of supporting
ANSI standard labelled tapes under
UNIX, and propose a solution.

\item[4.3] Design a scheme for providing index
sequential access to files.

\item[4.4] The emergence of the ``sticky bit''
(see ``CHMOD(I)'' in the PM) confirms
that there are some residual advantages
in allocating all the space for a file
contiguously. Discuss the merits of
making ``contiguous files'' more generally available.

\item[4.5] Devise a technique to measure the
efficiency of pipes. Apply the technique and report your results.

\item[4.6] Devise modifications to ``pipe.c''
which will make pipes more efficient
according to the following scheme:
whenever the ``read'' pointer is greater
than 512, rotate the non-null block
numbers in the ``inode'' ana decrease
both the ``read'' and ``write'' pointers by
512.

\item[5.1] By monitoring the number of free
buffers or otherwise, determine whether
the number of character buffers provided at your installation is adequate.

\item[5.2] Perform measurements and/or experiments
to determine whether the character buffer
blocks would be more efficiently utilised if they consisted of
four or eight characters, rather than
six, per block.

\item[5.3] Redesign the line printer driver to
handle overprinting and backspacing
more efficiently in the sense of
minimising the number of print cycles.

\item[5.4] Document ``mmread'' (0916) and
``mmwrite'' (9042).
\ed
\sbs{General}

\bd
\item[6.1] The easiest way to vary the main
memory space used by the operating system is to
vary ``NBUF''. If this is forbidden, propose the best way to:

\bd
\item[(a)] reduce the space required by 500
words;

\item[(b)] utilise an additional 500 words.
\ed

\item[6.2] Discuss the merits of ``C'' as a systems programming language. What
features are missing? or superfluous?
\ed

\end{document}
