\startcomponent cons
\product cl-impl-analysis

\chapter{构造过程}

CL 实现的构造过程可以分成两类:一类尽量用 Lisp 实现一切,一类则
用 C 等其它语言实现基础功能。Kea-CL, CMU CL, SBCL, Clozure CL 属于前者。

从零实现 Common Lisp,至少要具备如下功能:

\startitemize[1,packed,broad]
\item 整型、字符、字符串、symbol、function、package、cons、list 等数据类型
\item read, readtable 以及输入流和输出流
\item eval, 三种 form 的求值规则
\stopitemize

\section{Parrot CL}

Parrot CL 使用 PIR (Parrot Intermedia Representation, 相当于 Parrot 上的
宏汇编语言) 实现了上述三条。


\section{Kea-CL}

Kea-CL 汲取了 CMU CL 的实现思路,首先用 Lisp 实现一个 lisp -> PIR 的编译
器,借助已有的 CMU CL 运行这个编译器,将编译器本身编译到 PIR,然后用这个
编译器编译 CL runtime,此时得到一个 CL 实现,就可以用这个 CL 实现编译
lisp -> PIR 编译器,从而实现自举。源码中的 kea-cl.pod 和 README.text 描述了实现
过程。

\section{CMU CL}

\section{SBCL}

\section{CCL}

\stopcomponent

