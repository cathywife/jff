\environment env

\useURL[author-email][lyanry@gmail.com][][lyanry@gmail.com]
\useURL[wiki][http://wiki.contextgarden.net][][Wiki]
\useURL[yanry][mailto:lyanry@gmail.com][][lyanry@gmail.com]

\starttext

\title{你好,\CONTEXT!}

这是一份用于展示 \CONTEXT\ 基本排版功能的示范文档,希望你能喜欢。更多的 \CONTEXT\ 排版知识请查阅 \from[wiki],绿色的文本是可点击的,它可以带你去那里。

\subject{这是列表}

下面是一份列表排版示例:

\startitemize[n]
\item 吾生有崖,而知无崖,以有崖求无崖,殆哉矣
\item 居然真的会有人相信鲨鱼鳍和鸟口水有什么营养价值
\item 月球上是看不到长城的,也许只能看到伟大的防火墙
\stopitemize

\subject{程序代码}

若要在文档中排版一些程序代码,行内代码需要用~\type{\type{...}}~控制序列,行间代码需要用:

\starttyping

\starttyping

我是代码 wa ha ha

\stoptyping

\stoptyping

\subject{数学}

行内的公式是像 $e^{\pi i}+1=0$ 这个样子的,行间的公式则是这样:

\startformula
\int_0^\infty t^4 e^{-t}\,dt = 24.
\stopformula

还可以带编号:

\placeformula[eq:factorial-example]
\startformula
\int_0^\infty t^5 e^{-t}\,dt = 120.
\stopformula

可通过公式名去引用它。例如引用上述公式 {\tt factorial-example},即公式 \in[eq:factorial-example],\CONTEXT\ 会自动确定其编号,并且在交互模式开启的情况下,可以点击公式编号,它会将你带往那个公式的所在位置。

\subject{图文并茂}

现在看一下如何在文档中插入图片。

\placefigure[right,none][fig:pixiu]{}{\externalfigure[img/dujiao][width=.15\textwidth]}

相传貔貅是一种凶猛瑞兽,而这种猛兽分为雌性及雄性,雄性名貔,雌性名貅。但现在流传下来的都没有分为雌雄了。在古时这种瑞兽是分一角和两角的,一角的称为天禄,两角的称为辟邪。在《汉书•西域传》上有这样的记载:“乌戈山离国有桃拔、狮子、犀牛。”孟康注曰:“桃拔,一曰符拔,似鹿尾长,独角者称为天鹿,两角者称为辟邪。”辟邪便是貔貅了。山海经记载貔貅:“龙头、马身、麟脚、形如狮,会飞。”后来再没有分一角或两角,多以一角造型为主。 在南方,一般人是喜欢称这种瑞兽为貔貅,而在北方则依然称为辟邪。貔貅是以财为食的,纳食四方之财。中国传统是有貔貅的习俗,和龙狮一样,有将这地方的邪气赶走、带来欢乐及好运的作用。

\placefigure[force][fig:pixiu]{貔貅的传统造型}{\externalfigure[img/pixiu][width=.3\textwidth]}

\vfill

\noindent 
\framed[corner=round, width=\textwidth,height=1in,backgroundcolor=gray,background=color]{这份文档采用 Public Domain 许可协议发布,因此你可以改进它,与他人分享,或者做任何你想做的事情。提意见是受欢迎的,你可以将意见发到 \from[yanry] (Li Yanrui)。}

\stoptext
