% 加载字体定义文件
\usetypescriptfile[zhfonts]

% 设定西文字体
\usetypescript[serif][pagella]
\usetypescript[sans][heros]
\usetypescript[mono][cursor]

% 加载字体定义
\usetypescript[myfont]

% 设置文档默认字体
\setupbodyfont[myfont,rm,12pt]

% 启用中文断行规则
\setscript[hanzi]

% 使用本地化 label [可选]
\mainlanguage[cn]

% 首行缩进与行间距设置
\setupindenting[always,2em,first]
\setupinterlinespace[big]
\setuphead[chapter][indentnext=yes]

\starttext

\chapter{MkIV 中文排版支持的最近更新介绍}

Wolfgang 写了一份新的 zhfonts.tex,默认定义了 pagella, palatino, termes, times 四种西文字体以及 Adobe 宋体、仿宋、楷体、黑体四种中文字体。我对这份文件进行了一点小修改,硬性设定了中文与西文字符的间距大小,并且禁止在数学公式中直接使用中文字体,其下载见附件。

Hans 对 CJK 框架进行了调整,并且已将这部分内容的变动写入了 MkIV Reference 文档中,见 http://www.pragma-ade.com/general/manuals/mk.pdf

最近,Hans 在最新的 ConTeXt Minimals Beta 版本中添加了中文 label 支持,也就是说,现在基本上可以使用“第 x 章”、"图 x"、"表 x" 这样的中文本地化 label 的内建功能了。

\stoptext
