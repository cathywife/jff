\startcomponent repository
\product git-way-explained

\section{版本库}

版本库就是 .git 目录,用于保存{\bf 全局}版本演化图的
{\bf 子集}和标签。git clone A B,然后 A 库和 B 库内
各自修改提交了,从全局看,A 的修改者和 B 的修改者
都是在扩展一个全局的版本图,只是两个库各自只包含了
这个全局版本图的子集:

\startitemize[n,packed,broad]
\item 起始情况

    A 库版本图:
\starttyping
        a -- b => master
\stoptyping

    全局版本图:
\starttyping
        a -- b
\stoptyping

\item git clone A B 后,A 和 B 各提交一次

    A 库版本图:
\starttyping
        a -- b -- c => master
\stoptyping

    B 库版本图:
%        a -- b -- c' => master
%             \
%              origin/master
\starttyping
        a -- b -- c' => master (tracking branch)
                   \
           origin/master (remote branch)
\stoptyping

    全局版本图:
%        a -- b -- c
%              \
%               c'
\starttyping
        a -- b -- c
                 \
                   c'
\stoptyping
\stopitemize
    
A 和 B 之间同步,以完善到包含全局版本图,就是通过
push 和 fetch 命令实现的(pull = fetch + merge)。

这里有几个需要注意的地方:
\startitemize[1,packed,broad]
\item head 和 tag ref 都是只用于在某一个版本库之内标记,因为
  它们用一个字符串标记版本,而不同库里这个字符串可以都
  用到,比如都有 master 分支,上面故意没有在全局版本图上
  标注 head 就是为了强调 head 和 tag ref 的私有性,实际在
  表达全局版本图时往往都会标出各自的 head;
  我们平时说 linux kernel-2.6 的上游分支,其实是因为社会
  关系,以 Linus Torvalds 那个公开代码库里的 master 分支
  为准,而纯粹从技术上讲,并不存在一个全局的 master 分支,
  这跟集中式的版本管理工具比如 Subversion 全局主分支是
  很不一样的;

\item origin/master 是 git clone 自动建立的分支,用于跟踪
  clone 来源的分支情况,存放在 .git/refs/remotes/origin/
  下,用 git fetch 或者 git remote update 更新, 在
  .git/config 中 origin 定义了对方库地址;

\item B 库里的 master 也是 git clone 自动建立的分支,用于
  跟踪 origin/master 分支的变化(git branch --track 可以用于
  建立这样的分支),在 .git/config 里定义了 master 分支的
  跟踪关系;

\item GIT 的文档里称 origin/master 这样的为 remote branch,这个
  叫法相当易混淆,从上面的图看来,分支首先分为远端库里的分支
  和本地库里的分支,之间可以重名; 其次,每一个库里的分支又
  分为 remote branch, tracking branch, normal branch。
  remote branch 完全是用来同步对方库里的分支的,因此绝对
  不要往 remote branch 上提交;
  tracking branch 在无参数的 git pull 时会在同步完 remote
  branch 后,将 remote branch 往 tracking branch 上合并,
  在多人开发时这往往会冲突,因此也不建议往 tracking branch
  上提交。

  [ XXX: tracking branch 太无厘头了, 实际上 remote branch
   改叫 tracking branch 才合适,而原来的 tracking branch 以及
   pull 命令不应该存在]

\item git branch -a 显示的 remote branch 列表并没有 remotes 前缀,
  而在 git 里分支只是用 .git/refs/heads 下某个文件记录的,因此
  可以建立一个 origin/master 分支,这样会导致 git branch -a
  的输出很迷惑人,可以用 remotes/origin/master 来显式指代
  remote branch。

  [ XXX: refs, refs/heads, refs/tags 都可以省略,又加上
   symbolic ref 添乱,太容易弄迷糊了]

\stopitemize

理解 GIT 里的 branch(准确说是 branch head) 只是各人在全局版本图上的动态
标签,以及 GIT repository 只是保存了这个全局版本图的子集,对于理解分布式
版本控制中的分支关系是非常重要的,千万不要把 GIT 里的 branch 理解成像
Subversion 里那样一个线性的版本序列。GIT 里所称呼的 remote branch,不过
是私有的一个标记,用来跟踪其它代码库里的某个 head,remote branch 需要
不时的用 git fetch 同步才能和对应的 head 指向{\bf 全局}版本图中的同一个版本。

\stopcomponent

