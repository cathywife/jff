\startcomponent cache
\product git-way-explained

\section{混乱之源——index}

在 git 里文件在逻辑上有三个地方保存:repository, index,
work tree。repository 就是指 .git,这里保存了提交过的各个
版本的文件。work tree 就是工作目录,index 用来暂存修改
(实际只是存了个索引 .git/index,真实数据在 .git/objects/ 目录下)
以及记录工作目录里文件状态,其实早先 index 就被称做 cache,
现在在命令行选项中也能看到 cache 的字样。

正是 index 的这个暂存修改功能,导致 git 比起 svn 复杂
多了,svn 里的 .svn 只用于记录工作目录里文件状态,并保存
一份干净的基础版本。

下面列举一下涉及 index 的几个命令,你就能明白为什么有很多人都建议
对 GIT 用户隐藏 index 缓存文件这个机制了。

\definedescription[gitcmd][location=top,headstyle=bold,width=broad]
\startgitcmd{git add}
将文件从 work tree 加入 index。注意与 subversion 的 add 命令
不同,git add 是把文件内容加入 index,在每次修改 work tree
中文件后,都要 git add 才能让 git commit 把修改提交到 repository 里。
\stopgitcmd

\startgitcmd{git rm}
git rm 是从 index 和 work tree 中删除文件,git rm --cached 只从
index 中删除文件。
\stopgitcmd

\startgitcmd{git commit}
git commit 是把 index 中文件提交到 repository 里,git commit -a 是
把 work tree 中所有修改过的文件都 git add 到 index 里,然后把 index
中文件提交到 repository 里。
\stopgitcmd

\startgitcmd{git reset}
git reset <commit> -- paths... 是从 repository 中的 commit 版本取 paths
指定的文件覆盖 index 中的对应文件,也就是撤销 index 中暂存的修改。注意
输出信息中的“needs update”是说文件目前状态没有被“git update-index”命令
更新到 index 中,此处的“update”跟 CVS、Subversion 里的 update 完全
是两码事。

git reset --hard <commit> 是取 repository 中的 commit 版本覆盖 index 和
work tree 中的所有文件,也就是撤销 index 和 work tree 中的修改。注意这个
命令不允许指定 paths。
\stopgitcmd

\startgitcmd{git checkout}
git checkout <commit-ish> 是切换分支,取 repository 中指定版本的文件
覆盖 index 中文件,并覆盖 work tree 中文件(如果有区别要用 -f 强制覆盖,
用 -m 来与 work tree 合并)。

git checkout -- paths... 从 index 取文件覆盖 work tree 中文件。

git checkout <tree-ish> -- paths... 从 repository 中的 commit 版本取文件
覆盖 index 以及 work tree 中文件。

需要注意的是与 Subversion 里的 revert 命令等价的是
\starttyping
git reset -- paths... && git checkout -- paths...
\stoptyping
或者
\starttyping
git checkout HEAD -- paths...
\stoptyping
而 git revert 相当于 Subversion 里撤销已经提交的修改:提交一个新版本,
新版本做的修改等效于撤销指定版本的修改。

在用 git checkout 的第一种形式切换分支时,如果工作目录中有修改没提交,
那么最好用 git stash 先暂存起来,因为可能出现这样的尴尬境地:源分支
有文件 A、B、C,其中 C 有修改没提交,目标分支中 A、B 、C 都被修改或者删除,
那么切换分支会因为 C 的修改而告失败(默认不会覆盖或者合并),但在处理
C 时已经处理完 A 和 B,所以最后分支没切换成功,但 git status 显示 A、B、
C 都被修改了(A、B 是中断的分支切换操作误引起的,C 是切换分支前的正常修改
),如果文件很多,你就分不出来哪些是切换分支前修改过的。此时的挽救办法是
执行 git diff --name-status <目标分支>,输出的就是切换分支前修改的文件,
排除这些,git status 显示的就是失败的分支切换误引起的。
\stopgitcmd

\startgitcmd{git status}
git status 的输出分为三个部分,第一部分是“Changes to be committed”,
列举的是 index 中的文件状态,用不带 -a 的 git commit 会把这些文件
提交到 repository 中;第二部分是“Changed but not updated”,这部分
列举的是 work tree 中被 git 管理的文件状态,用 git add 会把这些文件
加入到 index 中,这里的“update”一词来源于 GIT 底层命令 update-index;
第三部分是“untracked files”,这部分列举的是 work tree 中没有被 git 管理的文件。

在 git status 第一部分输出中有“unstage”这个词,“stage”这个字眼
是为了避免“index”这个太易混淆的词,stage 是指 git add/rm,unstage 指
git reset,staged files 就是指 index 中暂存的文件修改。
\stopgitcmd

\startgitcmd{git diff}

由于在 repository 中有多份文件树,在 index 和 work tree
中各有一份文件树,因此 diff 的功能非常丰富:

git diff 比较 index 和 work tree

git diff --cached tree-ish 比较 index 和 tree-ish,后者缺省为 HEAD

git diff tree-ish 比较 tree-ish 和 work tree

git diff tree1-ish tree2-ish 比较两个 tree

tree-ish 可以是文件树的一个部分,比如
\starttyping
git diff v2.6.21:init v2.6.27:init
\stoptyping
\stopgitcmd

上面命令末尾都可以添加 -- path...,表示只输出牵涉到这些路径的
结果。

[ XXX: 繁复,无视正交设计原则。其实可以直接用一个 index 引用来标记,
限制用户不能自己建立 index 引用, 类似于 Subversion 的 -r BASE 里的 BASE,
造成现在这个局面无非是懒:懒输入 index 所以就用隐含参数和选项,
懒得输入 -r 所以要用 -- 分隔版本号和文件名 ]

\stopcomponent

