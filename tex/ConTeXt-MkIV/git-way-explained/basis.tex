\startcomponent basis
\product git-way-explained

\section[basis]{版本控制的基本概念}

版本控制(version control)的目的就是保存一个文件树在不同时刻的状态,其基
本概念是相当直白的:

\definedescription[concept][location=top,headstyle=bold,width=broad]
\startconcept{文件(file)}
普通的包含数据的文件。

用途:保存数据。
\stopconcept

\startconcept{目录(directory)}
文件名、目录名的集合,目录的这种递归包含结构构成了整个文件树。

用途:保存文件路径清单。
\stopconcept

\startconcept{版本(version, revision)}
记录某一时刻整个文件树的状态,并且会包含一些额外信息,比如时间、作者、备
注、从哪个版本演化而来的。版本的记法有很多种,比如用递增的数字标记
(Subversion),用点分数字如1.2.3 标记(CVS, RCS),用版本信息的摘要值标记
(GIT, Mercurial)等等。

用途:记录文件树状态变更整个过程,文件树状态演化形成一个单向无循环图
(DAG)。
\stopconcept

\startconcept{标签(tag, label)}
用一个容易记忆的名字标记某个版本。

用途:帮助记忆。
\stopconcept

标签可以分成两种:静态标签(static tag) 和浮动标签(float tag),在各种版本
控制工具中,“标签”一词往往指静态标签,而将浮动标签称为“分支头(branch
head)”,或简称为“头(head)”,分支头的名字(也即浮动标签的名字)命名了一
个分支。需要注意的是版本控制工具中往往用大写的 HEAD 指代{\em 当前}分支对
应的头部。

所谓静态标签就是指必须显式修改标签内容(比如版本控制工具的 tag 命令),才
能让其标记另外一个版本,浮动标签就是指从它指代的版本派生新版本时(比如版
本控制工具的commit 命令),标签自动被修改,以指向新版本。

而关于“分支(branch)”的定义,则是各有微妙分歧,下图中小写字母表示版本,
也就是文件树状态以及关于这个状态的备注信息,连线表示其版本变迁关系,时间
轴从左向右:

% a --- b --- c --- f --- g ==> master
%        \         /
%         d ----- e ==> test
\placefigure[][fig:revgraph]{一个简单的版本演化图}{
\starttyping
            a --- b --- c --- f --- g ==> master
                        \                /
                          d ---- e ==> test
\stoptyping
}

这个版本图中,文件树从起始的 a 版本,用 master 浮动标签来标记,由于修改
意图的分歧,在 b 版本之后分成两个修改方向,b -> c 方向仍然用 master 标记
,b->d 用 test 标记,在 c 和 e 之后,两个修改方向取长补短又合并成一个版
本 f,之后经过修改后又达到版本 g。 图中最后 master 和 test两个浮动标签分
别指代 g 和 e,它们命名了两个分支:master分支和 test 分支。

那么“分支”指代什么呢?比如 master 分支,有如下几种说法:
\startitemize[1,packed,broad]
\item 从分歧点算起,master 分支指 f, g 这两个版本,或者只算 g;
\item 指 master 这个标签曾经指向的版本,也就是 master 这个动态标签的“轨
      迹”──a, b, c, f, g;
\item 指从 master 标记的版本回溯,可以达到的版本,也即g, f, c, e, d, b,
      a (注意 f 是一次合并,它可以回溯到e 和 c)。
\stopitemize
这种分支含义的分歧直接导致版本控制工具对于“分支历史” 定义的分歧,比如
log 命令的输出。Subversion 使用第二种语意,GIT 则采用第三种语意,这两种
不同语意正好反映了集中式版本管理和分布式版本管理的核心区别:分支的地位问题。

由于{\em 分支头}命名了{\em 分支},所以往往将二者笼统的都称呼为{\em 分支}。

关于分支还有另外一个分歧,就是合并的对称性,考虑\in{图}[fig:revgraph]中
的版本图,在 e 版本的基础上合并 c(也即通常说的在 test分支{\em 上}合并分
支 master),和在 c 版本的基础上合并e(也即通常说的在 master 分支{\em 上}
合并分支 test) 所带来的结果是否等价呢?

而其实“版本”的定义也是有分歧的,从上面可以看到,标签主要是助记用的,它
的内容可以被修改,那么“版本”记录的状态里是否要包含当时标签的状态呢?
\footnote{GIT 的版本历史不包括动态标签和静态标签的状态,Subversion 则都
包括,Mercurial 只包括静态标签的状态}

各种术语称呼差别,命令叫法不一,模型定义分歧,往往导致从一种版本控制工具
切换到另一种时极为别扭,不知所措。

\stopcomponent

