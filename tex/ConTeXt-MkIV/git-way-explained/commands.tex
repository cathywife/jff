\startcomponent commands
\product git-way-explained

\section{常用命令简介}

\definedescription[gitcmd][location=top,headstyle=bold,width=broad]
\startgitcmd{git add}
往 index 里添加文件内容。加上 -i 选项可以交互式的选择需要 add 哪些内容,
细到 patch hunk 级别,非常实用。
\stopgitcmd

\startgitcmd{am}
从邮箱里提取补丁并应用。git apply-mbox 是过时的命令。
\stopgitcmd

\startgitcmd{apply}
应用补丁。
\stopgitcmd

\startgitcmd{archive}
将指定版本的文件树打包,支持 tar 和 zip 格式。
\stopgitcmd

\startgitcmd{bisect}
对半查找版本树中引入 bug 的版本,由于版本树往往不是线性的,手工记录
验证过的版本并且剪枝是很琐碎的。
\stopgitcmd

\startgitcmd{blame}
查看每一行是从哪个版本来的,谁修改的。git annotate 是过时的命令。
\stopgitcmd

\startgitcmd{bundle}
将指定版本的各种对象以及 refs 打包,方便离线开发。
\stopgitcmd

\startgitcmd{cat-file}
判断对象类型,以及原样或者格式化输出对象内容,注意对于 tree 对象
要用 ls-tree 或者 cat-file -p 或者 show 命令查看,cat-file 的原样
输出是二进制数据。
\stopgitcmd

\startgitcmd{checkout}
切换分支或者撤销修改。
\stopgitcmd

\startgitcmd{cherry-pick}
相当于将某次提交打个补丁应用到当前分支,用来从其它分支提取所需的
修改而不用合并整个分支。
\stopgitcmd

\startgitcmd{clean}
删除工作目录中没有被 git 管理的文件,非常方便。
\stopgitcmd

\startgitcmd{clone}
复制代码库,如果目标库和源库在同一台机器上,-s 选项能够很节约磁盘空间。
另外 --depth 选项可以避免复制代码库中包含的整个版本历史。
\stopgitcmd

\startgitcmd{commit}
提交一个新版本。 
\stopgitcmd

\startgitcmd{config}
配置 git 的行为,有三种配置:system 范围的,指 \type{/etc/gitconfig},global
范围的,指 \type{$HOME/.gitconfig},repository 范围的,指 \type{.git/config},
三者的优先级依次增高。config 命令默认操作的是 \type{.git/config}。repo-config
是过时的命令。 

config 命令可以用来设置命令别名,比如将 b 设置为 branch 的别名:
\starttyping
git config --global alias.b branch
\stoptyping

一份示例配置 \type{$HOME/.gitconfig},具体含义参考 \type{git help config}:
\starttyping
[user]
        email = YourEmail
        name = YourName
[color]
        ui = auto
[alias]
        ci = commit
        st = status
        co = checkout
        b = branch
[diff]
        renames = copy
[rerere]
        enabled = true
[i18n]
        commitEncoding = utf-8
        logOutputEncoding = utf-8
\stoptyping
\stopgitcmd

\startgitcmd{count-objects}
统计 loose object 个数以及所占空间。 
\stopgitcmd

\startgitcmd{describe}
将某个版本用最近的 tag 加上 \type{~} 和 \type{^} 记法来显示,方便识别,
有一点点类似 Subversion 里的整数版本号。
\stopgitcmd

\startgitcmd{diff}
查看文件内容修改情况。
\stopgitcmd

\startgitcmd{fetch}
从远端代码库同步版本图到本地代码库,并修改本地库里指定的 head 跟远端库里
对应的 head 一致,这个修改要求本地的 head 指示的分支{\bf 真包含于} 远端库
里对应 head 指示的分支,否则要加 -f 选项。
\stopgitcmd

\startgitcmd{format-patch}
将分支上的修改生成补丁序列,跟 am、imap-send、send-email 命令搭配构成了利用
Email 进行协作分布式开发的工作流程。
\stopgitcmd

\startgitcmd{gc}
由于 GIT 里存储对象分为 loose object 和 packed object,因此需要不时用
repack 将 loose objects 打包成 pack 格式存储的 packed objects,以节省
磁盘空间并提高访问效率。此外还有一些其它需要清理的文件,gc 作为一个高层
命令统管了这些杂事。
\stopgitcmd

\startgitcmd{grep}
在指定版本的文件树里搜索包含特定模式的行。
\stopgitcmd

\startgitcmd{gui}
GIT 的图形界面,个人觉得不大好用。
\stopgitcmd

\startgitcmd{init}
新建一个 GIT 版本库。init-db 是过时的命令。
\stopgitcmd

\startgitcmd{log}
查看版本历史。
\stopgitcmd

\startgitcmd{ls-files}
显示 index 和 work tree 中的文件列表,常用于 Shell 编程。
\stopgitcmd

\startgitcmd{merge}
分支合并。
\stopgitcmd

\startgitcmd{mergetool}
在合并冲突发生后,此命令会对每个冲突的文件调用第三方合并工具,比如
kdiff3(最好的图形三路合并工具)来进行可视化的合并,非常方便。
\stopgitcmd

\startgitcmd{mv}
重命名、移动文件或者目录,这个命令只是为了方便,GIT 并不在版本库里
记录这个操作。
\stopgitcmd

\startgitcmd{pull}
pull = fetch + merge,不推荐用这个命令。
\stopgitcmd

\startgitcmd{push}
与 fetch 功能类似,命令行格式一致,只是操作的方向相反,也接收 -f 选项。
\stopgitcmd

\startgitcmd{rebase}
将一个分支上的版本序列提取出来在另一个分支上重新提交一遍,这个命令可以
用来保证上游分支历史是直线的,有一点类似于 CVS 和 Subversion
里的 update 命令,只是更强大,能一次处理多个版本,在 CVS 和 Subversion
里,在某一个分支上工作的操作序列是这样的:

\starttyping
checkout <branch>
...modify....
update
commit
\stoptyping
update 操作保证了 commit 的提交总是在最新版本的基础上,这样分支历史
就是一条直线。

在 GIT 里,类似功能的操作序列是这样的:
\starttyping
git clone <remote_repos> <local_repos>
checkout -b <my_branch> <remotes/upstream_branch>
...can modify and commit many times...
...`fetch` or `remote update` to update <remotes/upstream_branch>
rebase <remotes/upstream_branch>
commit
push remote_repos <my_branch>:<upstream_branch>
\stoptyping

由于 GIT 的分布式版本控制特性,类似功能的命令 GIT 显得要繁琐的多。
\stopgitcmd

\startgitcmd{remote}
管理 remote branches。
\stopgitcmd

\startgitcmd{reset}
撤销 index 和 work tree 中未提交的修改。
\stopgitcmd

\startgitcmd{revert}
撤销已经提交的修改。 
\stopgitcmd

\startgitcmd{rm}
删除文件。
\stopgitcmd

\startgitcmd{shortlog}
统计提交情况。
\stopgitcmd

\startgitcmd{show-branch}
显示各个分支的提交情况,方便查看是否有提交需要合并或者 cherry-pick。
\stopgitcmd

\startgitcmd{stash}
暂存 index 和 work tree 的状态,稍候可以再恢复,非常有用的命令。
\stopgitcmd

\startgitcmd{status}
查看 index 和 work tree 中文件状态。
\stopgitcmd

\startgitcmd{svn}
与 Subversion 的双向交互,这样可以用 GIT 来当作更强更快的 Subversion 的客户端了。
\stopgitcmd

\startgitcmd{tag}
管理静态标签。
\stopgitcmd


\stopcomponent

