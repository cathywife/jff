\startcomponent preface
\product git-way-explained

\section{前言}

本文面向有一定版本控制经验的人群,对 GIT\footnote{\from[GIT]} 有基本了解,
如果有 Subversion\footnote{\from[Subversion]}、Mercurial\footnote{\from[Mercurial]}
等使用经验更好。文中从工具的原理和设计乃至实现出发,讲述了 GIT的用法,并
以个人愚见展示了一些 GIT 不完美的方面。文章名字本来想戏谑的起名《深入浅
出 GIT》,意指原理讲的较深(但愿),而具体单个命令用法讲的很浅,因为后者有
手册可查,但后来还是觉得有辱没类似名字命名的著作且混乱成语用法的嫌疑,遂
作罢。

以前初学 GIT 不久写过一篇《GIT 之五分钟教程》,当时以为自己对 GIT 了解的比
较好,但后来在工作中真正的用上 GIT 后,才发觉还是学习的很肤浅,遇到问题经常
觉得无所适从。GIT 秉承了 UNIX 的优良传统——小工具堆叠,“条条道路通罗马”,
可惜我常常不知道哪条路是最好走的。经过近一年的实战使用,我仍觉得 GIT 这玩意
还是没能运用自如,但总算有些心得体会,记录成文,希望对 GIT 用户有所帮助。

GIT 的命令行界面我觉得至今还是不能让人满意,我比较喜欢 Subversion、
Mercurial 的命令行界面:一致、简洁。GIT 属于那种每个人都想拼命往里面
塞功能,每个人都想让 GIT 具备自己喜欢特性的工具,结果就导致 GIT 如同
Shell 编程一般,“一切皆有可能”,虽然是颇有恶趣味,但也常让人厌烦。
这绝不会只是我个人的感受,遍观繁多的 GIT 包装工具就知道了,而且有一些
GIT 包装工具的做法被 GIT 吸收,可以说 GIT 正在成为一个怪物,一个让人又爱
又恨的怪物\footnote{我身边这种怪物还有 Perl、VIM。},它的运行速度非
常快——真的是非常快,它的设计思想非常简洁有效。

如果你要纳入版本控制的文件树规模不大\footnote{大于 500 MB 你就要慎重考虑
了,Subversion 没有文件复制自动探测,用户如果粗心的用 svn add 来替代 svn
copy,那么会导致版本库迅速膨胀,而且 Subversion 的工作拷贝实现原理比较低
效,类似操作比起 GIT 慢得多,著名的开源项目 FreeBSD 从 CVS 转向 Subversion,
我是觉得很可惜的。},如果你不关心分布式开发,那么对于版本控制工具,我向
你推荐Subversion, 集中式版本控制工具的佼佼者,有着非常友好的命令行和图
形使用界面,否则,如果你不是必须使用 GIT,那么我推荐Mercurial,目前分布
式版本控制工具里唯一可以跟 GIT 抗衡的选手\footnote{虽然我欣赏 Canonical
的Ubuntu Linux 发行版,但是对它家的 Bazaar 并没有多大好感,从几次使用印
象来看这东西效率太太太低了,表面友好的使用界面掩盖不住其繁复的设计。},
Sun的OpenJDK、OpenSolaris 这样的大规模项目都是使用Mercurial 管理的。这两
者都已经被用在大量开源项目里,而且可移植性非常好。

啰嗦了这么多,该说感谢致辞了,首先要感谢 NewSMTH BBS 的 donated 和 garfileo。
大概是一年前,donated 非常耐心的给我介绍了 \ConTeXt,但那时能方便使用新
字体的 \ConTeXt\ MkIV 还没有成熟,我也就没提起兴趣继续学习。不久前我阅读
了 garfileo 的《\ConTeXt\ 学习笔记》\footnote{\from[ctxnotes]},
才知道\ConTeXt\ MkIV 居然已堪实用,蒙这篇笔记的引导才开始慢慢学习
\ConTeXt,本文就是我正式用 \ConTeXt\ 排版的第一篇文章\footnote{就这版面设计
还很“憨厚”,另外 \ConTeXt\ MkIV 中文排版还有些问题,有待改进。}。其次我要感谢我的同事,我觍颜作为他们的
GIT 传教者以及版本库维护者,正是由于跟他们的合作促使我认真思索应该如何使
用 GIT——从“怎么用”到“怎么用好”\footnote{希望以后能明白“怎么用最好
”:-)}是个很大的转变。

\stopcomponent

