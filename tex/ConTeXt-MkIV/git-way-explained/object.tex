\startcomponent object
\product git-way-explained

\section{对象记法}

关于版本的记法在 git help rev-parse 中有详细描述,必须学会的是
\type{~} 和 \type{^} 记法的含义。

在 git 文档里常能碰上 commit-ish 和 tree-ish 这两个词,commit-ish
的意思是能推导出版本号的字符串,比如 SHA1 摘要,包含 \type{~} 和 \type{^} 记法
的版本号,版本的日期记法,标签对象的摘要,标签名字,分支名字,以及
这些记法的组合。tree-ish 含义以此类推。

指定某个版本的文件树中内容用: tree-ish:fullpath 语法,
比如 v2.6.27:init/main.c 就表示 linux-v2.6.27 源码中
的 init/main.c 文件,注意 init 前面没有“/”,而且也
不能是相对于当前目录的相对路径名,比如在 init/ 目录下
也不能用 v2.6.27:./main.c。

[ XXX: 如果目录树很深,不能用相对路径很不方便; 冒号记法
  可能跟 Windows 盘符记法冲突,比如有个分支 C,那么可能
  不小心引用了 C: 盘的文件 ]

查看对象内容可以用 git cat-file -p xxx 或者 git show xxx 。

由于 git 的命令行指定版本并没有用 -r REV 这样的方式,
而是直接用 REV 作为参数,如果这个命令同时还能接收文件名,
那么往往就会将 REV 和 file 参数混淆,此时需要在 file 参数
前面添加“--”参数。

[ XXX: git 有时能自动分清,有时不行]

由于 git 自己实现了命令行选项解析,因此选项不能像 rm aa -f
那样放到后面。

[ XXX: 重造车轮]


git log 参数是个 commit-ish,其含义是说查看从这个版本能够
回溯到的所有版本的日志信息,这跟 svn log 看单个分支的直线
历史是不一样的。git log 的这种行为导致看非线性历史的日志
时完全不可用,虽然 git log 后面可以用 .. 和 \type{^} 记号做版本树
剪枝,但还是没有 tig 和 gitk 好用。 而且由于 git merge
的 fast forward 特性,导致即使 git log --first-parent 看的
也未必是分支{\bf 头}曾经指代的那个轨迹。

由于 git log 的版本树回溯特性,因此看某个版本之后 N 个版本
的日志很不方便,如果要查看的这 N 个版本是线性演化的,那么
可以用下面的命令 \$ver 和 \$ver 之后第 N 个版本的区别:
\starttyping
git diff $ver `git rev-list $ver.. | tail -$n | head -1`
\stoptyping

\stopcomponent

