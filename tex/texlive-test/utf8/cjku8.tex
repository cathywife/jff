\documentclass[12pt]{article}
%\usepackage{CJK}
\usepackage{CJKutf8}
\usepackage[dvipdfm,unicode,bookmarks=true,bookmarksnumbered=true]{hyperref}
\usepackage{ccmap} %ʹpdflatexɲǩȷƣ
\begin{document}
\begin{CJK*}{UTF8}{song}

\title{哀江南賦並序}
\author{庾信}
\date{}
\maketitle \tableofcontents
\section{將進酒}
粵以戊辰之年,建亥之月,大盜移國,金陵瓦解。余乃竄身荒谷,
公私塗炭。華陽奔命,有去無歸:中興道銷,窮於甲戌。三日哭
於都亭,三年囚於別館。天道周星,物極不反。傅燮之但悲身世,
無處求生;袁安之每念王室,自然流涕。

昔桓君山之志事,杜元凱之平生,並有著書,咸能自序。潘岳之
文采,始述家風;陸機之辭賦,先陳世德。信年始二毛,即逢喪
亂,藐是流離,至於暮齒。《燕歌》遠別,悲不自勝;楚老相逢
,泣將何及!畏南山之雨,忽踐秦庭;讓東海之濱,遂餐周粟。
下亭漂泊,高橋羈旅。楚歌非取樂之方,魯酒無忘憂之用。追為
此賦,聊以記言;不無危苦之辭,惟以悲哀為主。

日暮途遠,人間何世?將軍一去,大樹飄零;壯士不還,寒風蕭
瑟。荊璧睨柱,受連城而見欺;載書橫階,捧珠盤而不定。鍾儀
君子,入就南冠之囚;季孫行人,留守西河之館。申包胥之頓地,
碎之以首;蔡威公之淚盡,加之以血。釣臺移柳,非玉關之可望;
華亭鶴唳,豈河橋之可聞。

孫策以天下為三分,眾纔一旅,項籍用江東之子弟,人惟八千,
遂乃分裂山河,宰割天下;豈有百萬義師,一朝卷甲;芟夷斬伐,
如草木焉!江淮無涯岸之阻,亭壁無籓籬之固。頭會箕歛者合從
締交,鋤耰棘矜者因利乘便;將非江表王氣,終於三百年乎?

是知并吞六合,不免軹道之災;混一車書,無救平陽之禍。嗚呼!
山嶽崩頹,既履危亡之運;春秋迭代,必有去故之悲:天意人事,
可以悽愴傷心者矣。況復舟楫路窮,星漢非乘槎可上;風飆道阻,
蓬萊無可到之期。窮者欲達其言,勞者須歌其事。陸士衡聞而撫
掌,是所甘心;張平子見而陋之,固其宜矣!

\end{CJK*}
\end{document}
