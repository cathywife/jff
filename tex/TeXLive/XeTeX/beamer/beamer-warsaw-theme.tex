\documentclass[CJK]{beamer}
\usepackage[boldfont,slantfont,CJKnumber,CJKtextspaces]{xeCJK}

\punctstyle{kaiming}
\setCJKmainfont{SimSun}
\setCJKmonofont{STFangsong}

\mode<presentation>
{
 % 设置背景
 \setbeamertemplate{background canvas}[vertical shading][bottom=red!10,top=blue!10]

 % 设置block的特征
 \setbeamertemplate{blocks}[rounded][shadow=true]

 % 设置主题
 \usetheme{Warsaw}

 % 设置footline显示页码,但是由于warsaw本身已经定义了一个footline,所以
 % 这个定义就会覆盖warsaw的定义。另一方面说明,slides的每一部分都是可以
 % 自己定制的。
 % \setbeamertemplate{footline}[frame number]

 % 设置覆盖的效果,透明
 \setbeamercovered{transparent}

 \usefonttheme[onlysmall]{structurebold}

 % 设置数学公式的字体
 \usefonttheme[onlymath]{serif}
}

\usepackage{latexsym}
\usepackage{amssymb}

\setlength{\parindent}{2em}

% 设置用acrobat打开就会全屏显示
\hypersetup{pdfpagemode=FullScreen}

% 设置logo
% If you have a file called "university-logo-filename.xxx", where xxx
% is a graphic format that can be processed by latex or pdflatex,
% resp., then you can add a logo as follows:

% \pgfdeclareimage[height=0.5cm]{university-logo}{pku_logo}
% \logo{\pgfuseimage{university-logo}}



%%%%%%%%%%%%%%%%%%%%%%%%%%%%%%%%%%%%%%%%%%%%%%%
%%
%% Beginning of document
%%
%%%%%%%%%%%%%%%%%%%%%%%%%%%%%%%%%%%%%%%%%%%%%%%
\begin{document}

% Delete this, if you do not want the table of contents to pop up at
% the beginning of each subsection:
\AtBeginSubsection[]
{
 \begin{frame}<beamer>
   \frametitle{主要内容}
   \tableofcontents[currentsection,currentsubsection]
 \end{frame}
}


\title[中文Slides模板]
{\huge 模态逻辑(model logic)\\ \normalsize 中文Slides模板}

\author[沈坚]{沈 坚 \\ \small shenjian@net.pku.edu.cn}

\institute[北京大学]
{
 网络实验室 \\
 北京大学
}

\date[]{2004-11-11}

%%%%%%%%%%%%%%%%%%%%%%%%%%%%%%%%%%%%%%%%%%%%%%%%%
\begin{frame}
 \titlepage
\end{frame}

%%%%%%%%%%%%%%%%%%%%%%%%%%%%%%%%%%%%%%%%%%%%%%%%%
\begin{frame}
 \frametitle{主要内容}
 \tableofcontents
 % You might wish to add the option [pausesections]
\end{frame}

% \begin{frame}
%   \frametitle{主要内容}
%   \tableofcontents[currentsection]
% \end{frame}

%%%%%%%%%%%%%%%%%%%%%%%%%%%%%%%%%%%%%%%%%%%%%%%%%

\section{引言:从模态词说起}

\begin{frame}
 \frametitle{为什么需要模态逻辑}
 \begin{itemize}
 \item<1-> 从推理的角度来说
   \begin{itemize}
   \item 如果认为模态词反映的只是“趋势”、“信念”,不是逻辑概念,那么在日常推理中就寸步难行,因为 “A真但是未必A必然真”
   \end{itemize}
 \item<2-> 从历史的角度来说
   \begin{itemize}
   \item 模态逻辑的起源:\alert{ \large 亚里士多德}
   \end{itemize}
 \item<3-> 从功能性方面来说
   \begin{itemize}
   \item 是除了一阶谓词演算,计算机界谈论最多的逻辑系统
   \item 不仅是程序语义描述的有力工具,时序逻辑和动态逻辑的理论基础,而且在知识的形式表示方面表现出越来越多的优越性
   \end{itemize}
 \end{itemize}
\end{frame}

%%%%%%%%%%%%%%%%%%%%%%%%%%%%%%%%%%%%%%%%%%%%%%%%%

\section{模态逻辑的非形式讨论}

\subsection{必然A,可能A的含义和性质}

\begin{frame}
 \frametitle{$\Box A$和$\Diamond A$的意义}
 \begin{enumerate}
 \item 必然A(necessarily A),记作$\Box A$
   \begin{itemize}
   \item 意思是:无论在什么场合(现实的场合或者可以想象到的非现实的场合)均有事实A
   \end{itemize}
 \item 可能A(possibly A),记作$\Diamond A$
   \begin{itemize}
   \item 意思是:对某些个场合(也许只是一个,甚至只是想象得到的某个场合)有事实A
   \end{itemize}
 \item 公式的例子

\begin{displaymath}
 g_{C_o}(c_i) = \frac{|\{\bar{d}_j\in C_o | ca_{ij} = 1\}|}{|\{\bar{d}_j\in C_o\}|}
\end{displaymath}

 \end{enumerate}
\end{frame}

%%%%%%%%%%%%%%%%%%%%%%%%%%%%%%%%%%%%%%%%%%%%%%%%%

\subsection{演化的克里普克语义结构}

\begin{frame}
 \frametitle{克里普克语义结构}
克里普克结构$\mathcal{K}$可用三元矢$<U,D,I>$来表示
\begin{itemize}
\item 非空集合$U$,称为\alert{空间},$U$上定义了一个偏序关系$\leqslant$。$U$中元素称为阶段
\item $U$到一非空集合上的映射$D$,即对每一$k\in U$, $D_k$为一非空集合,称为阶段$k$的\alert{个体域}。$D$还满足,对任意$k,l\in U$,若$k \leqslant l $,那么$D_k \subseteq D_l$
\item $U$到解释的集合上的映射$I$,即对每一$k\in U$,$I_k$表示一个相关于阶段$k$的\alert{解释}
\end{itemize}
\end{frame}

%%%%%%%%%%%%%%%%%%%%%%%%%%%%%%%%%%%%%%%%%%%%%%%%%
\section{模态逻辑正规系统及其语义}

\subsection{模态语言及模态逻辑正规系统NSK}

\begin{frame}
 \frametitle{几个概念}
 \begin{itemize}
 \item 模态语言:模态命题演算的语言的简称
 \item \alert{正规系统NS:模态逻辑系统NSK及其所有扩充系统}
 \item 模态逻辑(正规)系统(normal system)NSK
 \item 模态逻辑(正规)系统的语言$\mathcal{L}$(NSK)

$\mathcal{L}$(NSK)是命题演算形式系统  FSPC的语言$\mathcal{L}$(FSPC)的扩充
 \end{itemize}
\end{frame}
%%%%%%%%%%%%%%%%%%%%%%%%%%%%%%%%%%%%%%%%%%%%%%%%%

\begin{frame}
 \frametitle{NSK的定理}
 \begin{block}{定理7.1}
   如果公式$A$是NSK的定理,那么$\Box A$也是NSK的定理,即 $\vdash_{NSK} A$ 蕴含 $\vdash_{NSK} \Box A$
 \end{block}
 \begin{block}{定理7.2}
   设 $A_1,A_2,\cdots,A_n,A$ 均为NSK的公式,如果  $\vdash_{NSK}(A_1 \to (A_2 \to \cdots \to (A_n \to A)\cdots))$ (或 $\vdash_{NSK} A_1 \wedge \cdots \wedge A_n \to A$),那么 $(\Box A_1 \to (\Box A_2 \to \cdots \to (\Box A_n \to A)\cdots))$ (或 $\Box A_1 \wedge \Box A_2 \cdots \wedge A_n \to A$)
 \end{block}
 \begin{block}{定理7.3}
   如果公式 $\neg A$ 是NSK的定理,那么 $\neg \Diamond A$ 也是NSK的定理
 \end{block}
\end{frame}

%%%%%%%%%%%%%%%%%%%%%%%%%%%%%%%%%%%%%%%%%%%%%%%%

\subsection{正规结构}

\begin{frame}
 \frametitle{NSK的正规结构}
正规系统的语义结构称为\alert{正规结构}(称模态逻辑系统NSK及其所有扩充系统为正规系统)

\begin{block}{定义7.2}
模态逻辑正规系统的正规结构,是指三元矢$<U,R,I>$,其中:

$U$为一非空集合,称为宇宙,其成员称为可能世界,可能世界用$w$,$w'$,$w_1$,$w_2$等表示;

$R$为$U$上的一个二元关系,称为可能世界间的可到达关系(注意,$R$未必为偏序关系);

$I $为$U\times\{P_1,P_2,P_3,\cdots\}$到$\{0,1\}$上的映射,即对每一个可能世界$w$,对每一原子命题赋值。$ I(w, P_1)=1$表示在可能世界$w$中给$P_1$赋值真。$I$常用$U$的一个子集序列$\mathcal{B}_1,\mathcal{B}_2, \mathcal{B}_3,\cdots$ 来表示,使 $I(w, P_i)=1$ 当且仅当 $w\in \mathcal{B}_i (i=1,2,3,\cdots)$。换言之,$\mathcal{B}_i$是给原子命题$P_i$赋值真的可能世界的集合。
\end{block}
\end{frame}

%%%%%%%%%%%%%%%%%%%%%%%%%%%%%%%%%%%%%%%%%%%%%%%%%

\begin{frame}
 \frametitle{正规系统NSK的性质}
证明$\Box A \to \Diamond A$不是NSK中的定理:
\vspace{0.5em}

设$ \mathcal{K}$中有可能世界$w$,它与所有可能世界(包括它自身)均无$R$关系,因此对任何公式$A$必有$\vDash^w_k \Box A$(对所有$w'$, 若$wRw'$ 则 $\vDash^{w'}_k A$ 真, 以及$\nvDash^w_k \Diamond A$ (存在$w'$,$wRw'$ 且$\nvDash^{w'}_k A$ 假),因此$\nvDash^w_k \Box A \to \Diamond A$。
\vspace{0.5em}

从这个例子可以看出,要使 $\Box A \to \Diamond A$ 永真,必须对正规结构加以限制,即缩小$\mathcal{K}$,使每一结构的可达到关系$R$是“连续”的,即对每一可能世界$w$均有$w'$使$wRw'$。
\end{frame}

%%%%%%%%%%%%%%%%%%%%%%%%%%%%%%%%%%%%%%%%%%%%%%%%%
\subsection{关于正规系统的重要元定理}


\begin{frame}
 \frametitle{关于正规系统的重要元定理}
 \begin{block}{定理7.8(演绎定理)}
   对任何NSK中公式集$\Gamma$及公式$A,B$,,$\Gamma \vdash_{NSK} A \to B$,当且仅当$\Gamma$;$A \vdash_{NSK} B$.
 \end{block}
 \begin{block}{定理7.9(替换定理)}
   在NSK中,如果$A$ H $B$,$A$是$C$的子公式,而$D$是$C$中将$A$的若干出现换成$B$的替换式,那么$C$ H $D$
 \end{block}
 \begin{block}{定理7.11}
   NSK是可判定的
 \end{block}

\end{frame}

%%%%%%%%%%%%%%%%%%%%%%%%%%%%%%%%%%%%%%%%%%%%%%%%%

\section*{总结与回顾}

\begin{frame}
 \frametitle{总结与回顾}
 \begin{itemize}
 \item 引言:从模态词说起
 \item 模态逻辑的非形式讨论
   \begin{itemize}
   \item 必然A,可能A的含义和性质
   \item 演化的克里普克语义结构
   \end{itemize}
 \item 模态逻辑正规系统及其语义
   \begin{itemize}
   \item 模态语言及模态逻辑正规系统NSK
   \item 正规结构
   \item 关于正规系统的重要元定理
   \end{itemize}
 \end{itemize}
\end{frame}

%%%%%%%%%%%%%%%%%%%%%%%%%%%%%%%%%%%%%%%%%%%%%%%%%
\begin{frame}
 \begin{center}
   \huge Thanks for your attention! \\ Q \& A
 \end{center}
\end{frame}

%%%%%%%%%%%%%%%%%%%%%%%%%%%%%%%%%%%%%%%%%%%%%%%%%

\end{document}

