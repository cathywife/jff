\documentclass{beamer}

\mode<presentation>
{
  \setbeamertemplate{background canvas}[vertical shading][bottom=red!10,top=blue!10]
  \setbeamertemplate{blocks}[rounded][shadow=true]
  \usetheme{Warsaw}
  \setbeamercovered{transparent}
  \usefonttheme[onlysmall]{structurebold}
}


\usepackage[english]{babel}
\usepackage[latin1]{inputenc}
\usepackage{times}
\usepackage[T1]{fontenc}
\usepackage{listings}
\lstloadlanguages{Perl,bash}
\lstset{language=Perl,numbers=left, numberstyle=\tiny, stepnumber=1, numbersep=5pt,
        frame=lines, captionpos=b, basicstyle=\scriptsize}


\title{Perl Tutorial}
\author{Yubao Liu \\ \texttt{liuyb@yahoo-inc.com}}
\institute{Yahoo! Global R \& D Center, Beijing}
\date{2011-02-26}


\hypersetup{pdfpagemode=FullScreen}
\subject{Perl Tutorial}

%\pgfdeclareimage[height=0.5cm]{yahoo-logo}{yahoo-logo.jpg}
%\logo{\pgfuseimage{yahoo-logo}}

\AtBeginSection[]
{
  \begin{frame}<beamer>{Outline}
    \tableofcontents[currentsection,currentsubsection]
  \end{frame}
}


\begin{document}

\begin{frame}
  \titlepage
\end{frame}

\begin{frame}{Outline}
  \tableofcontents
  % You might wish to add the option [pausesections]
\end{frame}

\section{Preface}

\begin{frame}{How to get help}
  \begin{itemize}
    \item perldoc perl
    \item perldoc perlfunc, perldoc perlvar, perldoc perlre
    \item perldoc -f open
    \item perldoc File::Find, perldoc Carp
  \end{itemize}
\end{frame}

\begin{frame}{Books}
  \begin{itemize}
    \item  Learning Perl, Learning Perl on Win32 System
    \item  Advanced Perl Programming
    \item  Programming Perl
    \item  Intermediate Perl, Mastering Perl
    \item  Perl Best Practices
    \item  Modern Perl
  \end{itemize}
\end{frame}

\section{Basic Syntax}

\begin{frame}[containsverbatim]{Hello World!}
\begin{lstlisting}[caption=Greeting from Perl]
#!/usr/bin/perl
use strict;
use warnings;

my $name = <STDIN>;
chomp($name);
greet($name);

sub greet {
    print "Hello $_[0]!\n";
}
\end{lstlisting}

Running: \texttt{perl hello.pl}
\end{frame}

\begin{frame}[containsverbatim]{Variable}
  \begin{description}
    \item[Scalar]   \lstinline!$num = 3.3; $name = "Jack"!
    \item[Array]    \lstinline!@array = (1, 3, "Jack")!, \lstinline!$var[0]!
    \item[Hash]     \lstinline!%hash = ("k1" => "v1", "k2" => "v2")!, \lstinline!$hash{"k1"}!,
                    key must be number or string, value must be number, string or reference
  \end{description}
\end{frame}

\begin{frame}[containsverbatim]{Useful Operators and subroutines}
  \begin{itemize}
    \item  \$s x 3, \$a . \$b, chomp(), split(), defined(), undef()
    \item  scalar(@array), \$\#array, grep(), map(), reverse(), sort(), join(),
           shift(), unshift(), pop(), push()
    \item  keys(\%hash), values(\%hash), each(\%hash), exists(), delete()
    \item  \lstinline!@array[@indices]!, \lstinline!@hash{@keys}!
  \end{itemize}
\end{frame}

\begin{frame}[containsverbatim]{Conversion among variables}
  \begin{itemize}
    \item \lstinline!my @array = ($a, $b); my ($c, $d) = @array;!
    \item \lstinline!($a, $b) = ($b, $a); @a = (@b, @c); ($a, @a) = @c!
    \item \lstinline!%hash = @a;!
  \end{itemize}
\end{frame}

\begin{frame}{Contexts}
  \begin{itemize}
    \item Void context, \texttt{print "...."}
    \item Scalar context, \texttt{\$i < @a; \$i < scalar(@a);}
    \item Array context, \texttt{(stat \$f)[7]}
    \item Numeric context, \texttt{\$a + \$b}
    \item String context, \texttt{\$a\ .\ \$b}
    \item Dual var, \texttt{\$!, Scalar::Util::dualvar()}
    \item wantarray()
  \end{itemize}
\end{frame}

\begin{frame}[containsverbatim]{Clause}
 \begin{itemize}
  \item \lstinline!if (...) {...} elsif (...) {...} else {....}!
  \item \lstinline!while (...) {...}!
  \item \lstinline!for my $a (@a) {...}; for (my $i = 0; $i < @a; ++$i) {...}!
  \item \lstinline!... if ...; ... while ...;!
  \item \lstinline!next!, \lstinline!last!
  \item \lstinline!BEGIN {...}!, \lstinline!END {...}!
 \end{itemize}
\end{frame}

\begin{frame}[containsverbatim]{Subroutine}
  \begin{itemize}
    \item \lstinline!$_!, \lstinline!@_!
    \item \lstinline!sub foo {my @args = @_; ...}!
    \item \lstinline!my $i = 0; my $f = sub {print "Now ", ++$i, "\n"}; $f->(); &$f()!
  \end{itemize}
\end{frame}

\begin{frame}[containsverbatim]{I/O}
  \begin{itemize}
    \item  STDIN, STDOUT, STDERR
    \item  \lstinline{print STDERR "Can't find command!\n";}
    \item  \lstinline{my $fh = \*STDIN; print $fh "hello\n";}
    \item  \lstinline{open my $fh, "/etc/passwd" or die "Can't open: $!\n";}
    \item  \lstinline!while (<$fh>) { chomp; print "$_\n"; }!
    \item  \lstinline!close($fh)!
  \end{itemize}
\end{frame}

\begin{frame}[containsverbatim]{I/O (cont.)}
  \begin{itemize}
    \item  Buffered IO: open, <>, print, printf, seek, tell
    \item  Unbuffered IO: sysopen, sysread, syswrite, sysseek
    \item  opendir(), readdir(), closedir(), glob()
    \item  IO::File, IO::Dir
  \end{itemize}
\end{frame}

\section{Intermediate Syntax}

\begin{frame}[containsverbatim]{Reference}
  \begin{itemize}
    \item \lstinline{$a_ref = \$a; $$a_ref = 3;}
    \item \lstinline{$a_ref = \@a; @$a_ref = (1, 2); $a_ref->[0] = 2;}
    \item \lstinline!$a_ref = \%h; %$a_ref = (1 => 2); $a_ref->{1} = 3;!
    \item \lstinline!$a_ref = \&func_foo; &$a_ref(); $a_ref->();!
  \end{itemize}
\end{frame}

\begin{frame}[containsverbatim]{Reference (cont.)}
  \begin{itemize}
    \item \lstinline{$a_ref = [1, 2, 3]; $a_ref->[0] *= 2;}
    \item \lstinline!$a_ref = {1 => 2}; $a_ref->{1} = 3;!
    \item Autovivification: \lstinline!my $a_ref; $a_ref->[1]{name} = "Jack";!
    \item \lstinline!${$a[0]}, @{$a[0]}, %{$a[0]}!
  \end{itemize}
\end{frame}

\begin{frame}[containsverbatim]{Calling External Program}
  \begin{itemize}
    \item \lstinline{my $content = `ls`; $content = qx/ls/}
    \item \lstinline{system()}
    \item \lstinline{open()}
    \item \lstinline{fork() + exec() + pipe()}
    \item IPC::Open2, IPC::Open3, IPC::Cmd, IPC::Run
  \end{itemize}
\end{frame}

\begin{frame}[containsverbatim]{Exception}
\begin{lstlisting}[caption=Vanilla exception processing]
eval {
  open my $fh, "some_file" or die "Can't open";
  die "Empty!" if -z $fh;
  close $fh;
};

if ($@) {
  if ($@ =~ /Can't open/) {
  } elsif ($@ =~ /Empty!/) {
  } else {
  }
}
\end{lstlisting}
\end{frame}

\begin{frame}[containsverbatim]{Exception (cont.)}
  \begin{itemize}
    \item Try::Tiny, better than vanilla exception processing
    \item Exception::Class
  \end{itemize}
\end{frame}

\begin{frame}[containsverbatim]{Error Indicators}
  \begin{description}
    \item[\$@] Errors detected by Perl interpreter
    \item[\$!] Errors reported by C library, like "errno"
    \item[\$?] 16-bit status code of external program
  \end{description}
\end{frame}

\begin{frame}[containsverbatim]{Installing Modules}
  \begin{itemize}
    \item local::lib
    \item cpan, cpanp
    \item cpanm, App::cpanminus
    \item \url{http://search.cpan.org}
    \item \url{http://search.metacpan.org}
    \item \url{http://deps.cpantesters.org}
  \end{itemize}
\end{frame}

\begin{frame}[containsverbatim]{Using Modules}
  \begin{itemize}
    \item \texttt{perldoc -q @INC}
    \item use Carp qw/cluck/;
    \item use CGI ();
    \item use Fcntl qw(:DEFAULT :flock);
    \item require, do
    \item \%INC
  \end{itemize}
\end{frame}

\begin{frame}[containsverbatim]{Writing Modules}
\begin{lstlisting}[caption=Some/Module.pm]
package Some::Module;
use strict;
use warnings;
use Exporter 'import';

our $VERSION = 1.00;
our @EXPORT  = qw/&subA &subB $varA @varB/;
our @EXPORT_OK = qw/&subC &subD/;
our %EXPORT_TAGS = (tag1 => [ qw/..../ ], tag2 => [ qw/..../ ]);

our ($varA, @varB);

sub subA {
}

1;
\end{lstlisting}
\end{frame}

\begin{frame}[containsverbatim]{Object Oriented Perl}
  \begin{itemize}
    \item Class in Perl is represented by package
    \item A subroutine usually named \texttt{new} returns blessed reference
    \item Method is just subroutine whose first argument is object
    \item Special \texttt{DESTROY} subroutine
    \item class \emph{UNIVERSAL}
  \end{itemize}
\end{frame}

\begin{frame}[containsverbatim]{Object Oriented Perl (cont.)}
\begin{lstlisting}[caption=Define a class]
package Some::Class;
use strict;
use warnings;

our $VERSION = 1.0;
our @ISA = qw/BaseA BaseB/;  # or "use parent ...;" or "use base ...;"

sub new {
  my ($class, @args) = @_;
  bless {}, $class;
}

sub methodA {
  my ($self, @args) = @_;
}

1;
\end{lstlisting}
\end{frame}

\begin{frame}[containsverbatim]{Object Oriented Perl (cont.)}
\begin{lstlisting}[caption=Use a class]
#!/usr/bin/perl
use strict;
use warnings;
use Some::Class;

my $obj = Some::Class->new();
$obj->methodA();
\end{lstlisting}
\end{frame}

\begin{frame}[containsverbatim]{Others}
  \begin{itemize}
    \item Regular Expression, perldoc perlrequick/perlretut/
          perlre/perlreref/perlrebackslash/perlrecharclass
    \item Thread support, perldoc threads, Thread::Queue, Thread::Semaphore
    \item Unicode support, byte string vs. text string, \lstinline{use utf8; use Encode;}
  \end{itemize}
\end{frame}

\begin{frame}[containsverbatim]{Recommended Modules}
  \begin{itemize}
    \item Data::Dumper, Carp, CGI::Carp
    \item Devel::Cover, Devel::NYTProf, Devel::Size
    \item Config::Any, Config::General, Config::MVP
    \item Log::Any, Log::Dispatch, Log::Log4Perl
    \item HTTP::Tiny, LWP, WWW::Mechanize
    \item HTML::TreeBuilder, HTML::TreeBuilder::XPath, XML::Simple, XML::LibXML
    \item Template::Toolkit, HTML::Template::Pro, HTML::Template::Compiled
  \end{itemize}
\end{frame}

\begin{frame}[containsverbatim]{Recommended Modules (cont.)}
  \begin{itemize}
    \item AnyDBM\_File, BerkeleyDB
    \item DBI, DBIx::Class, DBIx::Simple
    \item Class::Struct, Moose, Mouse, Any::Moose
    \item Parallel::Prefork
    \item Cache::Cache, CHI
  \end{itemize}
\end{frame}

\begin{frame}
  \begin{center}
    \_\_END\_\_
  \end{center}
\end{frame}

\end{document}

